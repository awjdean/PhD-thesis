%%%%%%%%%%%%%%%%%%%%%%%%%%%%%%%%%%%%%%%%%%%%%%%%
\chapter{Further category theory}
%%%%%%%%%%%%%%%%%%%%%%%%%%%%%%%%%%%%%%%%%%%%%%%%
\section{Action labelling map}
%%%%%%%%%%%%%%%%%%%%%%%%%%%%%%%%%%%%%%%%%%%%%%%%
\subsection{Action labelling maps as a pushout}

Can we view the gluing of the world states in $\mathcal{F}(\hat{\mathscr{W}}_{\mathscr{A}}^{\bot})$ through the action labelling functor $L^{\bot}$ as a coequalizer.
The universal property of the coequalizer states that, for any category $\mathcal{C}$ and any functor $H: \mathcal{F}(\hat{\mathscr{W}}_{\mathscr{A}}^{\bot}) \to \mathcal{C}$ that identifies all objects (i.e., $H \circ F = H \circ G$), there is a unique functor $H': \textbf{B}\hat{A}^{*} \to \mathcal{C}$ with
\begin{equation}
    H' \circ q = H.
\end{equation}
The labelling functor $L^{\bot}$ is exactly such a functor $H$ since it sends every object in $\mathcal{F}(\hat{\mathscr{W}}_{\mathscr{A}}^{\bot})$ to the unique object $\bullet$ of $\textbf{B}\hat{A}^{*}$, and so we can identify $\textbf{B}\hat{A}^{*}$ as the coequalizer (i.e., the quotient) of $\mathcal{F}(\hat{\mathscr{W}}_{\mathscr{A}}^{\bot})$ by the equivalence that identifies all objects (i.e., $w \sim w'$ for all $w,w' \in W^{\bot}$ and relates morphisms accord to their labels (i.e., $f \sim g \iff L^{\bot}(f) = L^{\bot}(g)$).
As a diagram:
\begin{equation}
    % https://q.uiver.app/#q=WzAsOSxbMCwwLCJcXG1hdGhjYWx7SX0iXSxbMiwwLCJcXG1hdGhjYWx7Rn0oXFxoYXR7XFxtYXRoc2Nye1d9fV97XFxtYXRoc2Nye0F9fV57XFxib3R9KSJdLFszLDFdLFs0LDFdLFswLDIsIlxcbWF0aGNhbHtGfShcXGhhdHtcXG1hdGhzY3J7V319X3tcXG1hdGhzY3J7QX19XntcXGJvdH0pIl0sWzUsMiwiXFxtYXRoY2Fse0Z9KFxcaGF0e1xcbWF0aHNjcntXfX1fe1xcbWF0aHNjcntBfX1ee1xcYm90fSkiXSxbNSwwLCJcXG1hdGhjYWx7SX0iXSxbNywwLCJcXG1hdGhjYWx7Rn0oXFxoYXR7XFxtYXRoc2Nye1d9fV97XFxtYXRoc2Nye0F9fV57XFxib3R9KSJdLFs3LDIsIlxcdGV4dGJme0J9XFxoYXR7QX1eeyp9Il0sWzAsMSwicF8xIiwwLHsib2Zmc2V0IjotMX1dLFsyLDMsIlxcdGV4dHtwdXNob3V0fSJdLFswLDQsInBfezJ9IiwyLHsib2Zmc2V0IjoxfV0sWzYsNSwicF97Mn0iLDJdLFs2LDcsInBfezF9Il0sWzcsOCwiTF57XFxib3R9Il0sWzUsOCwiTF57XFxib3R9IiwyXV0=
\begin{tikzcd}[ampersand replacement=\&]
    {\mathcal{I}} \&\& {\mathcal{F}(\hat{\mathscr{W}}_{\mathscr{A}}^{\bot})} \&\&\& {\mathcal{I}} \&\& {\mathcal{F}(\hat{\mathscr{W}}_{\mathscr{A}}^{\bot})} \\
    \&\&\& {} \& {} \\
    {\mathcal{F}(\hat{\mathscr{W}}_{\mathscr{A}}^{\bot})} \&\&\&\&\& {\mathcal{F}(\hat{\mathscr{W}}_{\mathscr{A}}^{\bot})} \&\& {\textbf{B}\hat{A}^{*}}
    \arrow["{p_1}", shift left, from=1-1, to=1-3]
    \arrow["{p_{2}}"', shift right, from=1-1, to=3-1]
    \arrow["{p_{1}}", from=1-6, to=1-8]
    \arrow["{p_{2}}"', from=1-6, to=3-6]
    \arrow["{L^{\bot}}", from=1-8, to=3-8]
    \arrow["{\text{pushout}}", from=2-4, to=2-5]
    \arrow["{L^{\bot}}"', from=3-6, to=3-8]
\end{tikzcd}
\end{equation}
\draftnote{blue}{Consider}{Does this perspective mean that we can construct $A^{\bot}$ as a universal map from $\textbf{B}\hat{A}^{*}$ to $\mathcal{D}$ from the universal property of the coequalizer.}






%%%%%%%%%%%%%%%%%%%%%%%%%%%%%%%%%%%%%%%%%%%%%%%%
\subsection{Pullback Diagram for Recovering $\ast^{\bot}$:}

\begin{equation}
\begin{tikzcd}[row sep=large, column sep=large]
\mathcal{P} \ar[r, "p_2"] \ar[d, "p_1"'] \ar[dr, phantom, "\lrcorner", very near start] & \mathcal{F}(\hat{\mathscr{W}}_{\mathscr{A}}^{\bot}) \ar[d, "L^{\bot}"] \\
\mathbf{B}\hat{A}^* \times \mathcal{F}(\hat{\mathscr{W}}_{\mathscr{A}}^{\bot}) \ar[r, "\pi_2"'] & \mathbf{B}\hat{A}^*
\end{tikzcd}
\end{equation}
This is saying that the totalised action effect $\ast^{\bot}$ can be recovered by pulling back along the labelling functor $L^{\bot}$.
The pullback object $\mathcal{P}$ collects exactly those pairs $(a,d)$ where $a$ is the label of $d$ (i.e., $l(d) = a$).

%%%%%%%%%%%%%%%%%%%%%%%%%%%%%%%%%%%%%%%%%%%%%%%%
\section{Translating from the world to the agent's representation (category theory)}
\draftnote{blue}{Include}{
\begin{enumerate}
    \item Can we build an adjunction from the category of worlds (not the category of world states) to the category of representations ?
    \item What structures can we bring through the adjunction and what structures can we not ?
    \item What happens to the adjunction when we add noise to the sensors ?
    \begin{enumerate}
        \item I think a random "vector" gets added to the components of the $\eta$ and $\epsilon$ natural transformations.
    \end{enumerate}
\end{enumerate}
}


%%%%%%%%%%%%%%%%%%%%%%%%%%%%%%%%%%%%%%%%%%%%%%%%
\section{Translating between different representations}
\draftnote{blue}{Consider}{
Can we use adjunctions to transfer between different representations ?
\begin{enumerate}
    \item Transfer between different types of representations of the same world with the same learning algorithm (e.g., vector space vs set).
    \item Transfer between different representations that are of the same type (e.g., representations in $\cat{Vect}$) and are learning the same world but using different inference processes.
\end{enumerate}
}
Any category $\mathcal{D}$ with the required structure of products and a basepoint-like object (e.g., $\cat{Set}_{\bot}$) can be used in the following diagram
\begin{equation}
\begin{tikzcd}[row sep=large, column sep=large]
\mathbf{B}\hat{A}^* \times \mathcal{F}(\hat{\mathscr{W}}_{\mathscr{A}}^{\bot}) \ar[r, "\mathcal{A}^\bot"] \ar[d, "U \times U"'] & \mathcal{F}(\hat{\mathscr{W}}_{\mathscr{A}}^{\bot}) \ar[d, "U"] \\
\mathbf{B}\hat{A}^* \times \mathcal{D} \ar[r, "A^\bot"] & \mathcal{D}
\end{tikzcd}
\end{equation}
where
\begin{equation}
    \mathcal{A}^{\bot}: \mathbf{B}\hat{A}^* \times \mathcal{F}(\hat{\mathscr{W}}_{\mathscr{A}}^{\bot}) \to \mathcal{F}(\hat{\mathscr{W}}_{\mathscr{A}}^{\bot})
\end{equation}
is the internal action of the free monoid on the world, $U$ is a forgetful functor that forgets the extra structure in $\mathcal{F}(\hat{\mathscr{W}}_{\mathscr{A}}^{\bot})$ to give $\mathcal{D}$, and
\begin{equation}
    A^{\bot}: \mathbf{B}\hat{A}^* \times \mathcal{D} \to \mathcal{D}
\end{equation}
is the $\mathcal{D}$-level interpretation of the action.

\draftnote{purple}{(PS) Consider}{
If $\mathcal{D}$ has enough "free" objects (such as when $\mathcal{D}$ is chosen to be something like $\cat{Set}_{\bot}$), we can define a left adjoint function $F$ so that $U$ is part of an adjunction.
}

What we really want to know is whether it is possible to build an adjunction so we can transfer between different representations of the internal action $\mathcal{A}^{\bot}: \mathbf{B}\hat{A}^* \times \mathcal{F}(\hat{\mathscr{W}}_{\mathscr{A}}^{\bot}) \to \mathcal{F}(\hat{\mathscr{W}}_{\mathscr{A}}^{\bot})$ (e.g., can we construct an adjunction so that we can move between action functors $A^{\bot}: \mathbf{B}\hat{A}^* \times \mathcal{D} \to \mathcal{D}$ where, for example, $\mathcal{D} = \cat{Set}$ and $\mathcal{D} = \cat{Vect}$)?

Consider the internal action
\begin{equation}
    \mathcal{A}^{\bot}: \mathbf{B}\hat{A}^* \times \mathcal{F}(\hat{\mathscr{W}}_{\mathscr{A}}^{\bot}) \to \mathcal{F}(\hat{\mathscr{W}}_{\mathscr{A}}^{\bot}).
\end{equation}
Suppose we have an action functor
\begin{equation}
    A^{\bot}_{\mathcal{D}}: \textbf{B}\hat{A}^{*} \times \mathcal{D} \to \mathcal{D},
\end{equation}
where $\mathcal{D}$ hosts a representation of the internal action $\mathcal{A}^{\bot}$, and an action functor
\begin{equation}
    A^{\bot}_{\mathcal{E}}: \textbf{B}\hat{A}^{*} \times \mathcal{E} \to \mathcal{E},
\end{equation}
where $\mathcal{E}$ hosts a representation of the action $\mathcal{A}^{\bot}$.

If there exists a free-forgetful adjunction
\begin{equation}
    F: \mathcal{D} \leftrightarrows \mathcal{E} :U,
\end{equation}
then we can often transfer the action from $\mathcal{D}$ to $\mathcal{E}$:
\begin{equation}
    % https://q.uiver.app/#q=WzAsNixbMiwwLCJcXHRleHR7SW4gfSBcXG1hdGhjYWx7RX06Il0sWzIsMSwieCJdLFswLDAsIlxcdGV4dHtJbiB9IFxcbWF0aGNhbHtEfToiXSxbMCwxLCJVKHgpIl0sWzAsMywiQV57XFxib3R9X3tcXG1hdGhjYWx7RH19KGEsIFUoeCkpIl0sWzIsMywiRihBXntcXGJvdH1fe1xcbWF0aGNhbHtEfX0oYSwgVSh4KSkpIl0sWzEsMywiVSIsMSx7ImNvbG91ciI6WzEyMCw2MCw2MF19LFsxMjAsNjAsNjAsMV1dLFszLDQsIkFee1xcYm90fV97XFxtYXRoY2Fse0R9fShhLCB4KSJdLFs0LDUsIkYiLDEseyJjb2xvdXIiOlswLDYwLDYwXX0sWzAsNjAsNjAsMV1dLFs1LDEsIlxcZXBzaWxvbl97eH0iXSxbMSw1LCJBXntcXGJvdH1fe1xcbWF0aGNhbHtFfX0oYSx4KSIsMCx7Im9mZnNldCI6LTMsInN0eWxlIjp7ImJvZHkiOnsibmFtZSI6ImRhc2hlZCJ9fX1dXQ==
\begin{tikzcd}[ampersand replacement=\&]
    {\text{In } \mathcal{D}:} \&\& {\text{In } \mathcal{E}:} \\
    {U(x)} \&\& x \\
    \\
    {A^{\bot}_{\mathcal{D}}(a, U(x))} \&\& {F(A^{\bot}_{\mathcal{D}}(a, U(x)))}
    \arrow["{A^{\bot}_{\mathcal{D}}(a, x)}", from=2-1, to=4-1]
    \arrow["U"{description}, color={rgb,255:red,92;green,214;blue,92}, from=2-3, to=2-1]
    \arrow["{A^{\bot}_{\mathcal{E}}(a,x)}", shift left=3, dashed, from=2-3, to=4-3]
    \arrow["F"{description}, color={rgb,255:red,214;green,92;blue,92}, from=4-1, to=4-3]
    \arrow["{\epsilon_{x}}", from=4-3, to=2-3]
\end{tikzcd}
\end{equation}
where we have derived the new functor
\begin{equation}
\begin{aligned}
    & A^{\bot}_{\mathcal{E}}: \textbf{B}\hat{A}^{*} \times \mathcal{E} \to \mathcal{E} \quad \text{such that} \\
    & A^{\bot}_{\mathcal{E}}(a,x) := F\big(A^{\bot}_{\mathcal{D}}(a, U(x))\big)
\end{aligned}
\end{equation}
Because $F$ is left adjoint to $U$, this procedure preserves the algebraic structure of the action \draftnote{purple}{(PS) To do}{Explain this.}.


%%%%%%%%%%%%%%%%%%%%%%%%%%%%%%%%%%%%%%%%%%%%%%%%
\section{Enriched category theory stuff - come back to this}
% https://chat.deepseek.com/a/chat/s/0197bb5e-6c47-4c1e-9899-bec91faa1ff2
%%%%%%%%%%%%%%%%%%%%%%%%%%%%%%%%%%%%%%%%%%%%%%%%
\subsection{Monoidal categories}


\paragraph{Delooping of a monoid $M$.}
A category with one object whose morphisms form the monoid $M$.
This is the category version of the monoid $M$; in other words, $M$ is encoded as the morphisms of a one-object category.
Formally, consider a monoid $(M, \cdot, e)$.
The category $\textbf{B}M$ consists of
\begin{enumerate}
    \item \textbf{Objects:}
    A single object $\bullet$.
    \begin{equation}
        \text{Ob}(\textbf{B}M) = \{ \bullet \}
    \end{equation}

    \item \textbf{Morphisms:}
    \begin{equation}
        \text{Hom}_{\textbf{B}M}(\bullet, \bullet) = M.
    \end{equation}
    Composition of morphisms is given by the monoid operation $\cdot$ in $M$.
\end{enumerate}


\paragraph{Monoidal category.}
A category $\mathcal{C}$ with an internal tensor product (binary operations on objects and morphisms), a unit object, associativity constraints etc...
This is a category equipped with monoid-like structure.
Formally a monoidal category $(\mathcal{C}, \otimes, I)$ is a category where
\begin{enumerate}
    \item \textbf{Objects:}
    There is, in general, more than one object.

    \item A bifunctor call the tensor product
    \begin{equation}
        \otimes: \mathcal{C} \times \mathcal{C} \to \mathcal{C}
    \end{equation}
    which acts on objects and morphisms
    \begin{enumerate}
        \item \textbf{Objects:}
        \begin{equation}
            A,B \mapsto A \otimes B
        \end{equation}
        \item \textbf{Morphisms:}
        \begin{equation}
            \big(f:A \to A', g: B \to B' \big) \mapsto \big(f \otimes g: A \otimes B \to A' \otimes B' \big).
        \end{equation}
        NB: not necessarily the Cartesian product (it's the Cartesian product when the objects are sets).
    \end{enumerate}

    \item A unit object
    \begin{equation}
        I \in \mathcal{C}.
    \end{equation}
    which is the identity for the tensor product.
    The following natural isomorphisms (unitors) exist:
    \begin{align}
        & \lambda_{A}: I \otimes A \to A \\
        & \rho_{A}: A \otimes I \to A
    \end{align}

    \item \textbf{Associator.}
    Since $\otimes$ is not required to be strictly associative, we add a natural isomorphism
    \begin{equation}
        \alpha_{A,B,C}: (A \otimes B) \otimes C \xrightarrow{\cong} A \otimes (B \otimes C)
    \end{equation}
    i.e., different bracketing gives canonically isomorphic objects, but not necessarily equal objects.

    \item Mac Lane's coherence theorem - pentagon identity, and triangle identity.
\end{enumerate}

For a strict monoidal category, the associativity and unit conditions are equally equal not just isomorphic (no unitors or associator isomorphisms needed).

Mac Lane’s strictification theorem: Every (weak) monoidal category is monoidally equivalent to a strict one.
Therefore we can assume strictness for the purposes of reasoning.

%%%%%%%%%%%%%%%%%%%%%%%%%%%%%%%%%%%%%%%%%%%%%%%%
\paragraph{Delooped monoid}
For $a, b \in \hat{A}^{*}$, the monoid multiplication in $\textbf{B}\hat{A}^{*}$ satisfies
\begin{center}  
\begin{tikzcd}
    \bullet \ar[r, "a"] \ar[rr, "a \circ b", bend right=30] & \bullet \ar[r, "b"] & \bullet
\end{tikzcd}
\end{center}


\paragraph{Horizontal composition.}
\draftnote{red}{To do}{
\begin{enumerate}
    \item I think $\mathcal{A}(a): C^{\bot} \to C^{\bot}$ basically shifts the structure around making some world states closer and some further away.
    For an action $a \in \hat{A}^{*}$, $\mathcal{A}(a)$ shifts $w \mapsto a \ast w$ for all $w \in W$.
    \item Does horizontal composition do something with the labels too ?
\end{enumerate}
}


For an action $a \in \hat{A}^{*}$ as an endofunctor $\mathcal{A}(a): C^{\bot} \to C^{\bot}$, and a transformation $d \in C^{\bot}$ in $d: w \to w'$, the horizontal composition $a \ast d$ is the morphism $a \ast w \xrightarrow{a \ast d} a \ast w'$:
\begin{equation}
    \mathcal{A}(a): (w \xrightarrow{d} w') \mapsto (a \ast w \xrightarrow{a \ast d} a \ast w')
\end{equation}
OR
\begin{equation}
    \mathcal{A}(a): (d: w \to w') \mapsto (a \ast d: a \ast w \to a \ast w')
\end{equation}
\begin{equation}
    \mathcal{A}(a): \Big( w \xrightarrow{d} w' \Big) \mapsto \Big( a \ast w \xrightarrow{a \ast d} a \ast w' \Big).
\end{equation}


Horizontal composition is the application of $\mathcal{A}(a)$ to the morphism $d$.
The $\ast$ in $a \ast d$ is the horizontal composition (i.e., the functorial action on morphisms), while $a \ast w$ is the functorial action on objects.
This is analogous to the "whiskering" operation in 2-category theory.


\paragraph{Action functor.}
We have a functor
\begin{equation}
    \mathcal{A}: \textbf{B}\hat{A}^{*} \to \textbf{End}(C^{\bot})
\end{equation}
that acts as follows
\begin{enumerate}
    \item \textbf{Objects:}
    The single object of $\textbf{B}\hat{A}^{*}$ is mapped to the identity endofunctor
    \begin{equation}
        \mathcal{A}(\bullet) = \text{Id}_{C}
    \end{equation}

    \item \textbf{Morphisms:}
    Each $a \in \hat{A}^{*}$ maps to an endofunctor $\mathcal{A}(a): C \to C$ satisfying the properties
    \begin{enumerate}
        \item \textbf{Identity.}
        \begin{equation}
            \mathcal{A}(\varepsilon) = \mathrm{Id}_{\mathcal{C}^{\bot}}
        \end{equation}
        \item \textbf{Composition.}
        \begin{equation}
            \mathcal{A}(a' \circ a) = \mathcal{A}(a') \circ \mathcal{A}(a)
        \end{equation}
    \end{enumerate}
\end{enumerate}


The category $\textbf{End}(C^{\bot})$ consists of
\begin{enumerate}
    \item \textbf{Objects:}
    The endofunctors
    \begin{equation}
        F : C^{\bot} \to C^{\bot}
    \end{equation}
    For example $\mathcal{A}(a)$ for $a \in \hat{A}^{*}$.

    \item \textbf{Morphisms:}
    Natural transformations
    \begin{equation}
        \eta: F \Rightarrow G
    \end{equation}
    between endofunctors $F$ and $G$.
    For example, the identity natural transformation $\text{Id}_{C} \Rightarrow \text{Id}_{C}$.
\end{enumerate}


Each of the endofunctors $\mathcal{A}(a): C \to C$ act as follows:
\begin{enumerate}
    \item \textbf{Objects:}
    For $a \in \hat{A}^{*}$, $\mathcal{A}(a)$ maps objects in $C^{\bot}$ via
    \begin{equation}
        \mathcal{A}(a)(w) = a \ast w \quad \text{and} \quad \mathcal{A}(a)(\bot) = \bot
    \end{equation}

    \item \textbf{Morphisms:}
    For $d: w \to w'$ in $C^{\bot}$, $\mathcal{A}(a)(d) = a \ast d$ is defined by
    \begin{center}
    \begin{tikzcd}
        w \ar[r, "d"] \ar[d, "a \ast (-)"'] & w' \ar[d, "a \ast (-)"] \\
        a \ast w \ar[r, "a \ast d"] & a \ast w'
    \end{tikzcd}
    \end{center}
    This diagram commutes by the definition of horizontal composition.
\end{enumerate}



\draftnote{red}{DIVIDER}{}


\begin{equation}
\mathcal{A}(a)(w) = a \ast w.
\end{equation}

So the morphisms $\mathcal{A}(a)(w)$ act as $\mathcal{A}(a)(w) = a \ast w$.





\begin{equation}
\mathcal{A}(a)(w) = a \ast w \quad \text{and} \quad \mathcal{A}(a)(\bot) = \bot.
\end{equation}

\begin{equation}
    \mathcal{A}(a)(d) = a \ast d \quad \text{for all $d: w \to w'$ in $C$}
\end{equation}










\paragraph{Natural transformations from labelling.}
The labelling system induces natural transformations (for each $a \in \hat{A}^{*}$?)
\begin{equation}
    \alpha_{a}: \text{Id}_{\mathcal{C}^{\bot}} \Rightarrow \mathcal{A}(a)
\end{equation}
with components
\begin{equation}
    \alpha_{a}(w): w \to a \ast w.
\end{equation}
The naturality condition requires the following diagram to commute for all $d: w \to w'$
\begin{center}
\begin{tikzcd}
    w \ar[r, "\alpha_a(w)"] \ar[d, "d"'] & a \ast w \ar[d, "a \ast d"] \\
    w' \ar[r, "\alpha_a(w')"'] & a \ast w'
\end{tikzcd}
\end{center}
where $\alpha_{a}(w)$ is the morphism labelled by $a$ from $w$ to $a \ast w$.


For each $w \in C$, $\alpha_{a}(w)$ is the unique morphism $d_{a} \in D_{A}$ labelled by $a$ such that
\begin{equation}
    d_a: w \to a \ast w \quad \text{with} \quad l(d_a) = a.
\end{equation}

The naturality condition ensures compatibility: for any $d: w \to w'$, the diagram
\begin{center}
\begin{tikzcd}
w \ar[r, "d_a"] \ar[d, "d"'] & a \ast w \ar[d, "a \ast d"] \\
w' \ar[r, "d_a'"'] & a \ast w'
\end{tikzcd}
\end{center}
commutes; in other words
\begin{equation}
    (a \ast d) \circ d_{a} = d'_{a} \circ d.
\end{equation}


The natural transformation $\alpha_{a}$ encodes the deterministic effect of the action $a \in \hat{A}^{*}$ across all world states; each component $\alpha_{a}(w)$ of $\alpha_{a}$ encodes the effect of $a$ on the world state $w \in W$.



%%%%%%%%%%%%%%%%%%%%%%%%%%%%%%%%%%%%%%%%%%%%%%%%
\section{Other}
\draftnote{red}{}{
\begin{enumerate}
    \item \textbf{Category theory on reachable subworlds.}
    \begin{enumerate}
        \item Reachable subworld is an object in the category of categories (or category of worlds (which is a subcategory of the category of categories)?).
        \item Possible construction:
        \begin{enumerate}
            \item Strict endofunctors on reachable subworld categories for reversible actions.
            \item Non-strict endofunctors on (infinite) reachable subworld categories that are action-homogeneous - does this mean we can tell if a world is action-homogeneous by just looking at its reachable subworld category ?
            \item Functors for irreversible actions.
        \end{enumerate}
        \item How to deal with reachability in a disjoint world ?
    \end{enumerate}
\end{enumerate}
}


\draftnote{red}{Consider}{
\begin{enumerate}
    \item Would taking the dual give us anything ?
    \begin{enumerate}
        \item Doesn't make sense to take the dual of worlds $\mathcal{F}(\hat{\mathscr{W}}_{\mathscr{A}})$.
    \end{enumerate}
    \item Other equivalences
    \begin{enumerate}
        \item A congruence relation $\sim$ on a category $\mathcal{C}$ is a relation on the morphism of $\mathcal{C}$ that respects domain and codomain (only compares morphisms with the same source and target), and is compatible with composition:
        \begin{equation}
            f \sim f' \text{ and } g \sim g' \implies g \circ f \sim g' \circ f'.
        \end{equation}
    \end{enumerate}
\end{enumerate}
}