\draftnote{blue}{}{
\section{Overview}
}

%%%%%%%%%%%%%%%%%%%%%%%%%%%%%%%%%%%%%%%%%%%%%%%%
\section{Motivation}


\draftnote{red}{Include}{
\begin{enumerate}
    \item Physics has moved to caring predominately about the transformations (including the symmetries) of object instead of the objects themselves; in this way, category theory is the physics of mathematics - it cares about the relations and transformations between objects.
    \item Category theory is a minimalistic language for composition.
\end{enumerate}
}

\draftnote{red}{Discussion}{
\begin{enumerate}
    \item We have shown that the natural language for this framework is Category theory.
    \item Agent learning the structure of the world can map the transformation structure from one world to another or from (e.g., moving around the world $\to$ moving an object).
    \item The use of pointed sets to deal with the undefined state $\bot$ agrees with the idea that the agent has to learn a map that tells the agent if an action is going to be undefined.
    \item "We will replace set-theoretic and graph-theoretic constructions with their categorical analogues."
\end{enumerate}
}

Category theory studies the structure of mathematical objects and their relationships; it provides methods to study the structure of objects in a category using the interactions of that object with the other objects in the category.
A category consists of objects connected by arrows, which represent structure-preserving maps called \textit{morphisms} between the objects.
One of the most important concepts in category theory is the \textit{functor}.
A functor is a structure-preserving mapping between categories; functors are ways to transform a category to another category while preserving relationships between the objects and arrows of the original category.
\textit{Natural transforms} are ways to transform one functor into another while preserving the structure of the categories involved in the functor.
Simply, category theory provides a way of organising mathematical objects and the relationships between them as categories, which can be transformed into other categories using functions and compared to other categories using natural transforms.

A fundamental result in category theory is the Yoneda lemma, which produces the result that the properties of mathematical objects are completely determined by their relationships to other objects \autocite{riehl2017category,barr1990category}.
This result is similar to the shift in perspective in AI representations from studying objects to gaining insight into the structure of an object by studying the transformations of that object.
Due to the Yoneda Lemma, category theory already has this approach of considering the transformation properties of objects built in; this makes category theory appear to be the natural choice to describe the transformations of an agent's representation.


