\chapter{
Categorification of our framework
\draftnote{blue}{V0.5}{}
}

\draftnote{purple}{(PS) Consider}{
\begin{enumerate}
    \item Put pushout symbols in category theory diagrams.
    \item Decide on standardised way of defining categories (put in a different box type ?).
    \item Decide on a standardised way of defining functors (put in a different box type ?).
    \item Example for simple two world state systems with actions $a, a^{-1}$.
    \item Change arrows for faithful functors to be hooked in equations and diagrams.
    \item Decide on a symbol convention for morphisms, functors, natural transforms, categories.
    \item Talk about how faithfulness is essential for determinism (i.e., not stochastic transformations).
    \item Can we use the adjunction between $\cat{Quiv}$ and $\cat{Cat}$ to prove that it's impossible to go from $\hat{\mathscr{W}}$ to $\hat{\mathscr{W}}_{A}$ directly ?
    \begin{enumerate}
        \item We can show this process as $\hat{\mathscr{W}} \in \cat{Quiv}$ to 
        \item \textbf{Process:}
        We start with a quiver $Q$, then generate its free category using the free category functor $F(Q)$, then take the quiver of that free category using the forgetful functor $U(F(Q))$, then take the subset $Q_{A}$ of $U(F(Q))$ which is labelled by some labelling map $l: U(F(Q)) \to U(\text{B}A)$ (i.e., filter $U(F(Q))$ via $l$), then generate the free category $F(Q_{A})$
        \item Can we use this process to construct a morphism in Cat from $F(Q)$ to $F(Q_{A})$ ?
    \end{enumerate}
    \item Something about how working in $\cat{Quiv}$ is like we're working with the atomic multigraphs, and that the adjunction between $\cat{Quiv}$ and $\cat{Cat}$ means we can work in $\cat{Quiv}$ (for some stuff).
    \item The minimal quiver that generates $\mathcal{F}(Q)$
    \begin{enumerate}
        \item Can multiple quivers generate the same free category or do those quivers only become identified when $\sim$ is applied ?
        \item Is there a property that shows if the quiver $Q$ is the smallest necessary quiver that generates $\mathcal{F}(Q)$ ?
        Perhaps something like $\eta_{Q}$ being the identity morphism ?
    \end{enumerate}
    \item Convert the totalisation pushout to be done in $\cat{Quiv}$ and then lifted into $\cat{Cat}$ using the free-forgetful adjunction.
    \begin{enumerate}
        \item I think this should be much more elegant.
        \item No need to include identities in the quivers.
        \item Watch out with the use of $\hat{A}$ vs $\hat{A}^{*}$ in the quiver.
        \item Footnote about how the lifted labelling maps gain the ability to label empty paths with $\varepsilon$ even though the atomic labelling map $\hat{l}$ does not do so.
        \item Show all the diagrams.
    \end{enumerate}
    \item Change "action functor" to "action representation functor" ?
    \item Create a more involved proof that $L^{\bot}$ inherits faithfulness from $L$ and $L_{\mathcal{S}}$.
    \item Explicitly define the kernel category used for the equivalence.
    \item Reachable subworld as the connected component of an object ?
\end{enumerate}
}

\draftnote{red}{Explore}{
\begin{enumerate}
    \item Extensions to not using $\cat{Set}$ or $\cat{Set}_{\bot}$:
    \begin{enumerate}
        \item Representations of quivers as a natural model for describing the world using vector spaces (the world states) and linear maps between them (the transformations).
        \item Generalising the objects of the category away from sets to other things (e.g., vector spaces - quiver representations, local algebras etc...). Can put any property and the category will have the same structure (possibly with added redundancy).
        \item Action on a general category instead of $\cat{Set}$.
    \end{enumerate}
    \item Is the category $\mathscr{W}$ with the labelling map/functor $l$/$L$ an enriched category ?
    \item Category of worlds ?
    \begin{enumerate}
        \item Do worlds have any extra structure ?
        \item Is this the category of enriched categories with labelling maps ?
    \end{enumerate}
    \item Action-homogeneous condition is a condition for the world states to be deemed to be identical in category theory (from the Yoneda lemma we can work out all info from the morphisms of objects) and therefore we can "glue together" the objects of the groupoid to form a group.
\end{enumerate}
}

%%%%%%%%%%%%%%%%%%%%%%%%%%%%%%%%%%%%%%%%%%%%%%%%
\draftnote{blue}{}{
\section{Overview}
}

%%%%%%%%%%%%%%%%%%%%%%%%%%%%%%%%%%%%%%%%%%%%%%%%
\section{Motivation}


\draftnote{red}{Include}{
\begin{enumerate}
    \item Physics has moved to caring predominately about the transformations (including the symmetries) of object instead of the objects themselves; in this way, category theory is the physics of mathematics - it cares about the relations and transformations between objects.
    \item Category theory is a minimalistic language for composition.
\end{enumerate}
}

\draftnote{red}{Discussion}{
\begin{enumerate}
    \item We have shown that the natural language for this framework is Category theory.
    \item Agent learning the structure of the world can map the transformation structure from one world to another or from (e.g., moving around the world $\to$ moving an object).
    \item The use of pointed sets to deal with the undefined state $\bot$ agrees with the idea that the agent has to learn a map that tells the agent if an action is going to be undefined.
    \item "We will replace set-theoretic and graph-theoretic constructions with their categorical analogues."
\end{enumerate}
}

Category theory studies the structure of mathematical objects and their relationships; it provides methods to study the structure of objects in a category using the interactions of that object with the other objects in the category.
A category consists of objects connected by arrows, which represent structure-preserving maps called \textit{morphisms} between the objects.
One of the most important concepts in category theory is the \textit{functor}.
A functor is a structure-preserving mapping between categories; functors are ways to transform a category to another category while preserving relationships between the objects and arrows of the original category.
\textit{Natural transforms} are ways to transform one functor into another while preserving the structure of the categories involved in the functor.
Simply, category theory provides a way of organising mathematical objects and the relationships between them as categories, which can be transformed into other categories using functions and compared to other categories using natural transforms.

A fundamental result in category theory is the Yoneda lemma, which produces the result that the properties of mathematical objects are completely determined by their relationships to other objects \autocite{riehl2017category,barr1990category}.
This result is similar to the shift in perspective in AI representations from studying objects to gaining insight into the structure of an object by studying the transformations of that object.
Due to the Yoneda Lemma, category theory already has this approach of considering the transformation properties of objects built in; this makes category theory appear to be the natural choice to describe the transformations of an agent's representation.



%%%%%%%%%%%%%%%%%%%%%%%%%%%%%%%%%%%%%%%%%%%%%%%%

%%%%%%%%%%%%%%%%%%%%%%%%%%%%%%%%%%%%%%%%%%%%%%%%
\section{Preliminaries}
%%%%%%%%%%%%%%%%%%%%%%%%%%%%%%%%%%%%%%%%%%%%%%%
\subsection{Categories and morphisms}

\newthought{A \emph{category} $\mathcal{C}$ consists} of
\begin{enumerate}
    \item \textbf{Objects:}
    a class $\text{Ob}(\mathcal{C})$ of objects;
    
    \item \textbf{Morphisms:}\footnote{
    A \emph{morphism} is a structure preserving map between objects.
    }
    for each pair $X$, $Y$ of objects, a class $\text{Hom}(X,Y)$ of morphisms $\alpha : X \to Y$ that satisfy the following:
    \begin{enumerate}
        \item \textbf{Composition law:}
        Given two morphisms $\alpha \in \text{Hom}(X,Y)$ and $\beta \in \text{Hom}(Y,Z)$, there exists a morphism
        \begin{equation}
            \beta \circ \alpha \in \text{Hom}(X,Z)
        \end{equation}
        called the \emph{composition} of $\alpha$ and $\beta$.
        
        \item \textbf{Existence of identity morphisms:}
        Given an object $X$, there exists a morphism $\text{id}_{X} \in \text{Hom}(X,X)$ such that for any morphism $\alpha \in \text{Hom}(X, A)$,
        \begin{equation}
            \alpha \circ \text{id}_{X} = \alpha
        \end{equation}
        and for any morphism $\beta \in \text{Hom}(B,X)$,
        \begin{equation}
            \text{id}_{X} \circ \beta = \beta.
        \end{equation}
        
        \item \textbf{Associativity.}
        For any three morphisms $\alpha \in \text{Hom}(X,Y)$, $\beta \in \text{Hom}(Y,Z)$, $\gamma \in \text{Hom}(Z,U)$, the following associative law is satisfied:
        \begin{equation}
            \gamma \circ ( \beta \circ \alpha) = (\gamma \circ \beta) \circ \alpha.
        \end{equation}
    \end{enumerate}
\end{enumerate}

A category $\mathcal{C}$ is a \emph{small category} if its collection of objects $\text{Ob}(\mathcal{C})$ is a set, and its collections of morphisms $\text{Hom}_{\mathcal{C}}(X, Y)$, for $X,Y \in \text{Ob}(\mathcal{C})$, are also sets.

A \emph{discrete category} $\mathcal{C}$ is a category in which the only morphisms are identity morphisms; in other words, for $A, B \in \text{Ob}(\mathcal{C})$,
\begin{align}
    & \text{Hom}(A, A) = \{\text{id}_A\} \\
    & \text{Hom}(A, B) =  \emptyset \quad \text{for $A \neq B$}.
\end{align}


\newthought{A morphism $f: X \to Y$ in} a category $\mathcal{C}$ is a \emph{split monomorphism} if there exists a morphism $g: Y \to X$ such that
\begin{equation}
    g \circ f = \text{id}_{X}.
\end{equation}
$g$ is called a \emph{left inverse} of $f$.
A morphism $f: X \to Y$ is a \emph{split epimorphism} if there exists a morphism $h: Y \to X$ such that
\begin{equation}
    f \circ h = \text{id}_{Y}.
\end{equation}
$h$ is called a \emph{right inverse} of $f$.

A morphism $\alpha : X \to Y$ in a category $\mathcal{C}$ is an \emph{isomorphism} if it is both a split monomorphism and a split epimorphism is an isomorphism; formally, a morphism $\alpha : X \to Y$ in a category $\mathcal{C}$ is an isomorphism if there exists another morphism $\beta : X \to Y$ in $\mathcal{C}$ such that
\begin{align}
    & \beta \circ \alpha = \text{id}_{X} \\
    \text{and } & \alpha \circ \beta = \text{id}_{Y}
\end{align}
\begin{notation}
    An isomorphism $\alpha : X \to Y$ can be denoted by $X \overset{\alpha}{\cong} Y$.
\end{notation}


%%%%%%%%%%%%%%%%%%%%%%%%%%%%%%%%%%%%%%%%%%%%%%%
\subsection{Functors}

\newthought{A \emph{functor} is} a structure-preserving map between two categories.
For two categories $\mathcal{A}$ and $\mathcal{B}$, a functor $F: \mathcal{A} \to \mathcal{B}$ from $\mathcal{A}$ to $\mathcal{B}$ assigns to each object in $\mathcal{A}$ an object in $\mathcal{B}$ and to each morphism in $\mathcal{A}$ a morphism in $\mathcal{B}$ such that the composition of morphisms and the identity morphisms are preserved.
A functor transforms objects and morphism from one category to another in a way that preserves the structure of the original category.

Formally, a functor $F: \mathcal{A} \to \mathcal{B}$ maps:
\begin{enumerate}
    \item each object $A \in \text{Ob}(\mathcal{A})$ to an object $F(A) \in \mathcal{B}$
    \begin{equation}
        F: \text{Ob}(\mathcal{A}) \to \text{Ob}(\mathcal{B});
    \end{equation}

    \item each morphism $f: A \to B$ in $\mathcal{A}$ to a morphism $F(f): F(A) \to F(B)$ in $\mathcal{B}$
    \begin{equation}
        F: \text{Hom}_{\mathcal{A}}(X, Y) \to \text{Hom}_{\mathcal{B}}(F(X), F(Y)),
    \end{equation}
    while preserving
    \begin{enumerate}
        \item \textbf{Composition:} for any two composable morphisms $f,g \in \mathcal{A}$,
        \begin{equation}
            F(f \circ g) = F(f) \circ F(g);
        \end{equation}

        \item \textbf{Identities:} for any object $X$ in $\mathcal{A}$,
        \begin{equation}
            F(\text{id}_{X}) = \text{id}_{F(X)}.
        \end{equation}
    \end{enumerate}
\end{enumerate}
A \emph{strict functor} satisfies these rules exactly, not just up to isomorphism.
The \emph{data} of a functor consists of the object mapping and the morphism mapping specific to that functor.

A functor $F : \mathcal{C} \to \mathcal{D}$ is called \textbf{faithful} if, for every pair of objects $X, Y \in \mathcal{C}$, the function\footnote{
    Faithfulness ensures that the functor does not "collapse" distinct morphisms.
}
\begin{equation}
    F_{X,Y} : \text{Hom}_{\mathcal{C}}(X,Y) \to \text{Hom}_{\mathcal{D}}(F(X), F(Y))
\end{equation}
is injective. That is, if $f, g : X \to Y$ are two morphisms in $\mathcal{C}$ with $f \neq g$, then it must be that 
\begin{equation}
    F(f) \neq F(g) \quad \text{in } \mathcal{D}.
\end{equation}

A \emph{concrete category} $\mathcal{C}$ over a base category $\mathcal{B}$ is a pair $(\mathcal{C}, U)$ where $U: \mathcal{C} \to \mathcal{B}$ is a faithful functor.

An \emph{endofunctor} is a functor $F: \mathcal{C} \to \mathcal{C}$ that maps a category to itself.

The \emph{kernel congruence} of a functor tell us which morphisms are indistinguishable to the functor because the functor treats them as equivalent.
Formally, the kernel congruence of a functor $F: \mathcal{C} \to \mathcal{D}$ is the equivalence relation on the morphisms of $\mathcal{C}$ that identifies $f,g: X \to Y$ in $\mathcal{C}$ if $F(f) = F(g)$ in $\mathcal{D}$.


%%%%%%%%%%%%%%%%%%%%%%%%%%%%%%%%%%%%%%%%%%%%%%%
\subsection{Natural transformations}

A \emph{natural transformation} is a structure-preserving map between functors.
Formally, for two functors
\begin{equation}
    F,G: \mathcal{C} \to \mathcal{D}
\end{equation}
between categories $\mathcal{C}$ and $\mathcal{D}$, a \emph{natural transform}
\begin{equation}
    \eta: F \Rightarrow G
\end{equation}
between $F$ and $G$ consists of a family of morphisms
\begin{equation}
    \{\eta_{X}: F(X) \to G(X)\}_{X \in \text{Ob}(\mathcal{C})}
\end{equation}
in $\mathcal{D}$ such that, for every morphism $f: X \to Y$ in $\mathcal{C}$, the following diagram commutes
\begin{equation}
    % https://q.uiver.app/#q=WzAsNCxbMCwwLCJGKFgpIl0sWzAsMiwiRyhYKSJdLFsyLDAsIkYoWSkiXSxbMiwyLCJHKFkpIl0sWzAsMiwiRihmKSJdLFsyLDMsIlxcZXRhX3tZfSJdLFsxLDMsIkcoZikiXSxbMCwxLCJcXGV0YV97WH0iXV0=
\begin{tikzcd}[ampersand replacement=\&]
    {F(X)} \&\& {F(Y)} \\
    \\
    {G(X)} \&\& {G(Y)}
    \arrow["{F(f)}", from=1-1, to=1-3]
    \arrow["{\eta_{X}}", from=1-1, to=3-1]
    \arrow["{\eta_{Y}}", from=1-3, to=3-3]
    \arrow["{G(f)}", from=3-1, to=3-3]
\end{tikzcd}
\end{equation}
in other words
\begin{equation}
    \eta_{Y} \circ F(f) = G(f) \circ \eta_{X}.
\end{equation}

A \emph{natural isomorphism} is a natural transformation where each morphism is an isomorphism.

The \emph{functor category} $[\mathcal{C}, \mathcal{D}]$ of two categories $\mathcal{C}$ and $\mathcal{D}$ is the category consisting of
\begin{enumerate}
    \item \textbf{Objects:}
    objects that are functors
    \begin{equation}
        F: \mathcal{C} \to \mathcal{D};
    \end{equation}
    \item \textbf{Morphisms:}
    morphisms that are natural transforms between these functors $F$.
\end{enumerate}


%%%%%%%%%%%%%%%%%%%%%%%%%%%%%%%%%%%%%%%%%%%%%%%
\subsection{Algebraic structures as categories}

A standard way of "categorising" an algebraic structure $A$ is to construct the \emph{delooped category} of $A$.
Given an algebraic structure $A$, the delooped category $\textbf{B}A$ is the category whose morphisms correspond to the elements of $A$ with the relevant composition:
\begin{equation}
    A \xrightarrow{\text{deloop}} \textbf{B}A
\end{equation}
where $\textbf{B}$ is called the \emph{base} of the category and this base contains the necessary number of objects for the morphisms of $\textbf{B}A$ to correspond to the elements of $A$.

A \emph{monoid} is a category with a single object.

\begin{definition}[Group]
    A group is a category that has a single object and in which every morphism is an isomorphism (\textit{i.e.,} every morphism has an inverse).
\end{definition}

%%%%%%%%%%%%%%%%%%%%%%%%%%%%%%%%%%%%%%%%%%%%%%%
\subsection{Adjunctions}
\draftnote{purple}{(PS) Include?}{
\begin{enumerate}
    \item Projections.
    \item Inclusions.
    \item Terminal objects.
    \item Limits and colimits formally (in terms of diagrams $\mathcal{D}: \mathcal{J} \to \mathcal{C}$ from an index category $\mathcal{J}$ to a category $\mathcal{C}$).
    \item Adjunctions in general.
\end{enumerate}
}

\newthought{\emph{Adjunctions} can be} though of as the most general way of converting between two perspectives.
Adjunctions consists of two functors and a natural transformation between them that lets you translate objects and morphisms between two categories.
Formally, an adjunction $F \dashv G$ between two categories $\mathcal{C}$ and $\mathcal{D}$ consists of
\begin{enumerate}
    \item \textbf{Functors.}
    Two functors: the left adjoint
    \begin{equation}
        F: \mathcal{C} \to \mathcal{D}
    \end{equation}
    and the right adjoint
    \begin{equation}
        G: \mathcal{D} \to \mathcal{C}
    \end{equation}

    \item \textbf{Natural isomorphism.}
    For all $C \in \mathcal{C}$ and for all $D \in \mathcal{D}$, a natural isomorphism between two Hom-sets
    \begin{equation}
        \text{Hom}_{\mathcal{D}}(F(C, D) \cong \text{Hom}_{\mathcal{C}}(C, G(D));
    \end{equation}
    this isomorphism is natural in both directions\draftnote{purple}{PS}{Explain this.}, universal (the most efficient translation between $\mathcal{C}$ and $\mathcal{D}$ and therefore unique), and constructive (gives canonical maps).
    This means a morphism from $F(C)$ to $D$ in $\mathcal{D}$ corresponds uniquely and naturally to a morphism from $C$ to $G(C)$ in $\mathcal{C}$.

    \item \textbf{Natural transformations.}
    The natural isomorphism can be equivalently encodes using two natural transformations:
    \begin{enumerate}
        \item The unit natural transformation
        \begin{equation}
            \eta: \text{Id}_{\mathcal{C}} \Rightarrow G \circ F.
        \end{equation}
        \item The counit natural transformation
        \begin{equation}
            \epsilon: F \circ G \Rightarrow \text{Id}_{\mathcal{D}}.
        \end{equation}
    \end{enumerate}
    $\eta$ and $\epsilon$ satisfy the following triangle identities
    \begin{align}
        & \epsilon_{F(C)} \circ F(\eta_{C}) = \text{id}_{F(C)} \quad \text{for all $C \in \mathcal{C}$} \\
        & G(\epsilon_{D}) \circ \eta_{G(D)} = \text{id}_{G(D)} \quad \text{for all $D \in \mathcal{D}$}.
    \end{align}
    
    These two natural transformations describe how we can embed and lift objects and morphisms using the functors $F$ and $G$.
\end{enumerate}

\begin{notation}
    Adjunctions are denoted by $F \dashv G$, which is read as "$F$ is left adjoint to $G$".
\end{notation}

\newthought{Right adjoints are} \emph{continuous}, which means that when right adjoints are applied they preserve all limits (e.g., products, pullbacks, coequalizers etc...) that exist in their domain.\draftnote{purple}{(PS) Consider}{Define the preservation of limits under right adjoints formally.}

\newthought{Left adjoints are} \emph{cocontinuous}, which means that when left adjoints are applied they preserve all colimits (e.g., coproducts, pushouts, kernel pairs, equalizers etc...) that exist in their domain.\draftnote{purple}{(PS) Consider}{Define the preservation of colimits under left adjoints formally.}


%%%%%%%%%%%%%%%%%%%%%%%%%%%%%%%%%%%%%%%%%%%%%%%%%%
\paragraph{The universal property of free categories.}
\newthought{There is a} \emph{free category functor}\footnote{
    $\cat{Quiv}$ is the category of multidigraphs (also known as \emph{quivers}), and $\cat{Cat}$ is the category of categories.
}
\begin{equation}
    \mathcal{F}: \cat{Quiv} \to \cat{Cat}.
\end{equation}
that sends a multidigraph $Q$ to the \emph{free category} $\mathcal{F}(Q)$ generated by that multidigraph; $\mathcal{F}$ constructs the free category $\mathcal{F}(Q)$ by freely generating paths\footnote{
In category theory, the terms "walk" and "path" are used interchangeably, but both mean sequences of composable edges with repeated vertices and edges allowed; this is aligned with the term "walk" in the graph theory context used in \draftnote{blue}{section}{???}).
In graph theory, a walk allows repetition of vertices and edges, but a path does not.
}
between vertices of the multidigraph $Q$, treating these paths as morphism, and defining composition as the concatenation of paths.
The data of the free category $\mathcal{F}(Q)$ generated by a multidigraph $Q$ are:
\begin{enumerate}
    \item \textbf{Objects:}
    The vertices $Q_{0}$ of the multidigraph.
    \begin{equation}
        \text{Ob}(\mathcal{F}(\hat{\mathscr{W}})) := Q_{0}
    \end{equation}
    \item \textbf{Morphisms:}
    The finite paths of arrows $Q_{1}$
    \begin{equation}
        \text{Hom}_{\mathcal{F}(Q)}(x, y) = \{ \text{finite paths in $Q$ from $x$ to $y$} \}.
    \end{equation}
    where the identity morphisms of $\mathcal{F}(Q)$ are the trivial paths, the paths of length 1 are the arrows in $Q_{1}$, and the the paths of length $\geq 2$ are chains of composable arrows.
    Composition of morphisms is concatenation of paths.
\end{enumerate}

We also have a \emph{forgetful functor}
\begin{equation}
    U: \cat{Cat} \to \cat{Quiv},
\end{equation}
which sends a category $\mathcal{C}$ to the underlying multidigraph $U(\mathcal{C})$ of $\mathcal{C}$.
$U(\mathcal{C})$ is the multidigraph where
\begin{enumerate}
    \item vertices are the same as the objects of the category $\mathcal{C}$:
    \begin{equation}
        U(\mathcal{C})_{0} = \text{Ob}(\mathcal{C}).
    \end{equation}
    \item arrows are the morphisms of the category in such a way that respects the source and target of morphisms and arrows (i.e., a morphism $f:X \to Y$ in $\mathcal{C}$ becomes an arrow in $U(\mathcal{C})$ with source $X$ and target $Y$)\footnote{
        Importantly, the data of composition and the identity morphism relationships of $\mathcal{C}$ are forgotten by $U(\mathcal{C})$.
        The identity morphisms themselves are present in $U(\mathcal{C})$ are arrows but they have forgotten their special status as identities.
    }:
    \begin{equation}
        U(\mathcal{C})_{1} = \{f \mid \text{$f$ is a morphism in $\mathcal{C}$}\};
    \end{equation}
\end{enumerate}

The free category functor $\mathcal{F}$ and the forgetful functor $U$ form an adjunction
\begin{equation}
    \mathcal{F}: \cat{Quiv} \leftrightarrows \cat{Cat}: U.
\end{equation}
where the free category functor $\mathcal{F}$ is left adjoint to a forgetful functor
\begin{equation}
    \mathcal{F} \dashv U.
\end{equation}

This adjunction comes with a unit natural transformation\footnote{
    The adjunction also comes with a counit natural transformation
    \begin{equation}
        \epsilon: \mathcal{F} \circ U  \Rightarrow \text{id}_{\cat{Cat}}.
    \end{equation}
}
\begin{equation}
    \eta: \text{Id}_{\cat{Quiv}} \Rightarrow U \circ \mathcal{F}.
\end{equation}
The components of $\eta$ are the multidigraph morphisms from each multidigraph $Q \in \cat{Quiv}$ to the underlying multidigraph $U(\mathcal{F}(Q))$ of the free category $\mathcal{F}(Q)$ of the multidigraph $Q$:
\begin{equation}
    \eta_{Q}: Q \to U(\mathcal{F}(Q)).
\end{equation}
This gives us
\begin{equation}
% https://q.uiver.app/#q=WzAsNSxbMCwxLCJRIl0sWzAsMiwiVShcXG1hdGhjYWx7Rn0oUSkpIl0sWzIsMSwiXFxtYXRoY2Fse0Z9KFEpIl0sWzAsMCwiXFx0ZXh0e0luICRcXGNhdHtRdWl2fSQ6fSJdLFsyLDAsIlxcdGV4dHtJbiAkXFxjYXR7Q2F0fSR9OiJdLFswLDEsIlxcZXRhX3tRfSJdLFswLDIsIkYiLDAseyJjdXJ2ZSI6LTIsImNvbG91ciI6WzAsNjAsNjBdfSxbMCw2MCw2MCwxXV0sWzIsMSwiVSIsMix7ImNvbG91ciI6WzEyMCw2MCw2MF19LFsxMjAsNjAsNjAsMV1dXQ==
\begin{tikzcd}[ampersand replacement=\&]
    {\text{In $\cat{Quiv}$:}} \&\& {\text{In $\cat{Cat}$}:} \\
    Q \&\& {\mathcal{F}(Q)} \\
    {U(\mathcal{F}(Q))}
    \arrow["F", color={rgb,255:red,214;green,92;blue,92}, curve={height=-12pt}, from=2-1, to=2-3]
    \arrow["{\eta_{Q}}", from=2-1, to=3-1]
    \arrow["U"', color={rgb,255:red,92;green,214;blue,92}, from=2-3, to=3-1]
\end{tikzcd}
\end{equation}

The universal property of free categories says that for any category $\mathcal{C} \in \cat{Cat}$ and multidigraph morphism $G: Q \to U(\mathcal{C})$, there exists a unique functor $G' : \mathcal{F}(Q) \to \mathcal{C}$ such that\footnote{
    Explicitly, the functor $G': \mathcal{F}(Q) \to \mathcal{C}$ is uniquely defined as:
    \begin{enumerate}
        \item \textbf{Objects:}
        Since $\text{Ob}(\mathcal{F}(Q)) = Q_{0}$, where $Q_{0}$ is the set of vertices of the multidigraph $Q$,
        \begin{equation}
            G'(x) := G(x) \quad \text{for all $x \in Q_{0}$};
        \end{equation}
        \item \textbf{Morphisms:}
        For each object $x \in Q_{0}$,
        \begin{equation}
            G'(\text{id}_{x}) := \text{id}_{G(x)}.
        \end{equation}
        For each arrow $b \in Q_{1}$,
        \begin{equation}
            G'(b) := G(b).
        \end{equation}
        For each path of arrows $b_{n} \circ \dots \circ b_{1}$ that make up a morphism of $\mathcal{F}(C)$,
        \begin{equation}
            G'(b_{n} \circ \dots \circ b_{1}) := G(b_{n}) \circ \dots \circ G(b_{1}).
        \end{equation}
    \end{enumerate}

    Once $G'$ has been constructed, we can apply the forgetful functor $U: \cat{Cat} \to \cat{Quiv}$ to obtain $U(G')$.
}
\begin{equation}
    U(G') \circ \eta_{Q} = G.
\end{equation}
\begin{equation}
% https://q.uiver.app/#q=WzAsNyxbMiwxLCJRIl0sWzIsMiwiVShcXG1hdGhjYWx7Rn0oUSkpIl0sWzQsMSwiXFxtYXRoY2Fse0Z9KFEpIl0sWzIsMCwiXFx0ZXh0e0luICRcXGNhdHtRdWl2fSQ6fSJdLFs0LDAsIlxcdGV4dHtJbiAkXFxjYXR7Q2F0fSR9OiJdLFswLDIsIlUoXFxtYXRoY2Fse0N9KSJdLFs0LDIsIlxcbWF0aGNhbHtDfSJdLFswLDEsIlxcZXRhX3tRfSJdLFswLDIsIkYiLDAseyJjdXJ2ZSI6LTIsImNvbG91ciI6WzAsNjAsNjBdfSxbMCw2MCw2MCwxXV0sWzIsMSwiVSIsMix7ImNvbG91ciI6WzEyMCw2MCw2MF19LFsxMjAsNjAsNjAsMV1dLFswLDUsIkciLDJdLFs2LDUsIlUiLDAseyJvZmZzZXQiOi0xLCJjdXJ2ZSI6LTIsImNvbG91ciI6WzEyMCw2MCw2MF19LFsxMjAsNjAsNjAsMV1dLFsxLDUsIlxcZXhpc3RzISBcXDsgVShHJykiLDIseyJsYWJlbF9wb3NpdGlvbiI6NDAsInN0eWxlIjp7ImJvZHkiOnsibmFtZSI6ImRhc2hlZCJ9fX1dLFsyLDYsIlxcZXhpc3RzISBcXDsgRyciLDAseyJzdHlsZSI6eyJib2R5Ijp7Im5hbWUiOiJkYXNoZWQifX19XV0=
\begin{tikzcd}[ampersand replacement=\&]
    \&\& {\text{In $\cat{Quiv}$:}} \&\& {\text{In $\cat{Cat}$}:} \\
    \&\& Q \&\& {\mathcal{F}(Q)} \\
    {U(\mathcal{C})} \&\& {U(\mathcal{F}(Q))} \&\& {\mathcal{C}}
    \arrow["F", color={rgb,255:red,214;green,92;blue,92}, curve={height=-12pt}, from=2-3, to=2-5]
    \arrow["G"', from=2-3, to=3-1]
    \arrow["{\eta_{Q}}", from=2-3, to=3-3]
    \arrow["U"', color={rgb,255:red,92;green,214;blue,92}, from=2-5, to=3-3]
    \arrow["{\exists! \; G'}", dashed, from=2-5, to=3-5]
    \arrow["{\exists! \; U(G')}"'{pos=0.4}, dashed, from=3-3, to=3-1]
    \arrow["U", shift left, color={rgb,255:red,92;green,214;blue,92}, curve={height=-12pt}, from=3-5, to=3-1]
\end{tikzcd}
\end{equation}

This universal property is equivalent to the existence of a natural bijection\footnote{
    The unit natural transformation $\eta: \text{id}_{\cat{Quiv}} \Rightarrow U \circ \mathcal{F}$ of an adjunction always produces a natural bijection.
    
    The counit natural transformation $\epsilon: \mathcal{F} \circ U  \Rightarrow \text{id}_{\cat{Cat}}$ of an adjunction produces the same natural bijection as the unit natural transformation.
}
\begin{equation}
    \text{Hom}_{\cat{Quiv}}(Q, U(\mathcal{C})) \cong \text{Hom}_{\cat{Cat}}(\mathcal{F}(Q), \mathcal{C}),
\end{equation}
which means that
\begin{enumerate}
    \item every functor $\mathcal{F}(Q) \to \mathcal{C}$ in $\cat{Cat}$ corresponds to a unique multidigraph morphism $Q \to U(\mathcal{C})$ (namely $G = U(G') \circ \eta_{Q}$) in $\cat{Quiv}$; and
    \item every multidigraph morphism $Q \to U(\mathcal{C})$ in $\cat{Quiv}$ corresponds to a unique functor $\mathcal{F}(Q) \to \mathcal{C}$ in $\cat{Cat}$.
\end{enumerate}


%%%%%%%%%%%%%%%%%%%%%%%%%%%%%%%%%%%%%%%%%%%%%%%%%%
\paragraph{Currying.}

\newthought{\emph{Currying} can be} thought of as taking a two-valued map and converting it into a collection of one-valued maps; it is the most general way to reinterpret morphisms from a product as morphisms into an exponential object.

Formally, for any objects $X$, $Y$, $Z$ in a category $\mathcal{C}$, if the exponential object $Z^{Y}$ exists as an object in $\mathcal{C}$, then $Z^{Y}$ come equipped with a morphism
\begin{equation}
    \text{eval}: Z^{Y} \times Y \to Z.
\end{equation}
From the universal property of exponential objects, for any morphism $\alpha: X \times Y \to Z$ there exists a unique morphism
\begin{equation}
    \tilde{\alpha}: X \to Z^{B},
\end{equation}
called the \emph{curried form} of $\alpha$, such that the following diagram commutes:\footnote{
    The morphism $\tilde{\alpha}: X \to Z^{B}$ is the universal, unique morphism from the universal property; the morphism $\tilde{\alpha} \times \text{id}_{Y}$ is uniquely constructed after $\tilde{\alpha}$ is determined.
}
\begin{equation}
    % https://q.uiver.app/#q=WzAsMyxbMCwwLCJYIFxcdGltZXMgWSJdLFsyLDAsIloiXSxbMCwyLCJaXntCfSBcXHRpbWVzIFkiXSxbMCwxLCJcXGFscGhhIl0sWzAsMiwiXFx0aWxkZXtcXGFscGhhfVxcdGltZXNcXHRleHR7aWR9X3tZfSIsMix7InN0eWxlIjp7ImJvZHkiOnsibmFtZSI6ImRhc2hlZCJ9fX1dLFsyLDEsIlxcdGV4dHtldmFsfSIsMl1d
\begin{tikzcd}[ampersand replacement=\&]
    {X \times Y} \&\& Z \\
    \\
    {Z^{B} \times Y}
    \arrow["\alpha", from=1-1, to=1-3]
    \arrow["{\tilde{\alpha}\times\text{id}_{Y}}"', dashed, from=1-1, to=3-1]
    \arrow["{\text{eval}}"', from=3-1, to=1-3]
\end{tikzcd}
\end{equation}
where\footnote{
\textbf{Example of currying in $\cat{Set}$.}
Suppose we have sets
\begin{align}
    & X = \{x_{1}, x_{2}\} \\
    & Y = \{y_{1}, y_{2}\} \\
    & Z = \{z_{1}, z_{2}\},
\end{align}
and a function
\begin{align}
    \alpha: X \times Y \to Z \quad \text{such that} \\
    \alpha(x_{1}, y_{1}) = z_{1}, \alpha(x_{1}, y_{2}) = z_{2} \\
    \alpha(x_{2}, y_{1}) = z_{2}, \alpha(x_{2}, y_{2}) = z_{1}.
\end{align}
We curry $\alpha$:
\begin{equation}
    (\alpha: X \times Y \to Z) \xrightarrow{\text{curry}} (\tilde{\alpha}: X \to Z^{Y});
\end{equation}
for each $x \in X$, we produce a function
\begin{equation}
\begin{aligned}
    & \tilde{\alpha}(x): Y \to Z \quad \text{defined by} \\
    & \tilde{\alpha}(x)(y) := \alpha(x, y).
\end{aligned}
\end{equation}
Therefore, $\tilde{\alpha}(x_{1})$ is the function
\begin{equation}
\begin{aligned}
    & \tilde{\alpha}(x_{1}): Y \to Z \quad \text{such that} \\
    & \tilde{\alpha}(x_{1})(y_{1}) = z_{1}, \tilde{\alpha}(x_{1})(y_{2}) = z_{2};
\end{aligned}
\end{equation}
and $\tilde{\alpha}(x_{2})$ is the function
\begin{equation}
\begin{aligned}
    & \tilde{\alpha}(x_{2}): Y \to Z \quad \text{such that} \\
    & \tilde{\alpha}(x_{2})(y_{1}) = z_{2}, \tilde{\alpha}(x_{2})(y_{2}) = z_{1}.
\end{aligned}
\end{equation}
}
\begin{equation}
    \alpha = \text{eval} \circ (\tilde{\alpha} \times \text{id}_{Y});
\end{equation}
a natural isomorphism
\begin{equation}
    \text{Hom}_{\mathcal{C}}(X \times Y, Z) \cong \text{Hom}_{\mathcal{C}}(X, Z^{Y}),
\end{equation}
can be obtained from this condition.
\draftnote{purple}{PS}{How can we obtain the natural isomorphism of functors from this condition ?}

Canonically, currying relies on this natural isomorphism of functors, called the \emph{exponential adjunction}\footnote{
    In a category $\mathcal{C}$, the exponential adjunction is an adjunction between the product functor $- \times Y: \mathcal{C} \to \mathcal{C}$ and the Hom functor $\text{Hom}(Y, -): \mathcal{C} \to \mathcal{C}$.
    This adjunction only holds if category $\mathcal{C}$ is a Cartesian closed category.
}, which says that, for objects $X$, $Y$ and $Z$ in a category $\mathcal{C}$,
\begin{equation}
    \text{Hom}_{\mathcal{C}}(X \times Y, Z) \cong \text{Hom}_{\mathcal{C}}(X, Z^{Y}),
\end{equation}
where there exists an exponential object $Z^{Y}$.

\newthought{In a \emph{Cartesian closed category}} $\mathcal{C}$,
\begin{enumerate}
    \item \textbf{Products:}
    For any two objects $X$ and $Y$ in $\mathcal{C}$, the product $X \times Y$ exists in $\mathcal{C}$;
    \item \textbf{Exponentials:}
    For any two objects $X$ and $Y$ in $\mathcal{C}$, the exponential object $B^{A}$ exists in $\mathcal{C}$;
    \item \textbf{Terminal object:}
    There exists a terminal object $1$ in $\mathcal{C}$.
\end{enumerate}
Since all pairs of objects can make products and exponentials, the exponential adjunction holds naturally for all objects in the $\mathcal{C}$.


%%%%%%%%%%%%%%%%%%%%%%%%%%%%%%%%%%%%%%%%%%%%%%%%%%
\subsection{Universal constructions from limits.}

\newthought{\emph{Limits} can be} thought of as extracting a common structure from multiple pieces of data in a way that preserves the relationships between those pieces of data\footnote{
The formal definition of limits is not required to understand this work, but for interested parties:
A limit is a terminal object in the category of cones over a diagram.
}; in short, limits are intersections.

Limits provide the most general way to extract what is common among pieces of data, while respecting how those pieces of data relate to each other.
Since the limit object is the most general common structure of some pieces of data, any object formed by extracting common structure from the same pieces of data must uniquely factor through the limit\footnote{
This is the \emph{universal property} of the limit.
}.

\paragraph{Pullbacks.}
\newthought{The \emph{pullback} of} a pair of morphisms $f: A \to C$ and $g: B \to C$ is the most general way to extract a new object from $f$ and $g$ that consists of the parts of two objects $A$ and $B$ that are the same when mapped into the object $C$.
Formally, consider two morphisms $f: A \to C$ and $g: B \to C$ in a category $\mathcal{C}$.
A pullback $(P, p_{A}, p_{B})$ of the morphisms $f, g$ is an object $P$ together with \emph{pullback projection} morphisms
\begin{align}
    & p_{A}: P \to A \\
    & p_{B}: P \to B
\end{align}
such that the following diagram commutes:\footnote{
In diagrams, constructing a pullback $(P, p_{A}, p_{B})$ looks like:
\begin{equation}
% https://q.uiver.app/#q=WzAsOSxbNCwwLCJQIl0sWzUsMCwiQSJdLFs0LDEsIkIiXSxbNSwxLCJDIl0sWzMsMV0sWzIsMV0sWzEsMCwiQSJdLFsxLDEsIkMiXSxbMCwxLCJCIl0sWzAsMSwicF8xIl0sWzAsMiwicF8yIiwyXSxbMSwzLCJmIl0sWzIsMywiZyIsMl0sWzUsNCwiXFx0ZXh0e3B1bGxiYWNrfSJdLFs2LDcsImYiXSxbOCw3LCJnIiwyXV0=
\begin{tikzcd}[ampersand replacement=\&]
	\& A \&\&\& P \& A \\
	B \& C \& {} \& {} \& B \& C
	\arrow["f", from=1-2, to=2-2]
	\arrow["{p_A}", from=1-5, to=1-6]
	\arrow["{p_B}"', from=1-5, to=2-5]
	\arrow["f", from=1-6, to=2-6]
	\arrow["g"', from=2-1, to=2-2]
	\arrow["{\text{pullback}}", from=2-3, to=2-4]
	\arrow["g"', from=2-5, to=2-6]
\end{tikzcd}
\end{equation}
}
\begin{equation}
% https://q.uiver.app/#q=WzAsNCxbMCwwLCJLIl0sWzIsMCwiWCJdLFswLDIsIlgiXSxbMiwyLCJZIl0sWzAsMSwicF97MX0iLDAseyJvZmZzZXQiOi0xfV0sWzAsMiwicF97Mn0iLDIseyJvZmZzZXQiOjF9XSxbMSwzLCJmIl0sWzIsMywiZiJdLFswLDMsIlxcZXhpc3RzICEgXFw7IGgiLDAseyJzdHlsZSI6eyJib2R5Ijp7Im5hbWUiOiJkYXNoZWQifX19XV0=
\begin{tikzcd}[ampersand replacement=\&]
    P \&\& A \\
    \\
    B \&\& C
    \arrow["{p_{A}}", shift left, from=1-1, to=1-3]
    \arrow["{p_{B}}"', shift right, from=1-1, to=3-1]
    \arrow["f", from=1-3, to=3-3]
    \arrow["g", from=3-1, to=3-3]
\end{tikzcd}
\end{equation}
where\footnote{
\textbf{Example of a pullback in $\cat{Set}$.}
Suppose we have sets
\begin{align}
    & A = \{a_{1}, a_{2}\} \\
    & B = \{b_{1}, b_{2}, b_{3}\} \\
    & C = \{c_{1}, c_{2}\},
\end{align}
and functions
\begin{equation}
\begin{aligned}
    & f: A \to C \quad \text{such that} \\
    & f(a_{1}) = c_{1}, f(a_{2}) = c_{2}
\end{aligned}
\end{equation}
and
\begin{equation}
\begin{aligned}
    & g: B \to C \quad \text{such that} \\
    & g(b_{1}) = c_{1}, g(b_{2}) = c_{2}, g(b_{3}) = c_{2}.
\end{aligned}
\end{equation}
The pullback object $A \times_{C} B$ is
\begin{align}
    & A \times_{C} B = \{(a,b) \in A \times B \mid f(a) = g(b)\} \\
    & = \{(a_{1}, b_{1}), (a_{2}, b_{2}), (a_{2}, b_{3})\}.
\end{align}

So the pullback has associated $a_{1}$ with $b_{1}$, $a_{2}$ with $b_{2}$, and $a_{2}$ with $b_{3}$ because they match in $C$.
The pullback object $A \times_{C} B \subseteq A \times B$ is a subset of the Cartesian product, filtered by agreement over $C$.
}
\begin{equation}
    f \circ p_{A} = g \circ p_{B}.
\end{equation}

\begin{notation}
    The pullback object $P$ of the morphisms $f: A \to C$ and $g: B \to C$ is often denoted by $A \times_{C} B$, which is read as the "pullback of $A$ and $B$ over $C$"\footnote{
    $\times_{C}$ is also sometimes called the \emph{fibre product} over $C$.
    }.
\end{notation}

The pullback object $P$ is universal with the property: for any other object $Q$ with morphisms $q_{A}: Q \to A$ and $q_{B}: Q \to B$ such that $f \circ q_{A} = g \circ q_{B}$, there exists a unique morphism
\begin{align}
    & u: Q \to P \quad \text{such that} \\
    & p_{A} \circ u = q_{A} \\
    & p_{B} \circ u = q_{B},
\end{align}
which in diagram form is:
\begin{equation}
% https://q.uiver.app/#q=WzAsNSxbMCwwLCJRIl0sWzEsMSwiUCJdLFszLDEsIkEiXSxbMSwzLCJCIl0sWzMsMywiQyJdLFswLDEsIlxcZXhpc3RzICEgXFw7dSIsMCx7InN0eWxlIjp7ImJvZHkiOnsibmFtZSI6ImRhc2hlZCJ9fX1dLFswLDIsInFfMSIsMCx7ImN1cnZlIjotMX1dLFswLDMsInFfMiIsMix7ImN1cnZlIjoxfV0sWzEsMiwicF8xIl0sWzEsMywicF8yIiwyXSxbMiw0LCJmIl0sWzMsNCwiZyIsMl1d
\begin{tikzcd}[ampersand replacement=\&]
	Q \\
	\& P \&\& A \\
	\\
	\& B \&\& C
	\arrow["{\exists ! \;u}", dashed, from=1-1, to=2-2]
	\arrow["{q_A}", curve={height=-6pt}, from=1-1, to=2-4]
	\arrow["{q_B}"', curve={height=6pt}, from=1-1, to=4-2]
	\arrow["{p_A}", from=2-2, to=2-4]
	\arrow["{p_B}"', from=2-2, to=4-2]
	\arrow["f", from=2-4, to=4-4]
	\arrow["g"', from=4-2, to=4-4]
\end{tikzcd}
\end{equation}


\paragraph{Products.}
\newthought{A \emph{product} is} the most general way to collect and coordinate morphisms into multiple target objects.
Formally, the \emph{product} $(A \times B, p_{A}, p_{B})$ of two objects $A$ and $B$ in a category $\mathcal{C}$ is another object $A \times B$ together with two \emph{projection} morphisms\footnote{
    The \emph{product} $(A \times B, p_{A}, p_{B})$ of two objects $A$ and $B$ in a category $\mathcal{C}$ is the pullback of the morphisms $A \xrightarrow{!_{A}} 1 \xleftarrow{!_{B}} B$, where $1$ is the terminal object in $\mathcal{C}$, $!_{A}: A \to 1$ is the unique morphism from $A$ to the terminal object, and $!_{B}: B \to 1$ is the unique morphism from $B$ to the terminal object:
    \begin{equation}
    % https://q.uiver.app/#q=WzAsOSxbNCwwLCJBIFxcdGltZXMgQiJdLFs1LDAsIkEiXSxbNCwxLCJCIl0sWzUsMSwiMSJdLFszLDFdLFsyLDFdLFsxLDAsIkEiXSxbMSwxLCIxIl0sWzAsMSwiQiJdLFswLDEsInBfe0F9Il0sWzAsMiwicF97Qn0iLDJdLFsxLDMsIiFfe0F9Il0sWzIsMywiIV97Qn0iLDJdLFs1LDQsIlxcdGV4dHtwdWxsYmFja30iXSxbNiw3LCIhX3tBfSJdLFs4LDcsIiFfe0J9IiwyXV0=
    \begin{tikzcd}[ampersand replacement=\&]
        \& A \&\&\& {A \times B} \& A \\
        B \& 1 \& {} \& {} \& B \& 1
        \arrow["{!_{A}}", from=1-2, to=2-2]
        \arrow["{p_{A}}", from=1-5, to=1-6]
        \arrow["{p_{B}}"', from=1-5, to=2-5]
        \arrow["{!_{A}}", from=1-6, to=2-6]
        \arrow["{!_{B}}"', from=2-1, to=2-2]
        \arrow["{\text{pullback}}", from=2-3, to=2-4]
        \arrow["{!_{B}}"', from=2-5, to=2-6]
    \end{tikzcd}
    \end{equation}
}
\begin{align}
    & p_{A}: A \times B \to A \\
    & p_{B}: A \times B \to B.
\end{align}
As a diagram, $A \times B$ is shown as\footnote{
    \textbf{Example of a product in $\cat{Set}$.}
    Suppose we have sets
    \begin{align}
        & A = \{a_{1}, a_{2}\} \\
        & B = \{b_{1}, b_{2}\}.
    \end{align}
    The product $A \times B$ is
    \begin{align}
        A \times B & = \{(a,b) \mid a \in A, b \in B\} \\
        & = \{(a_{1}, b_{1}), (a_{1}, b_{2}), (a_{2}, b_{1}), (a_{2}, b_{2})\}.
    \end{align}
    The projections are defined as
    \begin{align}
        p_{A}((a, b)) = a \\
        p_{B}((a, b)) = b.
    \end{align}
    The product $A \times B$ is the Cartesian product in $\cat{Set}$.
}
\begin{equation}
    % https://q.uiver.app/#q=WzAsMyxbMCwwLCJBIFxcdGltZXMgQiJdLFsxLDAsIkEiXSxbMCwxLCJCIl0sWzAsMSwicF97QX0iXSxbMCwyLCJwX3tCfSIsMl1d
    \begin{tikzcd}[ampersand replacement=\&]
        {A \times B} \& A \\
        B
        \arrow["{p_{A}}", from=1-1, to=1-2]
        \arrow["{p_{B}}"', from=1-1, to=2-1]
    \end{tikzcd}
\end{equation}


\paragraph{Kernel pairs.}
\newthought{The \emph{kernel pair} of} a morphism $f: A \to B$ is the most general way to extract a new object from $f$ that consists of the parts of $A$ that are the same in $B$.
Formally, the kernel pair $(K, p_{1}, p_{2})$ of a morphism $f: A \to B$ in a category $\mathcal{C}$ is the pullback of $f$ with itself\footnote{
    In diagrams, constructing a kernel pair $(K, p_{1}, p_{2})$ looks like:
    \begin{equation}
    % https://q.uiver.app/#q=WzAsOSxbNCwwLCJLIl0sWzUsMCwiQSJdLFs0LDEsIkEiXSxbNSwxLCJCIl0sWzMsMV0sWzIsMV0sWzEsMCwiQSJdLFsxLDEsIkIiXSxbMCwxLCJBIl0sWzAsMSwicF8xIl0sWzAsMiwicF8yIiwyXSxbMSwzLCJmIl0sWzIsMywiZiIsMl0sWzUsNCwiXFx0ZXh0e3B1bGxiYWNrfSJdLFs2LDcsImYiXSxbOCw3LCJmIiwyXV0=
    \begin{tikzcd}[ampersand replacement=\&]
        \& A \&\&\& K \& A \\
        A \& B \& {} \& {} \& A \& B
        \arrow["f", from=1-2, to=2-2]
        \arrow["{p_1}", from=1-5, to=1-6]
        \arrow["{p_2}"', from=1-5, to=2-5]
        \arrow["f", from=1-6, to=2-6]
        \arrow["f"', from=2-1, to=2-2]
        \arrow["{\text{pullback}}", from=2-3, to=2-4]
        \arrow["f"', from=2-5, to=2-6]
    \end{tikzcd}
    \end{equation}
}:
\begin{equation}
% https://q.uiver.app/#q=WzAsNCxbMCwwLCJLIl0sWzIsMCwiQSJdLFswLDIsIkEiXSxbMiwyLCJCIl0sWzAsMSwicF8xIl0sWzAsMiwicF8yIiwyXSxbMSwzLCJmIl0sWzIsMywiZiIsMl1d
\begin{tikzcd}[ampersand replacement=\&]
	K \&\& A \\
	\\
	A \&\& B
	\arrow["{p_1}", from=1-1, to=1-3]
	\arrow["{p_2}"', from=1-1, to=3-1]
	\arrow["f", from=1-3, to=3-3]
	\arrow["f"', from=3-1, to=3-3]
\end{tikzcd}
\end{equation}
where $p_{1}, p_{2}: K \rightrightarrows A$ are the pullback projection morphisms that satisfy
\begin{equation}
    f \circ p_{1} = f \circ p_{2}
\end{equation}
to make the pullback diagram commute, and $K$ is the kernel pair object where
\begin{equation}
    K = A \times_{B} A.
\end{equation}


%%%%%%%%%%%%%%%%%%%%%%%%%%%%%%%%%%%%%%%%%%%%%%%%%%
\subsection{Universal constructions from colimits.}

\newthought{\emph{Colimits}\footnote{
    Colimits are the categorical \emph{dual} of limits; this means that if you reverse all the arrows of a limit you get a colimit.
} can be} thought of as constructing new structures from existing structures by gluing pieces of data together along shared overlaps or connections, in a way that respects how those pieces of data are related\footnote{
The formal definition of colimits is not required to understand this work, but for interested parties:
A colimit is an initial object in the category of cocones over a diagram.
}; in short, colimits are amalgamations.

Colimits provide the most general way to glue pieces of data together as specified by the morphisms in a diagram, while preserving the structure of the connections.
Since the colimit object is the most general structure constructed by gluing together pieces of data, any object constructed by gluing together the same pieces of data must uniquely factor through the colimit\footnote{
    This is the \emph{universal property} of the colimit.
}.


\paragraph{Pushouts.}
\newthought{The \emph{pushout} of} a pair of morphisms $f: A \to B$ and $g: A \to C$ is the most general way to construct a new object by gluing the two objects $B$ and $C$ together along a shared subobject $A$.
Formally, consider two morphisms $f: A \to B$ and $g: A \to C$ in a category $\mathcal{C}$.
A pushout $(P, i_{B}, i_{C})$ of the morphisms $f,g$ is an object $P$ together with \emph{pushout coprojection} morphisms
\begin{align}
    & i_{B}: B \to P \\
    & i_{C}: C \to P
\end{align}
such that the following diagram commutes:\footnote{
    As a diagram, constructing a pushout looks like:
    \begin{equation}
    % https://q.uiver.app/#q=WzAsOSxbNCwwLCJBIl0sWzUsMCwiQiJdLFs0LDEsIkMiXSxbNSwxLCJQIl0sWzMsMV0sWzIsMV0sWzAsMCwiQSJdLFsxLDAsIkIiXSxbMCwxLCJDIl0sWzAsMSwiZiJdLFswLDIsImciLDJdLFsyLDMsImlfe0N9IiwyXSxbMSwzLCJpX3tCfSJdLFs1LDQsIlxcdGV4dHtwdXNob3V0fSJdLFs2LDcsImYiXSxbNiw4LCJnIiwyXV0=
    \begin{tikzcd}[ampersand replacement=\&]
        A \& B \&\&\& A \& B \\
        C \&\& {} \& {} \& C \& P
        \arrow["f", from=1-1, to=1-2]
        \arrow["g"', from=1-1, to=2-1]
        \arrow["f", from=1-5, to=1-6]
        \arrow["g"', from=1-5, to=2-5]
        \arrow["{i_{B}}", from=1-6, to=2-6]
        \arrow["{\text{pushout}}", from=2-3, to=2-4]
        \arrow["{i_{C}}"', from=2-5, to=2-6]
    \end{tikzcd}
    \end{equation}
}
\begin{equation}
% https://q.uiver.app/#q=WzAsNCxbMCwwLCJBIl0sWzIsMCwiQiJdLFswLDIsIkMiXSxbMiwyLCJQIl0sWzAsMSwiZiJdLFswLDIsImciLDJdLFsyLDMsImlfe0N9IiwyXSxbMSwzLCJpX3tCfSJdXQ==
\begin{tikzcd}[ampersand replacement=\&]
    A \&\& B \\
    \\
    C \&\& P
    \arrow["f", from=1-1, to=1-3]
    \arrow["g"', from=1-1, to=3-1]
    \arrow["{i_{B}}", from=1-3, to=3-3]
    \arrow["{i_{C}}"', from=3-1, to=3-3]
\end{tikzcd}
\end{equation}
where\footnote{
    \textbf{Example of a pushout in $\cat{Set}$.}
    Suppose we have sets
    \begin{align}
        & A = \{a_{1}, a_{2}\} \\
        & B = \{b_{1}, b_{2}\} \\
        & C = \{c_{1}, c_{2}\},
    \end{align}
    and functions
    \begin{equation}
    \begin{aligned}
        & f: A \to B \quad \text{such that} \\
        & f(a_{1}) = b_{1}, f(a_{2}) = b_{2}
    \end{aligned}
    \end{equation}
    and
    \begin{equation}
    \begin{aligned}
        & g: A \to C \quad \text{such that} \\
        & g(a_{1}) = c_{1}, g(a_{2}) = c_{1}.
    \end{aligned}
    \end{equation}
    For each $a_{i} \in A$, the elements in $B$ and $C$ with $f(a_{i})=g(a_{i})$ are identified with each other in $B \sqcup_{A} C$
    \begin{align}
        & f(a_{1}) = g(a_{1}) \\
        \implies & \text{glue $b_{1}$ to $c_{1}$} \\
        \implies & b_{1} \sim c_{1} \text{ in $B \sqcup_{A} C$}
    \end{align}
    and
    \begin{align}
        & f(a_{2}) = g(a_{2}) \\
        \implies & \text{glue $b_{2}$ to $c_{1}$} \\
        \implies & b_{2} \sim c_{1} \text{ in $B \sqcup_{A} C$},
    \end{align}
    and so in $B \sqcup_{A} C$ we have
    \begin{equation}
        b_{1} \sim c_{1} \quad \text{and} \quad b_{2} \sim c_{1}.
    \end{equation}
    The pushout object $B \sqcup_{A} C$ is
    \begin{align}
        B \sqcup_{A} C &= ((B \times \{0\}) \cup (C \times \{1\}))/\{f(a) \sim g(a)\}\\
        & = \{[c_{1}]_{\sim}, c_{2}\}
    \end{align}
    where $[c_{1}]_{\sim} = \{c_{1}, b_{1}, b_{2}\}$.
    
    So the pushout has "glued" $B$ and $C$ together by identifying parts of $B$ and $C$ using the images of $A$ under $f$ and $g$.
    The pushout is the union of $B$ and $C$ with $f(a) \sim g(a)$ for all $a \in A$.
}
\begin{equation}
    i_{B} \circ f = i_{C} \circ g.
\end{equation}

\begin{notation}
    The pushout object $P$ of the morphisms $f: A \to B$ and $g: A \to C$ is often denoted by $B \sqcup_{A} C$, which is read as the "pushout of $B$ and $C$ over $A$".
\end{notation}

The pushout object $P$ is universal with the property: for any other object $Q$ with morphisms $j_{B}: B \to Q$ and $j_{C}: C \to Q$ such that $j_{B} \circ f = j_{C} \circ g$, there exists a unique morphism
\begin{align}
    & u: P \to Q \quad \text{such that} \\
    & u \circ i_{B} = j_{B} \\
    & u \circ i_{C} = j_{C},
\end{align}
which as a diagram is:
\begin{equation}
% https://q.uiver.app/#q=WzAsNSxbMCwwLCJBIl0sWzIsMCwiQiJdLFswLDIsIkMiXSxbMiwyLCJQIl0sWzMsMywiUSJdLFswLDEsImYiXSxbMCwyLCJnIiwyXSxbMiwzLCJpX3tDfSIsMl0sWzEsMywiaV97Qn0iXSxbMSw0LCJqX3tCfSIsMCx7ImN1cnZlIjotMn1dLFsyLDQsImpfe0N9IiwyLHsiY3VydmUiOjJ9XSxbMyw0LCJcXGV4aXN0cyAhIFxcOyB1IiwwLHsic3R5bGUiOnsiYm9keSI6eyJuYW1lIjoiZGFzaGVkIn19fV1d
\begin{tikzcd}[ampersand replacement=\&]
    A \&\& B \\
    \\
    C \&\& P \\
    \&\&\& Q
    \arrow["f", from=1-1, to=1-3]
    \arrow["g"', from=1-1, to=3-1]
    \arrow["{i_{B}}", from=1-3, to=3-3]
    \arrow["{j_{B}}", curve={height=-12pt}, from=1-3, to=4-4]
    \arrow["{i_{C}}"', from=3-1, to=3-3]
    \arrow["{j_{C}}"', curve={height=12pt}, from=3-1, to=4-4]
    \arrow["{\exists ! \; u}", dashed, from=3-3, to=4-4]
\end{tikzcd}
\end{equation}
Pushouts automatically "glue" the compositions in the structure they construct, so we don't need to worry about that.


\paragraph{Coequalizer.}
\newthought{The \emph{coequalizer} of} two parallel morphisms $f,g: A \to B$ gives us the most efficient way to glue together parts of $B$ so that the two different ways of mapping $A \to B$ become indistinguishable; this generalises the notion of identifying equivalent elements, which are related by a pair of parallel morphisms, by "gluing together" the images of the morphisms.

Formally, the coequalizer $(Q, q)$ of a pair of parallel morphisms $f,g: A \rightrightarrows B$ in a category $\mathcal{C}$ is the pushout of the morphisms $f,g$:
\begin{equation}
% https://q.uiver.app/#q=WzAsNCxbMCwwLCJBIl0sWzIsMCwiQiJdLFswLDIsIkIiXSxbMiwyLCJRIl0sWzAsMSwiZiJdLFswLDIsImciLDJdLFsyLDMsInEiLDJdLFsxLDMsInEiXV0=
\begin{tikzcd}[ampersand replacement=\&]
    A \&\& B \\
    \\
    B \&\& Q
    \arrow["f", from=1-1, to=1-3]
    \arrow["g"', from=1-1, to=3-1]
    \arrow["q", from=1-3, to=3-3]
    \arrow["q"', from=3-1, to=3-3]
\end{tikzcd}
\end{equation}
where $q: B \to Q$ is the coequalizer morphism that satisfies\footnote{
    \textbf{Example of a coequalizer in $\cat{Set}$.}
    Suppose we have sets
    \begin{align}
        & A = \{a_{1}, a_{2}\} \\
        & B = \{b_{1}, b_{2}, b_{3}, b_{4}\},
    \end{align}
    and functions
    \begin{align}
        & f,g: A \to B \quad \text{such that} \\
        & f(a_{1}) = b_{1}, f(a_{2}) = b_{2} \text{ and} \\
        & g(a_{1}) = b_{2}, g(a_{2}) = b_{3}
    \end{align}
    For each $a_{i} \in A$, the coequalizer morphism $q$ forces elements in $B$ with $f(a_{i})=g(a_{i})$ to be the same in $B \sqcup_{A} B$:
    \begin{align}
        & f(a_{1}) = g(a_{1}) \\
        \implies & \text{glue $b_{1}$ to $b_{2}$} \\
        \implies & b_{1} \sim b_{2} \text{ in $B \sqcup_{A} B$}
    \end{align}
    and
    \begin{align}
        & f(a_{2}) = g(a_{2}) \\
        \implies & \text{glue $b_{2}$ to $b_{3}$} \\
        \implies & b_{2} \sim b_{3} \text{ in $B \sqcup_{A} B$},
    \end{align}
    and so in $B \sqcup_{A} B$ we have
    \begin{equation}
        b_{1} \sim b_{2} \sim b_{3}.
    \end{equation}
    The coequalizer object $B \sqcup_{A} B$ is
    \begin{equation}
        B \sqcup_{A} B = \{[b_{1}]_{\sim}, [b_{4}]_{\sim}\},
    \end{equation}
    where $[b_{1}]_{\sim} = \{b_{1}, b_{2}, b_{3}\}$ and $[b_{4}]_{\sim} = \{b_{4}\}$.
    
    So the coequalizer has constructed the object $B \sqcup_{A} B$ by "gluing" $B$ to itself at the parts of $B$ that give the same result in $A$ under $f$ and $g$.
}
\begin{equation}
    q \circ f = q \circ g;
\end{equation}
and $Q$ is the coequalizer object with
\begin{equation}
    Q = B \sqcup_{A} B.
\end{equation}
The diagram for the coequalizer of the morphisms $f,g: A \to B$ is usually drawn as\footnote{
In diagrams, constructing a coequalizer looks like
\begin{equation}
% https://q.uiver.app/#q=WzAsNyxbNCwwLCJBIl0sWzUsMCwiQiJdLFs2LDAsIlEiXSxbMCwwLCJBIl0sWzEsMCwiQiJdLFsyLDBdLFszLDBdLFswLDEsImciLDIseyJvZmZzZXQiOjF9XSxbMCwxLCJmIiwwLHsib2Zmc2V0IjotMX1dLFsxLDIsInEiXSxbMyw0LCIiLDAseyJvZmZzZXQiOjF9XSxbMyw0LCIiLDIseyJvZmZzZXQiOi0xfV0sWzUsNiwiXFx0ZXh0e2NvZXF1YWxpemVyfSJdXQ==
\begin{tikzcd}[ampersand replacement=\&]
    A \& B \& {} \& {} \& A \& B \& Q
    \arrow[shift right, from=1-1, to=1-2]
    \arrow[shift left, from=1-1, to=1-2]
    \arrow["{\text{coequalizer}}", from=1-3, to=1-4]
    \arrow["g"', shift right, from=1-5, to=1-6]
    \arrow["f", shift left, from=1-5, to=1-6]
    \arrow["q", from=1-6, to=1-7]
\end{tikzcd}
\end{equation}
}:
\begin{equation}
% https://q.uiver.app/#q=WzAsMyxbMCwwLCJBIl0sWzIsMCwiQiJdLFs0LDAsIlEiXSxbMCwxLCJnIiwyLHsib2Zmc2V0IjoxfV0sWzAsMSwiZiIsMCx7Im9mZnNldCI6LTF9XSxbMSwyLCJxIl1d
\begin{tikzcd}[ampersand replacement=\&]
    A \&\& B \&\& Q
    \arrow["g"', shift right, from=1-1, to=1-3]
    \arrow["f", shift left, from=1-1, to=1-3]
    \arrow["q", from=1-3, to=1-5]
\end{tikzcd}
\end{equation}
We can then use the universal property of coequalizers to construct the unique morphism:
for any other morphism $q': B \to Q'$ satisfying
\begin{equation}
    q' \circ f = q' \circ g,
\end{equation}
there exists a unique morphism $u: Q \to Q'$ such that the following diagram commutes
\begin{equation}
% https://q.uiver.app/#q=WzAsNCxbMCwwLCJBIl0sWzIsMCwiQiJdLFs0LDAsIlEiXSxbNCwyLCJRJyJdLFswLDEsImciLDIseyJvZmZzZXQiOjF9XSxbMCwxLCJmIiwwLHsib2Zmc2V0IjotMX1dLFsxLDIsInEiXSxbMSwzLCJxJyJdLFsyLDMsIlxcZXhpc3RzICEgXFw7IHUiLDAseyJzdHlsZSI6eyJib2R5Ijp7Im5hbWUiOiJkYXNoZWQifX19XV0=
\begin{tikzcd}[ampersand replacement=\&]
    A \&\& B \&\& Q \\
    \\
    \&\&\&\& {Q'}
    \arrow["g"', shift right, from=1-1, to=1-3]
    \arrow["f", shift left, from=1-1, to=1-3]
    \arrow["q", from=1-3, to=1-5]
    \arrow["{q'}", from=1-3, to=3-5]
    \arrow["{\exists ! \; u}", dashed, from=1-5, to=3-5]
\end{tikzcd}
\end{equation}
where
\begin{equation}
    q' = u \circ q.
\end{equation}


\draftnote{red}{DIVIDER}{}\draftnote{red}{DIVIDER}{}\draftnote{red}{DIVIDER}{}\draftnote{red}{DIVIDER}{}\draftnote{red}{DIVIDER}{}

\begin{definition}[Categorical product]
    The \emph{categorical product} of two categories $C_{1}$ and $C_{2}$ is the category $C_{1} \times C_{2}$.
    The objects of $C_{1} \times C_{2}$ are the pairs of objects $(B_{1}, B_{2})$, where $B_{1}$ is an object in $C_{1}$ and $B_{2}$ is an object in $C_{2}$.
    The morphisms in $C_{1} \times C_{2}$ are the pairs of morphisms $(f_{1}, f_{2})$, where $f_{1}$ is a morphism in $C_{1}$ and $f_{2}$ is a morphism in $C_{2}$.
    The composition of morphisms is defined component-wise.
\end{definition}

\begin{definition}[Sub-functors of sub-categories]
    We can define a functor $F: C_{1} \times C_{2} \to C$ as follows:
    \begin{enumerate}
        \item For each object $(B_{1}, B_{2})$ in $C_{1} \times C_{2}$, $F(B_{1}, B_{2})$ is the object in $C$ that corresponds to the pair $(B_{1}, B_{2})$.
    
        \item For each morphism $(f_{1}, f_{2}): (B_{1}, B_{2}) \to (B'_{1}, B'_{2})$ in $G_{1} \times G_{2}$, $F(f_{1}, f_{2})$ is the morphism in $G$ that corresponds to the pair $(f_{1}, f_{2})$.
    \end{enumerate}
    
    We can now define sub-functors $F_{1}$ of $G_{1}$ and $F_{2}$ of $G_{2}$ on an object $B$ of $G$ as follows:
    \begin{enumerate}
        \item $F_{1}(g_{1}, B) = F(g_{1}, 1_{G_{2}})(B)$ for all $g_{1} \in G_{1}$.
        \item $F_{2}(g_{2}, B) = F(1_{G_{1}}, g_{2})(B)$ for all $g_{1} \in G_{1}$.
    \end{enumerate}
    
    We can decompose $F$ into the sub-functors $F_{1}$ and $F_{2}$ as $F = F_{1} \times F_{2}$, if there is a decomposition $B = B_{1} \times B_{2}$ of $B$ into two sub-objects $B_{1}$ and $B_{2}$ such that:
    \begin{enumerate}
        \item For all $g_{1} \in G_{1}$ and $B_{2} \in G_{2}$, we have $F_{1}(g_{1}, B_{1} \times B_{2}) = F_{1}(g_{1}, B_{1} \times B_{2}$.
    
        \item For all $g_{2} \in G_{2}$ and $B_{1} \in G_{1}$, we have $F_{2}(g_{2}, B_{1} \times B_{2}) = B_{1} \times F_{2}(g_{2}, B_{2})$.
    \end{enumerate}
\end{definition}

\draftnote{red}{DIVIDER}{}\draftnote{red}{DIVIDER}{}\draftnote{red}{DIVIDER}{}\draftnote{red}{DIVIDER}{}\draftnote{red}{DIVIDER}{}
%%%%%%%%%%%%%%%%%%%%%%%%%%%%%%%%%%%%%%%%%%%%%%%%

%%%%%%%%%%%%%%%%%%%%%%%%%%%%%%%%%%%%%%%%%%%%%%%%
%%%%%%%%%%%%%%%%%%%%%%%%%%%%%%%%%%%%%%%%%%%%%%%%
\section{
Categorification of world states and their transformations
}
%%%%%%%%%%%%%%%%%%%%%%%%%%%%%%%%%%%%%%%%%%%%%%%%
\subsection{Recap.}
In our original framework, we defined an atomic multidigraph
\begin{equation}
    \hat{\mathscr{W}} = (W, \hat{D}, \hat{s}, \hat{t})
\end{equation}
consisting of
\begin{enumerate}
    \item a set $W$ of world states;
    \item a set $\hat{D}$ of atomic transformations; and
    \item source and target maps $\hat{s}, \hat{t}: \hat{D} \to W$.
\end{enumerate}

We generate the set $D$ of transformations by taking all finite directed walks in $\hat{\mathscr{W}}$ with composition $\circ$ given by concatenation of directed walks.
The trivial walks $1_{w} \in D$ for each $w \in W$ (called trivial transformations) serve as identities.
We then define the world $\mathscr{W}$ as the multidigraph
\begin{equation}
    \mathscr{W} = (W, D, s, t)
\end{equation}
where $s, t: D \to W$ are source and target maps.


%%%%%%%%%%%%%%%%%%%%%%%%%%%%%%%%%%%%%%%%%%%%%%%%
\subsection{Categorical conversion.}

In the category theoretic version of our framework, a world $\mathscr{W}$ is the \emph{free category} $\mathcal{F}(\hat{\mathscr{W}})$ generated by the atomic multidigraph $\hat{\mathscr{W}}$:\footnote{
    In both the construction of the world $\mathscr{W}$ from the atomic multidigraph $\hat{\mathscr{W}}$ in our original framework and in the construction of the free category $\mathcal{F}(\hat{\mathscr{W}})$ from the atomic multidigraph $\hat{\mathscr{W}}$ in the categorical conversion of our framework, we freely generate the walks in the graph theory sense (i.e., allowing repetition of vertices and edges) over $\hat{\mathscr{W}}$.
    }
\begin{equation}
    \mathscr{W} := \mathcal{F}(\hat{\mathscr{W}}).
\end{equation}
where $\mathcal{F}$ is the free category functor.
We call $\mathcal{F}(\hat{\mathscr{W}})$ a \emph{category of world states}.
The data of the category $\mathcal{F}(\hat{\mathscr{W}})$ of world states are
\begin{enumerate}
    \item \textbf{Objects:}
    The world states in $W$.
    \begin{equation}
        \text{Ob}(\mathcal{F}(\hat{\mathscr{W}})) := W
    \end{equation}
    \item \textbf{Morphisms:}
    The transformations in $D$.
    \begin{equation}
        \text{Hom}_{\mathcal{F}(\hat{\mathscr{W}})}(w, w') = \{ d \in D \mid s(d) = w \text{ and } t(d) = w' \}.
    \end{equation}
    The identity morphisms of $\mathcal{F}(\hat{\mathscr{W}})$ are the trivial transformations $1_{w}$, and composition of morphisms is concatenation of paths.
\end{enumerate}

%%%%%%%%%%%%%%%%%%%%%%%%%%%%%%%%%%%%%%%%%%%%%%%%
\paragraph{Subworlds.}

\newthought{The conditions for} a subworld in our framework are precisely the conditions that define a subcategory:
\begin{equation}
    \mathscr{W}' \subseteq \mathscr{W} \iff \mathcal{F}(\hat{\mathscr{W}}') \subseteq \mathcal{F}(\hat{\mathscr{W}})
\end{equation}

The reachable subworld $\mathscr{W}^{\to}(w)$ of a world $\mathscr{W}$ is the \emph{full subcategory} of $\mathscr{W}$ on the objects $W^{\to}(w)$.


%%%%%%%%%%%%%%%%%%%%%%%%%%%%%%%%%%%%%%%%%%%%%%%%
\section{
Categorification of the actions of an agent
}
%%%%%%%%%%%%%%%%%%%%%%%%%%%%%%%%%%%%%%%%%%%%%%%%
\subsection{Recap.}
We introduce an agent
\begin{equation}
    \mathscr{A} = (\hat{A}, \mathscr{Z}, b, h),
\end{equation}
where $\hat{A}$ is the set of atomic actions of the agent, $\mathscr{Z}$ is the agent's representation, $b$ is the agent's observation process, and $h$ is the agent's inference process.

The set $\hat{A}^{*}$ of actions of the agent is generated by taking the Kleene star operator of the set $\hat{A}$ of atomic actions.

Let $\hat{D}_{A} \subset D$ be the set of transformations that are caused by the atomic actions in $\hat{A}$; we call $\hat{D}_{A}$ the set of \emph{atomic action transformations}.
We define a labelling map
\begin{equation}
    \hat{l}: \hat{D}_{A} \to \hat{A}
\end{equation}
such that any two distinct atomic transformations leaving the same world state are labelled with different actions:
\begin{equation}
  \text{If } s(d) = s(d'), \text{ then } \hat{l}(d) \neq \hat{l}(d') \quad \text{for any } d, d' \in \hat{D}_{A}.
\end{equation}
We call $\hat{l}$ the \emph{atomic action labelling map}.
We extend $\hat{l}$ to an \emph{action labelling map} $l: D_{A} \to \hat{A}^{*}$ such that $l$ satisfies the following conditions:
\begin{enumerate}
    \item \textbf{Uniqueness.}
    For any $d,d' \in D_{A}$ with $s(d)=s(d')$, $l(d) \neq l(d')$.

    \item \textbf{Identity.}
    $l(1_{w}) = \varepsilon$ for all $w \in W$.

    \item \textbf{Composition consistency.}
    If $d = \hat{d}_{n} \circ ... \circ \hat{d}_{1}$ then $l(d) = \hat{l}(\hat{d}_{n}) ... \hat{l}(\hat{d}_{1})$ for all $d \in D_{A}$ and for all $\hat{d_{1}}, \dots , \hat{d_{n}} \in \hat{D}_{A}$.
\end{enumerate}

To construct the world with transformations due to the actions of the agent $\mathscr{A}$ we take the set $\hat{D}_{A} \subseteq D$ of transformations due to the atomic actions of the agent and use it to construct the atomic multidigraph of the actions of the agent
\begin{equation}
    \hat{\mathscr{W}}_{\mathscr{A}} = (W, \hat{D}_{A}, s, t).
\end{equation}
Then we construct the world with transformations due to the actions of the agent by taking all finite direct walks in $\hat{\mathscr{W}}_{\mathscr{A}}$ and constructing the multidigraph
\begin{equation}
    \mathscr{W}_{\mathscr{A}} = (W, D_{A}, s, t).
\end{equation}
where $\mathscr{W}_{\mathscr{A}} \subseteq \mathscr{W}$.


%%%%%%%%%%%%%%%%%%%%%%%%%%%%%%%%%%%%%%%%%%%%%%%%
\subsection{Categorical conversion.}
\newthought{We construct the} free category\footnote{
The free category $\mathcal{F}(Q)$ of a multidigraph is the most general category that can be made from a multidigraph by allowing composition of walks \draftnote{blue}{continue}{}
}
\begin{equation}
    \mathcal{F}(\hat{\mathscr{W}}_{\mathscr{A}}).
\end{equation}
which is is a subcategory of $\mathcal{F}(\hat{\mathscr{W}})$:
\begin{equation}
    \mathcal{F}(\hat{\mathscr{W}}_{\mathscr{A}}) \subseteq \mathcal{F}(\hat{\mathscr{W}}).
\end{equation}

%%%%%%%%%%%%%%%%%%%%%%%%%%%%%%%%%%%%%%%%%%%%%%%%
\paragraph{Categorising the monoid $\hat{A}^{*}$.}

\newthought{To "categorify" the} monoid $\hat{A}^{*}$, we construct the delooping of $\hat{A}^{*}$,
\begin{equation}
    \hat{A}^{*} \xrightarrow{\text{deloop}} \mathbf{B}\hat{A}^{*},
\end{equation}
which is the category $\mathbf{B}\hat{A}^{*}$ consisting of
\begin{enumerate}
    \item \textbf{Objects.}
    There is a single object
    \begin{equation}
        \text{Ob}(\mathbf{B}\hat{A}^{*}) = \{ \bullet \}
    \end{equation}
    \item \textbf{Morphism.}
    The hom-set $\text{Hom}(\bullet, \bullet)$ is given by the elements of $\hat{A}^{*}$
    \begin{equation}
        \text{Hom}(\bullet, \bullet) = \hat{A}^{*}
    \end{equation}
    \item \textbf{Composition.}
    The composition of morphisms is defined by the action composition operator $\circ$.
    \item \textbf{Identity.}
    The identity morphism is the empty action $\varepsilon$.
\end{enumerate}


\newthought{We can use} the atomic labelling map
\begin{equation}
    \hat{l}: \hat{D}_{A} \to \hat{A}
\end{equation}
to construct a multidigraph morphism $l_{Q}: \hat{\mathscr{W}}_{\mathscr{A}} \to U(\mathbf{B}\hat{A}^{*})$ defined by:
\begin{enumerate}
    \item \textbf{Objects:}
    For every $w \in W$,
    \begin{equation}
        l_{Q}(w) = \bullet;
    \end{equation}
    \item \textbf{Arrows:}
    For every $\hat{d} \in \hat{D}_{A}$,
    \begin{equation}
        l_{Q}(\hat{d}) = \hat{l}(\hat{d}).
    \end{equation}
\end{enumerate}
Due to the free-forgetful adjunction
\begin{equation}
    \mathcal{F}: \cat{Quiv} \leftrightarrows \cat{Cat}: U.
\end{equation}
formed by the free category functor $\mathcal{F}$ and the forgetful functor $U$, we have, from the universal property of free categories, that for the category $\mathbf{B}\hat{A}^{*} \in \cat{Cat}$ and multidigraph morphism $l_{Q}: \hat{\mathscr{W}}_{\mathscr{A}} \to U(\mathbf{B}\hat{A}^{*})$ in $\cat{Quiv}$, there exists a unique functor\footnote{
    The universal property of free categories also provides the natural bijection
    \begin{equation}
        \text{Hom}_{\cat{Quiv}}(\hat{\mathscr{W}}, U(\mathbf{B}\hat{A}^{*})) \cong \text{Hom}_{\cat{Cat}}(\mathcal{F}(\hat{\mathscr{W}}), \mathbf{B}\hat{A}^{*}),
    \end{equation}
    which means that
    \begin{enumerate}
        \item every functor $\mathcal{F}(\hat{\mathscr{W}}) \to \mathbf{B}\hat{A}^{*}$ in $\cat{Cat}$ corresponds to a unique multidigraph morphism $\hat{\mathscr{W}} \to U(\mathbf{B}\hat{A}^{*})$ (namely $G = U(G') \circ \eta_{\hat{\mathscr{W}}}$) in $\cat{Quiv}$; and
        \item every multidigraph morphism $\hat{\mathscr{W}} \to U(\mathbf{B}\hat{A}^{*})$ in $\cat{Quiv}$ corresponds to a unique functor $\mathcal{F}(\hat{\mathscr{W}}) \to \mathbf{B}\hat{A}^{*}$ in $\cat{Cat}$.
    \end{enumerate}
}
\begin{equation}
    L : \mathcal{F}(\hat{\mathscr{W}}_{\mathscr{A}}) \to \mathbf{B}\hat{A}^{*}
\end{equation}
such that\footnote{
    $\eta_{\hat{\mathscr{W}}_{\mathscr{A}}}$ is the $\hat{\mathscr{W}}_{\mathscr{A}}$-component of the unit natural transformation $\eta: \text{id}_{\cat{Quiv}} \Rightarrow U \circ \mathcal{F}$ of the adjunction $\mathcal{F}: \cat{Quiv} \leftrightarrows \cat{Cat}: U$.
}
\begin{equation}
    U(L) \circ \eta_{\hat{\mathscr{W}}_{\mathscr{A}}} = l_{Q};
\end{equation}
this universal property gives the following diagram:\footnote{
    $U(\mathcal{F}(\hat{\mathscr{W}}_{\mathscr{A}}))$ is exactly the multidigraph $\mathscr{W}_{\mathscr{A}}$ (without the composition $\circ$).
}
\begin{equation}
    % https://q.uiver.app/#q=WzAsNyxbMiwxLCJcXGhhdHtcXG1hdGhzY3J7V319X3tcXG1hdGhzY3J7QX19Il0sWzIsMiwiVShcXG1hdGhjYWx7Rn0oXFxoYXR7XFxtYXRoc2Nye1d9fV97XFxtYXRoc2Nye0F9fSkpIl0sWzQsMSwiXFxtYXRoY2Fse0Z9KFxcaGF0e1xcbWF0aHNjcntXfX1fe1xcbWF0aHNjcntBfX0pIl0sWzIsMCwiXFx0ZXh0e0luICRcXGNhdHtRdWl2fSQ6fSJdLFs0LDAsIlxcdGV4dHtJbiAkXFxjYXR7Q2F0fSR9OiJdLFswLDIsIlUoXFx0ZXh0YmZ7Qn1cXGhhdHtBfV57Kn0pIl0sWzQsMiwiXFx0ZXh0YmZ7Qn1cXGhhdHtBfV57Kn0iXSxbMCwxLCJcXGV0YV97XFxoYXR7XFxtYXRoc2Nye1d9fV97XFxtYXRoc2Nye0F9fX0iXSxbMCwyLCJGIiwwLHsiY3VydmUiOi0yLCJjb2xvdXIiOlswLDYwLDYwXX0sWzAsNjAsNjAsMV1dLFsyLDEsIlUiLDIseyJjb2xvdXIiOlsxMjAsNjAsNjBdfSxbMTIwLDYwLDYwLDFdXSxbMCw1LCJsX3tRfSIsMl0sWzYsNSwiVSIsMCx7Im9mZnNldCI6LTEsImN1cnZlIjotMiwiY29sb3VyIjpbMTIwLDYwLDYwXX0sWzEyMCw2MCw2MCwxXV0sWzEsNSwiXFxleGlzdHMhIFxcOyBVKEwpIiwyLHsibGFiZWxfcG9zaXRpb24iOjQwLCJzdHlsZSI6eyJib2R5Ijp7Im5hbWUiOiJkYXNoZWQifX19XSxbMiw2LCJcXGV4aXN0cyEgXFw7IEwiLDAseyJzdHlsZSI6eyJib2R5Ijp7Im5hbWUiOiJkYXNoZWQifX19XV0=
\begin{tikzcd}[ampersand replacement=\&]
    \&\& {\text{In $\cat{Quiv}$:}} \&\& {\text{In $\cat{Cat}$}:} \\
    \&\& {\hat{\mathscr{W}}_{\mathscr{A}}} \&\& {\mathcal{F}(\hat{\mathscr{W}}_{\mathscr{A}})} \\
    {U(\mathbf{B}\hat{A}^{*})} \&\& {U(\mathcal{F}(\hat{\mathscr{W}}_{\mathscr{A}}))} \&\& {\mathbf{B}\hat{A}^{*}}
    \arrow["F", color={rgb,255:red,214;green,92;blue,92}, curve={height=-12pt}, from=2-3, to=2-5]
    \arrow["{l_{Q}}"', from=2-3, to=3-1]
    \arrow["{\eta_{\hat{\mathscr{W}}_{\mathscr{A}}}}", from=2-3, to=3-3]
    \arrow["U"', color={rgb,255:red,92;green,214;blue,92}, from=2-5, to=3-3]
    \arrow["{\exists! \; L}", dashed, from=2-5, to=3-5]
    \arrow["{\exists! \; U(L)}"'{pos=0.4}, dashed, from=3-3, to=3-1]
    \arrow["U", shift left, color={rgb,255:red,92;green,214;blue,92}, curve={height=-12pt}, from=3-5, to=3-1]
\end{tikzcd}
\end{equation}

We call the unique functor $L$ that we have constructed the \emph{action labelling functor}, which is defined as:
\begin{enumerate}
    \item \textbf{Objects:}
    each object $w \in W$ of $\mathcal{F}(\hat{\mathscr{W}}_{\mathscr{A}})$ to $\bullet \in \mathbf{B}$ of $\mathbf{B}\hat{A}^{*}$
    \begin{equation}
        L(w) = \bullet \quad \text{for all $w \in W$};
    \end{equation}
    
    \item \textbf{Morphisms:}
    each morphism $d \in \text{Hom}(w,w')$ in $\mathcal{F}(\hat{\mathscr{W}}_{\mathscr{A}})$ to a morphism $L(d) \in \hat{A}^{*}$ in $\mathbf{B}\hat{A}^{*}$ such that\footnote{
        Note that the unique functor $L$ automatically satisfies the identity and composition consistency conditions of our action labelling map $l$ !
    }
    \begin{enumerate}
        \item \textbf{Identity:}
        \begin{equation}
            L(1_{w}) = \varepsilon \quad \text{for all $w \in W$};
        \end{equation}
        \item \textbf{Composition:}
        for $d = \hat{d}_{n} \circ \dots \circ \hat{d}_{1}$,
        \begin{equation}
            L(d) = \hat{l}(\hat{d}_{n}) \circ \dots \circ \hat{l}(\hat{d}_{1}).
        \end{equation}
    \end{enumerate}
\end{enumerate}

The uniqueness condition on the atomic labelling map $\hat{l}$ is inherited by $L$; in categorical terms, the action labelling functor $L$ is \emph{faithful}, which means that for any two objects $w, w' \in W$, the induced map on Hom-sets
\begin{equation}
    L_{w,w'} : \text{Hom}_{\mathcal{F}(\hat{\mathscr{W}}_{\mathscr{A}})}(w, w') \to \text{Hom}_{\mathbf{B}\hat{A}^{*}}(\bullet, \bullet) = \hat{A}^{*}
\end{equation}
is injective\footnote{
    A consequence of this is that no distinct sequences of atomic actions are "identified" (i.e., treated the same) under $L$.
}.
The faithfulness of $L$ means that $L$ embeds the structure of $\mathcal{F}(\hat{\mathscr{W}}_{\mathscr{A}})$ into $\mathbf{B}\hat{A}^{*}$, and so the transformations (morphisms) in $\mathcal{F}(\hat{\mathscr{W}}_{\mathscr{A}})$ inherit the algebraic structure of $\hat{A}^{*}$ through the labelling.

On the level of endomorphisms, $L$ gives a monoid homomorphism
\begin{equation}
    L_{w}: \text{End}(w) \to \hat{A}^{*}
\end{equation}
that embeds the action monoid at a state into the free monoid $\hat{A}^{*}$.

\begin{notation}
    In category theory diagrams, we use a $\hookrightarrow$ to denote an \emph{inclusion functor}, which is a functor that is faithful embedding.
    Therefore, we can write the action labelling functor $L$ as
    \begin{equation}
        L: \mathcal{F}(\hat{\mathscr{W}}_{\mathscr{A}}) \hookrightarrow \mathbf{B}\hat{A}^{*}.
    \end{equation}
\end{notation}


%%%%%%%%%%%%%%%%%%%%%%%%%%%%%%%%%%%%%%%%%%%%%%%%
\section{
Categorification of totalisation and the effect of actions
}
\draftnote{purple}{To do}{
\begin{enumerate}
    \item Prove in original framework that $\ast^{\bot}$ satisfies
    \begin{equation}
        (a' \circ a) \ast^{\bot} w = a' \ast^{\bot} (a \ast^{\bot} w)
    \end{equation}
    for all $a, a' \in \hat{A}^{*}$ and for all $w \in W$
    \item Need that $\Delta^{\bot}$ is closed under composition with $\delta_{a}$.
    So, explicitly define composition rules for composing $\delta_{a}$ and $d^{\bot}$.
\end{enumerate}
}

%%%%%%%%%%%%%%%%%%%%%%%%%%%%%%%%%%%%%%%%%%%%%%%%
\subsection{Recap.}
\newthought{We have an} action effect operator
\begin{equation}
    \ast: \hat{A}^{*} \times W \to W,
\end{equation}
such that
\begin{enumerate}
	\item if there exists a $d: w \xrightarrow{a} t(d)$, then $a \ast w = t(d)$; and
	\item if there does not exist a $d: w \xrightarrow{a} t(d)$, then we say that $a \ast w$ is \emph{undefined}.
\end{enumerate}
that encodes how actions in $\hat{A}^{*}$ acts on states in $W$.
Since $\hat{A}^{*}$ is a monoid, $\ast$ is a partial monoid action on the set $W$ of world states.

\newthought{We can \emph{curry}}\footnote{
    Currying is an operation that allows us to transform an operation with multiple arguments into a collection of functions with single arguments by fixing all but one of the operation's arguments.
} the action effect operator $\ast : \hat{A}^{*} \times W \to W$
\begin{equation}
	\operatorname{Curry}: (\ast: \hat{A}^{*} \times W \to W) \to (f_{a}: W \to W)
\end{equation}
to obtain a collection of (partial) functions
\begin{equation}
	\mathcal{T}_{\hat{A}^{*}} = \{f_{a}: W \to W \mid f_{a}(w) = a \ast w \text{ where defined, for each } a \in \hat{A}^{*} \}.
\end{equation}

\newthought{To handle undefined actions} we totalised our action effect operator $\ast$ by 
\begin{enumerate}
    \item augmenting our set $W$ with an undefined state $\bot$ to give
    \begin{equation}
        W^{\bot} = W \cup \bot;
    \end{equation}
    \item augmenting the set $D_{A}$ of transformations due to the actions of the agent $\mathscr{A}$ to a set $D_{A}^{\bot}$, which includes transformations for the behaviour of $\ast^{\bot}$ where previously $\ast$ was undefined:
    \begin{alignat}{2}
        & U {}={} && \{ (a,w) \in \hat{A}^{*} \times W \mid \not\exists (d: w \xrightarrow{a} t(d)) \in D_{A} \} \\
        & \Delta^{\bot} {}={} && \{ d^{\bot}: w \xrightarrow{a} \bot \mid (a, w) \in U \}                                       \\
                      &       && \cup \{ d^{\bot}: \bot \xrightarrow{a} \bot \text{ for all $a \in \hat{A}^{*}$} \} \\
        & D_{A}^{\bot} = && D \cup \Delta^{\bot};
    \end{alignat}
    \item extending our action labelling map $l: D_{A} \to \hat{A}^{*}$ label our new $\bot$-terminating transformations in $D_{A}^{\bot} $:
    \begin{align}
         & l^{\bot} : D_{A}^{\bot} \to \hat{A}^{*} \quad\text{such that} \\
         & l^{\bot}(d) =
        \begin{cases}
            l(d) & \text{if $d \in D_{A}$}, \\
            a    & \text{if $d \in \Delta^{\bot}$ and $d: s(d) \xrightarrow{a} \bot$.}
        \end{cases}
    \end{align}
\end{enumerate}

These augmentations allow us to extend the action effect operator $\ast$ to a total operator
\begin{equation}
    \ast^{\bot}: \hat{A}^{*} \times W^{\bot} \to W^{\bot}.
\end{equation}
These extensions of $W$, $D_{A}$, and $l$ to $W^{\bot}$, $D_{A}^{\bot}$, and $l^{\bot}$ mean $\ast^{\bot}$ is a total monoid action on $W^{\bot}$.

%%%%%%%%%%%%%%%%%%%%%%%%%%%%%%%%%%%%%%%%%%%%%%%%
\subsection{Categorical conversion.}
%%%%%%%%%%%%%%%%%%%%%%%%%%%%%%%%%%%%%%%%%%%%%%%%%%%
\paragraph{Totalisation of the category of world states.}
\newthought{To handle totalisation}, we need to adjoin an undefined object $\bot$ and to adjoin morphisms that send an object $w \in W$ in $\mathcal{F}(\hat{\mathscr{W}}_{\mathscr{A}})$ to the undefined $\bot$ whenever $a \ast w$ is undefined; we will do this by constructing a pushout in the category $\cat{Cat}$ of categories:

To handle undefined actions, we define a discrete category $\mathcal{U}$ with objects that represent undefined action-state pairs $(a, w)$:
\begin{equation}
    \text{Ob}(\mathcal{U}) = \{(a,w) \in \hat{A}^{*} \times W \mid \centernot\exists d:w \xrightarrow{a} t(d) \in D_{A}\}.
\end{equation}

To handle $\bot$ and the relevant transformations to $\bot$, we define a star-shaped category $\mathcal{S}$ with:
\begin{enumerate}
    \item \textbf{Objects:}
    A central proto-terminal object $\bot$ and, for each $(a,w) \in \mathcal{U}$, an object $x_{(a,w)}$:
    \begin{equation}
        \text{Ob}(\mathcal{S}) = \{\bot\} \cup \{x_{(a,w)} \mid (a,w) \in \text{Ob}(\mathcal{U}).
    \end{equation}
    \item \textbf{Morphisms:}
    For each $(a,w) \in \text{Ob}(\mathcal{U})$, a morphism $d^{\bot}_{(a,w)}: x_{(a,w)} \to \bot$.
    For each $a \in \hat{A}^{*}$, a morphism $\delta_{a}: \bot \to \bot$.
    For each object $x_{(a,w)}$, a morphism $1_{x_{(a,w)}}$.
    A morphism $1_{\bot}$.
\end{enumerate}

We define two functors:
\begin{enumerate}
    \item a functor $F_{1}: \mathcal{U} \to \mathcal{F}(\hat{\mathscr{W}}_{\mathscr{A}})$ that maps each $(a,w) \in \mathcal{U}$ to the object $w \in \mathcal{F}(\hat{\mathscr{W}}_{\mathscr{A}})$, and maps each identity morphism $1_{(a,w)} \in \mathcal{U}$ to the identity morphism $1_{w} \in \mathcal{F}(\hat{\mathscr{W}}_{\mathscr{A}})$;
    \item a functor $F_{2}: \mathcal{U} \to \mathcal{S}$ that maps each $(a,w) \in \mathcal{U}$ to the object $x_{(a,w)} \in \mathcal{S}$, and maps each identity morphism $1_{(a,w)} \in \mathcal{U}$ to the identity morphism $1_{x_{(a,w)}} \in \mathcal{S}$.
\end{enumerate}

The pushout $(\mathcal{F}(\hat{\mathscr{W}}_{\mathscr{A}}^{\bot}), i_{1}, i_{2})$ of $\mathcal{F}(\hat{\mathscr{W}}_{\mathscr{A}})$ and $\mathcal{S}$ over $\mathcal{U}$ completes the diagram:\footnote{
    This pushout glues each $x_{(a,w)} \in \mathcal{S}$ to $w \in \mathcal{F}(\hat{\mathscr{W}}_{\mathscr{A}})$ for each $(a,w) \in \mathcal{U}$.
    This collapses each $x_{(a,w)}$ into $w$, turning $d^{\bot}_{(a,w)}: x_{(a,w)} \to \bot \in \mathcal{S}$ into $d^{\bot}_{(a,w)}: w \to \bot \in \mathcal{F}(\hat{\mathscr{W}}_{\mathscr{A}}^{\bot})$.
}
\begin{equation}
    % https://q.uiver.app/#q=WzAsOSxbMCwwLCJcXG1hdGhjYWx7VX0iXSxbMiwwLCJcXG1hdGhjYWx7Rn0oXFxoYXR7XFxtYXRoc2Nye1d9fV97XFxtYXRoc2Nye0F9fSkiXSxbMCwyLCJcXG1hdGhjYWx7U30iXSxbMywxXSxbNCwxXSxbNSwwLCJcXG1hdGhjYWx7VX0iXSxbNywwLCJcXG1hdGhjYWx7Rn0oXFxoYXR7XFxtYXRoc2Nye1d9fV97XFxtYXRoc2Nye0F9fSkiXSxbNSwyLCJcXG1hdGhjYWx7U30iXSxbNywyLCJcXG1hdGhjYWx7Rn0oXFxoYXR7XFxtYXRoc2Nye1d9fV97XFxtYXRoc2Nye0F9fV57XFxib3R9KSJdLFswLDEsIkZfezF9Il0sWzAsMiwiRl97Mn0iLDJdLFszLDQsIlxcdGV4dHtwdXNob3V0fSJdLFs1LDYsIkZfezF9Il0sWzUsNywiRl97Mn0iLDJdLFs3LDgsImlfezJ9IiwyXSxbNiw4LCJpX3sxfSJdXQ==
\begin{tikzcd}[ampersand replacement=\&]
    {\mathcal{U}} \&\& {\mathcal{F}(\hat{\mathscr{W}}_{\mathscr{A}})} \&\&\& {\mathcal{U}} \&\& {\mathcal{F}(\hat{\mathscr{W}}_{\mathscr{A}})} \\
    \&\&\& {} \& {} \\
    {\mathcal{S}} \&\&\&\&\& {\mathcal{S}} \&\& {\mathcal{F}(\hat{\mathscr{W}}_{\mathscr{A}}^{\bot})}
    \arrow["{F_{1}}", from=1-1, to=1-3]
    \arrow["{F_{2}}"', from=1-1, to=3-1]
    \arrow["{F_{1}}", from=1-6, to=1-8]
    \arrow["{F_{2}}"', from=1-6, to=3-6]
    \arrow["{i_{1}}", from=1-8, to=3-8]
    \arrow["{\text{pushout}}", from=2-4, to=2-5]
    \arrow["{i_{2}}"', from=3-6, to=3-8]
\end{tikzcd}
\end{equation}
The category $\mathcal{F}(\hat{\mathscr{W}}_{\mathscr{A}}^{\bot})$ is universal, so any category with maps from $\mathcal{F}(\hat{\mathscr{W}}_{\mathscr{A}})$ and $\mathcal{S}$ that agree on $\mathcal{U}$ factors uniquely through $\mathcal{F}(\hat{\mathscr{W}}_{\mathscr{A}}^{\bot})$.
The resulting structure $\mathcal{F}(\hat{\mathscr{W}}_{\mathscr{A}}^{\bot})$ consists of:
\begin{enumerate}
    \item \textbf{Objects:}
    $W \cup \{\bot\}$.
    \item \textbf{Morphisms:}
    The original morphisms in $\mathcal{F}(\hat{\mathscr{W}}_{\mathscr{A}})$, the new "glued" morphisms $d^{\bot}_{(a,w)}: w \to \bot$ for each $(a,w) \in \mathcal{U}$, and the morphisms $\delta_{a}: \bot \to \bot$ for each $a \in \hat{A}^{*}$.
\end{enumerate}

%%%%%%%%%%%%%%%%%%%%%%%%%%%%%%%%%%%%%%%%%%%%%%%%%%%
\paragraph{Totalisation of the labelling functor.}
\newthought{We also need} to construct a totalised labelling functor $L^{\bot}: \mathcal{F}(\hat{\mathscr{W}}_{\mathscr{A}}^{\bot}) \to \mathbf{B}\hat{A}^{*}$ from the labelling functor $L: \mathcal{F}(\hat{\mathscr{W}}_{\mathscr{A}}) \to \mathbf{B}\hat{A}^{*}$.

First we define a labelling functor $L_{\mathcal{S}}: \mathcal{S} \to \mathbf{B}\hat{A}^{*}$ on the star-shaped category $\mathcal{S}$ that acts as follows:
\begin{enumerate}
    \item \textbf{On objects:}
    All the objects in $\mathcal{S}$ are sent to the single object $\bullet \in \mathbf{B}\hat{A}^{*}$:
    \begin{align}
        & L_{\mathcal{S}}(x_{(a,w)}) = \bullet \\
        & L_{\mathcal{S}}(\bot) = \bullet;
    \end{align}

    \item \textbf{On morphisms:}
    For each $d^{\bot}_{(a,w)}: x_{(a,w)} \to \bot$, assign
    \begin{equation}
        L_{\mathcal{S}}(d^{\bot}_{(a,w)}) = a;
    \end{equation}
    For each $\delta_{a}: \bot \to \bot$, assign
    \begin{equation}
        L_{\mathcal{S}}(\delta_{a}) = a
    \end{equation}
    For the identities, assign
    \begin{align}
        L_{\mathcal{S}}(1_{x_{(a,w)}}) = \varepsilon \\
        L_{\mathcal{S}}(1_{\bot}) = \varepsilon.
    \end{align}
\end{enumerate}

So we currently have the diagram:
\begin{equation}
    % https://q.uiver.app/#q=WzAsNSxbMCwwLCJcXG1hdGhjYWx7VX0iXSxbMiwwLCJcXG1hdGhjYWx7Rn0oXFxoYXR7XFxtYXRoc2Nye1d9fV97XFxtYXRoc2Nye0F9fSkiXSxbMCwyLCJcXG1hdGhjYWx7U30iXSxbMiwyLCJcXG1hdGhjYWx7Rn0oXFxoYXR7XFxtYXRoc2Nye1d9fV97XFxtYXRoc2Nye0F9fV57XFxib3R9KSJdLFszLDMsIlxcdGV4dGJme0J9XFxoYXR7QX1eeyp9Il0sWzAsMSwiRl97MX0iXSxbMCwyLCJGX3syfSIsMl0sWzIsMywiaV97Mn0iLDJdLFsxLDMsImlfezF9Il0sWzEsNCwiTCIsMCx7ImN1cnZlIjotMn1dLFsyLDQsIkxfe1xcbWF0aGNhbHtTfX0iLDIseyJjdXJ2ZSI6Mn1dXQ==
\begin{tikzcd}[ampersand replacement=\&]
    {\mathcal{U}} \&\& {\mathcal{F}(\hat{\mathscr{W}}_{\mathscr{A}})} \\
    \\
    {\mathcal{S}} \&\& {\mathcal{F}(\hat{\mathscr{W}}_{\mathscr{A}}^{\bot})} \\
    \&\&\& {\mathbf{B}\hat{A}^{*}}
    \arrow["{F_{1}}", from=1-1, to=1-3]
    \arrow["{F_{2}}"', from=1-1, to=3-1]
    \arrow["{i_{1}}", from=1-3, to=3-3]
    \arrow["L", curve={height=-12pt}, from=1-3, to=4-4]
    \arrow["{i_{2}}"', from=3-1, to=3-3]
    \arrow["{L_{\mathcal{S}}}"', curve={height=12pt}, from=3-1, to=4-4]
\end{tikzcd}
\end{equation}
So if we prove that the functors $L$ and $L_{\mathcal{S}}$ agree on the category $\mathcal{U}$, by proving the consistency condition,
\begin{equation}
    L \circ F_{1} = L_{\mathcal{S}} \circ F_{2},
\end{equation}
then we can construct $L^{\bot}$ as the unique functor from the universal property of the pushout.

\begin{propositionE}
\begin{equation}
    L \circ F_{1} = L_{\mathcal{S}} \circ F_{2}
\end{equation}
\end{propositionE}
\begin{proofE}
\begin{enumerate}
    \item \textbf{For objects:}
    For an arbitrary $(a,w) \in \mathcal{U}$, we have
    \begin{equation}
        L(F_{1}(a,w)) = L(w) = \bullet
    \end{equation}
    and
    \begin{equation}
        L_{\mathcal{S}}(F_{2}(a,w)) = L_{\mathcal{S}}(x_{(a,w)}) = \bullet.
    \end{equation}
    \item \textbf{For morphisms:}
    $\mathcal{U}$ is a discrete category, therefore there are only identities in $\mathcal{U}$.
    For an arbitrary $(a,w) \in \mathcal{U}$, we have
    \begin{equation}
        L(F_{1}(1_{(a,w)}) = L(1_{w}) = \varepsilon
    \end{equation}
    and
    \begin{equation}
        L_{\mathcal{S}}(F_{2}(a,w)) = L_{\mathcal{S}}(1_{x_{(a,w)}}) = \varepsilon.
    \end{equation}
    \item \textbf{Conclusion.}
    Since $L \circ F_{1}$ and $L_{\mathcal{S}} \circ F_{2}$ agree for all objects and morphisms in $\mathcal{U}$, it follows that
    \begin{equation}
            L \circ F_{1} = L_{\mathcal{S}} \circ F_{2}.
    \end{equation}
\end{enumerate}
\end{proofE}

Therefore, by the universal property of the pushout, there exists a unique functor
\begin{equation}
    L^{\bot}: \mathcal{F}(\hat{\mathscr{W}}_{\mathscr{A}}^{\bot}) \to \mathbf{B}\hat{A}^{*}
\end{equation}
such that
\begin{align}
    & L^{\bot} \circ i_{1} = L \\
    & L^{\bot} \circ i_{2} = L_{\mathcal{S}};
\end{align}
as a diagram, this looks like:
\begin{equation}
    % https://q.uiver.app/#q=WzAsNSxbMCwwLCJcXG1hdGhjYWx7VX0iXSxbMiwwLCJcXG1hdGhjYWx7Rn0oXFxoYXR7XFxtYXRoc2Nye1d9fV97XFxtYXRoc2Nye0F9fSkiXSxbMCwyLCJcXG1hdGhjYWx7U30iXSxbMiwyLCJcXG1hdGhjYWx7Rn0oXFxoYXR7XFxtYXRoc2Nye1d9fV97XFxtYXRoc2Nye0F9fV57XFxib3R9KSJdLFszLDMsIlxcdGV4dGJme0J9XFxoYXR7QX1eeyp9Il0sWzAsMSwiRl97MX0iXSxbMCwyLCJGX3syfSIsMl0sWzIsMywiaV97Mn0iLDJdLFsxLDMsImlfezF9Il0sWzEsNCwiTCIsMCx7ImN1cnZlIjotMn1dLFsyLDQsIkxfe1xcbWF0aGNhbHtTfX0iLDIseyJjdXJ2ZSI6Mn1dLFszLDQsIkxee1xcYm90fSIsMCx7InN0eWxlIjp7ImJvZHkiOnsibmFtZSI6ImRhc2hlZCJ9fX1dXQ==
\begin{tikzcd}[ampersand replacement=\&]
    {\mathcal{U}} \&\& {\mathcal{F}(\hat{\mathscr{W}}_{\mathscr{A}})} \\
    \\
    {\mathcal{S}} \&\& {\mathcal{F}(\hat{\mathscr{W}}_{\mathscr{A}}^{\bot})} \\
    \&\&\& {\mathbf{B}\hat{A}^{*}}
    \arrow["{F_{1}}", from=1-1, to=1-3]
    \arrow["{F_{2}}"', from=1-1, to=3-1]
    \arrow["{i_{1}}", from=1-3, to=3-3]
    \arrow["L", curve={height=-12pt}, from=1-3, to=4-4]
    \arrow["{i_{2}}"', from=3-1, to=3-3]
    \arrow["{L_{\mathcal{S}}}"', curve={height=12pt}, from=3-1, to=4-4]
    \arrow["{L^{\bot}}", dashed, from=3-3, to=4-4]
\end{tikzcd}
\end{equation}

Explicitly, the augmented action labelling functor $L^{\bot}: \mathcal{F}(\hat{\mathscr{W}}_{\mathscr{A}}^{\bot}) \to \mathbf{B}\hat{A}^{*}$ acts as follows:
\begin{enumerate}
    \item \textbf{On objects:}
    \begin{align}
        & L^{\bot}(w) = \bullet \quad \text{for all $w \in W$} \\
        & L^{\bot}(\bot) = \bullet
    \end{align}
    \item \textbf{On morphisms:}
    For the morphisms $d: w \to w'$ in $\mathcal{F}(\hat{\mathscr{W}}_{\mathscr{A}})$:
    \begin{equation}
        L^{\bot}(d) = L(d);
    \end{equation}
    For the new morphisms $d^{\bot}_{(a,w)}: w \to \bot$:
    \begin{equation}
        L^{\bot}(d^{\bot}_{(a,w)}) = a;
    \end{equation}
    For the morphisms $\delta_{a}: \bot \to \bot$ in $\mathcal{S}$:
    \begin{equation}
        L^{\bot}(\delta_{a}) = a.
    \end{equation}
\end{enumerate}
$L^{\bot}$ inherits faithfulness by construction because $L$ and $L_{\mathcal{S}}$ are both faithful and the pushout glues morphisms consistently.


\newthought{Since the free category} functor $\mathcal{F}$ is left adjoint to the forgetful functor $U$ (i.e., $\mathcal{F} \dashv U$), any construction that is free (e.g., functor $\mathcal{F}(Q) \to \mathcal{C}$) or is built as a colimit (colimits are preserved by left adjoints) is uniquely determined by its restriction to the underlying quivers $Q$ and $U(C)$ via $U$.
Since we only used free categories, functors from free categories, and colimits in our construction of $L^{\bot}$, we could have constructed the pushout using quivers in $\cat{Quiv}$ and then used $\mathcal{F}$ and the universal property of free categories to lift these into $\cat{Cat}$ to obtain the totalisation category constructions we constructed directly in $\cat{Cat}$. \draftnote{purple}{PS}{
    After changing construction to use quivers then lift via the adjunction, make this more general and put it in the preliminaries
}
The following is a diagram containing these constructions:
\begin{fullwidth}
\begin{equation}
    % https://q.uiver.app/#q=WzAsMTUsWzEsNCwiXFxoYXR7XFxtYXRoc2Nye1d9fV97XFxtYXRoc2Nye0F9fV57XFxib3R9Il0sWzEsMSwiVShcXG1hdGhjYWx7Rn0oXFxoYXR7XFxtYXRoc2Nye1d9fV97XFxtYXRoc2Nye0F9fV57XFxib3R9KSkiXSxbNSw0LCJcXG1hdGhjYWx7Rn0oXFxoYXR7XFxtYXRoc2Nye1d9fV97XFxtYXRoc2Nye0F9fV57XFxib3R9KSJdLFsyLDAsIlxcdGV4dHtJbiAkXFxjYXR7UXVpdn0kOn0iXSxbMywwLCJcXHRleHR7SW4gJFxcY2F0e0NhdH0kfToiXSxbMiw0LCJVKFxcdGV4dGJme0J9XFxoYXR7QX1eeyp9KSJdLFszLDQsIlxcdGV4dGJme0J9XFxoYXR7QX1eeyp9Il0sWzAsMiwiXFxoYXR7XFxtYXRoc2Nye1d9fV97XFxtYXRoc2Nye0F9fSJdLFswLDYsIlMiXSxbMCw0LCJVIl0sWzYsNiwiXFxtYXRoY2Fse1N9ID0gXFxtYXRoY2Fse0Z9KFMpIl0sWzYsMiwiXFxtYXRoY2Fse0Z9KFxcaGF0e1xcbWF0aHNjcntXfX1fe1xcbWF0aHNjcntBfX0pIl0sWzYsNCwiXFxtYXRoY2Fse1V9ID0gXFxtYXRoY2Fse0Z9KFUpIl0sWzIsMSwiVShcXG1hdGhjYWx7Rn0oXFxoYXR7XFxtYXRoc2Nye1d9fV97XFxtYXRoc2Nye0F9fSkpIl0sWzIsNiwiVShcXG1hdGhjYWx7Rn0oUykpIl0sWzAsMSwiXFxldGFfe1xcaGF0e1xcbWF0aHNjcntXfX1fe1xcbWF0aHNjcntBfX1ee1xcYm90fX0iLDFdLFsyLDEsIlUiLDEseyJvZmZzZXQiOjEsImNvbG91ciI6WzEyMCw2MCw2MF19LFsxMjAsNjAsNjAsMV1dLFswLDUsImxfe1F9XntcXGJvdH0iLDFdLFs2LDUsIlUiLDEseyJjb2xvdXIiOlsxMjAsNjAsNjBdfSxbMTIwLDYwLDYwLDFdXSxbMSw1LCJcXGV4aXN0cyEgXFw7IFUoTF57XFxib3R9KSIsMSx7InN0eWxlIjp7ImJvZHkiOnsibmFtZSI6ImRhc2hlZCJ9fX1dLFs3LDBdLFs4LDBdLFs5LDddLFswLDksIiIsMSx7InN0eWxlIjp7Im5hbWUiOiJjb3JuZXIifX1dLFsxMCwyLCJpX3syfSIsMV0sWzExLDIsImlfezF9IiwxXSxbMTIsMTEsIkZfezF9IiwxXSxbMTIsMTAsIkZfezJ9IiwxXSxbMTAsNiwiXFxleGlzdHMgISBcXDsgTF97XFxtYXRoY2Fse1N9fSIsMSx7ImN1cnZlIjotMiwic3R5bGUiOnsiYm9keSI6eyJuYW1lIjoiZGFzaGVkIn19fV0sWzExLDYsIlxcZXhpc3RzICEgXFw7TCIsMSx7ImN1cnZlIjoyLCJzdHlsZSI6eyJib2R5Ijp7Im5hbWUiOiJkYXNoZWQifX19XSxbNywxMSwiXFxtYXRoY2Fse0Z9IiwxLHsiY3VydmUiOi0xLCJjb2xvdXIiOlswLDYwLDYwXX0sWzAsNjAsNjAsMV1dLFs5LDEyLCJcXG1hdGhjYWx7Rn0iLDEseyJjdXJ2ZSI6NCwiY29sb3VyIjpbMCw2MCw2MF19LFswLDYwLDYwLDFdXSxbOCwxMCwiXFxtYXRoY2Fse0Z9IiwxLHsiY3VydmUiOjMsImNvbG91ciI6WzAsNjAsNjBdfSxbMCw2MCw2MCwxXV0sWzcsNSwibF97UX0iLDEseyJsYWJlbF9wb3NpdGlvbiI6NjB9XSxbMiwxMiwiIiwxLHsic3R5bGUiOnsibmFtZSI6ImNvcm5lciJ9fV0sWzcsMTMsIlxcZXRhX3tcXGhhdHtcXG1hdGhzY3J7V319X3tcXG1hdGhzY3J7QX19fSIsMSx7ImxhYmVsX3Bvc2l0aW9uIjozMH1dLFsxMSwxMywiVSIsMSx7ImNvbG91ciI6WzEyMCw2MCw2MF19LFsxMjAsNjAsNjAsMV1dLFsxMyw1LCJcXGV4aXN0cyAhIFxcOyBVKEwpIiwxLHsic3R5bGUiOnsiYm9keSI6eyJuYW1lIjoiZGFzaGVkIn19fV0sWzE0LDUsIlxcZXhpc3RzICEgXFw7IFUoTF97XFxtYXRoY2Fse1N9fSkiLDEseyJzdHlsZSI6eyJib2R5Ijp7Im5hbWUiOiJkYXNoZWQifX19XSxbOCwxNCwiXFxldGFfe1N9IiwxXSxbMTAsMTQsIlUiLDEseyJjb2xvdXIiOlsxMjAsNjAsNjBdfSxbMTIwLDYwLDYwLDFdXSxbOSw4XSxbMCwyLCJcXG1hdGhjYWx7Rn0iLDEseyJvZmZzZXQiOjEsImN1cnZlIjotNCwiY29sb3VyIjpbMCw2MCw2MF19LFswLDYwLDYwLDFdXSxbMiw2LCJcXGV4aXN0cyEgXFw7IExee1xcYm90fSIsMSx7InN0eWxlIjp7ImJvZHkiOnsibmFtZSI6ImRhc2hlZCJ9fX1dLFs4LDUsImxfe1EsU30iLDFdXQ==
\begin{tikzcd}[ampersand replacement=\&]
    \&\& {\text{In $\cat{Quiv}$:}} \& {\text{In $\cat{Cat}$}:} \\
    \& {U(\mathcal{F}(\hat{\mathscr{W}}_{\mathscr{A}}^{\bot}))} \& {U(\mathcal{F}(\hat{\mathscr{W}}_{\mathscr{A}}))} \\
    {\hat{\mathscr{W}}_{\mathscr{A}}} \&\&\&\&\&\& {\mathcal{F}(\hat{\mathscr{W}}_{\mathscr{A}})} \\
    \\
    U \& {\hat{\mathscr{W}}_{\mathscr{A}}^{\bot}} \& {U(\mathbf{B}\hat{A}^{*})} \& {\mathbf{B}\hat{A}^{*}} \&\& {\mathcal{F}(\hat{\mathscr{W}}_{\mathscr{A}}^{\bot})} \& {\mathcal{U} = \mathcal{F}(U)} \\
    \\
    S \&\& {U(\mathcal{F}(S))} \&\&\&\& {\mathcal{S} = \mathcal{F}(S)}
    \arrow["{\exists! \; U(L^{\bot})}"{description}, dashed, from=2-2, to=5-3]
    \arrow["{\exists ! \; U(L)}"{description}, dashed, from=2-3, to=5-3]
    \arrow["{\eta_{\hat{\mathscr{W}}_{\mathscr{A}}}}"{description, pos=0.3}, from=3-1, to=2-3]
    \arrow["{\mathcal{F}}"{description}, color={rgb,255:red,214;green,92;blue,92}, curve={height=-6pt}, from=3-1, to=3-7]
    \arrow[from=3-1, to=5-2]
    \arrow["{l_{Q}}"{description, pos=0.6}, from=3-1, to=5-3]
    \arrow["U"{description}, color={rgb,255:red,92;green,214;blue,92}, from=3-7, to=2-3]
    \arrow["{\exists ! \;L}"{description}, curve={height=12pt}, dashed, from=3-7, to=5-4]
    \arrow["{i_{1}}"{description}, from=3-7, to=5-6]
    \arrow[from=5-1, to=3-1]
    \arrow["{\mathcal{F}}"{description}, color={rgb,255:red,214;green,92;blue,92}, curve={height=24pt}, from=5-1, to=5-7]
    \arrow[from=5-1, to=7-1]
    \arrow["{\eta_{\hat{\mathscr{W}}_{\mathscr{A}}^{\bot}}}"{description}, from=5-2, to=2-2]
    \arrow["\lrcorner"{anchor=center, pos=0.125, rotate=-135}, draw=none, from=5-2, to=5-1]
    \arrow["{l_{Q}^{\bot}}"{description}, from=5-2, to=5-3]
    \arrow["{\mathcal{F}}"{description}, shift right, color={rgb,255:red,214;green,92;blue,92}, curve={height=-24pt}, from=5-2, to=5-6]
    \arrow["U"{description}, color={rgb,255:red,92;green,214;blue,92}, from=5-4, to=5-3]
    \arrow["U"{description}, shift right, color={rgb,255:red,92;green,214;blue,92}, from=5-6, to=2-2]
    \arrow["{\exists! \; L^{\bot}}"{description}, dashed, from=5-6, to=5-4]
    \arrow["\lrcorner"{anchor=center, pos=0.125, rotate=45}, draw=none, from=5-6, to=5-7]
    \arrow["{F_{1}}"{description}, from=5-7, to=3-7]
    \arrow["{F_{2}}"{description}, from=5-7, to=7-7]
    \arrow[from=7-1, to=5-2]
    \arrow["{l_{Q,S}}"{description}, from=7-1, to=5-3]
    \arrow["{\eta_{S}}"{description}, from=7-1, to=7-3]
    \arrow["{\mathcal{F}}"{description}, color={rgb,255:red,214;green,92;blue,92}, curve={height=18pt}, from=7-1, to=7-7]
    \arrow["{\exists ! \; U(L_{\mathcal{S}})}"{description}, dashed, from=7-3, to=5-3]
    \arrow["{\exists ! \; L_{\mathcal{S}}}"{description}, curve={height=-12pt}, dashed, from=7-7, to=5-4]
    \arrow["{i_{2}}"{description}, from=7-7, to=5-6]
    \arrow["U"{description}, color={rgb,255:red,92;green,214;blue,92}, from=7-7, to=7-3]
\end{tikzcd}
\end{equation}
\end{fullwidth}

%%%%%%%%%%%%%%%%%%%%%%%%%%%%%%%%%%%%%%%%%%%%%%%%%
\paragraph{Categorising the totalised action effect operator.}
\newthought{We now want} to categorify our totalised action effect operator $\ast^{\bot}: \hat{A}^{*} \times W^{\bot} \to W^{\bot}$ by currying it into a functor
\begin{equation}
    A^{\bot}: \mathbf{B}\hat{A}^{*} \to \cat{Set}_{\bot},
\end{equation}
where $\cat{Set}_{\bot}$ is the category of pointed sets and basepoint-preserving maps with the undefined state $\bot$ as the basepoint\footnote{
    In the category $\cat{Set}_{\bot}$ of pointed sets and basepoint-preserving maps:
    \begin{enumerate}
        \item \textbf{Objects:}
        Objects are pairs 
        \begin{equation}
            (X, \bot_{X}),
        \end{equation}
        where $X$ is a set and $\bot_{X} \in X$ is a distinguished "undefined" basepoint in $X$.
        \item \textbf{Morphisms:}
        Morphisms are basepoint-preserving functions
        \begin{equation}
            f: (X, \bot_{X}) \to (Y, \bot_{Y})
        \end{equation}
        satisfying
        \begin{equation}
            f(\bot_{X}) = \bot_{Y}.
        \end{equation}
    \end{enumerate}
    This is useful in our case because we always want morphisms to map the undefined state to another undefined state rather than a different element in the relevant set.
}.

Unfortunately, the category $\cat{Set}_{\bot}$ of pointed sets is not a Cartesian closed category\footnote{
    Specifically, the Cartesian product is not compatible with basepoints since exponential object do not have basepoints and so exponential objects of pointed sets are not objects in $\cat{Set}_{\bot}$.
    However, exponential objects of pointed sets are objects in $\cat{Set}$.
} because, for an object $X$ in $\cat{Set}_{\bot}$, the exponential object $X^{X}$ need not exist in $\cat{Set}_{\bot}$, so currying does not work internally in $\cat{Set}_{\bot}$.
However, $\cat{Set}$ is a Cartesian closed category, so we can curry in $\cat{Set}$; we also have an adjunction
\begin{equation}
    F: \cat{Set} \leftrightarrows \cat{Set}_{\bot}: U
\end{equation}
between $\cat{Set}$ and $\cat{Set}_{\bot}$, where $F$ is left adjoint to $U$:
\begin{equation}
    F \dashv U.
\end{equation}
This adjunction $F \dashv U$ consists of
\begin{enumerate}
    \item a free functor
    \begin{equation}
        F: \cat{Set} \to \cat{Set}_{\bot}
    \end{equation}
    which freely adds a basepoint $\bot$ to a set:
    \begin{equation}
        F(X) = (X \sqcup \{\bot\}, \bot)
    \end{equation}
    where $(X \sqcup \{\bot\}, \bot)$ is the pointed set obtained by adjoining a new, disjoint basepoint $\bot$ to $X$\footnote{
        Note that $F(X)$ has one more element that $X$ (the adjoined disjoint basepoint $\bot$), therefore
        \begin{equation}
            U(F(X)) \neq X.
        \end{equation}
    }; and

    \item a forgetful functor
    \begin{equation}
        U: \cat{Set}_{\bot} \to \cat{Set},
    \end{equation}
    which forgets the extra structure of the basepoint:
    \begin{equation}
        U((Y, \bot)) = Y
    \end{equation}
    where $Y$ is the underlying set of the pointed set $(Y, \bot)$\footnote{
        If a set $X$ does not have a distinguished basepoint, then $U$ has no affect $U$ on $X$:
        \begin{equation}
            U(X) = X.
        \end{equation}
    }. \draftnote{blue}{(PS) Consider}{Change to $(Y, y_{0})$ ?}
\end{enumerate}
The adjunction $F \dashv U$ means that every object in $\cat{Set}_{\bot}$ corresponds with an object in $\cat{Set}$, and every morphism in $\cat{Set}_{\bot}$ corresponds with a morphism in $\cat{Set}$ (but not the reverse).

So we want to (1) embed our totalised action effect operator $\ast^{\bot}: \hat{A}^{*} \times W^{\bot} \to W^{\bot}$ into $\cat{Set}$ from $\cat{Set}_{\bot}$ using the adjunction $F \dashv U$, then (2) curry the totalised action effect operator in $\cat{Set}$, and finally (3) lift the curried totalised action effect operator back into $\cat{Set}_{\bot}$ using the adjunction $F \dashv U$.

\begin{enumerate}
    \item \textbf{Embed $\ast^{\bot}$ into $\cat{Set}$ via $F \dashv U$.}
    To do this we, define $\ast^{\bot}: \hat{A}^{*} \times W^{\bot} \to W^{\bot}$ from our original framework as the morphism
    \begin{align}
        & \ast^{\bot}: F(\hat{A}^{*}) \times F(W) \to F(W) \quad \text{such that} \\
        & \ast^{\bot}(a,w) :=
                \begin{cases}
                  a \ast w, & \text{if } \exists d: w \xrightarrow{a} t(d) \\
                  \bot, & \text{if } \centernot\exists d: w \xrightarrow{a} t(d) \\
                  \bot, & \text{if } a = \bot \text{ or } w = \bot
                \end{cases}
    \end{align}
    in $\cat{Set}_{\bot}$.
    Using the forgetful functor $U$, we embed our morphism $\ast^{\bot}: F(\hat{A}^{*}) \times F(W) \to F(W)$ into the morphism\footnote{
        We have
        \begin{equation}
            U(F(\hat{A}^{*}) \times F(W)) \cong U(F(\hat{A}^{*})) \times U(F(W))
        \end{equation}
        because $U: \cat{Set}_{\bot} \to \cat{Set}$ is a right adjoint, therefore it preserves products (and all universal constructions of limits).
        \draftnote{purple}{PS}{Check if $ U(F(\hat{A}^{*}) \times F(W)) = U(F(\hat{A}^{*})) \times U(F(W))$ or if $ U(F(\hat{A}^{*}) \times F(W)) \cong U(F(\hat{A}^{*})) \times U(F(W))$.}
    }
    \begin{equation}
        U(\ast^{\bot}): U(F(\hat{A}^{*})) \times U(F(W)) \to U(F(W))
    \end{equation}
    in $\cat{Set}$:\footnote{
        It's worth taking a moment to consider what $U(F(W))$ and $U(F(\hat{A}^{*}))$ are.
        From the definitions of $U$ and $F$,
        \begin{align}
            U(F(W)) & = U(W \sqcup \{\bot\}, \bot) \\
            & = W \sqcup \{\bot\} \\
            & = W^{\bot};
        \end{align}
        so $U(F(W))$ is just our set $W^{\bot}$ of world states that has been augmented with the undefined state $\bot$.

        \begin{align}
            U(F(\hat{A}^{*})) & = U(\hat{A}^{*} \sqcup \{\bot\}, \bot) \\
            & = \hat{A}^{*} \sqcup \{\bot\};
        \end{align}
        so our set $\hat{A}^{*}$ of actions has gained an element $\bot$.
        However, since
        \begin{equation}
            \ast^{\bot}(\bot, w) = \bot \quad \text{for all $w \in W^{\bot}$},
        \end{equation}
        this additional element makes no difference to the behaviour of $\ast^{\bot}$.
    }
    \begin{fullwidth}
    \begin{equation}
    % https://q.uiver.app/#q=WzAsOCxbMywwLCJcXHRleHR7SW4gfSBcXGNhdHtTZXR9X3tcXGJvdH06Il0sWzEsMCwiXFx0ZXh0e0luIH0gXFxjYXR7U2V0fToiXSxbMywxLCJGKFxcaGF0e0F9XnsqfSkgXFx0aW1lcyBGKFcpIl0sWzMsMywiRihXKSJdLFswLDEsIlUoRihcXGhhdHtBfV57Kn0pKSBcXHRpbWVzIFUoRihXKSkiXSxbMCwzLCJVKEYoVykpIl0sWzEsMSwiVShGKFxcaGF0e0F9XnsqfSkgXFx0aW1lcyBGKFcpKSJdLFsxLDMsIlUoRihXKSkiXSxbNCw1LCJVKFxcYXN0XntcXGJvdH0pIl0sWzIsMywiXFxhc3Ree1xcYm90fSJdLFsyLDYsIlUiLDEseyJjb2xvdXIiOlsxMjAsNjAsNjBdfSxbMTIwLDYwLDYwLDFdXSxbMyw3LCJVIiwxLHsiY29sb3VyIjpbMTIwLDYwLDYwXX0sWzEyMCw2MCw2MCwxXV0sWzcsNSwiPSJdLFs2LDQsIlxcY29uZyJdLFs2LDcsIlUoXFxhc3Ree1xcYm90fSkiXV0=
    \begin{tikzcd}[ampersand replacement=\&]
        \& {\text{In } \cat{Set}:} \&\& {\text{In } \cat{Set}_{\bot}:} \\
        {U(F(\hat{A}^{*})) \times U(F(W))} \& {U(F(\hat{A}^{*}) \times F(W))} \&\& {F(\hat{A}^{*}) \times F(W)} \\
        \\
        {U(F(W))} \& {U(F(W))} \&\& {F(W)}
        \arrow["{U(\ast^{\bot})}", from=2-1, to=4-1]
        \arrow["\cong", from=2-2, to=2-1]
        \arrow["{U(\ast^{\bot})}", from=2-2, to=4-2]
        \arrow["U"{description}, color={rgb,255:red,92;green,214;blue,92}, from=2-4, to=2-2]
        \arrow["{\ast^{\bot}}", from=2-4, to=4-4]
        \arrow["{=}", from=4-2, to=4-1]
        \arrow["U"{description}, color={rgb,255:red,92;green,214;blue,92}, from=4-4, to=4-2]
    \end{tikzcd}
    \end{equation}
    \end{fullwidth}
    \draftnote{purple}{PS}{Can we lift the $\ast$ morphism from $\cat{Set}$ into $\cat{Set}_{\bot}$ in a categorical way?}

    \item \textbf{Curry $U(\ast^{\bot})$.}
    Since $\cat{Set}$ is a Cartesian closed category, there is a canonical isomorphism
    \begin{equation}
        \text{Hom}_{\cat{Set}}(U(F(\hat{A}^{*})) \times U(F(W)), U(F(W)) \cong \text{Hom}_{\cat{Set}}(F(\hat{A}^{*}), U(F(W))^{U(F(W))}),
    \end{equation}
    which we use to curry $U(\ast^{\bot})$ to obtain the curried morphism
    \begin{equation}
        \tilde{\ast}^{\bot}: U(F(\hat{A}^{*})) \to U(F(W))^{U(F(W))}
    \end{equation}
    \begin{fullwidth}
    \begin{equation}
    % https://q.uiver.app/#q=WzAsMTAsWzQsMCwiXFx0ZXh0e0luIH0gXFxjYXR7U2V0fV97XFxib3R9OiJdLFsyLDAsIlxcdGV4dHtJbiB9IFxcY2F0e1NldH06Il0sWzQsMSwiRihcXGhhdHtBfV57Kn0pIFxcdGltZXMgRihXKSJdLFs0LDMsIkYoVykiXSxbMSwxLCJVKEYoXFxoYXR7QX1eeyp9KSkgXFx0aW1lcyBVKEYoVykpIl0sWzEsMywiVShGKFcpKSJdLFswLDMsIlUoRihXKSlee1UoRihXKSl9Il0sWzAsMSwiVShGKFxcaGF0e0F9XnsqfSkpIl0sWzIsMSwiVShGKFxcaGF0e0F9XnsqfSkgXFx0aW1lcyBGKFcpKSJdLFsyLDMsIlUoRihXKSJdLFs0LDUsIlUoXFxhc3Ree1xcYm90fSkiXSxbMiwzLCJcXGFzdF57XFxib3R9Il0sWzQsNywiXFxjb25nIiwwLHsic3R5bGUiOnsidGFpbCI6eyJuYW1lIjoiYXJyb3doZWFkIn19fV0sWzUsNiwiXFxjb25nIiwwLHsic3R5bGUiOnsidGFpbCI6eyJuYW1lIjoiYXJyb3doZWFkIn19fV0sWzcsNiwiXFx0aWxkZXtcXGFzdH1ee1xcYm90fSJdLFsyLDgsIlUiLDEseyJjb2xvdXIiOlsxMjAsNjAsNjBdfSxbMTIwLDYwLDYwLDFdXSxbMyw5LCJVIiwxLHsiY29sb3VyIjpbMTIwLDYwLDYwXX0sWzEyMCw2MCw2MCwxXV0sWzgsOSwiVShcXGFzdF57XFxib3R9KSJdLFs5LDUsIj0iXSxbOCw0LCJcXGNvbmciXV0=
    \begin{tikzcd}[ampersand replacement=\&]
        \&\& {\text{In } \cat{Set}:} \&\& {\text{In } \cat{Set}_{\bot}:} \\
        {U(F(\hat{A}^{*}))} \& {U(F(\hat{A}^{*})) \times U(F(W))} \& {U(F(\hat{A}^{*}) \times F(W))} \&\& {F(\hat{A}^{*}) \times F(W)} \\
        \\
        {U(F(W))^{U(F(W))}} \& {U(F(W))} \& {U(F(W)} \&\& {F(W)}
        \arrow["{\tilde{\ast}^{\bot}}", from=2-1, to=4-1]
        \arrow["\cong", tail reversed, from=2-2, to=2-1]
        \arrow["{U(\ast^{\bot})}", from=2-2, to=4-2]
        \arrow["\cong", from=2-3, to=2-2]
        \arrow["{U(\ast^{\bot})}", from=2-3, to=4-3]
        \arrow["U"{description}, color={rgb,255:red,92;green,214;blue,92}, from=2-5, to=2-3]
        \arrow["{\ast^{\bot}}", from=2-5, to=4-5]
        \arrow["\cong", tail reversed, from=4-2, to=4-1]
        \arrow["{=}", from=4-3, to=4-2]
        \arrow["U"{description}, color={rgb,255:red,92;green,214;blue,92}, from=4-5, to=4-3]
    \end{tikzcd}
    \end{equation}
    \end{fullwidth}
    For each $a \in U(F(\hat{A}^{*}))$, we now have a function
    \begin{equation}
        \tilde{\ast}^{\bot}(a): U(F(W)) \to U(F(W))
    \end{equation}
    satisfying
    \begin{equation}
        \tilde{\ast}^{\bot}(a)(w) = \begin{cases}
                    \bot, & \text{if $a = \bot$} \\
                  a \ast^{\bot} w, & \text{otherwise}
                \end{cases}
    \end{equation}
    Since, $\ast^{\bot}$ is basepoint-preserving, we have
    \begin{equation}
        \tilde{\ast}^{\bot}(a)(\bot) = \bot.
    \end{equation}
    Therefore, each $\tilde{\ast}^{\bot}(a)$ is a basepoint-preserving map
    \begin{equation}
        \tilde{\ast}^{\bot}(a) \in \{f: U(F(W)) \to U(F(W)) \mid f(\bot) = \bot\}
    \end{equation}


    \item \textbf{Lift $\tilde{\ast}^{\bot}$ into $\cat{Set}_{\bot}$ via $F \dashv U$.}
    To lift $\tilde{\ast}^{\bot}$ into $\cat{Set}_{\bot}$, we're going to consider each curried morphism $\tilde{\ast}^{\bot}(a): U(F(W)) \to U(F(W))$ individually, and show we can use these morphisms to construct an expression for a morphism from $F(W) \to F(W)$, which we will call $\phi(a)$.
    
    Starting at $W$ in $\cat{Set}$, we apply $F$ to $W$,
    \begin{equation}
        % https://q.uiver.app/#q=WzAsNCxbMCwwLCJcXHRleHR7SW4gfSBcXGNhdHtTZXR9OiJdLFsyLDAsIlxcdGV4dHtJbiB9IFxcY2F0e1NldH1fe1xcYm90fToiXSxbMiwzLCJGKFcpIl0sWzAsMSwiVyJdLFszLDIsIkYiLDEseyJjb2xvdXIiOlswLDYwLDYwXX0sWzAsNjAsNjAsMV1dXQ==
    \begin{tikzcd}[ampersand replacement=\&]
        {\text{In } \cat{Set}:} \&\& {\text{In } \cat{Set}_{\bot}:} \\
        W \\
        \\
        \&\& {F(W)}
        \arrow["F"{description}, color={rgb,255:red,214;green,92;blue,92}, from=2-1, to=4-3]
    \end{tikzcd}
    \end{equation}
    then we apply $U$ to $F(W)$, constructing the relevant map $\eta_{W}$ as we do, and adding the curried morphism $\tilde{\ast}^{\bot}$ that corresponds to $a \in \hat{A}^{*}$:
    \begin{equation}
        % https://q.uiver.app/#q=WzAsNSxbMCwwLCJcXHRleHR7SW4gfSBcXGNhdHtTZXR9OiJdLFsyLDAsIlxcdGV4dHtJbiB9IFxcY2F0e1NldH1fe1xcYm90fToiXSxbMiwzLCJGKFcpIl0sWzAsMywiVShGKFcpKSJdLFswLDEsIlciXSxbMiwzLCJVIiwxLHsiY29sb3VyIjpbMTIwLDYwLDYwXX0sWzEyMCw2MCw2MCwxXV0sWzQsMiwiRiIsMSx7ImNvbG91ciI6WzAsNjAsNjBdfSxbMCw2MCw2MCwxXV0sWzQsMywiXFxldGFfe1d9Il0sWzMsMywiXFx0aWxkZXtcXGFzdH1ee1xcYm90fShhKSIsMCx7Im9mZnNldCI6LTUsImFuZ2xlIjotOTB9XV0=
    \begin{tikzcd}[ampersand replacement=\&]
        {\text{In } \cat{Set}:} \&\& {\text{In } \cat{Set}_{\bot}:} \\
        W \\
        \\
        {U(F(W))} \&\& {F(W)}
        \arrow["{\eta_{W}}", from=2-1, to=4-1]
        \arrow["F"{description}, color={rgb,255:red,214;green,92;blue,92}, from=2-1, to=4-3]
        \arrow["{\tilde{\ast}^{\bot}(a)}", shift left=5, from=4-1, to=4-1, loop, in=145, out=215, distance=10mm]
        \arrow["U"{description}, color={rgb,255:red,92;green,214;blue,92}, from=4-3, to=4-1]
    \end{tikzcd}
    \end{equation}
    Then we apply $F$ to $U(F(W))$, construct the relevant morphism $\epsilon_{F(W)}$, and transport $\tilde{\ast}^{\bot}(a)$ and $\eta_{W}$ over to $\cat{Set}_{\bot}$:
    \begin{equation}
        % https://q.uiver.app/#q=WzAsNixbMCwwLCJcXHRleHR7SW4gfSBcXGNhdHtTZXR9OiJdLFsyLDAsIlxcdGV4dHtJbiB9IFxcY2F0e1NldH1fe1xcYm90fToiXSxbMiwzLCJGKFcpIl0sWzIsNSwiRihVKEYoVykpKSJdLFswLDMsIlUoRihXKSkiXSxbMCwxLCJXIl0sWzIsNCwiVSIsMSx7ImNvbG91ciI6WzEyMCw2MCw2MF19LFsxMjAsNjAsNjAsMV1dLFs0LDMsIkYiLDEseyJjb2xvdXIiOlswLDYwLDYwXX0sWzAsNjAsNjAsMV1dLFszLDIsIlxcZXBzaWxvbl97RihXKX0iXSxbMywzLCJGKFxcdGlsZGV7XFxhc3R9XntcXGJvdH0oYSkpIiwwLHsiYW5nbGUiOi0xODB9XSxbNSwyLCJGIiwxLHsiY29sb3VyIjpbMCw2MCw2MF19LFswLDYwLDYwLDFdXSxbNSw0LCJcXGV0YV97V30iXSxbNCw0LCJcXHRpbGRle1xcYXN0fV57XFxib3R9KGEpIiwwLHsib2Zmc2V0IjotNSwiYW5nbGUiOi05MH1dLFsyLDMsIkYoXFxldGFfe1d9KSIsMCx7Im9mZnNldCI6LTN9XV0=
    \begin{tikzcd}[ampersand replacement=\&]
        {\text{In } \cat{Set}:} \&\& {\text{In } \cat{Set}_{\bot}:} \\
        W \\
        \\
        {U(F(W))} \&\& {F(W)} \\
        \\
        \&\& {F(U(F(W)))}
        \arrow["{\eta_{W}}", from=2-1, to=4-1]
        \arrow["F"{description}, color={rgb,255:red,214;green,92;blue,92}, from=2-1, to=4-3]
        \arrow["{\tilde{\ast}^{\bot}(a)}", shift left=5, from=4-1, to=4-1, loop, in=145, out=215, distance=10mm]
        \arrow["F"{description}, color={rgb,255:red,214;green,92;blue,92}, from=4-1, to=6-3]
        \arrow["U"{description}, color={rgb,255:red,92;green,214;blue,92}, from=4-3, to=4-1]
        \arrow["{F(\eta_{W})}", shift left=3, from=4-3, to=6-3]
        \arrow["{\epsilon_{F(W)}}", from=6-3, to=4-3]
        \arrow["{F(\tilde{\ast}^{\bot}(a))}", from=6-3, to=6-3, loop, in=235, out=305, distance=10mm]
    \end{tikzcd}
    \end{equation}
    Finally, from the universal property of the adjunction $F \dashv U$, the unique lift of the curried map $\tilde{\ast}^{\bot}(a): U(F(W)) \to U(F(W))$ to a pointed map $\phi(a): W^{\bot} \to W^{\bot}$ can be constructed:
    \begin{equation}
        % https://q.uiver.app/#q=WzAsNixbMCwwLCJcXHRleHR7SW4gfSBcXGNhdHtTZXR9OiJdLFsyLDAsIlxcdGV4dHtJbiB9IFxcY2F0e1NldH1fe1xcYm90fToiXSxbMiwzLCJGKFcpIl0sWzIsNSwiRihVKEYoVykpKSJdLFswLDMsIlUoRihXKSkiXSxbMCwxLCJXIl0sWzIsNCwiVSIsMSx7ImNvbG91ciI6WzEyMCw2MCw2MF19LFsxMjAsNjAsNjAsMV1dLFs0LDMsIkYiLDEseyJjb2xvdXIiOlswLDYwLDYwXX0sWzAsNjAsNjAsMV1dLFszLDIsIlxcZXBzaWxvbl97RihXKX0iXSxbMywzLCJGKFxcdGlsZGV7XFxhc3R9XntcXGJvdH0oYSkpIiwwLHsiYW5nbGUiOi0xODB9XSxbNSwyLCJGIiwxLHsiY29sb3VyIjpbMCw2MCw2MF19LFswLDYwLDYwLDFdXSxbNSw0LCJcXGV0YV97V30iXSxbNCw0LCJcXHRpbGRle1xcYXN0fV57XFxib3R9KGEpIFxcXFw9IFUoXFxwaGkoYSkpIiwwLHsib2Zmc2V0IjotNSwiYW5nbGUiOi05MH1dLFsyLDMsIkYoXFxldGFfe1d9KSIsMCx7Im9mZnNldCI6LTN9XSxbMiwyLCJcXGV4aXN0cyAhIFxcOyBcXHBoaShhKSIsMCx7InN0eWxlIjp7ImJvZHkiOnsibmFtZSI6ImRhc2hlZCJ9fX1dXQ==
    \begin{tikzcd}[ampersand replacement=\&]
        {\text{In } \cat{Set}:} \&\& {\text{In } \cat{Set}_{\bot}:} \\
        W \\
        \\
        {U(F(W))} \&\& {F(W)} \\
        \\
        \&\& {F(U(F(W)))}
        \arrow["{\eta_{W}}", from=2-1, to=4-1]
        \arrow["F"{description}, color={rgb,255:red,214;green,92;blue,92}, from=2-1, to=4-3]
        \arrow["\begin{array}{c} \tilde{\ast}^{\bot}(a) \\= U(\phi(a)) \end{array}", shift left=5, from=4-1, to=4-1, loop, in=145, out=215, distance=10mm]
        \arrow["F"{description}, color={rgb,255:red,214;green,92;blue,92}, from=4-1, to=6-3]
        \arrow["U"{description}, color={rgb,255:red,92;green,214;blue,92}, from=4-3, to=4-1]
        \arrow["{\exists ! \; \phi(a)}", dashed, from=4-3, to=4-3, loop, in=55, out=125, distance=10mm]
        \arrow["{F(\eta_{W})}", shift left=3, from=4-3, to=6-3]
        \arrow["{\epsilon_{F(W)}}", from=6-3, to=4-3]
        \arrow["{F(\tilde{\ast}^{\bot}(a))}", from=6-3, to=6-3, loop, in=235, out=305, distance=10mm]
    \end{tikzcd}
    \end{equation}
    where
    \begin{equation}
        \phi(a) = \epsilon_{F(W)} \circ F(\tilde{\ast}^{\bot}(a)) \circ F(\eta_{W})
    \end{equation}
    is the unique morphism in $\cat{Set}_{\bot}$ such that the underlying set function is $\tilde{\ast}^{\bot}(a)$:
    \begin{equation}
        U(\phi(a)) = \tilde{\ast}^{\bot}(a).
    \end{equation}
    This means that $\phi(a)$ is the unique lift of the function
    \begin{equation}
        (w \mapsto a \ast^{\bot} w) \in \text{End}(W^{\bot})
    \end{equation}
    from $\cat{Set}$ to $\cat{Set}_{\bot}$, and, therefore,
    \begin{equation}
    \begin{aligned}
        & \phi: \hat{A}^{*} \to \text{End}(W^{\bot}) \\
        & \phi(a)(w) = a \ast^{\bot} w \quad \text{for all $w \in W$ and for all $a \in \hat{A}^{*}$}.
    \end{aligned}
    \end{equation}
\end{enumerate}


\newthought{In categorical terms,} currying the action $\ast^{\bot}$ encodes it as as a functor from the one-object category $\mathbf{B}\hat{A}^{*}$ into $\cat{Set}_{\bot}$:
\begin{equation}
    A^{\bot}: \mathbf{B}\hat{A}^{*} \to \cat{Set}_{\bot}.
\end{equation}
where $A^{\bot}$ acts as follows:
\begin{enumerate}
    \item \textbf{On objects:}
    the single object $\bullet$ is mapped to the pointed set $(W^{\bot}, \bot)$
    \begin{equation}
        A^{\bot}(\bullet) := (W^{\bot}, \bot).
    \end{equation}
    \item \textbf{On morphisms:}
    each morphism $a \in \hat{A}^{*}$ is mapped to the world state-transition function $w \mapsto a \ast^{\bot} w$, respecting $\bot$,
    \begin{equation}
    \begin{aligned}
        A^{\bot}(a) & := \phi(a) \quad \text{such that}
    \end{aligned}
    \end{equation}
    and so\footnote{
        \begin{equation}
             % https://q.uiver.app/#q=WzAsNixbMSwwLCJcXGJ1bGxldCJdLFszLDAsIlxcYnVsbGV0Il0sWzEsMSwiKFdee1xcYm90fSwgXFxib3QpIl0sWzMsMSwiKFdee1xcYm90fSwgXFxib3QpIl0sWzAsMCwiXFx0ZXh0e0luICB9IFxcdGV4dGJme0J9XFxoYXR7QX1eeyp9OiJdLFswLDEsIlxcdGV4dHtJbiAgfSBcXGNhdHtTZXR9X3tcXGJvdH0iXSxbMCwxLCJhIl0sWzAsMiwiQV57XFxib3R9KGEpIiwwLHsic3R5bGUiOnsidGFpbCI6eyJuYW1lIjoibWFwcyB0byJ9fX1dLFsxLDMsIkFee1xcYm90fShhKSIsMCx7InN0eWxlIjp7InRhaWwiOnsibmFtZSI6Im1hcHMgdG8ifX19XSxbMiwzLCJBXntcXGJvdH0oYSkiXV0=
        \begin{tikzcd}[ampersand replacement=\&]
            {\text{In  } \mathbf{B}\hat{A}^{*}:} \& \bullet \&\& \bullet \\
            {\text{In  } \cat{Set}_{\bot}} \& {(W^{\bot}, \bot)} \&\& {(W^{\bot}, \bot)}
            \arrow["a", from=1-2, to=1-4]
            \arrow["{A^{\bot}(a)}", maps to, from=1-2, to=2-2]
            \arrow["{A^{\bot}(a)}", maps to, from=1-4, to=2-4]
            \arrow["{A^{\bot}(a)}", from=2-2, to=2-4]
        \end{tikzcd}
        \end{equation}
    }
    \begin{align}
    \begin{aligned}
        & A^{\bot}(a): W^{\bot} \to W^{\bot} \quad \text{such that} \\
        & A^{\bot}(a)(w) = a \ast^{\bot} w.
    \end{aligned}
    \end{align}
     in $\cat{Set}_{\bot}$.
\end{enumerate}


\begin{propositionE}
    $A^{\bot}: \mathbf{B}\hat{A}^{*} \to \cat{Set}_{\bot}$ is a functor.
\end{propositionE}
\begin{proofE}
\begin{enumerate}
    \item \textbf{Identity.}
    For any $w \in W^{\bot}$, we have
    \begin{align}
        A^{\bot}(\varepsilon)(w) & = \varepsilon \ast^{\bot} w \\
        & = w.
    \end{align}
    Since this holds for any $w \in W^{\bot}$, we have
    \begin{equation}
        A^{\bot}(\varepsilon) = \text{id}_{W^{\bot}}.
    \end{equation}

    \item \textbf{Composition.}
    For any $w \in W^{\bot}$, we have
    \begin{align}
        A^{\bot}(a' \circ a)(w) & = (a' \circ a) \ast^{\bot} w \\
        & = a' \ast^{\bot} (a \ast^{\bot} w) \\
        & = (A^{\bot}(a') \circ A^{\bot}(a))(w).
    \end{align}
    Since this holds for any $w \in W^{\bot}$, we have
    \begin{equation}
        A^{\bot}(a' \circ a) = A^{\bot}(a') \circ A^{\bot}(a).
    \end{equation}
\end{enumerate}
\end{proofE}

The functor $A^{\bot}$ is faithful because distinct actions induce distinct functions; this is ensured by the uniqueness condition in the labelling map $l$.\draftnote{purple}{To do}{Prove this.}

Our indirect currying method allows us to
\begin{enumerate}
    \item package the data of $\ast^{\bot}$ into a functor
    \begin{equation}
        A^{\bot}: \mathbf{B}\hat{A}^{*} \to \cat{Set}_{\bot},
    \end{equation}
    which contains the total action as a monoid representation in $\cat{Set}_{\bot}$; and
    \item reinterpret the totalised action operator $\ast^{\bot}$ as a map into the endomorphism set $\text{End}(W^{\bot})$; the endomorphism viewpoint is the pointwise version of the functor $A^{\bot}$ (i.e., the functor $A^{\bot}$ acts on each element of the object $W^{\bot}$ individually instead of manipulating the entire structure $W^{\bot}$ as a whole).
\end{enumerate}


%%%%%%%%%%%%%%%%%%%%%%%%%%%%%%%%%%%%%%%
\paragraph{The action functor and the labelling functor.}
\newthought{We can compose} our action functor
\begin{equation}
    A^{\bot}: \mathbf{B}\hat{A}^{*} \to \cat{Set}_{\bot}
\end{equation}
with our totalised action labelling map
\begin{equation}
    L^{\bot}: \mathcal{F}(\hat{\mathscr{W}}_{\mathscr{A}}^{\bot}) \to \mathbf{B}\hat{A}^{*}
\end{equation}
to give the composite functor
\begin{equation}
    A^{\bot} \circ L^{\bot}: \mathcal{F}(\hat{\mathscr{W}}_{\mathscr{A}}^{\bot}) \to \cat{Set}_{\bot},
\end{equation}
\begin{equation}
    % https://q.uiver.app/#q=WzAsMyxbMCwxLCJcXG1hdGhjYWx7Rn0oXFxoYXR7XFxtYXRoc2Nye1d9fV97XFxtYXRoc2Nye0F9fV57XFxib3R9KSJdLFsyLDAsIlxcdGV4dGJme0J9XFxoYXR7QX1eeyp9Il0sWzQsMSwiXFxjYXR7U2V0fV97XFxib3R9Il0sWzAsMSwie0xee1xcYm90fX0iXSxbMSwyLCJ7QV57XFxib3R9fSJdLFswLDIsIlxcZXhpc3RzICEgXFw7IHtBXntcXGJvdH0gXFxjaXJjIExee1xcYm90fX0iLDIseyJzdHlsZSI6eyJib2R5Ijp7Im5hbWUiOiJkYXNoZWQifX19XV0=
    \begin{tikzcd}[ampersand replacement=\&]
        \&\& {\mathbf{B}\hat{A}^{*}} \\
        {\mathcal{F}(\hat{\mathscr{W}}_{\mathscr{A}}^{\bot})} \&\&\&\& {\cat{Set}_{\bot}}
        \arrow["{{A^{\bot}}}", from=1-3, to=2-5]
        \arrow["{{L^{\bot}}}", from=2-1, to=1-3]
        \arrow["{\exists ! \; {A^{\bot} \circ L^{\bot}}}"', dashed, from=2-1, to=2-5]
    \end{tikzcd}
\end{equation}
where $A^{\bot} \circ L^{\bot}$ is unique because both $L^{\bot}$ is uniquely determined by the atomic labelling $\hat{l}$ and the pushout, and $A^{\bot}$ is uniquely determined by $\ast^{\bot}$ and the adjunction.
$A^{\bot} \circ L^{\bot}$ being unique means there is no ambiguity in the labelling of $\mathcal{F}(\hat{\mathscr{W}}_{\mathscr{A}}^{\bot})$ by $\mathbf{B}\hat{A}^{*}$.

$A^{\bot} \circ L^{\bot}$ takes a transformation in the totalised world, labels it with its corresponding action $a \in \hat{A}^{*}$, and then applies the effect of $a$ on $W^{\bot}$ via the monoid homomorphism $\phi$.
In this way, $\ast^{\bot}$ can be viewed as the \emph{concrete representation}\footnote{
    A \emph{concrete representation} is a representation in $\cat{Set}$.
} of the abstract structure of the free category $\mathcal{F}(\hat{\mathscr{W}}_{\mathscr{A}}^{\bot})$ via the functor $A^{\bot} \circ L^{\bot}$; $A^{\bot} \circ L^{\bot}$ captures how the abstract composition of actions (encoded categorically) actually operates on world states in a concrete (i.e., set-theoretic) manner.
\draftnote{purple}{(PS) Include}{
Talk about syntactic action vs semantic information.
Syntax is the free, combinatorial structure of the actions.
The semantics is the interpretation of these actions as actual transformations on $W^{\bot}$.
}

So what is happening when we apply the composite functor $A^{\bot} \circ L^{\bot}$ to the free category $\mathcal{F}(\hat{\mathscr{W}}_{\mathscr{A}}^{\bot})$?
The free category $\mathcal{F}(\hat{\mathscr{W}}_{\mathscr{A}}^{\bot})$ contains rich semantic data about the world\footnote{
    From a category theory perspective, $\mathcal{F}(\hat{\mathscr{W}}_{\mathscr{A}}^{\bot})$ actually contains all the information about the world via the Yoneda lemma \draftnote{purple}{(PS) To do}{Check this.}.
}.
The labelling functor $L^{\bot}$ "forgets" much of the rich semantic data present in  $\mathcal{F}(\hat{\mathscr{W}}_{\mathscr{A}}^{\bot})$ (e.g., details about individual world states and the intermediate steps of transformations), but it retains syntactic information about the compositional properties of the transformations in $\mathcal{F}(\hat{\mathscr{W}}_{\mathscr{A}}^{\bot})$ as reflected in the concatenation of actions in $\mathbf{B}\hat{A}^{*}$.
We can think of $L^{\bot}$ as forgetting this semantic information by "gluing" all the world states in $\mathcal{F}(\hat{\mathscr{W}}_{\mathscr{A}}^{\bot})$ together to form a single object $\bullet$ in $\mathbf{B}\hat{A}^{*}$, but preserving the essential compositional structure of the transformations in $\mathcal{F}(\hat{\mathscr{W}}_{\mathscr{A}}^{\bot})$ by imposing a syntactic structure through the labels; $A^{\bot}$ then restores semantic meaning by interpreting the syntactic labels as concrete transformations of the set $W^{\bot}$.
In this way, the composite functor $A^{\bot} \circ L^{\bot}$ forms a semantic–syntactic bridge.


%%%%%%%%%%%%%%%%%%%%%%%%%%%%%%%%%%%%%%%%%%%%%%%%
\section{
Categorification of the equivalence relation
}
%%%%%%%%%%%%%%%%%%%%%%%%%%%%%%%%%%%%%%%%%%%%%%%%
\subsection{Recap.}
We defined an equivalence relation $\sim$ on $\hat{A}^{*}$ such that, for two actions $a, a' \in \hat{A}^{*}$,
\begin{equation}
    a \sim a' \iff a \ast w = a' \ast w \quad \text{ for all $w \in W^{\bot}$}.
\end{equation}
$\sim$ is a congruence relation on the free monoid $(\hat{A}^{*}, \circ)$.

We then apply $\sim$ to the free monoid $(\hat{A}^{*}, \circ)$ through the monoid homomorphism
\begin{equation}
    \pi_{\sim}: \hat{A}^{*} \to \hat{A}^{*}/\sim
\end{equation}
to give the quotient monoid $(\hat{A}^{*}/\sim, \circ_{\sim})$, which reflects the global structure of the effect $\ast^{\bot}$ of the actions on the world.


%%%%%%%%%%%%%%%%%%%%%%%%%%%%%%%%%%%%%%%%%%%%%%%%
\subsection{Categorical conversion.}
\draftnote{purple}{Include}{
The equivalence relation $\sim$ is the kernel of the Yoneda embedding $\mathcal{Y}: \mathcal{F}(\hat{\mathscr{W}}_{\mathscr{A}}^{\bot}) \to \cat{Set}^{{\mathcal{F}(\hat{\mathscr{W}}_{\mathscr{A}}^{\bot})}^{\text{op}}}$, where $a \sim a'$ iff they induce the same natural transformation.
}


\newthought{The congruence relation} $\sim$ identifies precisely those pairs of actions $a, a' \in \hat{A}^{*}$ that the functor $A^{\bot}$ maps to the same function in $\cat{Set}_{\bot}$; in other words,\footnote{
    The Yoneda embedding maps states to their distinguishable (i.e., observable) behaviours
    \begin{equation}
        \mathcal{Y}(w) = \text{Hom}(-, w).
    \end{equation}
    Two states are \emph{observationally equivalent} (i.e., $a \ast w = a \ast w'$ for all $a \in \hat{A}^{*}$) if and only if $\mathcal{Y}(w) \cong \mathcal{Y}(w')$; therefore,
    \begin{equation}
        a \sim a' \iff \mathcal{Y}(w) \cong \mathcal{Y}(w'),
    \end{equation}
    and so we can view the equivalence relation $\sim$ as a characterisation of the Yoneda embedding.
}
\begin{align}
    & a \sim a' \\
    \iff & a \ast^{\bot} w = a' \ast^{\bot} w \quad \text{for all $w \in W^{\bot}$}
\end{align}
and, since $A^{\bot}(a)(w) = a \ast^{\bot} w$, we have
\begin{align}
    & a \ast^{\bot} w = a' \ast^{\bot} w \quad \text{for all $w \in W^{\bot}$} \\
    \iff & A^{\bot}(a)(w) = A^{\bot}(a')(w) \quad \text{for all $w \in W^{\bot}$} \\
    \iff & A^{\bot}(a) = A^{\bot}(a').
\end{align}
This makes $\sim$ the \emph{kernel congruence} of the functor $A^{\bot}$ - so the application of $\sim$ to the morphisms in $\mathbf{B}\hat{A}^{*}$ collapses all the morphisms that act identically on $W^{\bot}$.
The categorical manifestation of the kernel congruence $\sim$ is the \emph{kernel pair} $(\mathcal{K}, p_{1}, p_{2})$ of the action functor $A^{\bot}$:
\begin{equation}
    % https://q.uiver.app/#q=WzAsOSxbNSwwLCJcXG1hdGhjYWx7S30iXSxbNywwLCJcXHRleHRiZntCfVxcaGF0e0F9XnsqfSJdLFs1LDIsIlxcdGV4dGJme0J9XFxoYXR7QX1eeyp9Il0sWzcsMiwiXFxjYXR7U2V0fV97XFxib3R9Il0sWzQsMV0sWzMsMV0sWzIsMCwiXFx0ZXh0YmZ7Qn1cXGhhdHtBfV57Kn0iXSxbMiwyLCJcXGNhdHtTZXR9X3tcXGJvdH0iXSxbMCwyLCJcXHRleHRiZntCfVxcaGF0e0F9XnsqfSJdLFswLDEsInBfMSJdLFswLDIsInBfMiIsMl0sWzEsMywiQV57XFxib3R9Il0sWzIsMywiQV57XFxib3R9IiwyXSxbNSw0LCJcXHRleHR7cHVsbGJhY2t9Il0sWzYsNywiQV57XFxib3R9Il0sWzgsNywiQV57XFxib3R9IiwyXV0=
\begin{tikzcd}[ampersand replacement=\&]
    \&\& {\mathbf{B}\hat{A}^{*}} \&\&\& {\mathcal{K}} \&\& {\mathbf{B}\hat{A}^{*}} \\
    \&\&\& {} \& {} \\
    {\mathbf{B}\hat{A}^{*}} \&\& {\cat{Set}_{\bot}} \&\&\& {\mathbf{B}\hat{A}^{*}} \&\& {\cat{Set}_{\bot}}
    \arrow["{A^{\bot}}", from=1-3, to=3-3]
    \arrow["{p_1}", from=1-6, to=1-8]
    \arrow["{p_2}"', from=1-6, to=3-6]
    \arrow["{A^{\bot}}", from=1-8, to=3-8]
    \arrow["{\text{pullback}}", from=2-4, to=2-5]
    \arrow["{A^{\bot}}"', from=3-1, to=3-3]
    \arrow["{A^{\bot}}"', from=3-6, to=3-8]
\end{tikzcd}
\end{equation}
where $\mathcal{K}$ consists of
\begin{enumerate}
    \item \textbf{Objects:}
    \begin{equation}
        \text{Obj}(\mathcal{K}) = \{\bullet_{\mathcal{K}}\};
    \end{equation}
    \item \textbf{Morphisms:}\footnote{
    $\mathcal{K}$ encodes pairs $(a, a')$ with $A^{\bot}(a) = A^{\bot}(a')$ so that $a$ and $a'$ are sent to the same morphism in $\cat{Set}_{\bot}$.
}
    \begin{equation}
        \text{Hom}(\bullet_{\mathcal{K}}, \bullet_{\mathcal{K}}) = \{(a, a') \in \hat{A}^{*} \times \hat{A}^{*} \mid a \sim a'\};
    \end{equation}
\end{enumerate}
and the functors $p_{1}, p_{2}: \mathcal{K} \rightrightarrows \mathbf{B}\hat{A}^{*}$ are the pullback projection morphisms that send:
\begin{enumerate}
    \item \textbf{Objects:}
    \begin{align}
        & p_{1}(\bullet_{\mathcal{K}}) = \bullet \\
        & p_{2}(\bullet_{\mathcal{K}}) = \bullet
    \end{align}
    \item \textbf{Morphisms:}
    \begin{align}
        & p_{1}((a,a')) = a \\
        & p_{2}((a,a')) = a'.
    \end{align}
\end{enumerate}

The categorical manifestation of taking the quotient of the free monoid $\mathbf{B}\hat{A}^{*}$ under the equivalence relation $\sim$ is constructing the coequalizer $(\mathbf{B}(\hat{A}^{*}/\sim), \Pi_{\sim})$, where the functor $\Pi_{\sim}$ enforces the equivalence relation $\sim$ on $\mathbf{B}\hat{A}^{*}$.
We can construct the coequalizer $(\mathbf{B}(\hat{A}^{*}/\sim), \Pi_{\sim})$ as follows:
\begin{equation}
% https://q.uiver.app/#q=WzAsOSxbMCwwLCJcXG1hdGhjYWx7S30iXSxbMiwwLCJcXHRleHRiZntCfVxcaGF0e0F9XnsqfSJdLFszLDFdLFs0LDFdLFswLDIsIlxcdGV4dGJme0J9XFxoYXR7QX1eeyp9Il0sWzUsMiwiXFx0ZXh0YmZ7Qn1cXGhhdHtBfV57Kn0iXSxbNSwwLCJcXG1hdGhjYWx7S30iXSxbNywwLCJcXHRleHRiZntCfVxcaGF0e0F9XnsqfSJdLFs3LDIsIlxcdGV4dGJme0J9KFxcaGF0e0F9XnsqfS9cXHNpbSkiXSxbMCwxLCJwXzEiLDAseyJvZmZzZXQiOi0xfV0sWzIsMywiXFx0ZXh0e3B1c2hvdXR9Il0sWzAsNCwicF97Mn0iLDIseyJvZmZzZXQiOjF9XSxbNiw1LCJwX3syfSIsMl0sWzYsNywicF97MX0iXSxbNyw4LCJcXFBpX3tcXHNpbX0iXSxbNSw4LCJcXFBpX3tcXHNpbX0iLDJdXQ==
\begin{tikzcd}[ampersand replacement=\&]
    {\mathcal{K}} \&\& {\mathbf{B}\hat{A}^{*}} \&\&\& {\mathcal{K}} \&\& {\mathbf{B}\hat{A}^{*}} \\
    \&\&\& {} \& {} \\
    {\mathbf{B}\hat{A}^{*}} \&\&\&\&\& {\mathbf{B}\hat{A}^{*}} \&\& {\mathbf{B}(\hat{A}^{*}/\sim)}
    \arrow["{p_1}", shift left, from=1-1, to=1-3]
    \arrow["{p_{2}}"', shift right, from=1-1, to=3-1]
    \arrow["{p_{1}}", from=1-6, to=1-8]
    \arrow["{p_{2}}"', from=1-6, to=3-6]
    \arrow["{\Pi_{\sim}}", from=1-8, to=3-8]
    \arrow["{\text{pushout}}", from=2-4, to=2-5]
    \arrow["{\Pi_{\sim}}"', from=3-6, to=3-8]
\end{tikzcd}
\end{equation}
where the quotient functor\footnote{
    $\Pi_{\sim}$ is a strict monoidal functor because it preserves composition and identities.
}
\begin{equation}
    \Pi_{\sim}: \mathbf{B}\hat{A}^{*} \to \mathbf{B}(\hat{A}^{*}/\sim),
\end{equation}
sends\footnote{
    The functor $\Pi_{\sim}$ enforces the structure of $\mathcal{F}(\hat{\mathscr{W}}_{\mathscr{A}}^{\bot})$ onto $\mathbf{B}\hat{A}^{*}$ through the congruence relation $\sim$ by collapsing sequences of actions in $\hat{A}^{*}$ that are observationally indistinguishable in the world $\mathcal{F}(\hat{\mathscr{W}}_{\mathscr{A}}^{\bot})$.
    In other words, the coequalizer $(\mathbf{B}(\hat{A}^{*}/\sim), \Pi_{\sim})$ collapses the kernel pair of $A^{\bot}$.
}
\begin{enumerate}
    \item \textbf{Objects:}
    the single object $\bullet$ in $\mathbf{B}\hat{A}^{*}$ to the single object $\bullet$ in $\mathbf{B}(\hat{A}^{*}/\sim)$:
    \begin{equation}
        \Pi_{\sim}(\bullet) = \bullet.
    \end{equation}
    \item \textbf{Morphisms:}
    each morphism $a \in \hat{A}^{*}$ to its equivalence class $[a]_{\sim} \in \hat{A}^{*}/\sim$:
    \begin{equation}
        \Pi_{\sim}(a) = [a]_{\sim}.
    \end{equation}
\end{enumerate}
This gives us the diagram:
\begin{equation}
% https://q.uiver.app/#q=WzAsMyxbMCwwLCJcXG1hdGhjYWx7S30iXSxbMiwwLCJcXHRleHRiZntCfVxcaGF0e0F9XnsqfSJdLFs0LDAsIlxcdGV4dGJme0J9KFxcaGF0e0F9XnsqfS9cXHNpbSkiXSxbMCwxLCJwXzEiLDAseyJvZmZzZXQiOi0xfV0sWzEsMl0sWzAsMSwicF97Mn0iLDIseyJvZmZzZXQiOjF9XV0=
\begin{tikzcd}[ampersand replacement=\&]
    {\mathcal{K}} \&\& {\mathbf{B}\hat{A}^{*}} \&\& {\mathbf{B}(\hat{A}^{*}/\sim)}
    \arrow["{p_1}", shift left, from=1-1, to=1-3]
    \arrow["{p_{2}}"', shift right, from=1-1, to=1-3]
    \arrow["{\Pi_{\sim}}", from=1-3, to=1-5]
\end{tikzcd}
\end{equation}

The quotient map $\Pi_{\sim}$ forgets precisely the syntactic differences found in $\mathbf{B}\hat{A}^{*}$ that do not affect the semantic action on $W^{\bot}$.
This means that the functor $\Pi_{\sim}$ makes $\mathbf{B}\hat{A}^{*}$ into a structure $\mathbf{B}(\hat{A}^{*}/\sim)$ that is isomorphic to the structure of $\mathcal{F}(\hat{\mathscr{W}}_{\mathscr{A}}^{\bot})$ with respect to the equivalence $\sim$.


%%%%%%%%%%%%%%%%%%%%%%%%%%%%%%%%%%%%%%%%%%%%%%%%%%%%%%%%
\paragraph{
Labelling $\mathcal{F}(\hat{\mathscr{W}}_{\mathscr{A}}^{\bot})$ by the equivalence classes in $\mathbf{B}(\hat{A}^{*}/\sim)$.
}

The universal property of the coequalizer ensures that any functor from $\mathcal{F}(\hat{\mathscr{W}}_{\mathscr{A}}^\bot)$ into a category where the labels are already collapsed accord to $\sim$, must factor uniquely through $\Pi_{\sim}$.
Therefore, we can use our action labelling functor
\begin{equation}
    L^{\bot}: \mathcal{F}(\hat{\mathscr{W}}_{\mathscr{A}}^\bot) \to \mathbf{B}\hat{A}^{*}
\end{equation}
to construct the unique functor
\begin{equation}
    L^{\bot}_{\sim}: \mathcal{F}(\hat{\mathscr{W}}_{\mathscr{A}}^\bot) \to \mathbf{B}(\hat{A}^{*}/\sim)
\end{equation}
so that the diagram
\begin{equation}
    % https://q.uiver.app/#q=WzAsMyxbMCwwLCJcXG1hdGhjYWx7Rn0oXFxoYXR7XFxtYXRoc2Nye1d9fV97XFxtYXRoc2Nye0F9fV5cXGJvdCkiXSxbMiwwLCJcXG1hdGhiZntCfVxcaGF0e0F9XioiXSxbMiwyLCJcXG1hdGhiZntCfShcXGhhdHtBfV4qL1xcc2ltKSJdLFswLDEsIkxeXFxib3QiXSxbMCwyLCJcXGV4aXN0cyAhIFxcOyBMXlxcYm90X1xcc2ltIiwyLHsic3R5bGUiOnsiYm9keSI6eyJuYW1lIjoiZGFzaGVkIn19fV0sWzEsMiwiXFxQaV9cXHNpbSJdXQ==
\begin{tikzcd}[ampersand replacement=\&]
    {\mathcal{F}(\hat{\mathscr{W}}_{\mathscr{A}}^\bot)} \&\& {\mathbf{B}\hat{A}^*} \\
    \\
    \&\& {\mathbf{B}(\hat{A}^*/\sim)}
    \arrow["{L^\bot}", from=1-1, to=1-3]
    \arrow["{\exists ! \; L^\bot_\sim}"', dashed, from=1-1, to=3-3]
    \arrow["{\Pi_\sim}", from=1-3, to=3-3]
\end{tikzcd}
\end{equation}
commutes; in other words,
\begin{equation}
    L^{\bot}_{\sim} = \Pi_{\sim} \circ L^{\bot}.
\end{equation}
$L^{\bot}_{\sim}$ is the \emph{labelling functor under $\sim$} and it labels the morphisms in $\mathcal{F}(\hat{\mathscr{W}}_{\mathscr{A}}^\bot)$ with their associated equivalence classes in $\mathbf{B}(\hat{A}^{*}/\sim)$.

%%%%%%%%%%%%%%%%%%%%%%%%%%%%%%%%%%%%%%%%%%%%%%%%%%%%%%%
\paragraph{Constructing the action functor under $\sim$.}
We can now reattach the action functor $A^{\bot}: \mathbf{B}\hat{A}^{*} \to \cat{Set}_{\bot}$ to our diagram and construct the unique functor $A^{\bot}_{\sim}: \mathbf{B}(\hat{A}^{*}/\sim) \to \cat{Set}_{\bot}$ using the universal property of coequalizers:
\begin{equation}
    % https://q.uiver.app/#q=WzAsNCxbMCwwLCJcXG1hdGhjYWx7S30iXSxbMiwwLCJcXHRleHRiZntCfVxcaGF0e0F9XnsqfSJdLFs0LDAsIlxcdGV4dGJme0J9KFxcaGF0e0F9XnsqfS9cXHNpbSkiXSxbNCwyLCJcXHRleHRiZntTZXR9X3tcXGJvdH0iXSxbMCwxLCJwXzEiLDAseyJvZmZzZXQiOi0xfV0sWzEsMiwiXFxQaV97XFxzaW19Il0sWzAsMSwicF97Mn0iLDIseyJvZmZzZXQiOjF9XSxbMSwzLCJBXntcXGJvdH0iLDJdLFsyLDMsIlxcZXhpc3RzICEgXFw7IEFee1xcYm90fV97XFxzaW19IiwwLHsic3R5bGUiOnsiYm9keSI6eyJuYW1lIjoiZGFzaGVkIn19fV1d
\begin{tikzcd}[ampersand replacement=\&]
    {\mathcal{K}} \&\& {\mathbf{B}\hat{A}^{*}} \&\& {\mathbf{B}(\hat{A}^{*}/\sim)} \\
    \\
    \&\&\&\& {\textbf{Set}_{\bot}}
    \arrow["{p_1}", shift left, from=1-1, to=1-3]
    \arrow["{p_{2}}"', shift right, from=1-1, to=1-3]
    \arrow["{\Pi_{\sim}}", from=1-3, to=1-5]
    \arrow["{A^{\bot}}"', from=1-3, to=3-5]
    \arrow["{\exists ! \; A^{\bot}_{\sim}}", dashed, from=1-5, to=3-5]
\end{tikzcd}
\end{equation}
The unique functor $A^{\bot}_{\sim}$ is the functor associated with the action of the quotient $\mathbf{B}(\hat{A}^{*}/\sim)$ on the set $W^{\bot}$:\footnote{
    From the equalization property of the coequalizer, we have
    \begin{equation}
        \Pi_{\sim} \circ p_{1} = \Pi_{\sim} \circ p_{2}.
    \end{equation}
    This means $\Pi_{\sim}$ sends $p_{1}(a)$ and $p_{2}(a')$ to the same place in $\mathbf{B}(\hat{A}^{*}/\sim)$ for every $a \in \hat{A}^{*}$ (i.e., $\Pi_{\sim}$ "coequalizes" them).
    
    From the universal property of the coequalizer, we have the unique functor
    \begin{equation}
        A^{\bot} = A^{\bot}_{\sim} \circ \Pi_{\sim}.
    \end{equation}
}
\begin{equation}
    A^{\bot}_{\sim}: \mathbf{B}(\hat{A}^{*}/\sim) \to \cat{Set}_{\bot}
\end{equation}
that acts as follows:
\begin{enumerate}
    \item \textbf{On objects:}
    The single object $\bullet$ is sent to $W$
    \begin{equation}
        A^{\bot}_{\sim}(\bullet) = W^{\bot};
    \end{equation}
    \item \textbf{On morphisms:}
    Each action $a \in \hat{A}^{*}$ in the equivalence class $[a]_{\sim} \in \hat{A}^{*}/\sim$ is sent to the function in $\cat{Set}_{\bot}$ with
    \begin{align}
        & A^{\bot}_{\sim}([a]_{\sim}): W \to W \text{ such that } \\
        & A^{\bot}_{\sim}([a]_{\sim})(w) = [a]_{\sim} \ast w = a \ast w.
    \end{align}
    Since $A^{\bot}(a)(w) = a \ast w$, we have
    \begin{align}
        & A^{\bot}_{\sim}([a]_{\sim})(w) = A^{\bot}(a)(w) \quad \text{for all $w \in W$} \\
        \implies & A^{\bot}_{\sim}([a]_{\sim}) = A^{\bot}(a).
    \end{align}
\end{enumerate}


\newthought{A summary of} the most significant categories and functors we have derived up in $\cat{Cat}$ to this point is given by:

\draftnote{purple}{To do}{
\begin{enumerate}
    \item Include a diagram with all categories and functors (including the non-totalised categories and totalising functors) we have derived in $\cat{Cat}$.
    \item Include a diagram also showing the relationship to the objects and morphisms in $\cat{Quiv}$.
    \item Talk about the different possible composite functor from $\mathcal{F}(\hat{\mathscr{W}}_{\mathscr{A}}^{\bot})$ to $\cat{Set}_{\bot}$:
    \begin{equation}
        A^{\bot} \circ L^{\bot}: \mathcal{F}(\hat{\mathscr{W}}_{\mathscr{A}}^{\bot}) \to \cat{Set}_{\bot}
    \end{equation}
    and
    \begin{equation}
        A^{\bot}_{\sim} \circ \Pi_{\sim} \circ L^{\bot}: \mathcal{F}(\hat{\mathscr{W}}_{\mathscr{A}}^{\bot}) \to \cat{Set}_{\bot}
    \end{equation}
    \begin{enumerate}
        \item Are these two composite both unique ?
        \item How are they different ? Is it that $A^{\bot}_{\sim} \circ \Pi_{\sim} \circ L^{\bot}$ is faithful, while $A^{\bot} \circ L^{\bot}$ isn't because $A^{\bot} \circ L^{\bot}$ maps actions with the same representation in $\cat{Set}$ to the same elements) ?
    \end{enumerate}
\end{enumerate}
}

\begin{equation}
    % https://q.uiver.app/#q=WzAsNCxbMCwwLCJcXG1hdGhjYWx7Rn0oXFxoYXR7XFxtYXRoc2Nye1d9fV97XFxtYXRoc2Nye0F9fV57XFxib3R9KSJdLFsyLDAsIlxcbWF0aGJme0J9XFxoYXR7QX1eeyp9Il0sWzQsMCwiXFxtYXRoYmZ7Qn0oXFxoYXR7QX1eeyp9L1xcc2ltKSJdLFsyLDIsIlxcbWF0aGJme1NldH1fe1xcYm90fSJdLFswLDEsIkxee1xcYm90fSJdLFsxLDMsIkFee1xcYm90fSJdLFsxLDIsIlxcUGlfe1xcc2ltfSJdLFsyLDMsIkFee1xcYm90fV97XFxzaW19Il1d
\begin{tikzcd}[ampersand replacement=\&]
    {\mathcal{F}(\hat{\mathscr{W}}_{\mathscr{A}}^{\bot})} \&\& {\mathbf{B}\hat{A}^{*}} \&\& {\mathbf{B}(\hat{A}^{*}/\sim)} \\
    \\
    \&\& {\mathbf{Set}_{\bot}}
    \arrow["{L^{\bot}}", from=1-1, to=1-3]
    \arrow["{\Pi_{\sim}}", from=1-3, to=1-5]
    \arrow["{A^{\bot}}", from=1-3, to=3-3]
    \arrow["{A^{\bot}_{\sim}}", from=1-5, to=3-3]
\end{tikzcd}
\end{equation}


%%%%%%%%%%%%%%%%%%%%%%%%%%%%%%%%%%%%%%%%%%%%%%%%
\section{
Categorification of invertibility
}
%%%%%%%%%%%%%%%%%%%%%%%%%%%%%%%%%%%%%%%%%%%%%%%%
\subsection{Recap.}
An action $a \in \hat{A}^{*}$ is 
\begin{enumerate}
    \item \textbf{left invertible} from a world state $w \in W$ if there exists an action $a' \in \hat{A}^{*}$ such that
    \begin{equation}
        a' \ast (a \ast w) = w.
    \end{equation}
    An action that is left invertible from $w$ is also called \emph{reversible} from $w$;
    
    \item \textbf{right invertible} from a world state $w \in W$ if there exists an action $a' \in \hat{A}^{*}$ such that
    \begin{equation}
        a \ast (a' \ast w) = w;
    \end{equation}

    \item \textbf{full invertible} from a world state $w \in W$ if it is both left invertible from $w$ and right invertible from $w$.
\end{enumerate}
The invertibility of an action is
\begin{enumerate}
    \item \textbf{global} if it is left/right/fully invertible from all $w \in W$, but not necessarily by using the same $a'$.

    \item \textbf{consistently global} if it is globally left/right/fully invertible using the same $a'$; for example, an action $a \in \hat{A}^{*}$ is consistently globally left invertible if 
    \begin{equation}
        a' \ast (a \ast w) = w \quad \text{for all $w \in W$}.
    \end{equation}
\end{enumerate}

%%%%%%%%%%%%%%%%%%%%%%%%%%%%%%%%%%%%%%%%%%%%%%%%
\subsection{Categorical conversion.}
%%%%%%%%%%%%%%%%%%%%%%%%%%%%%%%%%%%%%%%%%%%%%%%%
\draftnote{purple}{(PS) To do}{
\begin{enumerate}
    \item Check.
    \item What is local invertibility ?
\end{enumerate}
}
\newthought{So far we} have come across three categorified structures that describe the dynamics of the world: (1) the free category $\mathcal{F}(\hat{\mathscr{W}}_{\mathscr{A}}^{\bot})$, which describes the syntactic and semantic structure of transformations, (2) the free monoid $\mathbf{B}\hat{A}^{*}$, which describes the syntactic structure of actions\footnote{
    The syntactic structure provided to $\mathbf{B}\hat{A}^{*}$ by $L^{\bot}$ is the same no matter the semantic structure of $\mathcal{F}(\hat{\mathscr{W}}_{\mathscr{A}}^{\bot})$, and so $\mathbf{B}\hat{A}^{*}$ contains no information about the structure of the world including the invertibility of actions in the world.
}, and (3) the quotient $\mathbf{B}(\hat{A}^{*}/\sim)$, which describes the syntactic and global semantic structure of actions.

%%%%%%%%%%%%%%%%%%%%%%%%%%%%%%%%%%%%%%%%%%%%%%%%
\paragraph{Reversibility in $\mathcal{F}(\hat{\mathscr{W}}_{\mathscr{A}}^{\bot})$.}
\newthought{Reversibility and irresistibility} are properties of transformations and therefore translate to properties of morphisms in $\mathcal{F}(\hat{\mathscr{W}}_{\mathscr{A}}^{\bot})$.
An action $a \in \hat{A}^{*}$ is \emph{reversible from a world state $w \in W^{\bot}$} if there exists a morphism $d': w \to a \ast w$ in $\mathcal{F}(\hat{\mathscr{W}}_{\mathscr{A}}^{\bot})$:
\begin{equation}
    % https://q.uiver.app/#q=WzAsNCxbMSwwLCJ3Il0sWzMsMCwiYSBcXGFzdCB3Il0sWzUsMCwidyJdLFswLDAsIlxcdGV4dHtJbiB9IFxcbWF0aGNhbHtGfShcXGhhdHtcXG1hdGhzY3J7V319X3tcXG1hdGhzY3J7QX19XntcXGJvdH0pOiJdLFswLDEsImQiXSxbMSwyLCJkJyJdXQ==
\begin{tikzcd}[ampersand replacement=\&]
    {\text{In } \mathcal{F}(\hat{\mathscr{W}}_{\mathscr{A}}^{\bot}):} \& w \&\& {a \ast w} \&\& w
    \arrow["d", from=1-2, to=1-4]
    \arrow["{d'}", from=1-4, to=1-6]
\end{tikzcd}
\end{equation}
If no such morphism exists, then the action $a \in \hat{A}^{*}$ is \emph{irreversible from the world state $w \in W^{\bot}$}.

\draftnote{blue}{Consider}{
Since $\mathcal{F}(\hat{\mathscr{W}}_{\mathscr{A}}^{\bot})$ is the free category over the underlying quiver, there are no relations between morphisms and so there are is no concept of equivalence of morphisms.
}


%%%%%%%%%%%%%%%%%%%%%%%%%%%%%%%%%%%%%%%%%%%%%%%%
\paragraph{
Global consistent invertibility as a representation in $\cat{Set}_{\bot}$.
}
\newthought{An action $a \in \hat{A}^{*}$ viewed} as a morphism in $\mathbf{B}\hat{A}^{*}$ is represented as a function $A^{\bot}(a): W^{\bot} \to W^{\bot}$.

An action $a \in \hat{A}^{*}$ is consistently globally left invertible in $\cat{Set}_{\bot}$ if there exists an action $a' \in \hat{A}^{*}$ such that the following diagram commutes:
\begin{equation}
    % https://q.uiver.app/#q=WzAsNCxbMSwwLCJXXntcXGJvdH0iXSxbMywwLCJXXntcXGJvdH0iXSxbNSwwLCJXXntcXGJvdH0iXSxbMCwwLCJcXHRleHR7SW4gfSBcXGNhdHtTZXR9X3tcXGJvdH06Il0sWzAsMSwiQV57XFxib3R9KGEpIl0sWzEsMiwiQV57XFxib3R9KGEnKSJdLFswLDIsIlxcdGV4dHtpZH1fVyIsMix7ImN1cnZlIjozfV1d
\begin{tikzcd}[ampersand replacement=\&]
    {\text{In } \cat{Set}_{\bot}:} \& {W^{\bot}} \&\& {W^{\bot}} \&\& {W^{\bot}}
    \arrow["{A^{\bot}(a)}", from=1-2, to=1-4]
    \arrow["{\text{id}_W}"', curve={height=18pt}, from=1-2, to=1-6]
    \arrow["{A^{\bot}(a')}", from=1-4, to=1-6]
\end{tikzcd}
\end{equation}
which means
\begin{equation}
    A^{\bot}(a') \circ A^{\bot}(a) = \text{id}_{W^{\bot}},
\end{equation}
where $a'$ is called the global left inverse of $a$.
In other words, the morphism $A^{\bot}(a): W^{\bot} \to W^{\bot}$ is a \emph{split monomorphism} in $\cat{Set}_{\bot}$.

An action $a \in \hat{A}^{*}$ viewed as a morphism in $\mathbf{B}\hat{A}^{*}$ is consistently globally right invertible in $\cat{Set}_{\bot}$ if there exists an action $a' \in \hat{A}^{*}$ such that the following diagram commutes:
\begin{equation}
    % https://q.uiver.app/#q=WzAsNCxbMSwwLCJXXntcXGJvdH0iXSxbMywwLCJXXntcXGJvdH0iXSxbNSwwLCJXXntcXGJvdH0iXSxbMCwwLCJcXHRleHR7SW4gfSBcXGNhdHtTZXR9X3tcXGJvdH06Il0sWzAsMSwiQV57XFxib3R9KGEnKSJdLFsxLDIsIkFee1xcYm90fShhKSJdLFswLDIsIlxcdGV4dHtpZH1fVyIsMix7ImN1cnZlIjozfV1d
\begin{tikzcd}[ampersand replacement=\&]
    {\text{In } \cat{Set}_{\bot}:} \& {W^{\bot}} \&\& {W^{\bot}} \&\& {W^{\bot}}
    \arrow["{A^{\bot}(a')}", from=1-2, to=1-4]
    \arrow["{\text{id}_W}"', curve={height=18pt}, from=1-2, to=1-6]
    \arrow["{A^{\bot}(a)}", from=1-4, to=1-6]
\end{tikzcd}
\end{equation}
which means
\begin{equation}
    A^{\bot}(a) \circ A^{\bot}(a') = \text{id}_{W^{\bot}},
\end{equation}
where $a'$ is called the global right inverse of $a$.
In other words, the morphism $A^{\bot}(a): W^{\bot} \to W^{\bot}$ is a \emph{split epimorphism} in $\cat{Set}_{\bot}$.

An action $a \in \hat{A}^{*}$ viewed as a morphism in $\mathbf{B}\hat{A}^{*}$ is consistently globally fully invertible in $\cat{Set}_{\bot}$ if there exists an action $a' \in \hat{A}^{*}$ such that
\begin{align}
    & A^{\bot}(a') \circ A^{\bot}(a) = \text{id}_{W^{\bot}} \\
    & A^{\bot}(a) \circ A^{\bot}(a') = \text{id}_{W^{\bot}},
\end{align}
where $a'$ is called the global full inverse of $a$.
In other words, the morphism $A^{\bot}(a): W^{\bot} \to W^{\bot}$ is an \emph{isomorphism} in $\cat{Set}_{\bot}$.


%%%%%%%%%%%%%%%%%%%%%%%%%%%%%%%%%%%%%%%%%%%%%%%%
\paragraph{
Global consistent invertibility in $\mathbf{B}(\hat{A}^{*}/\sim)$.
}
\newthought{In $\mathbf{B}(\hat{A}^{*}/\sim)$ every morphism} represents a global action.

A morphism $[a]_{\sim}$ in $\mathbf{B}(\hat{A}^{*}/\sim)$ is consistently globally left invertible in $\mathbf{B}(\hat{A}^{*}/\sim)$ if there exists a morphism $[a']_{\sim}$ in $\mathbf{B}(\hat{A}^{*}/\sim)$ such that the following diagram commutes:
\begin{equation}
    % https://q.uiver.app/#q=WzAsNCxbMSwwLCJcXGJ1bGxldCJdLFszLDAsIlxcYnVsbGV0Il0sWzUsMCwiXFxidWxsZXQiXSxbMCwwLCJcXHRleHR7SW4gfSBcXHRleHRiZntCfShcXGhhdHtBfV57Kn0vXFxzaW0pOiJdLFswLDEsIlthXV97XFxzaW19Il0sWzEsMiwiW2EnXV97XFxzaW19Il0sWzAsMiwiW1xcdmFyZXBzaWxvbl1fe1xcc2ltfSIsMix7ImN1cnZlIjozfV1d
\begin{tikzcd}[ampersand replacement=\&]
    {\text{In } \mathbf{B}(\hat{A}^{*}/\sim):} \& \bullet \&\& \bullet \&\& \bullet
    \arrow["{[a]_{\sim}}", from=1-2, to=1-4]
    \arrow["{[\varepsilon]_{\sim}}"', curve={height=18pt}, from=1-2, to=1-6]
    \arrow["{[a']_{\sim}}", from=1-4, to=1-6]
\end{tikzcd}
\end{equation}
which means
\begin{equation}
    [a']_{\sim} \circ_{\sim} [a]_{\sim} = [\epsilon]_{\sim},
\end{equation}
where $[a']_{\sim}$ is called the global left inverse of $[a]_{\sim}$.
In other words, the morphism $[a]_{\sim}: \bullet \to \bullet$ is a \emph{split monomorphism} in $\mathbf{B}(\hat{A}^{*}/\sim)$.

A morphism $[a]_{\sim}$ in $\mathbf{B}(\hat{A}^{*}/\sim)$ is consistently globally right invertible in $\mathbf{B}(\hat{A}^{*}/\sim)$ if there exists a morphism $[a']_{\sim}$ in $\mathbf{B}(\hat{A}^{*}/\sim)$ such that the following diagram commutes:
\begin{equation}
    % https://q.uiver.app/#q=WzAsNCxbMSwwLCJcXGJ1bGxldCJdLFszLDAsIlxcYnVsbGV0Il0sWzUsMCwiXFxidWxsZXQiXSxbMCwwLCJcXHRleHR7SW4gfSBcXHRleHRiZntCfShcXGhhdHtBfV57Kn0vXFxzaW0pOiJdLFswLDEsIlthJ11fe1xcc2ltfSJdLFsxLDIsIlthXV97XFxzaW19Il0sWzAsMiwiW1xcdmFyZXBzaWxvbl1fe1xcc2ltfSIsMix7ImN1cnZlIjozfV1d
\begin{tikzcd}[ampersand replacement=\&]
    {\text{In } \mathbf{B}(\hat{A}^{*}/\sim):} \& \bullet \&\& \bullet \&\& \bullet
    \arrow["{[a']_{\sim}}", from=1-2, to=1-4]
    \arrow["{[\varepsilon]_{\sim}}"', curve={height=18pt}, from=1-2, to=1-6]
    \arrow["{[a]_{\sim}}", from=1-4, to=1-6]
\end{tikzcd}
\end{equation}
which means
\begin{equation}
    [a]_{\sim} \circ_{\sim} [a']_{\sim} = [\epsilon]_{\sim},
\end{equation}
where $[a']_{\sim}$ is called the global right inverse of $[a]_{\sim}$.
In other words, the morphism $[a]_{\sim}: \bullet \to \bullet$ is a \emph{split epimorphism} in $\mathbf{B}(\hat{A}^{*}/\sim)$.

A morphism $[a]_{\sim}$ in $\mathbf{B}(\hat{A}^{*}/\sim)$ is consistently globally fully invertible in $\mathbf{B}(\hat{A}^{*}/\sim)$ if there exists a morphism $[a']_{\sim}$ in $\mathbf{B}(\hat{A}^{*}/\sim)$ such that:
\begin{align}
    & [a']_{\sim} \circ_{\sim} [a]_{\sim} = [\epsilon]_{\sim} \\
    & [a]_{\sim} \circ_{\sim} [a']_{\sim} = [\epsilon]_{\sim},
\end{align}
where $[a']_{\sim}$ is called the global full inverse of $[a]_{\sim}$.
In other words, the morphism $[a]_{\sim}: \bullet \to \bullet$ is an \emph{isomorphism} in $\mathbf{B}(\hat{A}^{*}/\sim)$.

\whendraft{
\draftnote{red}{Inside whendraft}{}
%%%%%%%%%%%%%%%%%%%%%%%%%%%%%%%%%%%%%%%%%%%%%%%%
\paragraph{Isomorphisms as natural transformations.}


\draftnote{blue}{Info}{
\begin{enumerate}
    \item Any assignment that is defined on each object in a way that is compatible with the morphisms (i.e. any \emph{consistent global condition}) automatically yields a natural transformation.
    \item When an action’s effect is the same across all morphisms (or is preserved under every structural change in the category), then the assignment
    \begin{equation}
        c\mapsto A^{\bot}_{\sim}([a]_\sim)\big|_{F(c)}
    \end{equation}
    is natural.
    \item In the special case when $\mathcal{C}$ is the delooping of a monoid (a one-object category), every action automatically induces a natural transformation. In the multi-object case the naturality condition ensures that the property is uniformly defined across all objects. 
    \item If a global condition is consistent and functorial, it can be turned into a natural transformation. This is precisely because the naturality condition is the mathematical formalization of the notion that the condition is preserved under the action of every morphism in the category.
    \item Non-Functorial Properties: Properties that cannot be expressed as relationships between functors (e.g., ad-hoc constraints not preserved under morphisms) do not correspond to natural transformations.
    \item Non-Uniformity: Properties that vary across states violate naturality conditions.
\end{enumerate}
}

\draftnote{red}{Approach}{
I'm currently leaning towards every action inducing a natural transformation:
"
In the functor category $[\mathbf{B}\hat{A}^{*}, \cat{Set}_{\bot}]$, each action $a \in \hat{A}^{*}$ corresponds to a natural transformation
\begin{equation}
    \eta_{a}: A^{\bot} \Rightarrow A^{\bot}
\end{equation}
with components
\begin{equation}
    \eta_{a}(w) = A^{\bot}(a)(w) = a \ast^{\bot} w
\end{equation}
for $w \in W$.
\draftnote{blue}{Consider}{
Are we sure the components of $\eta_{a}: A \Rightarrow A$ are $\eta_{w}$ and not $\eta_{a}$ ?
}
"
\begin{enumerate}
    \item Is every action actually the component of a natural transformation and is that natural transformation $\eta: A^{\bot} \Rightarrow A^{\bot}_{\sim}$ ?
    \begin{enumerate}
        \item This might be true for every action in the case of $\mathbf{B}(\hat{A}^{*}/\sim)$ being a monoid, but not for the case of $\mathbf{B}(\hat{A}^{*}/\sim)$ being a multi-object small category.
        \item For $(\hat{A}^{*}/\sim)$ being a monoid is there a special case where each action induces a natural transformation or is this a consequence of all morphisms being of the form $a: \bullet \to \bullet$ in the single-object case ?
        \item "The component of the natural transformation corresponding to an action is exactly the function that $A^{\bot}_{\sim}$ assigns to that action."
    \end{enumerate}
    \item Is it the case that consistent global properties correspond to natural transformations, but properties that are not consistent and global do not ?
    \begin{enumerate}
        \item Check for multi-object case first.
        \item Do congruences induce natural transformations ?
    \end{enumerate}
\end{enumerate}
}
\draftnote{red}{End of whendraft}{}
}


%%%%%%%%%%%%%%%%%%%%%%%%%%%%%%%%%%%%%%%%%%%%%%%%
\section{
The action functor with a general category as a target
}
\draftnote{blue}{Consider}{
Move into later section about extensions then add something about embeddings (representations) in $\cat{Vect}$.
}
\draftnote{purple}{To do}{
\begin{enumerate}
    \item Check.
    \item Functorial action $\mathcal{A}^{\bot}: \mathbf{B}\hat{A}^{*} \times \mathcal{F}(\hat{\mathscr{W}}_{\mathscr{A}}^{\bot}) \to \mathcal{F}(\hat{\mathscr{W}}_{\mathscr{A}}^{\bot})$.
\end{enumerate}
}
\newthought{In categorical terms,} the totalised action effect operator $\ast^{\bot}: \hat{A}^{*} \times W^{\bot} \to W^{\bot}$ is encoded as a functor $A^{\bot}$ from the one-object category $\mathbf{B}\hat{A}^{*}$ into a target category, which up until now has been $\cat{Set}_{\bot}$.

However, we can replace $\cat{Set}_{\bot}$ with a category $\mathcal{D}$ containing an object $X$; in this case, our action effect operator $\ast^{\bot}$ becomes a morphism
\begin{equation}
    \ast^{\bot}: \hat{A}^{*} \times X \to X
\end{equation}
in $\mathcal{D}$ where $X$ is an object in $\mathcal{D}$.
The set $\mathcal{D}$ needs to have products, so that $\ast^{\bot}$ can be interpreted as a morphism in $\mathcal{D}$, and a notion of a basepoint, so that totalisation can have occurred.

From the morphism $\ast^{\bot}$ we define, for each $a \in \hat{A}^{*}$, an endomorphism
\begin{equation}
    \phi(a): X \to X
\end{equation}
\draftnote{purple}{(PS) Consider}{
Is $\phi(a): X \to X$ always constructed by setting $\phi(a)(x) := a \ast^{\bot} x$ ?
}
Combining all the $\phi(a)$, we can construct the monoid homomorphism
\begin{equation}
    \phi: \hat{A}^{*} \to \text{End}_{\mathcal{D}}(X).
\end{equation}
which we then use to construct the appropriate action functor
\begin{equation}
    A^{\bot}: \mathbf{B}\hat{A}^{*} \to \mathcal{D}
\end{equation}
by defining that it acts as follows:
\begin{enumerate}
    \item \textbf{On objects:}
    \begin{equation}
        \mathcal{A}^{\bot}(\bullet) := X.
    \end{equation}
    \item \textbf{On morphisms:}
    \begin{equation}
        A^{\bot}(a) := \phi(a) \quad \text{for all $a \in \hat{A}^{*}$}.
    \end{equation}
\end{enumerate}
$\mathcal{A}^{\bot}$ provides a representation of the monoid $\mathbf{B}\hat{A}^{*}$ in $\mathcal{D}$.

There are two options for the construction of $\phi$:
\begin{enumerate}
    \item \textbf{$X^{X}$ exists as an object in $\mathcal{D}$.}
    If $X^{X}$ exists as an object in $\mathcal{D}$ (e.g., if $\mathcal{D}$ is Cartesian closed), then we can use the canonical exponential adjunction to curry the action $\ast^{\bot}$ inside $\mathcal{D}$ as
    \begin{equation}
        \tilde{\ast}^{\bot}: \hat{A} \to X^{X},
    \end{equation}
    where $X^{X}$ is is the object that internalises the set of endomorphism of $X$.
    In general,\draftnote{purple}{(PS)}{Prove this as a footnote: take the natural isomorphism for currying, then evaluate $\text{Hom}_{\mathcal{D}}(1, X^{X})$, where $1$ is the terminal object of $\mathcal{D}$.}
    \begin{equation}
        X^{X} \cong \text{End}(X)
    \end{equation}
    and so $\tilde{\ast}^{\bot}$ and $\phi$ have equivalent data
    \begin{equation}
        \text{Hom}_{\mathcal{D}}(1, X^{X}) \cong \text{End}_{\mathcal{D}}(X).
    \end{equation}
    
    \item \textbf{$X^{X}$ does not exist as an object in $\mathcal{D}$.}
    If $\mathcal{D}$ is not Cartesian closed, then for a given object $X \in \mathcal{D}$, the exponential object $X^{X}$ might not exist in $\mathcal{D}$.
    If $X^{X}$ does not exist as an object in $\mathcal{D}$, then we cannot canonically curry $X$ directly inside $\mathcal{D}$; in other words, we cannot use currying to internalise the endomorphism object (i.e., the assignment $a \mapsto \phi(a)$) as an object in $\mathcal{D}$, and so we must construct the functor $A^{\bot}$ externally at the level of Hom-sets by defining a monoid homomorphism
    \begin{equation}
        \phi: \hat{A}^{*} \to \text{End}_{\mathcal{D}}(X).
    \end{equation}
    
    If we can fully faithfully embed $X$ into a Cartesian closed category $\mathcal{G}$ through a functor $y$    \footnote{
        If $\mathcal{D}$ is small, then we can always fully faithfully embed it into its presheaf category $\cat{Set}^{\mathcal{D}^{\text{op}}}$, which is Cartesian closed, via the Yoneda embedding:
        \begin{equation}
            y: \mathcal{D} \to \cat{Set}^{\mathcal{D}^{\text{op}}}.
        \end{equation}
    }, then we can curry the embedded object $y(X)$ in $\mathcal{G}$ that corresponds to $X$ before trying to lift the result $y(X)^{y(X)}$ back into $\mathcal{D}$\footnote{
        This is what we did to curry the action $\ast^{\bot}$ in $\cat{Set}_{\bot}$.
    }; a canonical lifting of $y(X)^{y(X)}$ back into $\mathcal{D}$ is only possible if $y(X)^{y(X)}$ is isomorphic to a representable functor in $\mathcal{D}$.

    If an appropriate embedding and lifting cannot be found then, it might be possible to explicitly construct the functor $\mathcal{A}^{\bot}: \mathbf{B}\hat{A}^{*} \to \mathcal{D}$ by separately defining the object assignment and the morphism assignment.
\end{enumerate}

\draftnote{purple}{(PS) Include}{
If $\mathcal{D}$ is anything that isn't $\mathcal{F}(\hat{\mathscr{W}}_{\mathscr{A}}^{\bot})$ then some information about the structure of $\mathcal{F}(\hat{\mathscr{W}}_{\mathscr{A}}^{\bot})$ will be lost when the composite functor $A^{\bot} \circ L^{\bot}$ is applied.
}

\draftnote{purple}{Include}{
When generalising to a category $\mathcal{D}$, if $\mathcal{D}$ lacks exponentials, then define
\begin{equation}
    \phi: \hat{A}^{*} \to \text{End}_{\mathcal{D}}(X)
\end{equation}
externally.
If $\mathcal{D}$ embeds into a Cartesian closed category $\mathcal{G}$ (e.g., via Yoneda), then curry in $\mathcal{G}$ and restrict to $\mathcal{D}$.
}


%%%%%%%%%%%%%%%%%%%%%%%%%%%%%%%%%%%%%%%%%%%%%%%%
\section{Changing perspective}
\draftnote{red}{Consider}{
\begin{itemize}
    \item Restriction categories \cite{j_r_b_cockett2002restriction}.
    \item The labelling map functor $L$ might be different for the multi-object category case.
    \item Getting single-object categories form multi-object categories
    \begin{enumerate}
        \item When $\mathbf{B}\hat{A}^{*}$ is a group, do we actually go from $\mathbf{B}\hat{A}^{*}$ as a groupoid then notice that the actions are all globally consistent and invertible (i.e., the relevant natural transformations exist); at that point we can "glue" the different world states of the groupoid together to get a group.
        \item Similar argument for monoid: When $\mathbf{B}\hat{A}^{*}$ is a group, do we actually go from $\mathbf{B}\hat{A}^{*}$ as a small category then notice that the actions are all globally consistent (i.e., the relevant natural transformations exist); at that point we can "glue" the different world states of the small category together to get a monoid.
    \end{enumerate}
\end{itemize}
}

\draftnote{red}{Structure}{
\begin{enumerate}
    \item Here we want to view the delooped category $\mathbf{B}(\hat{A}^{*}/\sim)$ as a multi-object small category that acts on $W$ (not $W^{\bot}$.
    \item Therefore the action functor $A$ should be something like
    \begin{equation}
        A: \mathbf{B}(\hat{A}^{*}/\sim) \to \cat{Set}.
    \end{equation}
    \item Isomorphisms still correspond to natural transformations when $\mathbf{B}(\hat{A}^{*}/\sim)$ has multiple objects ?
\end{enumerate}
}
\draftnote{green}{DIVIDER}{}
\draftnote{purple}{(PS) To do}{
\begin{enumerate}
    \item Outline the step-by-step used to get $A_{\sim}$ + why we can reuse previous mechanisms used on totalised structures to get 
    \begin{enumerate}
        \item Give summary diagram with all the non-totalised structures + maps.
    \end{enumerate}
    \item Link to the relevant sections on "A different perspective" in chapter 2 (?)
    \item Define $\cat{Par(Set)}$ in footnote.
    \begin{enumerate}
        \item $\cat{Par(Set)}$ as a restriction category.
        \item Change notation from $\cat{Par(Set)}$ to $\cat{Set}_{p}$ ?
    \end{enumerate}
    \item The connection with the Grothendieck construction of the partial action functor: $\mathbf{B}^{+}(\hat{A}^{*}/\sim)$ is isomorphic to the Grothendieck construction of the functor $A_{\sim}: \mathbf{B}(\hat{A}^{*}/\sim) \to \cat{Par(Set))}$.
    \item Worked example in $\cat{Set}$.
    \begin{enumerate}
        \item Come up with a single example that can be used to illustrate this perspective and the other perspective.
    \end{enumerate}
\end{enumerate}
}
\draftnote{orange}{To do (other sections)}{
\begin{enumerate}
    \item If we apply the interpretation of the equivalence relation $\sim$ on transformations to the category  $\mathcal{F}(\hat{\mathscr{W}}_{\mathscr{A}}^\bot)$ of world states, then (I think) $\mathcal{F}(\hat{\mathscr{W}}_{\mathscr{A}}^\bot)$ looks exactly like the underlying quiver $\hat{\mathscr{W}}_{\mathscr{A}}^\bot$ (but with composition between transformations ?).
    \item Preliminaries
    \begin{enumerate}
        \item Splitting idempotents - Karoubi envelope as universal construction where all idempotents split.
    \end{enumerate}
\end{enumerate}
}
%%%%%%%%%%%%%%%%%%%%%%%%%%%%%%%%%%%%%%%%%%%%%%%%
\subsection{Recap}

Previously, we took the (partial) action effect operator
\begin{equation}
  \ast: \hat{A}^{*} \times W \to W,
\end{equation}
which is a partial map, and then totalised it to give 
\begin{equation}
    \ast^{\bot}: \hat{A}^{*} \times W^{\bot} \to W^{\bot};
\end{equation}
then we categorified $\ast^{\bot}$ into the totalised action functor
\begin{equation}
    A^{\bot}: \mathbf{B}\hat{A}^{*} \to \cat{Set}_{\bot},
\end{equation}
where $\mathbf{B}\hat{A}^{*}$ is a single-object category (i.e., a monoid).
Next, we applied the equivalence relation $\sim$ by factoring through the functor $\Pi_{\sim}: \mathbf{B}\hat{A}^{*} \to \mathbf{B}(\hat{A}^{*}/\sim)$ to give the totalised action functor $A^{\bot}_{\sim}$ under $\sim$:
\begin{equation}
    A^{\bot}_{\sim}: \mathbf{B}(\hat{A}^{*}/\sim) \to \cat{Set}_{\bot},
\end{equation}
where $\mathbf{B}(\hat{A}^{*}/\sim)$ is a a single-object category (i.e., a monoid).

%%%%%%%%%%%%%%%%%%%%%%%%%%%%%%%%%%%%%%%%%%%%%%%%
\subsection{A different approach}
In this section, we take a different approach and don't totalise $\ast$; this will lead us to a different perspective on the action functor.
Without totalisation, $\ast$ becomes categorified to the action functor
\begin{equation}
    A: \mathbf{B}\hat{A}^{*} \to \cat{Par(Set)},
\end{equation}
where $\cat{Par(Set)}$ is the category of sets and partial functions.
We then apply equivalence relation $\sim$ by factoring through the functor $\Pi_{\sim}: \mathbf{B}\hat{A}^{*} \to \mathbf{B}(\hat{A}^{*}/\sim)$ to give the action functor under $\sim$
\begin{equation}
    A_{\sim}: \mathbf{B}(\hat{A}^{*}/\sim) \to \cat{Par(Set)},
\end{equation}
where, for each $a \in \hat{A}^{*}$, $A_{\sim}(a)$ is a partial function
\begin{equation}
    A_{\sim}(a): W \rightharpoonup W
\end{equation}
from $\text{dom}(A_{\sim}(a)) \subseteq W$ to $W$.

We will then reinterpret the action functor $A_{\sim}$ of the action of the monoid $\mathbf{B}(\hat{A}^{*}/\sim)$ on the category $\cat{Par(Set)}$ as the functor 
\begin{equation}
    A^{+}_{\sim}: \mathbf{B}^{+}(\hat{A}^{*}/\sim) \to \cat{Set}
\end{equation}
of the action of a multi-object category $\mathbf{B}^{+}(\hat{A}^{*}/\sim)$ on the set $W$ of world states in $\cat{Set}$; for each $[a]_{\sim} \in \hat{A}^{*}/\sim$, $A_{\sim}([a]_{\sim})$ is a partial function
\begin{equation}
    A_{\sim}([a]_{\sim}): W \rightharpoonup W
\end{equation}
from $W$ to $W$.

We will obtain the multi-object category $\mathbf{B}^{+}(\hat{A}^{*}/\sim)$ by \emph{splitting the restriction idempotents}\footnote{
    In technical terms, this involves passing $\cat{Par(Set)}$ to its Karoubi envelope, which is the universal category where all idempotents split; in other words, for a category $\mathcal{C}$, there's a functor
    \begin{equation}
        \mathcal{C} \to \text{Kar}(\mathcal{C}),
    \end{equation}
    where $\text{Kar}(\mathcal{C})$ is the Karoubi envelope of $\mathcal{C}$, and this functor is fully faithful and initial among all functors from $\mathcal{C}$ into categories where idempotents split.
} in $\cat{Par(Set)}$; this replaces the single object $W$ in $\cat{Par(Set)}$ by its subobjects that are the domains on which the partial maps $A_{\sim}([a]_{\sim})$ are defined.
This process lifts the action functor to the \emph{split action functor}
\begin{equation}
    A^{+}_{\sim}: \mathbf{B}^{+}(\hat{A}^{*}/\sim) \to \cat{Set},
\end{equation}
where, for each $a \in \hat{A}^{*}/\sim$, $A^{+}_{\sim}([a]_{\sim})$ is a total function.

%%%%%%%%%%%%%%%%%%%%%%%%%%%%%%%%%%%%%%%%%%%%%%%%
\paragraph{Splitting the restriction idempotents.}

Each partial function $f: W \rightharpoonup W$ in $\cat{Par(Set)}$ comes with a \emph{restriction idempotent}\footnote{
    A \emph{restriction idempotent} $\overline{f}: A \rightharpoonup A$ is a partial identity function on $A$ that is defined only on the domain of $f$:
    \begin{equation}
        \overline{f}(a) = 
        \begin{cases}
            a & \text{if } a \in \operatorname{dom}(f) \\
            \text{undefined} & \text{otherwise}
        \end{cases}
    \end{equation}
}
\begin{equation}
    \overline{f}: W \to W,
\end{equation}
which is the partial identity on $\operatorname{dom}(f) \subseteq W$.

To split these idempotents\footnote{
    An idempotent $e : X \to X$ \emph{splits} if there exists:
    \begin{itemize}
      \item An object $Y$,
      \item A morphism $r : X \to Y$ \textbf{(retraction)},
      \item A morphism $s : Y \to X$ \textbf{(section)},
    \end{itemize}
    such that:
    \begin{align}
        & e = s \circ r \quad &\text{(Factorization)} \\
        & r \circ s = \mathrm{id}_Y \quad &\text{(Y is a retract of X)}.
    \end{align}
}, we pass to the Karoubi envelope of $\cat{Par(Set)}$; this process replaces the object $W$ by a collection of pairs $(W, e)$
\begin{equation}
    W \mapsto (W, e),
\end{equation}
where $e: W \to W$ is a restriction idempotent that splits as
\begin{align}
    & e = i \circ r \\
    & r \circ i = \text{id}_{\text{dom}(f)},
\end{align}
with $i : \text{dom}(f) \hookrightarrow W$ and $r : W \to \text{dom}(f)$.
This means the single object $W$ in $\cat{Par(Set)}$ splits into the collection of its subobjects given by
\begin{equation}
    \{\text{dom}([a]_{\sim}\ast(-)) \mid [a]_{\sim} \in \hat{A}^{*}/\sim\}.
\end{equation}

 Using the universal property of the Karoubi envelope, the functor
\begin{equation}
    A_{\sim} : \mathbf{B}(\hat{A}^{*}/\sim) \to \cat{Par(Set)}
\end{equation}
factorises through this splitting, to give the unique functor
\begin{equation}
    A^{+}_{\sim}: \mathbf{B}^{+}(\hat{A}^{*}/\sim) \to \cat{Set}
\end{equation}
as shown in the following diagram:
\begin{equation}
    % https://q.uiver.app/#q=WzAsNCxbMCwwLCJcXG1hdGhiZntCfShcXGhhdHtBfV4qL3tcXHNpbX0pIl0sWzIsMCwiXFxjYXR7UGFyKFNldCl9Il0sWzAsMiwiXFxtYXRoYmZ7Qn1eKyhcXGhhdHtBfV4qL3tcXHNpbX0pIl0sWzIsMiwiXFxjYXR7U2V0fSJdLFswLDEsIkFfXFxzaW0iXSxbMCwyLCJcXGlvdGEiLDJdLFsxLDMsIlxcbWF0aHJte0thcn0iXSxbMiwzLCJcXGV4aXN0cyAhIFxcOyBBXitfXFxzaW0iLDIseyJzdHlsZSI6eyJib2R5Ijp7Im5hbWUiOiJkYXNoZWQifX19XV0=
\begin{tikzcd}[ampersand replacement=\&]
    {\mathbf{B}(\hat{A}^*/{\sim})} \&\& {\cat{Par(Set)}} \\
    \\
    {\mathbf{B}^+(\hat{A}^*/{\sim})} \&\& {\cat{Set}}
    \arrow["{A_\sim}", from=1-1, to=1-3]
    \arrow["\iota"', from=1-1, to=3-1]
    \arrow["{\mathrm{Kar}}", from=1-3, to=3-3]
    \arrow["{\exists ! \; A^+_\sim}"', dashed, from=3-1, to=3-3]
\end{tikzcd}
\end{equation}
such that
\begin{equation}
    A^{+}_{\sim} \circ \iota = \text{Kar} \circ A_{\sim}
\end{equation}
where $\mathbf{B}^{+}(\hat{A}^{*}/\sim)$ is a multi-object category whose objects correspond to the subobjects of $W$ (i.e., the domains of definition of the partial functions $A_{\sim}([a]_{\sim})$), $\iota$ is the subcategory inclusion functor that embeds the single-object category $\mathbf{B}(\hat{A}^{*}/\sim)$ into the multi-object category $\mathbf{B}^{+}(\hat{A}^{*}/\sim)$, and $\text{Kar}$ is the Karoubi envelope construction functor.

\draftnote{blue}{To do}{
    Give the objects and morphisms of $\mathbf{B}^{+}(\hat{A}^{*}/\sim)$ explicitly.
}

The split action functor $A^{+}_{\sim}$ acts as follows:
\begin{enumerate}
    \item \textbf{On objects:}
    For each object $X \subseteq W$, where $X$ is a domain of definition,
    \begin{equation}
        A^{+}_{\sim}(X) = X.
    \end{equation}
    \item \textbf{On morphisms:}
    For each morphism corresponding to the restriction of the partial function $[a]_{\sim} \ast (-)$ to its domain $X$,
    \begin{align}
        & A^{+}_{\sim}([a]_{\sim}): X \to X' \quad \text{such that} \\
        & A^{+}_{\sim}([a]_{\sim})(d) = [a]_{\sim} \ast d,
    \end{align}
    where $A^{+}_{\sim}([a]_{\sim})$ is a total function on $X$.
\end{enumerate}

\draftnote{blue}{Consider}{
If we apply the forgetful functor $U$ to $A^{\bot}_{\sim}$ to give $A^{\bot}_{\sim}$ then we can have both perspective represented as embedding in $\cat{Set}$.
Can we form an adjunction between the perspectives ?
}
\draftnote{blue}{Include}{
Footnote about how the number of subobjects of $W$ is not necessarily equal to the number of world states in $W$ (e.g., there could be multiple world states on which all actions are defined; these world states would all be in the same subobject of $W$).
}





%%%%%%%%%%%%%%%%%%%%%%%%%%%%%%%%%%%%%%%%%%%%%%%%

%%%%%%%%%%%%%%%%%%%%%%%%%%%%%%%%%%%%%%%%%%%%%%%%
% %%%%%%%%%%%%%%%%%%%%%%%%%%%%%%%%%%%%%%%%%%%%%%%%
\chapter{Further category theory}
%%%%%%%%%%%%%%%%%%%%%%%%%%%%%%%%%%%%%%%%%%%%%%%%
\section{Action labelling map}
%%%%%%%%%%%%%%%%%%%%%%%%%%%%%%%%%%%%%%%%%%%%%%%%
\subsection{Action labelling maps as a pushout}

Can we view the gluing of the world states in $\mathcal{F}(\hat{\mathscr{W}}_{\mathscr{A}}^{\bot})$ through the action labelling functor $L^{\bot}$ as a coequalizer.
The universal property of the coequalizer states that, for any category $\mathcal{C}$ and any functor $H: \mathcal{F}(\hat{\mathscr{W}}_{\mathscr{A}}^{\bot}) \to \mathcal{C}$ that identifies all objects (i.e., $H \circ F = H \circ G$), there is a unique functor $H': \textbf{B}\hat{A}^{*} \to \mathcal{C}$ with
\begin{equation}
    H' \circ q = H.
\end{equation}
The labelling functor $L^{\bot}$ is exactly such a functor $H$ since it sends every object in $\mathcal{F}(\hat{\mathscr{W}}_{\mathscr{A}}^{\bot})$ to the unique object $\bullet$ of $\textbf{B}\hat{A}^{*}$, and so we can identify $\textbf{B}\hat{A}^{*}$ as the coequalizer (i.e., the quotient) of $\mathcal{F}(\hat{\mathscr{W}}_{\mathscr{A}}^{\bot})$ by the equivalence that identifies all objects (i.e., $w \sim w'$ for all $w,w' \in W^{\bot}$ and relates morphisms accord to their labels (i.e., $f \sim g \iff L^{\bot}(f) = L^{\bot}(g)$).
As a diagram:
\begin{equation}
    % https://q.uiver.app/#q=WzAsOSxbMCwwLCJcXG1hdGhjYWx7SX0iXSxbMiwwLCJcXG1hdGhjYWx7Rn0oXFxoYXR7XFxtYXRoc2Nye1d9fV97XFxtYXRoc2Nye0F9fV57XFxib3R9KSJdLFszLDFdLFs0LDFdLFswLDIsIlxcbWF0aGNhbHtGfShcXGhhdHtcXG1hdGhzY3J7V319X3tcXG1hdGhzY3J7QX19XntcXGJvdH0pIl0sWzUsMiwiXFxtYXRoY2Fse0Z9KFxcaGF0e1xcbWF0aHNjcntXfX1fe1xcbWF0aHNjcntBfX1ee1xcYm90fSkiXSxbNSwwLCJcXG1hdGhjYWx7SX0iXSxbNywwLCJcXG1hdGhjYWx7Rn0oXFxoYXR7XFxtYXRoc2Nye1d9fV97XFxtYXRoc2Nye0F9fV57XFxib3R9KSJdLFs3LDIsIlxcdGV4dGJme0J9XFxoYXR7QX1eeyp9Il0sWzAsMSwicF8xIiwwLHsib2Zmc2V0IjotMX1dLFsyLDMsIlxcdGV4dHtwdXNob3V0fSJdLFswLDQsInBfezJ9IiwyLHsib2Zmc2V0IjoxfV0sWzYsNSwicF97Mn0iLDJdLFs2LDcsInBfezF9Il0sWzcsOCwiTF57XFxib3R9Il0sWzUsOCwiTF57XFxib3R9IiwyXV0=
\begin{tikzcd}[ampersand replacement=\&]
    {\mathcal{I}} \&\& {\mathcal{F}(\hat{\mathscr{W}}_{\mathscr{A}}^{\bot})} \&\&\& {\mathcal{I}} \&\& {\mathcal{F}(\hat{\mathscr{W}}_{\mathscr{A}}^{\bot})} \\
    \&\&\& {} \& {} \\
    {\mathcal{F}(\hat{\mathscr{W}}_{\mathscr{A}}^{\bot})} \&\&\&\&\& {\mathcal{F}(\hat{\mathscr{W}}_{\mathscr{A}}^{\bot})} \&\& {\textbf{B}\hat{A}^{*}}
    \arrow["{p_1}", shift left, from=1-1, to=1-3]
    \arrow["{p_{2}}"', shift right, from=1-1, to=3-1]
    \arrow["{p_{1}}", from=1-6, to=1-8]
    \arrow["{p_{2}}"', from=1-6, to=3-6]
    \arrow["{L^{\bot}}", from=1-8, to=3-8]
    \arrow["{\text{pushout}}", from=2-4, to=2-5]
    \arrow["{L^{\bot}}"', from=3-6, to=3-8]
\end{tikzcd}
\end{equation}
\draftnote{blue}{Consider}{Does this perspective mean that we can construct $A^{\bot}$ as a universal map from $\textbf{B}\hat{A}^{*}$ to $\mathcal{D}$ from the universal property of the coequalizer.}






%%%%%%%%%%%%%%%%%%%%%%%%%%%%%%%%%%%%%%%%%%%%%%%%
\subsection{Pullback Diagram for Recovering $\ast^{\bot}$:}

\begin{equation}
\begin{tikzcd}[row sep=large, column sep=large]
\mathcal{P} \ar[r, "p_2"] \ar[d, "p_1"'] \ar[dr, phantom, "\lrcorner", very near start] & \mathcal{F}(\hat{\mathscr{W}}_{\mathscr{A}}^{\bot}) \ar[d, "L^{\bot}"] \\
\mathbf{B}\hat{A}^* \times \mathcal{F}(\hat{\mathscr{W}}_{\mathscr{A}}^{\bot}) \ar[r, "\pi_2"'] & \mathbf{B}\hat{A}^*
\end{tikzcd}
\end{equation}
This is saying that the totalised action effect $\ast^{\bot}$ can be recovered by pulling back along the labelling functor $L^{\bot}$.
The pullback object $\mathcal{P}$ collects exactly those pairs $(a,d)$ where $a$ is the label of $d$ (i.e., $l(d) = a$).

%%%%%%%%%%%%%%%%%%%%%%%%%%%%%%%%%%%%%%%%%%%%%%%%
\section{Translating from the world to the agent's representation (category theory)}
\draftnote{blue}{Include}{
\begin{enumerate}
    \item Can we build an adjunction from the category of worlds (not the category of world states) to the category of representations ?
    \item What structures can we bring through the adjunction and what structures can we not ?
    \item What happens to the adjunction when we add noise to the sensors ?
    \begin{enumerate}
        \item I think a random "vector" gets added to the components of the $\eta$ and $\epsilon$ natural transformations.
    \end{enumerate}
\end{enumerate}
}


%%%%%%%%%%%%%%%%%%%%%%%%%%%%%%%%%%%%%%%%%%%%%%%%
\section{Translating between different representations}
\draftnote{blue}{Consider}{
Can we use adjunctions to transfer between different representations ?
\begin{enumerate}
    \item Transfer between different types of representations of the same world with the same learning algorithm (e.g., vector space vs set).
    \item Transfer between different representations that are of the same type (e.g., representations in $\cat{Vect}$) and are learning the same world but using different inference processes.
\end{enumerate}
}
Any category $\mathcal{D}$ with the required structure of products and a basepoint-like object (e.g., $\cat{Set}_{\bot}$) can be used in the following diagram
\begin{equation}
\begin{tikzcd}[row sep=large, column sep=large]
\mathbf{B}\hat{A}^* \times \mathcal{F}(\hat{\mathscr{W}}_{\mathscr{A}}^{\bot}) \ar[r, "\mathcal{A}^\bot"] \ar[d, "U \times U"'] & \mathcal{F}(\hat{\mathscr{W}}_{\mathscr{A}}^{\bot}) \ar[d, "U"] \\
\mathbf{B}\hat{A}^* \times \mathcal{D} \ar[r, "A^\bot"] & \mathcal{D}
\end{tikzcd}
\end{equation}
where
\begin{equation}
    \mathcal{A}^{\bot}: \mathbf{B}\hat{A}^* \times \mathcal{F}(\hat{\mathscr{W}}_{\mathscr{A}}^{\bot}) \to \mathcal{F}(\hat{\mathscr{W}}_{\mathscr{A}}^{\bot})
\end{equation}
is the internal action of the free monoid on the world, $U$ is a forgetful functor that forgets the extra structure in $\mathcal{F}(\hat{\mathscr{W}}_{\mathscr{A}}^{\bot})$ to give $\mathcal{D}$, and
\begin{equation}
    A^{\bot}: \mathbf{B}\hat{A}^* \times \mathcal{D} \to \mathcal{D}
\end{equation}
is the $\mathcal{D}$-level interpretation of the action.

\draftnote{purple}{(PS) Consider}{
If $\mathcal{D}$ has enough "free" objects (such as when $\mathcal{D}$ is chosen to be something like $\cat{Set}_{\bot}$), we can define a left adjoint function $F$ so that $U$ is part of an adjunction.
}

What we really want to know is whether it is possible to build an adjunction so we can transfer between different representations of the internal action $\mathcal{A}^{\bot}: \mathbf{B}\hat{A}^* \times \mathcal{F}(\hat{\mathscr{W}}_{\mathscr{A}}^{\bot}) \to \mathcal{F}(\hat{\mathscr{W}}_{\mathscr{A}}^{\bot})$ (e.g., can we construct an adjunction so that we can move between action functors $A^{\bot}: \mathbf{B}\hat{A}^* \times \mathcal{D} \to \mathcal{D}$ where, for example, $\mathcal{D} = \cat{Set}$ and $\mathcal{D} = \cat{Vect}$)?

Consider the internal action
\begin{equation}
    \mathcal{A}^{\bot}: \mathbf{B}\hat{A}^* \times \mathcal{F}(\hat{\mathscr{W}}_{\mathscr{A}}^{\bot}) \to \mathcal{F}(\hat{\mathscr{W}}_{\mathscr{A}}^{\bot}).
\end{equation}
Suppose we have an action functor
\begin{equation}
    A^{\bot}_{\mathcal{D}}: \textbf{B}\hat{A}^{*} \times \mathcal{D} \to \mathcal{D},
\end{equation}
where $\mathcal{D}$ hosts a representation of the internal action $\mathcal{A}^{\bot}$, and an action functor
\begin{equation}
    A^{\bot}_{\mathcal{E}}: \textbf{B}\hat{A}^{*} \times \mathcal{E} \to \mathcal{E},
\end{equation}
where $\mathcal{E}$ hosts a representation of the action $\mathcal{A}^{\bot}$.

If there exists a free-forgetful adjunction
\begin{equation}
    F: \mathcal{D} \leftrightarrows \mathcal{E} :U,
\end{equation}
then we can often transfer the action from $\mathcal{D}$ to $\mathcal{E}$:
\begin{equation}
    % https://q.uiver.app/#q=WzAsNixbMiwwLCJcXHRleHR7SW4gfSBcXG1hdGhjYWx7RX06Il0sWzIsMSwieCJdLFswLDAsIlxcdGV4dHtJbiB9IFxcbWF0aGNhbHtEfToiXSxbMCwxLCJVKHgpIl0sWzAsMywiQV57XFxib3R9X3tcXG1hdGhjYWx7RH19KGEsIFUoeCkpIl0sWzIsMywiRihBXntcXGJvdH1fe1xcbWF0aGNhbHtEfX0oYSwgVSh4KSkpIl0sWzEsMywiVSIsMSx7ImNvbG91ciI6WzEyMCw2MCw2MF19LFsxMjAsNjAsNjAsMV1dLFszLDQsIkFee1xcYm90fV97XFxtYXRoY2Fse0R9fShhLCB4KSJdLFs0LDUsIkYiLDEseyJjb2xvdXIiOlswLDYwLDYwXX0sWzAsNjAsNjAsMV1dLFs1LDEsIlxcZXBzaWxvbl97eH0iXSxbMSw1LCJBXntcXGJvdH1fe1xcbWF0aGNhbHtFfX0oYSx4KSIsMCx7Im9mZnNldCI6LTMsInN0eWxlIjp7ImJvZHkiOnsibmFtZSI6ImRhc2hlZCJ9fX1dXQ==
\begin{tikzcd}[ampersand replacement=\&]
    {\text{In } \mathcal{D}:} \&\& {\text{In } \mathcal{E}:} \\
    {U(x)} \&\& x \\
    \\
    {A^{\bot}_{\mathcal{D}}(a, U(x))} \&\& {F(A^{\bot}_{\mathcal{D}}(a, U(x)))}
    \arrow["{A^{\bot}_{\mathcal{D}}(a, x)}", from=2-1, to=4-1]
    \arrow["U"{description}, color={rgb,255:red,92;green,214;blue,92}, from=2-3, to=2-1]
    \arrow["{A^{\bot}_{\mathcal{E}}(a,x)}", shift left=3, dashed, from=2-3, to=4-3]
    \arrow["F"{description}, color={rgb,255:red,214;green,92;blue,92}, from=4-1, to=4-3]
    \arrow["{\epsilon_{x}}", from=4-3, to=2-3]
\end{tikzcd}
\end{equation}
where we have derived the new functor
\begin{equation}
\begin{aligned}
    & A^{\bot}_{\mathcal{E}}: \textbf{B}\hat{A}^{*} \times \mathcal{E} \to \mathcal{E} \quad \text{such that} \\
    & A^{\bot}_{\mathcal{E}}(a,x) := F\big(A^{\bot}_{\mathcal{D}}(a, U(x))\big)
\end{aligned}
\end{equation}
Because $F$ is left adjoint to $U$, this procedure preserves the algebraic structure of the action \draftnote{purple}{(PS) To do}{Explain this.}.


%%%%%%%%%%%%%%%%%%%%%%%%%%%%%%%%%%%%%%%%%%%%%%%%
\section{Enriched category theory stuff - come back to this}
% https://chat.deepseek.com/a/chat/s/0197bb5e-6c47-4c1e-9899-bec91faa1ff2
%%%%%%%%%%%%%%%%%%%%%%%%%%%%%%%%%%%%%%%%%%%%%%%%
\subsection{Monoidal categories}


\paragraph{Delooping of a monoid $M$.}
A category with one object whose morphisms form the monoid $M$.
This is the category version of the monoid $M$; in other words, $M$ is encoded as the morphisms of a one-object category.
Formally, consider a monoid $(M, \cdot, e)$.
The category $\textbf{B}M$ consists of
\begin{enumerate}
    \item \textbf{Objects:}
    A single object $\bullet$.
    \begin{equation}
        \text{Ob}(\textbf{B}M) = \{ \bullet \}
    \end{equation}

    \item \textbf{Morphisms:}
    \begin{equation}
        \text{Hom}_{\textbf{B}M}(\bullet, \bullet) = M.
    \end{equation}
    Composition of morphisms is given by the monoid operation $\cdot$ in $M$.
\end{enumerate}


\paragraph{Monoidal category.}
A category $\mathcal{C}$ with an internal tensor product (binary operations on objects and morphisms), a unit object, associativity constraints etc...
This is a category equipped with monoid-like structure.
Formally a monoidal category $(\mathcal{C}, \otimes, I)$ is a category where
\begin{enumerate}
    \item \textbf{Objects:}
    There is, in general, more than one object.

    \item A bifunctor call the tensor product
    \begin{equation}
        \otimes: \mathcal{C} \times \mathcal{C} \to \mathcal{C}
    \end{equation}
    which acts on objects and morphisms
    \begin{enumerate}
        \item \textbf{Objects:}
        \begin{equation}
            A,B \mapsto A \otimes B
        \end{equation}
        \item \textbf{Morphisms:}
        \begin{equation}
            \big(f:A \to A', g: B \to B' \big) \mapsto \big(f \otimes g: A \otimes B \to A' \otimes B' \big).
        \end{equation}
        NB: not necessarily the Cartesian product (it's the Cartesian product when the objects are sets).
    \end{enumerate}

    \item A unit object
    \begin{equation}
        I \in \mathcal{C}.
    \end{equation}
    which is the identity for the tensor product.
    The following natural isomorphisms (unitors) exist:
    \begin{align}
        & \lambda_{A}: I \otimes A \to A \\
        & \rho_{A}: A \otimes I \to A
    \end{align}

    \item \textbf{Associator.}
    Since $\otimes$ is not required to be strictly associative, we add a natural isomorphism
    \begin{equation}
        \alpha_{A,B,C}: (A \otimes B) \otimes C \xrightarrow{\cong} A \otimes (B \otimes C)
    \end{equation}
    i.e., different bracketing gives canonically isomorphic objects, but not necessarily equal objects.

    \item Mac Lane's coherence theorem - pentagon identity, and triangle identity.
\end{enumerate}

For a strict monoidal category, the associativity and unit conditions are equally equal not just isomorphic (no unitors or associator isomorphisms needed).

Mac Lane’s strictification theorem: Every (weak) monoidal category is monoidally equivalent to a strict one.
Therefore we can assume strictness for the purposes of reasoning.

%%%%%%%%%%%%%%%%%%%%%%%%%%%%%%%%%%%%%%%%%%%%%%%%
\paragraph{Delooped monoid}
For $a, b \in \hat{A}^{*}$, the monoid multiplication in $\textbf{B}\hat{A}^{*}$ satisfies
\begin{center}  
\begin{tikzcd}
    \bullet \ar[r, "a"] \ar[rr, "a \circ b", bend right=30] & \bullet \ar[r, "b"] & \bullet
\end{tikzcd}
\end{center}


\paragraph{Horizontal composition.}
\draftnote{red}{To do}{
\begin{enumerate}
    \item I think $\mathcal{A}(a): C^{\bot} \to C^{\bot}$ basically shifts the structure around making some world states closer and some further away.
    For an action $a \in \hat{A}^{*}$, $\mathcal{A}(a)$ shifts $w \mapsto a \ast w$ for all $w \in W$.
    \item Does horizontal composition do something with the labels too ?
\end{enumerate}
}


For an action $a \in \hat{A}^{*}$ as an endofunctor $\mathcal{A}(a): C^{\bot} \to C^{\bot}$, and a transformation $d \in C^{\bot}$ in $d: w \to w'$, the horizontal composition $a \ast d$ is the morphism $a \ast w \xrightarrow{a \ast d} a \ast w'$:
\begin{equation}
    \mathcal{A}(a): (w \xrightarrow{d} w') \mapsto (a \ast w \xrightarrow{a \ast d} a \ast w')
\end{equation}
OR
\begin{equation}
    \mathcal{A}(a): (d: w \to w') \mapsto (a \ast d: a \ast w \to a \ast w')
\end{equation}
\begin{equation}
    \mathcal{A}(a): \Big( w \xrightarrow{d} w' \Big) \mapsto \Big( a \ast w \xrightarrow{a \ast d} a \ast w' \Big).
\end{equation}


Horizontal composition is the application of $\mathcal{A}(a)$ to the morphism $d$.
The $\ast$ in $a \ast d$ is the horizontal composition (i.e., the functorial action on morphisms), while $a \ast w$ is the functorial action on objects.
This is analogous to the "whiskering" operation in 2-category theory.


\paragraph{Action functor.}
We have a functor
\begin{equation}
    \mathcal{A}: \textbf{B}\hat{A}^{*} \to \textbf{End}(C^{\bot})
\end{equation}
that acts as follows
\begin{enumerate}
    \item \textbf{Objects:}
    The single object of $\textbf{B}\hat{A}^{*}$ is mapped to the identity endofunctor
    \begin{equation}
        \mathcal{A}(\bullet) = \text{Id}_{C}
    \end{equation}

    \item \textbf{Morphisms:}
    Each $a \in \hat{A}^{*}$ maps to an endofunctor $\mathcal{A}(a): C \to C$ satisfying the properties
    \begin{enumerate}
        \item \textbf{Identity.}
        \begin{equation}
            \mathcal{A}(\varepsilon) = \mathrm{Id}_{\mathcal{C}^{\bot}}
        \end{equation}
        \item \textbf{Composition.}
        \begin{equation}
            \mathcal{A}(a' \circ a) = \mathcal{A}(a') \circ \mathcal{A}(a)
        \end{equation}
    \end{enumerate}
\end{enumerate}


The category $\textbf{End}(C^{\bot})$ consists of
\begin{enumerate}
    \item \textbf{Objects:}
    The endofunctors
    \begin{equation}
        F : C^{\bot} \to C^{\bot}
    \end{equation}
    For example $\mathcal{A}(a)$ for $a \in \hat{A}^{*}$.

    \item \textbf{Morphisms:}
    Natural transformations
    \begin{equation}
        \eta: F \Rightarrow G
    \end{equation}
    between endofunctors $F$ and $G$.
    For example, the identity natural transformation $\text{Id}_{C} \Rightarrow \text{Id}_{C}$.
\end{enumerate}


Each of the endofunctors $\mathcal{A}(a): C \to C$ act as follows:
\begin{enumerate}
    \item \textbf{Objects:}
    For $a \in \hat{A}^{*}$, $\mathcal{A}(a)$ maps objects in $C^{\bot}$ via
    \begin{equation}
        \mathcal{A}(a)(w) = a \ast w \quad \text{and} \quad \mathcal{A}(a)(\bot) = \bot
    \end{equation}

    \item \textbf{Morphisms:}
    For $d: w \to w'$ in $C^{\bot}$, $\mathcal{A}(a)(d) = a \ast d$ is defined by
    \begin{center}
    \begin{tikzcd}
        w \ar[r, "d"] \ar[d, "a \ast (-)"'] & w' \ar[d, "a \ast (-)"] \\
        a \ast w \ar[r, "a \ast d"] & a \ast w'
    \end{tikzcd}
    \end{center}
    This diagram commutes by the definition of horizontal composition.
\end{enumerate}



\draftnote{red}{DIVIDER}{}


\begin{equation}
\mathcal{A}(a)(w) = a \ast w.
\end{equation}

So the morphisms $\mathcal{A}(a)(w)$ act as $\mathcal{A}(a)(w) = a \ast w$.





\begin{equation}
\mathcal{A}(a)(w) = a \ast w \quad \text{and} \quad \mathcal{A}(a)(\bot) = \bot.
\end{equation}

\begin{equation}
    \mathcal{A}(a)(d) = a \ast d \quad \text{for all $d: w \to w'$ in $C$}
\end{equation}










\paragraph{Natural transformations from labelling.}
The labelling system induces natural transformations (for each $a \in \hat{A}^{*}$?)
\begin{equation}
    \alpha_{a}: \text{Id}_{\mathcal{C}^{\bot}} \Rightarrow \mathcal{A}(a)
\end{equation}
with components
\begin{equation}
    \alpha_{a}(w): w \to a \ast w.
\end{equation}
The naturality condition requires the following diagram to commute for all $d: w \to w'$
\begin{center}
\begin{tikzcd}
    w \ar[r, "\alpha_a(w)"] \ar[d, "d"'] & a \ast w \ar[d, "a \ast d"] \\
    w' \ar[r, "\alpha_a(w')"'] & a \ast w'
\end{tikzcd}
\end{center}
where $\alpha_{a}(w)$ is the morphism labelled by $a$ from $w$ to $a \ast w$.


For each $w \in C$, $\alpha_{a}(w)$ is the unique morphism $d_{a} \in D_{A}$ labelled by $a$ such that
\begin{equation}
    d_a: w \to a \ast w \quad \text{with} \quad l(d_a) = a.
\end{equation}

The naturality condition ensures compatibility: for any $d: w \to w'$, the diagram
\begin{center}
\begin{tikzcd}
w \ar[r, "d_a"] \ar[d, "d"'] & a \ast w \ar[d, "a \ast d"] \\
w' \ar[r, "d_a'"'] & a \ast w'
\end{tikzcd}
\end{center}
commutes; in other words
\begin{equation}
    (a \ast d) \circ d_{a} = d'_{a} \circ d.
\end{equation}


The natural transformation $\alpha_{a}$ encodes the deterministic effect of the action $a \in \hat{A}^{*}$ across all world states; each component $\alpha_{a}(w)$ of $\alpha_{a}$ encodes the effect of $a$ on the world state $w \in W$.



%%%%%%%%%%%%%%%%%%%%%%%%%%%%%%%%%%%%%%%%%%%%%%%%
\section{Other}
\draftnote{red}{}{
\begin{enumerate}
    \item \textbf{Category theory on reachable subworlds.}
    \begin{enumerate}
        \item Reachable subworld is an object in the category of categories (or category of worlds (which is a subcategory of the category of categories)?).
        \item Possible construction:
        \begin{enumerate}
            \item Strict endofunctors on reachable subworld categories for reversible actions.
            \item Non-strict endofunctors on (infinite) reachable subworld categories that are action-homogeneous - does this mean we can tell if a world is action-homogeneous by just looking at its reachable subworld category ?
            \item Functors for irreversible actions.
        \end{enumerate}
        \item How to deal with reachability in a disjoint world ?
    \end{enumerate}
\end{enumerate}
}


\draftnote{red}{Consider}{
\begin{enumerate}
    \item Would taking the dual give us anything ?
    \begin{enumerate}
        \item Doesn't make sense to take the dual of worlds $\mathcal{F}(\hat{\mathscr{W}}_{\mathscr{A}})$.
    \end{enumerate}
    \item Other equivalences
    \begin{enumerate}
        \item A congruence relation $\sim$ on a category $\mathcal{C}$ is a relation on the morphism of $\mathcal{C}$ that respects domain and codomain (only compares morphisms with the same source and target), and is compatible with composition:
        \begin{equation}
            f \sim f' \text{ and } g \sim g' \implies g \circ f \sim g' \circ f'.
        \end{equation}
    \end{enumerate}
\end{enumerate}
}
%%%%%%%%%%%%%%%%%%%%%%%%%%%%%%%%%%%%%%%%%%%%%%%%

%%%%%%%%%%%%%%%%%%%%%%%%%%%%%%%%%%%%%%%%%%%%%%%%
\chapter{Category theory (OLD)}
\input{7CategoryTheoryI/old}
%%%%%%%%%%%%%%%%%%%%%%%%%%%%%%%%%%%%%%%%%%%%%%%%
