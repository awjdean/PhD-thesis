\section{
A local equivalence
}
\draftnote{blue}{Consider}{
Put into own chapter?
Beyond SBDRL II: local algebras
}
\draftnote{green}{Include}{
\begin{enumerate}
    \item Define $\sim_{w}$.
    \item Proof: If $a \sim_{w} a'$ for all $w \in W$, then $a \sim a'$.
    \item Properties of $\circ_{\sim_{w}}$.
    \item General properties of $(\hat{A}^{*}, \circ_{\sim_{w}})$.
\end{enumerate}
}




%%%%%%%%%%%%%%%%%%%%%%%%%%%%%%%%%%%%%%%%%%%%%
\section{Local algebra algorithm}
\draftnote{green}{Include}{
\begin{enumerate}
    \item Proof that the algorithm halts.
        \item Proof that when the algorithm halts it contains the complete algebra.
        \item (?) Put the full algorithm in the appendices.
\end{enumerate}
}

The latest version of the code for this section can be found at
\begin{center}
\url{https://github.com/awjdean/CayleyTableGeneration}
\end{center}

%%%%%%%%%%%%%%%%%%%%%%%%%%%%%%%%%%%%%%%%%%%%%
\subsection{Generating the equivalence classes}

To generate the local algebra, we altered algorithm \ref{alg:GenerateEquivClasses} to use a different $\Call{ComputeActionFunction}$ function.
Since our $\Call{LocalComputeActionFunction}$ requires an additional initial state parameter $w^{*}$, we must also feed that parameter into $\Call{GenerateEquivClasses}$ and $\Call{ProcessCandidate}$.
Other than those changes, the algorithm for generating the equivalence classes for the local algebra is the same as for the global algebra.

\begin{algorithm}[H]
	\caption{
		Compute the part of the action function $f_{a}: W \to W$ that sends $w^{*} \mapsto a \ast w^{*}$.
	}
	\hrulefill
	\begin{algorithmic}[1]
		\Procedure{LocalComputeActionFunction}{$a$, \; $\mathscr{W}$, \; $w^{*}$}
		\State $f_{a} \gets (\emptyset \to \emptyset)$
		\State $w_{a} \gets$ \Call{GenerateActionOutcome}{$a$, \; $w^{*}$, \; $\hat{\ast}$}
		\State $f_{a} \gets f_{a} \cup f_{a}'$ where $f_{a}': \{w^{*}\} \to \{w_{a}\}$ such that $f_{a}'(w^{*}) = w_{a}$
		\State \Return $f_{a}$
		\EndProcedure
	\end{algorithmic}
\end{algorithm}

%%%%%%%%%%%%%%%%%%%%%%%%%%%%%%%%%%%%%%%%%%%%%
\subsection{Generating the Cayley table}

To generate the local Cayley table, we use algorithm \ref{alg:GenerateCayley} with a different $\Call{ComputeCompositionActionFunction}$ function.

\begin{algorithm}[H]
	\caption{
		Compute the action function for the combination $l_{L} \circ l_{R}$ using \Call{LocalComputeActionFunction}.
	}
	\hrulefill
	\begin{algorithmic}[1]
		\Procedure{ComputeCompositionActionFunction}{$l_{L}$, \; $l_{R}$, \; $\mathcal{T}$, \; $\rho$, \; $\mathscr{W}$, \; $w^{*}$}
		      \State $a \gets \operatorname{Combine}(l_{L}, \; l_{R})$
                \State $f_{a} \gets$ \Call{LocalComputeActionFunction}{$a$, \; $\mathscr{W}$, \; $w^{*}$}
                \State \Return $f_{a}$
		\EndProcedure
	\end{algorithmic}
\end{algorithm}



%%%%%%%%%%%%%%%%%%%%%%%%%%%%%%%%%%%%%%%
\section{
Action homogeneity
}
%%%%%%%%%%%%%%%%%%%%%%%%%%%%%%%%%%%%%%%
\subsection{
A motivating example
}
\draftnote{green}{To do}{
\begin{enumerate}
    \item Look at $\mathscr{W}_{\alpha}$ vs $\mathscr{W}_{\beta}$ with identity treatment.
    \item $\mathscr{W}_{\alpha}$: Cayley tables + properties for $(\hat{A}^{*}/\sim_{w}, \circ_{\sim_{w}})$ for all four $w \in W$.
    \item $\mathscr{W}_{\beta}$ with identity treatment: Cayley tables + properties for $(\hat{A}^{*}/\sim_{w}, \circ_{\sim_{w}})$ for all four $w \in W$.
    \item Talk about differences.
\end{enumerate}
}


%%%%%%%%%%%%%%%%%%%%%%%%%%%%%%%%%%%%%%%
\subsection{
Homogenous vs inhomogeneous actions
}



%%%%%%%%%%%%%%%%%%%%%%%%%%%%%%%%%%%%%%%
\subsection{
Action homogeneous vs action inhomogeneous worlds.
}


%%%%%%%%%%%%%%%%%%%%%%%%%%%%%%%%%%%%%%%
\subsection{
An easy mistake: \autocite{caselles2020sensory}
}
\draftnote{green}{To do}{
\begin{enumerate}
    \item Explain mistake in \autocite{caselles2020sensory}.
    \item Globally reversible vs consistently globally reversible vs having a consistent global inverse (which is the group property).
    \item Made more difficult to spot because the world without objects in is action homogeneous and so (assuming finite) globally reversible $\implies$ having a consistent global inverse (see done proposition earlier).
    \item Check the PhD thesis that \autocite{caselles2020sensory} cited.
\end{enumerate}
}



