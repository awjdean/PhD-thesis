\draftnote{blue}{}{
\section{Overview}
}
\draftnote{blue}{Include in intro}{
\begin{enumerate}
    \item This thesis develops a novel mathematical foundation for....
\end{enumerate}
}

\newthought{In this chapter}, we describe a mathematical framework of the structure for the transformations of a world from the perspective of an agent; we consider an \emph{agent} to be an entity that can interact with our world and learn its structure.
We want our framework to be as general as possible, so we model a world as a multidigraph consisting of the states of the world (\emph{world states}) and the transformations between those world states.

\draftnote{blue}{To do}{
We then construct the arrows of this multidigraph from an underlying multigraph of atomic transformations between world states.
Then we will introduce our agent into our framework and how we treat the agent's representations.
We describe the actions of an agent as collections of transformations between world states and explore some of the properties of these actions.
In future chapters, we will use our framework to explore the types of structure that can be present in the transformations in the representation of our agent for an agent learning a representation for different worlds.
}

\draftnote{blue}{To do}{
We argue that graph algebras...
}

%%%%%%%%%%%%%%%%%%%%%%%%%%%%%%%%%%%%%%%%%%%%%%
\draftnote{blue}{}{
\section{Motivation}
}
\draftnote{blue}{Include}{
\begin{enumerate}
    \item Why is creating a mathematical framework useful ?
    \begin{itemize}
        \item Give examples in Physics, Chemistry, Biology, and AI of mathematical frameworks leading to a deeper understanding/novel discoveries.
        \item Note how we're interested in general learning algorithms not necessarily deep learning - put in intro ?
    \end{itemize}
    \item Why is creating a mathematical framework for studying the representations of agents useful ?
    \item Mention geometric deep learning - then in the category theory section point out how Bruno has translated this into Category theory. Can also show how you get geometric deep learning out of the generalised equivariance condition.
\end{enumerate}
}