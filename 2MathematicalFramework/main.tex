\chapter{
  A mathematical framework
  (V1.2)
 }

\draftnote{green}{To do}{
	\begin{enumerate}
		\item [Mention] Actions can be represented as tuples of their unique compositions of minimum actions.
		\item Define operator $\operatorname{Seq}: \hat{A}^{\ast} \to \hat{A}^{(n)}$ that sends an action to its tuple of minimum actions.
		      \begin{itemize}
			      \item $\operatorname{Seq}$ is a bijective mapping because each action has a unique minimum action tuple.
		      \end{itemize}
		\item Distinguish between the underlying minimum transformation worlds
	\end{enumerate}
}

%%%%%%%%%%%%%%%%%%%%%%%%%%%%%%%%%%%%%%%%%%%%%%
\section{Overview}

\newthought{In this chapter}, we describe a mathematical framework of the structure for the transformations of a world from the perspective of an agent; we consider an \emph{agent} to be an entity that can interact with our world and learn its structure.
We want our framework to be as general as possible, so we model a world as a directed multigraph consisting of the states of the world (\emph{world states}) and the transformations between those world states.





\draftnote{blue}{To do}{
We then construct the arrows of this directed multigraph from an underlying multigraph of atomic transformations between world states.
Then we will introduce our agent into our framework and how we treat the agent's representations.
We describe the actions of an agent as collections of transformations between world states and explore some of the properties of these actions.
In future chapters, we will use our framework to explore the types of structure that can be present in the transformations in the representation of our agent for an agent learning a representation for different worlds.
}


%%%%%%%%%%%%%%%%%%%%%%%%%%%%%%%%%%%%%%%%%%%%%%
\section{A mathematical treatment of worlds and their transformations}\label{sec:A mathematical treatment of worlds and their transformations}

%%%%%%%%%%%%%%%%%%%%%%%%%%%%%%%%%%%%%%%%%%%%%%
\subsection{What is the most general form of a world?}
A world contains world states, which are how the world is at a snapshot in time, and a way to move between world states when the world changes; we call a change of a world from one world state to another a \emph{tranformation}.

Mathematically, we can represent the collection of world states as a set $W$ of objects, the collection of tranformations as a set $D$ of objects that connect world states; individual world states in $W$ and individual transformations in $D$ are distinguishable in some way\footnote{
We are building up to using our framework to consider the transformation structure of the world from the perspective of an agent. So when we say \emph{distinguishable in some way} we mean that there is some way for our agent to distinguish between the world states or the transformations.
}.
We connect transformations with their associated world states using a source map
\begin{equation}
	s: D \to W,
\end{equation}
which sends each transformation to the world state it originates at, and a target map
\begin{equation}
	t: D \to W,
\end{equation}
which sends each transformation to the world state it ends at.
Note that we haven't enforced any structure on the world states or the transformations except that each transformation connects two world states.

\begin{definition}[\textbf{Change to Notation block}]
	We denote a transformation $d \in D$ with a source of $w$ and a target of $w'$ as $d: w \to w'$.
\end{definition}

These source and target maps allow us to represent a world as a type of graph $\mathscr{W}$ with arrows (the transformations in $D$) connecting vertices (the world states $W$).
Since it is possible to have a world where more than one transformation has the same source and the same target, our graph $\mathscr{W}$ must be a \emph{multigraph}.
Since it is possible to have a world where there is a transformation from a world state $w$ to a world state $w'$, but not have a transformation from $w'$ back to $w$\footnote{In other words, a world with an irreversible change.}, our multigraph $\mathscr{W}$ must be \emph{directed}.

\begin{definition}[\textbf{Change to Postulate block}]
	The most abstract mathematical representation of a world is a directed multigraph that consists of a collection $W$ of world states as vertices, a collection $D$ of tranformations as arrows, a source map $s$ that maps each arrow to the world state it is leaving, and a target map $t$ that maps each arrow to the world state it is entering.
\end{definition}

\begin{definition}[\textbf{Change to Notation block}]
	We denote a world $\mathscr{W}$ with a set $W$ of world states, a set $D$ of transformations, a source map $s$ and a target map $t$ by
	\begin{equation}
		\mathscr{W} = (W, D, s, t)
	\end{equation}
\end{definition}

%%%%%%%%%%%%%%%%%%%%%%%%%%%%%%%%%%%%%%%%%%%%%%
\subsection{Constructing worlds from atomic transformations}

\begin{definition}[\textbf{Change to Postulate block}]
	All transformations are made up of sequences of \emph{atomic transformations}, which cannot be broken down into smaller transformations.
\end{definition}

We will construct our world $\mathscr{W}$ from atomic transformations\footnote{
Note that, in a similar same way as for distinguishable world states, when we use our framework to describe the representation of an agent, these atomic transformations can be thought of as the transformations giving the smallest change to an aspect of a world state that is detectable by the agent.
}, but first we will establish some properties of atomic transformations.

\begin{definition}[\textbf{Change to Notation block}]
	We use a $\hat{ }$ to explicitly show that something is atomic, that a set contains atomic things, or that an operator acts exclusively on atomic things.
\end{definition}

Let $\hat{D}$ be the set of atomic transformations.
We define atomic source and target maps
\begin{equation}
	\hat{s}: \hat{D} \to W \quad \hat{t}: \hat{D} \to W
\end{equation}
such that for any atomic transformation $\hat{d} \in \hat{D}$
\begin{equation}
	\hat{d}: w \to w' \implies \hat{s}(d) = w \text{ and } \hat{t}(d) = w'.
\end{equation}

Using $W$, $\hat{D}$, $\hat{s}$, and $\hat{t}$ we can construct another directed multigraph $\hat{\mathscr{W}}$, which we call the \emph{atomic multigraph of the world}
\begin{equation}
	\hat{\mathscr{W}} = (W, \hat{D}, \hat{s}, \hat{t})
\end{equation}

Our set $D$ of all transformations of $\mathscr{W}$ is the set of all finite directed walks in $\mathscr{W}$\footnote{
	The atomic transformations are in $D$ as directed walks of length 1.
}.
This means that a transformation $d \in D$ is a sequence of atomic transformations\footnote{
	This operator $\hat{\circ}$ is the operator of concatenation of walks of length 1.
	}
\begin{equation}
	d = \hat{d}_{n} \hat{\circ} \hat{d}_{n-1} \hat{\circ} ... \hat{\circ} \hat{d}_{1}
\end{equation}
where
\begin{equation}
	\hat{\circ}: \hat{D} \times \hat{D} \to D
\end{equation}
is the atomic transformation composition operator that is defined if
\begin{equation}
	\hat{t}(\hat{d}_{i}) = \hat{s}(\hat{d}_{i+1}) \quad \text{for $i = 1, ..., n-1$}.
\end{equation}
The composition operator $\hat{\circ}$ is associative\footnote{
An operator $\cdot$ is \emph{associative} if it satisfies
\begin{equation}
	a \cdot (b \cdot c) = (a \cdot b) \cdot c.
\end{equation}
This means that when we perform the operation $\cdot$ on three or more elements, the way in which the elements are composed does not affect the outcome.
}
by the definition of concatenation of walks.


\paragraph{Trivial transformations.}
For each world state $w \in W$ of our directed multigraph $\hat{\mathscr{W}}$, there is a trivial walk that does not leave $w$.
We call the associated transformation of this trivial walk the \emph{trivial transformation at $w$} and denote it by $1_{w}$\footnote{Trivial transformations essentially represent no change happening in the world; we will hear more from trivial transformations when we look at the actions of agents.}
.

Therefore, for all $w \in W$, there exists a trivial transformation $1_{w} \in D$ with the following properties
\begin{align}
	\hat{s}(1_{w}) = w \\
	\hat{t}(1_{w}) = w \\
	|1_{w}| = 0
\end{align}
Trivial walks serve as neutral elements in the concatination of directed walks, and so transformation sequences involving trivial transformations are not distinct elements in $D$ (except for the sequences of containing only a single trivial transformation); for example, $\hat{d} \circ 1_{s(d)}$ is the same element in $\hat{D}$ as $\hat{d}$.
In other words, trivial transformations satisfy
\begin{align}
	& \hat{d} \circ 1_{\hat{s}(\hat{d})} = \hat{d} \quad \text{for all $\hat{d} \in \hat{D}$} \\
	& 1_{\hat{t}(\hat{d})} \circ \hat{d} = \hat{d} \quad \text{for all $\hat{d} \in \hat{D}$}
\end{align}
We denote the set of trival transformations by
\begin{equation}
	D_{\epsilon} = \{ 1_{w} \mid w \in W \}
\end{equation}

\paragraph{Extending from $\hat{\mathscr{W}}$ to $\mathscr{W}$.}
Each transformation in $D$ can now be uniquely expressed as a sequence of composed atomic tranformations.
\begin{equation}
	\operatorname{Seq}: D \to \hat{D}^{n}
\end{equation}
such that, for $d = \hat{d}_{n} \hat{\circ} \hat{d}_{n-1} \hat{\circ} \dots \hat{\circ} \hat{d}_{1}$
\begin{equation}
	\operatorname{Seq}(d) = (\hat{d}_{n}, \hat{d}_{n-1}, \dots, \hat{d}_{1})
\end{equation}

We can extend our atomic source map $\hat{s}$ and atomic target map $\hat{t}$ to our source map $s$ and target map $t$ as, for a transformation $d \in D$ with $\operatorname{Seq}(d) = (\hat{d}_{n}, \hat{d}_{n-1}, \dots, \hat{d}_{1})$,
\begin{align}
	& s(d) = \hat{s}(\hat{d}_{1}) \\
	& t(d) = \hat{t}(\hat{d}_{n}).
\end{align}

We can also extend the atomic transformation composition operator $\hat{\circ}$ to the transformation operator on the entire set $D$ as:
\begin{equation}
\begin{aligned}
	& \circ: D \times D \to D \\
	& \text{where $d' \circ d$ is defined if $t(d) = s(d')$.}  
\end{aligned}
\end{equation}
such that, for $\operatorname{Seq}(d) = (\hat{d}_{n}, \hat{d}_{n-1}, \dots, \hat{d}_{1})$ and $\operatorname{Seq}(d') = (\hat{d'}_{n}, \hat{d'}_{n-1}, \dots, \hat{d'}_{1})$
\begin{equation}
	\operatorname{Seq}(d' \circ d) = (\hat{d'}_{n}, \hat{d'}_{n-1}, \dots, \hat{d'}_{1}, \hat{d}_{n}, \hat{d}_{n-1}, \dots, \hat{d}_{1}).
\end{equation}
$\circ$ is associative due to $\hat{\circ}$ being associative\footnote{
$\circ$ is the operator of concatenation of walks.
	We can now naturally extend stuff like the neutral element condition for trivial transformations as
	\begin{align}
		& d \circ 1_{s(d)} = d \quad \text{for all $d \in D$} \\
		& 1_{t(d)} \circ d = d \quad \text{for all $d \in D$}
	\end{align}
}
.


\paragraph{Length of a transformation.}
The \emph{length} of a transformation $d \in D$ is the number of atomic transformations that make up its associated directed walk in $\hat{\mathscr{W}}$ \footnote{
	This means minimum transformations have a length of one.
}.
We denote the length of $d$ by $|d|$.



\draftnote{blue}{To do}{
Something about how we can refer to worlds by $\mathscr{W}$ or by $\hat{\mathscr{W}}$ ?
}


%%%%%%%%%%%%%%%%%%%%%%%%%%%%%%%%%%%%%%%%%%%%%%
\subsection{Simplifications}

We make the following simplifications to worlds in our framework.

\paragraph{Deterministic worlds.}
In a deterministic world, every change in the current world state has a single guaranteed effect.
A world that is not deterministic is called \emph{stochastic}; in a stochastic world, the same change to the current world state could produce different results.
We will initially consider deterministic worlds\footnote{Extensions to stochastic worlds will be discussed in \draftnote{blue}{section ???}.
}.

\paragraph{Discrete worlds.}
In a \emph{discrete} world, the number of states and the number of transformations between states are countable.
A world that is not discrete is called \emph{continuous}.
For simplicity, we only consider discrete worlds.
However, we can make the following argument that it is actually more natural to consider discrete worlds than contiuous worlds.
Consider a continuous world $\mathscr{W}$ and an agent with $n$ sensors $\theta_{i}$ that can produce a sensor observation $o_{i}$ of $\mathscr{W}$ as\footnote{
More on how we model the sensory observations of agents in \draftnote{blue}{section ???}.
}
\begin{equation}
	\theta_{i}(w) = o_{i}
\end{equation}
where $i \in \{1, ..., n\}$ and $w \in W$.
We hypothesis that there are transformations that cuase changes to their source world state that are so small that the changes cannot be perceived by the agent; in other words, there exist transformations $d \in D$ such that
\begin{equation}
	\theta_{i}(s(d)) = \theta_{i}(t(d))
\end{equation}
Such transformations do not exist from the perspective of the agent; therefore, from the persepctive of the agent there will be discrete jumps between perceptible world states.
We will initially consider discrete worlds\footnote{Extensions to continous worlds will be discussed in \draftnote{blue}{section ???}.
}.



\draftnote{red}{DIVIDER}{}





%%%%%%%%%%%%%%%%%%%%%%%%%%%%%%%%%%%%%%%%%%%%%%%%%%%%%
\subsection{Drawing worlds}

\newthought{We can visually} represent the structure of our directed multigraphs using \emph{world diagrams} consisting of a dots for the world states and arrows for the transformations.
These diagrams can make it easier to understand the structure and properties of worlds.

Since our set $D$ is can be very large\footnote{
	If we have transformations $d: s(d) \to t(d)$ and $d': s(d) \to t(d)$, then we can construct sequences like $d' \circ d$, $d' \circ d \circ d' \circ d$ etc..., which means $D$ has an infinite number of elements:
	\begin{center}
		\FloatBarrier
		\captionsetup{type=figure}
		\includegraphics[width=1.0\linewidth]{2MathematicalFramework/Images/D_commonly_large.jpg}
		% \captionof{figure}{Test caption.}
		\caption{A world diagram showing sequences of the transformations $d$ and $d'$ that are of the form $(d' \circ d)^{n}$.}
	\end{center}
}, we do not show all the transformations in $D$ in our world diagrams unless explicitly stated.

Let's look at some world diagrams show casing some of the properties we've already seen.

\begin{figure}[H]
	\centering
	\begin{subfigure}[b]{0.45\linewidth}
		\centering
		\includegraphics[width=\linewidth]{2MathematicalFramework/Images/transformation.jpg}
		\caption{A world diagram showing a transformation.}
		\label{fig:transformation}
	\end{subfigure}
	\hfill
	\begin{subfigure}[b]{0.45\linewidth}
		\centering
		\includegraphics[width=\linewidth]{2MathematicalFramework/Images/transformation_composition.jpg}
		\caption{A world diagram showing the composition of two transformations $d$ followed by $d'$ to give a transformation $d' \circ d$.}
		\label{fig:transformation_composition}
	\end{subfigure}
	\vskip\baselineskip
	\begin{subfigure}[b]{0.45\linewidth}
		\centering
		\includegraphics[width=\linewidth]{2MathematicalFramework/Images/left_trivial_transformation.jpg}
		\caption{A world diagram showing a transformation $d: s(d) \to t(d)$ and its left identity element, the trivial transformation $1_{t(d)}$. Composing these two transformations gives the transformation $d$.}
		\label{fig:left_trivial_transformation}
	\end{subfigure}
	\hfill
	\begin{subfigure}[b]{0.45\linewidth}
		\centering
		\includegraphics[width=\linewidth]{2MathematicalFramework/Images/right_trivial_transformation.jpg}
		\caption{A world diagram showing a transformation $d: s(d) \to t(d)$ and its right identity element, the trivial transformation $1_{s(d)}$. Composing these two transformations gives the transformation $d$.}
		\label{fig:right_trivial_transformation}
	\end{subfigure}
	\vskip\baselineskip
	\begin{subfigure}[b]{0.45\linewidth}
		\centering
		\includegraphics[width=\linewidth]{2MathematicalFramework/Images/trivial_transformations_example_all_transformations.jpg}
		\caption{A world diagram showing a transformation $d: s(d) \to t(d)$, its left identity $1_{t(d)}$, and its right identity $1_{s(d)}$.}
		\label{fig:trivial_transformations_example_all_transformations}
	\end{subfigure}
	\caption{Examples of world diagrams displaying some of the concepts we have already come across.}
\end{figure}

It's important to note that, in a world diagram, the positioning (relative or absolute) of the world states and the shapes and lengths of the arrows do not matter\footnote{Formally, two world diagrams are isomorphic if...}:
\begin{figure}
	\centering
	\includegraphics[width=0.5\linewidth]{2MathematicalFramework/Images/isomorphic_world_diagrams.jpeg}
	\caption{The properties of these world diagrams are the same (the world diagrams are isomorphic).}
	\label{fig:isomorphic_world_diagrams}
\end{figure}

\paragraph{Subworlds.}
A world $\mathscr{W}' = (W', \hat{D}', \hat{s}', \hat{t}')$ is a \emph{subworld} of the world $\mathscr{W} = (W, \hat{D}, \hat{s}, \hat{t})$ if
\begin{align}
	 & W' \subseteq W                                                                                                   \\
	 & \hat{D}' \subseteq \hat{D} \quad \text{such that $s(\hat{d}), t(\hat{d}) \in W'$ for all $\hat{d} \in \hat{D}'$} \\
	 & \hat{s}' = \hat{s} \big|_{\hat{D}'}                                                                              \\
	 & \hat{t}' = \hat{t} \big|_{\hat{D}'}
\end{align}
\footnote{
	This definition requires $\hat{D}' \subseteq \hat{D}$ and not $D' \subseteq D$ because removal of an element of $D$ which is not an element of $\hat{D}$ to create $D'$ would lead to sequences of minimum transformations not being transformations
	\draftnote{blue}{Include}{Wouldn't it also work the other way: removal of a non-minimum transformation but removing one of its constituent minimum transformations would break compositionality}
	; to avoid problems like this we always consider the minimum transformations then use them to generate the set of transformations.

	However, I think we could remove minimum transformations while keeping non-minimum transformations if composition still works (i.e., removing some minimum transformations would turn some non-minimum transformations into minimum transformations (size $|\hat{D}|$ would need to decrease or stay the same due to the removal of the minimum transformations.
	This could be a way of removing minimum transformations that can't be sensed by the agent - can we make this work for infinitely small minimum transformations ?
}.
\draftnote{blue}{To do}{
	Note that $1_{w} \in D$ are also in $D'$ because $D_{\epsilon}$ is separate.
	What about the $1_{w}$ for
}
We can denote that $\mathscr{W}'$ is a subworld of $\mathscr{W}$ using $\mathscr{W}' \subseteq \mathscr{W}$.
A subworld is a world.
\draftnote{blue}{Consider}{
	I think we might also not be able to remove an arbitrary element of $D_{A}$ ?
	Or do we need a proof that all types of actions produce subworlds that obey compositionality ?
	Or we say that a subworld is a sub-multi-digraph that obeys compositionality ?
}

\paragraph{Reachable and unreachable subworlds.}
A world $\mathscr{W}'$ is \emph{reachable} from a world $\mathscr{W}$ if there exists a transformation $d: w \to w'$ where $w \in W$ and $w' \in W'$.
Formally, we consider a world $\mathscr{W} = (W, \hat{D}, \hat{s}, \hat{t})$.
The set of world states that are reachable from a world state $w$ is given by
\begin{equation}
	W^{\to}(w) := \{ w' \in W \mid \text{there exists} \; d \in D \; \text{with} \; d: w \to w' \}
\end{equation}

The set of minimum transformations that are reachable from $w$ is given by\footnote{
	This is the largest subset of minimum transformations that stay within the set of reachable states.
}
\begin{equation}
	\hat{D}^{\to}(w) := \{ \hat{d} \in \hat{D} \mid s(\hat{d}) \in W^{\to}(w) \; \text{and} \; t(\hat{d}) \in W^{\to}(w) \}
\end{equation}

The reachable subworld of $\mathscr{W}$ from the state $w$ is given by
\begin{equation}
	\mathscr{W}^{\to}(w) := (W^{\to}(w), \hat{D}^{\to}(w), \hat{s} \big|_{\hat{D}^{\to}(w)}, \hat{t} \big|_{\hat{D}^{\to}(w)})
\end{equation}

We denote the current state of a world by $w^{*}$.
The subworld that is reachable from the current state of the world is called the \emph{natural reachable subworld}, which we denote by $\mathscr{W}^{\to}(w^{*})$.



\draftnote{blue}{To do somewhere in this chapter}{Proposition that if there exists a $d \in D$ with $d: s(d) \to t(d)$ then there exists $\hat{d} \in \hat{D}$ with $\hat{d}: s(d) \to t(\hat{d})$ and there exists $\hat{d}' \in \hat{D}$ with $\hat{d}: s(\hat{d}) \to t(d)$.
	Same, is true for $d \in D_{A}$
}

\draftnote{blue}{Note}{
	Need to consider $\hat{D}_{A}$ instead of $\hat{D}$ when looking at reachable worlds due to the actions of an agent.
}


\footnote{
	A sensible question is `what set of transformations contains $d$ ?'.
	If $\mathscr{W}' \subseteq \mathscr{W}$ then $d \in \mathscr{W}$.
	If $\mathscr{W}' \not\subseteq \mathscr{W}$ then we can consider a world $\mathscr{W}^{\aleph}$ containing all physically allowed world states and world state transformations; we then have that $\mathscr{W} \subseteq \mathscr{W}^{\aleph}$, $\mathscr{W}' \subseteq \mathscr{W}^{\aleph}$, and $d$ is an element of the set of transformations of $\mathscr{W}^{\aleph}$.
}.
If a subworld is not reachable, we say it is \emph{unreachable}.


\newthought{We are building} towards describing the structure of transformations in an agent's representation; late we will only be interested in the perspective of the agent and so will only care about the world states and transformations that the agent can come into contact with.
Unreachable and reachable worlds give us the language to describe and then disregard parts of worlds that agents will never explore and therefore are not relevant to the agent's representation.
For example, if an agent is in a maze and a section of the maze is inaccessible from the position that the agent is in, then that section of the maze would be disconnected from the section of the maze that the agent is in; if we want to study how the agent’s representation evolves as it learns, it makes sense to disregard world states where the agent is in the inaccessible part of the maze since those world states can never be reached and so those world states will not affect the agent's representation.
\footnote{
	But what about worlds that are unreachable from the current world state, but the agent has already visited?
	If we assume the representation evolution process is Markov, then all the necessary information will contained in the current representation space of the agent and so past world states which are now unreachable are irrelevant.
	Therefore, for a Markov representation evolution process, we only need to consider the world states in subworlds that are reachable from the current world state.

	If the representation evolution process is not Markov, we can transform the representation evolution process into a Markov process by augmenting the state space of the process to include any relevant historical information.
}
\begin{figure}
	\centering
	\includegraphics[width=0.5\linewidth]{2MathematicalFramework/Images/reachable_worlds.jpg}
	\caption{
		$\mathscr{W}_{C}$ is reachable from $\mathscr{W}_{A}$ and $\mathscr{W}_{B}$, but $\mathscr{W}_{A}$ is not reachable from $\mathscr{W}_{B}$ or $\mathscr{W}_{C}$.
	}
	\label{fig:reachable_worlds}
\end{figure}


\draftnote{green}{To do}{
	\begin{enumerate}
		\item \textbf{Mention:} we can have more than one minimum transformation between two states. Why we allow this becomes clear when we introduce the idea of agent actions.
		\item \textbf{Footnote:} Composition of transformations is read from left to right; the reason for this will be explained later.
		\item \textbf{Mention:} sometimes we don't include the trivial transformations for clarity.
		\item \textbf{Mention:} For the rest of this document, if we talk about a sets $D$, $\hat{D}$ of transformations or sets $W$ of worlds without explicitly defining an associated world $\mathscr{W}$, then the results are for a general world $\mathscr{W}$; we will explicitly define the world for specific examples.
		\item Define what a \emph{finite directed walk} is.
		\item Explain how to construct the set of all finite walks.
		\item \textbf{Diagrams:}
		      \begin{enumerate}
			      \item (?) No unique element in $D$ for sequence with trivial transformations.
		      \end{enumerate}
	\end{enumerate}
}

%%%%%%%%%%%%%%%%%%%%%%%%%%%%%%%%%%%%%%%%%%%%%%%%%%%%%%%%%%
\section{Introducing an agent}

\draftnote{red}{DIVIDER}{}
\draftnote{blue}{Put in Agents section}{
\paragraph{Fully observable worlds.}
If complete knowledge of the current world state for every possible current world state is known\footnote{This does not mean knowledge of the transformations between world states. Also, it does not mean knowledge of every world state, but knowledge of the current world state, no matter which world state is the current one}, then the world is called \emph{fully observable}; if not then the world is called \emph{partially observable}.
Partially observable worlds are common in many real-world scenarios, however, we will initially consider fully observable worlds.
\footnote{Move this section to agents section? "When an agent sensor is capable to sense or access the complete state of an agent at each point in time, it is said to be a fully observable environment else it is partially observable."}
We begin with fully observable worlds because:
\begin{itemize}
	\item their treatment is simpler than partially observable worlds, and
	\item the representation of an agent of a partially observable world should ideally be the same as the representation of the agent in the same world but fully observable; therefore, if we identify structures present in an agent's representation of fully observable worlds then we are also identifying structures that should be present in the agent's representation of the same world but partially observable without having to consider the complications of partial observability\footnote{???}.
\end{itemize}
}
\draftnote{red}{DIVIDER}{}



\newthought{We want to} study the structure of the representations of an agent that can interact with our worlds - we will now explain what an agent is in our framework.

\subsection{What is an agent ?}

\newthought{An \emph{agent} is} an autonomous entity\footnote{This could be a natural or artificial entity.} that can observe and interact with its environment\footnote{Our agent does not need to be embodied in the world - more on this later !}; the agents that we care about can also learn about their environments.

\subsection{Representations of an agent}

\newthought{An agent's \emph{representation states}} are the agent's internal representations of world states, through which the agent maintains its understanding of its world.
An agent's \emph{representation model} is the agent's internal model of the world, which includes how the agent encodes and organises information about its environment (e.g., as vectors, symbols, or other data structures) as representation states as well as the dynamics of how those representation states change due to actions that affect the world\footnote{The agent's representation model does not encode the dynamics of the world states, it encodes the dynamics of the representation states; a `good' representation will provide a good approximation to the dynamics of the world states. More on this later.}.

An agent's \emph{representation model} not only captures the current state of the world but also allows it to (attempt to) conceptualise future states by predicting the state of the world based on its observations and actions and plan actions accordingly, forming a comprehensive framework for decision-making and interaction.

\newthought{Agents use \emph{sensors}} to capture data from their world, enabling the agent to perceive the world; these sensors include real-world sensors (e.g., cameras on a robot, the human eye) and virtual inputs (e.g., data streams, user commands) in a software agent.
Sensors provide the agent with \emph{observation states}\footnote{Sometimes these are called sensory states such as in \cite{Ramstead2020}.}, which are the agent's internal representations of the information the sensors collect (e.g., data streams from the camera), and importantly are not the sensors themselves.
An \emph{agent's perceptual model} is how the agent encodes and organises information from its sensors (e.g., vectors, symbols, or other data structures) as well as how it conceptualises the structure of the landscape of possible observation states.

\newthought{Agents use \emph{effectors}} to interact with and alter their world, enabling the agent to carry out actions on the world; this can involve real-world effectors (e.g., motors and actuators on a robot) or virtual outputs (e.g., API calls or applying a transformation to a dataset the agent is manipulating) in a software agent.
These effectors receive \emph{effector signals}\footnote{Called \emph{control signals} in the field of robotics.}, which are the concrete instructions sent by the agent to execute actions (e.g., a voltage spike to a motor or a command to a software interface).
An agent also has an \emph{action model}, which is how the agent encodes and organises information about its actions (e.g., control signals, symbolic commands) as well as how it conceptualises how its actions will affect the world state.

An agent's action model is intrinsically linked to its representation model since the action model describes the movement between states of the representation model due to actions\footnote{These actions do not have to be only the actions of the agent itself, it can be a grouping of world state transformations that the agent understands to be from a particular cause - more on this later.}.

\draftnote{green}{Include}{
	\begin{enumerate}
		\item \textbf{Figures:}
		      \begin{itemize}
			      \item Diagram showing what $W$, $Z$, $A$, $w$, $z$, $a$ are.
			      \item Diagram showing mapping from $W$ to $O$ to $Z$. Also have diagram showing $W$ to $O$ to $Z$ for agent with ideal sensors.
			            \begin{figure}[H]
				            \centering
				            \includegraphics[scale = 0.35]{2MathematicalFramework/Images/W_to_O_to_Z_mapping.png}
				            \caption{The composite mapping from the set $W$ of world states to the set $Z$ of state representations via the set $O$ of observations.}
			            \end{figure}
		      \end{itemize}
	\end{enumerate}
}


\subsection{Agents as Markov blankets}

In our framework, we consider agents to be systems with internal states that interact with the world in ways that maintain a boundary between their internal states and the world.

Agent's in our framework have four types of state:
\begin{enumerate}
	\item \emph{Observation states} represent the agent's inputs from the environment (e.g., vision, sound, touch etc...).
	\item \emph{Effector states} represent how the agent influences the world state (i.e., the agent's external environment).
	\item \emph{(External) world states} consist of the states of the agent's external environment, which are the world states we introduced in section \ref{sec:A mathematical treatment of worlds and their transformations}.
	\item \emph{Agent's (internal) representation states} contain our agent's representations of the environment, memory model, knowledge models, goals, decision making processes etc...
	      These internal states can only interact with the external world states via observation states and effector states.
\end{enumerate}

Using these four types of states we can view our agents as Markov blankets (see figure \ref{fig:markov_blanket}).

\begin{figure}
	\centering
	\includegraphics[width=0.5\linewidth]{2MathematicalFramework/Images/markov_blanket.jpg}
	\caption{
		Complete caption.
		Arrows show the flow of information.
	}
	\label{fig:markov_blanket}
\end{figure}

The Markov blanket consists of the states that mediate interactions between our agent and its environment (the world) \cite{Ramstead2020}.

For our agent, the observation and effector states form the Markov blanket with the world states being outside the blanket and the the internal state being inside the blanket.
This Markov blanket acts as a boundary that separates the agent's internal representation states from the external world states.
The agent can not directly access the external world states, but can infer these world states by processing sensory inputs through its observation states and performing actions through its effector states; information about the world states crosses the Markov blanket through the agent's sensory states and the agent's affect on the world crosses the Markov blanket through the agent's effector states.

Technically, given the observation and effector states (i.e., the blanket), the internal representation states of the agent are \emph{conditionally independent} of the external world states; conditionally independent of the external world states means that the behaviour of the agent's internal representation depends only on the states of the agent's Markov blanket (i.e., the agent's observation and effector states) and not (directly) on the world states external to the agent because all the relevant information about the external world states are filtered through the observation states and the agent only affects the world through its effector states.

By framing an agent as a Markov blanket, we clearly show that the agent learns about its world only through its observation states and its actions.
One major implication of this approach for our framework is we consider the internal representation states of the agent to be separate from the world states introduced in section \ref{sec:A mathematical treatment of worlds and their transformations}; this means that if we swap an agent embodied in a world with another agent that is identical except for having a different internal representation state (or a different representation model), then the world state would remain unchanged.

Another implication is that, since an agent's only gains knowledge about the structure of the world through its sensors and its actions, world states with differences that are not detectable by the agent are identical from the agent's perspective and so can be treated as such when we are constructing a mathematical description of the agent's interaction with the world (i.e, those world states do not exist from the perspective of the agent)\footnote{More on this later !}.

\subsection{Sensory perception}

Our agents have an unspecified number of sensors that allows it to make observations of the world.
Information about the agent's current world state is delivered to the internal state of the agent as an observation state; this is called the \emph{observation process}\footnote{If the world state is modelled as a distribution then is process is also referred to as the agent \emph{sampling} the world state.}.
For example, the human eye (the sensor) converts information about the light entering the eye into
electrical and chemical energy in the optic nerve (the observation process).

Mathematically, we treat the observation process as a mapping $b: W \to O$ which produces an observation $o_{i}$.
The structure of $o_{i}$ depends on the agent's perceptual model; commonly, each observation $o_{i}$ is a vector of length $N$, where $N$ is the number of sensors the agent has.

\begin{figure}
	\centering
	\includegraphics[width=0.5\linewidth]{2MathematicalFramework/Images/observation_process_W_to_O.jpeg}
	\caption{
		Observation map
		\draftnote{blue}{awjdean}{Improve this caption.}
	}
	\label{fig:observation_process_W_to_O}
\end{figure}

Initially, we will consider our agents to have \emph{ideal sensors}; for an agent with ideal sensors:
\begin{enumerate}[(1)]
	\item information is not lost or modified during the observation process,
	\item there is no noise in the sensors,
	\item a change in the world state causes a change in the observation state \draftnote{blue}{awjdean}{This might be too strict? We want to allow cases where $b(w_{1})=b(w_{2})$ due to things such as occlusion? Should we discuss this separately later or should we introduce something like a world with bijective f ?} \draftnote{blue}{awjdean}{Include a note about how changes occur instantly ? (I'm thinking about the skipped intermediate world states example).}.
\end{enumerate}

In real world situations, it is common for information about the world state to be lost, modified or not picked up during the observation process; for example, the human eye only picks up certain wavelengths of light or a camera could take in a colour spectrum then output greyscale.
We will discuss agents with non-ideal sensors in section ???.

Something else to note is that world states with differences that are not detectable by the agent's sensors or actions are effectively identical from the agent's perspective and so can be treated as such when we are constructing a mathematical description of the agent's interaction with the world.

\subsection{Learning and inference}

We consider agents that learn about their world.
The end goal of the agent’s learning process is to map the useful aspects of the structure of the world to the structure of its representation; the useful aspects are those that enable the agent to complete whatever task the agent has.
The observation states from our agents' sensors are used by an \emph{inference process} to produce a representation of the current state of the world; the agent
then uses some internal mechanism to select an action to perform.
For example, the electrical signals from the optic nerve are given to the brain for inference; mechanisms in the brain then select an action to be performed.
Mathematically, we consider this inference process to be a mapping $h: O \to Z$.
\begin{figure}
	\centering
	\includegraphics[width=0.5\linewidth]{2MathematicalFramework/Images/inference_process_O_to_Z.jpeg}
	\caption{
		Inference process.
		\draftnote{blue}{awjdean}{Improve this caption}
	}
	\label{fig:inference_process_O_to_Z}
\end{figure}

For agents without ideal sensors, there are scenarios (e.g., in partially observable environments) where two world states that are distinct for the agent produce the same observation state (i.e., $h(o_{\alpha}) = w_{1}$ and $h(o_{\alpha}) = w_{2}$).
We will discuss these types of scenarios in section ???.

We can combine $b$ and $h$ to form a mapping $f = h \circ b$, from $W$ to $Z$, that describes the process of the agent getting information about the world.
For agents with ideal sensors, the map $f$ is bijective
\draftnote{blue}{awjdean}{Proof that $f$ is bijective if agent ideal sensors.
	1. Prove $b$ bijective.
	2. Prove $h$ bijective.
	3. Composition of two bijective maps is also a bijective map.
}, and therefore is invertible.
We will initially focus on the structure of an agent's representation due to these types of maps $f$.
\begin{figure}
	\centering
	\includegraphics[width=0.5\linewidth]{2MathematicalFramework/Images/W_to_Z.jpeg}
	\caption{
		Example of map $f$
		\draftnote{blue}{awjdean}{Improve this caption}
		\draftnote{blue}{awjdean}{$g$ should be $b$.}
	}
	\label{fig:W_to_Z}
\end{figure}


\subsection{Our approach}

We propose there are two approaches for understanding the learning of an agent: (1) we can study the learning process of the agent itself (i.e., how does the agent learn at each step of the learning process) or (2) we can study what we believe the end result of the agent's learning process should be.
We will initially focus on approach (2).

We assume that the best representation of the worlds we study should exactly mirror the structure of the worlds themselves in the structure of the agent's representation, and so the end result of the agent's learning process should be the relevant structure of the world being present in the structure of the agent's representation of the world
\draftnote{blue}{awjdean}{Can we prove that the representation 'world' should be a subworld of $\mathscr{W}$ ?}.
By initially studying the structure of the worlds themselves, rather than structure of an agent's representation, we simplify the problem we are tackling since we no longer have to consider each step of the learning process; this provides the opportunity to generalises our results somewhat - for example, if we prove that the structure of the world does not obey certain conditions then we might be able to show that this is true no matter the process used for learning
\draftnote{blue}{awjdean}{Flesh this out a bit more - talk about learning equilibria ?}
.

Once we have this foundation, we can introduce aspects of non ideal sensors to see how they affect the agent's representation when learning the structure of the same worlds we've already studied.

\draftnote{green}{To do}{
	\begin{enumerate}
		\item Sensory perception in an agent.
		      \begin{compactitem}
			      \item Look at Higgins distribution set up.
			      \item Cones figure - cone in $W$ leading to a single observation in $O$ - put in later discussion.
			      \item If the agent has ideal sensors (i.e., there is no noise from the sensors) the number of possible observation states is less than or equal to the number of world states (i.e., $|O| \leq |W|$) - put this should later with a proof.
		      \end{compactitem}
		\item Learning.
		      \begin{compactitem}
			      \item Brief discussion about scenarios where you can have $b(w_{1}) = b(w_{2})$ but $h(b(w_{1})) \neq h(b(w_{2}))$ - say we'll discuss this in more detail later.
			      \item Proof that the structure of an agent's representation should mirror the structure of the agent's representation for an agent with ideal sensors.
		      \end{compactitem}
	\end{enumerate}
	\textbf{Small things:}
	\begin{enumerate}
		\item Environment vs world.
		\item Mention at start of Agents section that we use a mix of papers for inspiration for our agent: Agent as a Markov blanker paper, Higgins, Caselles-Dupre.
		\item World states as a distribution - see Higgins.
		\item \textbf{References:}
		      \begin{enumerate}
			      \item Agent as a Markov blanket: \url{https://www.researchgate.net/figure/The-structure-of-a-Markov-blanket-A-Markov-blanket-highlights-open-systems-exchanging_fig1_343643283}
		      \end{enumerate}
	\end{enumerate}
	\textbf{Implemented notes:}
	\begin{enumerate}
		\item Do the 'sensory states' in the Markov Blanket represent different observation states in $O$ or different sensors ? They represent different observation states - need to make my formalism consistent. \textbf{Use $o$ for observation states.}
		\item Sensors can mean real-world sensors or virtual inputs (e.g., in a software agent).
	\end{enumerate}
}

%%%%%%%%%%%%%%%%%%%%%%%%%%%%%%%%%%%%%%%%%%%%%%%%%%%%%%%%%%%%%%%%%%%
\section{Actions of agents}

In this section, we define the actions of agents as collections of transformations that are labelled by their associated action, and then we define some relevant properties of these actions.

%%%%%%%%%%%%%%%%%%%%%%%%%%%%%%%%%%%%%%%%%%%%%%%%%%%%%%%%%%%%%%%%%%%
\subsection{Actions as collections of labelled transformations}

Consider a set $\hat{A}$ called the set of \emph{minimum actions} of an agent, where $\hat{A} = \{1, a_{1}, a_{2}, \dots, a_{n}\}$.

We now take the Kleene star operator of $\hat{A}$ \footnote{
	$\hat{A}^{\ast} = \{ \epsilon \} \cup \{ (a_1, a_2, \dots, a_n) \mid a_i \in \hat{A}, \, n \in \mathbb{N} \}$.
} to get the set of all possible finite sequences formed from elements of $\hat{A}$, including the empty sequence $\epsilon$.
We call $\hat{A}^{\ast}$ the set of \emph{actions} of an agent.

We define a composition operator $\circ: \hat{A}^{\ast} \times \hat{A}^{\ast} \to \hat{A}^{\ast}$ such that $(a_1, a_2, \dots, a_n) \in \hat{A}^{\ast}$ can be written as $a_1 \circ a_2 \circ \dots \circ a_n$.
Sometimes we write compositions like $a_1 \circ a_2 \circ \dots \circ a_n$ as $a_1 a_2 \dots a_n$.
The operator $\circ$ is closed under the definition of the Kleene star operator.

\paragraph{Properties of the empty sequence $\epsilon$.}
\begin{enumerate}
	\item \textbf{Uniqueness:} There is only one empty sequence $\epsilon$.
	\item \textbf{Identity element for concatenation:} the empty sequence $\epsilon$ serves as an identity element for concatenation of sequences:
	      $\epsilon \circ a = a \circ \epsilon = a$ for all $a \in \hat{A}^{\ast}$.
	\item \textbf{Length:} The empty sequence $\epsilon$ has a length of zero (i.e., $|\epsilon| = 0$).
\end{enumerate}

\paragraph{Linking actions with transformations.}
We now want to link our set of actions to the transformations of the world that would occur if an agent performed each action while in a current world state.
We do this by first labelling transformations in the set $D$ of transformations that are caused by minimum actions with the elements of $\hat{A}$, and then using those labellings to label other transformations in $D$ with the elements of $\hat{A}^{\ast}$ by considering how the elements of $\hat{A}$ combine to form the elements of $\hat{A}^{\ast}$.

Consider a set $\hat{D}_{A} \subset D$, where $1_{w} \in \hat{D}_{A}$ for all $w \in W$; we call $\hat{D}_{A}$ the set of \textit{minimum action transformations}.
We define a labelling map $\hat{l}: \hat{D}_{A} \to \hat{A}$ such that $\hat{l}$ satisfies the following conditions:
\begin{enumerate}
	\item \textbf{Uniqueness condition.}
	      Any two distinct minimum transformations leaving the same world state are labelled with different actions:
	      \begin{equation}
		      \text{If } s(d) = s(d'), \text{ then } \hat{l}(d) \neq \hat{l}(d') \quad \text{for any } d, d' \in \hat{D}_{A}.
	      \end{equation}

	\item \textbf{Identity condition.}
	      Trivial transformations, $1_{w}$ for all $w \in W$, are mapped to the empty sequence $\epsilon$:
	      \begin{equation}
		      \hat{l}(1_{w}) = \epsilon \quad \text{for all } w \in W.
	      \end{equation}
\end{enumerate}

So, for any world $\mathscr{W} = (W, \hat{D}, \hat{s}, \hat{t})$, there is a subworld $\mathscr{W}_{A} = (W, \hat{D}_{A}, \hat{s}, \hat{t})$.
In the same way as we did in section \ref{sec:A mathematical treatment of worlds and their transformations} to get $D$ from $\mathscr{W}$, we take the set of all finite directed walks in $\mathscr{W}_{A}$ and denote it by $D_{A}$; the elements of $D_{A}$ are called \emph{transformations due to the actions of the agent}.
As before, we extend $\hat{s}$ and $\hat{t}$ to $s$ and $t$ respectively\footnote{
	We can use $s$, $t$ instead of having to define sub maps $s_{A}$ and $t_{A}$ because $s$ and $t$ act on the elements of $D$ and $D_{A}$ is just $D$ with some elements removed.
	If we wanted to define such maps $s_{A}$ and $t_{A}$ they would be restrictions of the maps $s$ and $t$ to the set $D_{A}$.
	\draftnote{blue}{awjdean}{
		- Put this as a footnote for the definition of subworlds?
		does this slightly alter the definition of subworld ?
	}
}.

We extend the map $\hat{l}$ to the map $l: D_{A} \to \hat{A}^{\ast}$ such that:
if $d = \hat{d}_{n} \circ ... \circ \hat{d}_{1}$ then $l(d) = \hat{l}(\hat{d}_{n}) ... \hat{l}(\hat{d}_{1})$.
The extended map $l$ now satisfies the following conditions:
\begin{enumerate}
	\item Any two distinct transformations leaving the same world state are
	      labelled with different actions.
	      \begin{action_condition}[Uniqueness]\label{actcon:action_gives_single_outcome}
		      For any $d,d' \in D_{A}$ with $s(d)=s(d')$, $l(d) \neq l(d')$.
	      \end{action_condition}

	\item Trivial transformations, $1_{w}$ for all $w \in W$, are mapped to the empty sequence $\epsilon$:
	      \begin{action_condition}[Identity]\label{actcon:trivial_transformations_mapped_to_empty_sequence}
		      $l(1_{w}) = \epsilon$ for all $w \in W$.
	      \end{action_condition}

	\item Transformations are labelled consistently with the sequence of labels corresponding to their constituent minimum transformations.
	      \begin{action_condition}[Composition consistency]\label{actcon:composition_consistency}
		      If $d = \hat{d}_{n} \circ ... \circ \hat{d}_{1}$ then $l(d) = \hat{l}(\hat{d}_{n}) ... \hat{l}(\hat{d}_{1})$ for all $d \in D_{A}$ and for all $\hat{d_{1}}, \dots , \hat{d_{n}} \in \hat{D}_{A}$.
	      \end{action_condition}
\end{enumerate}

For $d \in D_{A}$ with $s(d) = w$, $t(d) = w'$ and $l(d) = a$, we will often denote $d$ by $d: w \xrightarrow{a} w'$.

A summary of the procedure we've taken to label the transformations of a world $\mathscr{W}$ that are due to the actions of an agent is given by figure \ref{fig:action_labelling_procedure}.

\begin{figure}
	\centering
	\includegraphics[width=\linewidth]{2MathematicalFramework/Images/action_labelling_procedure.jpeg}
	\caption{Procedure for labelling the transformations of the world with their associated action.}
	\label{fig:action_labelling_procedure}
\end{figure}


\paragraph{Effect of actions on world states.}
\draftnote{blue}{awjdean}{Define $\hat{\ast}$ first then extend to $\ast$ - we need $\hat{\ast}$ for Cayley algorithms.}
Unlike transformations, which are tied to individual world states, actions can have different outcomes when applied to different world states.
For example, it could be the case that applying the action $a$ followed by the action $b$ on a world state $w_{1}$ gives back the original world state, while applying $a$ followed by $b$ on world state $w_{2}$ could give a different world state $w_{3}$ - the action sequence $ba$ produces the original world state when applied to certain world states (e.g., $w_{1}$), but produces a world state that is not the original world state when applied to other world states (e.g., $w_{2}$).
Due to this behaviour, we need a way to describe how actions change world states.
We define the \emph{effect of an action $a \in \hat{A}^{\ast}$ on a world state $w \in W$} as

\begin{itemize}
	\item[] $\ast: \hat{A}^{\ast} \times W \to W$ such that
	      \begin{enumerate}[(1)]
		      \item if there exists a $d: w \xrightarrow{a} t(d)$, then $a \ast w = t(d)$; and
		      \item if there does not exist a $d: w \xrightarrow{a} t(d)$, then we say that $a \ast w$ is \emph{undefined}.
	      \end{enumerate}
\end{itemize}
From now on, unless stated otherwise, when we give something like $a \ast w$ we will assume that $a \ast w$ is defined \draftnote{blue}{awjdean}{Remove this ? - have introduced $\bot$ state.}.


The effect of actions on world states is well-defined due to action condition \ref{actcon:action_gives_single_outcome} \draftnote{blue}{awjdean}{Prove this explicitly?}.

The compatibility of $\ast$ and $\circ$ is given by:
\begin{equation}\label{eqn:circ_ast_compatibility}
	(a' \circ a) \ast w = a' \ast (a \ast w) \quad \text{for any $a, a' \in \hat{A}^{\ast}$ and $w \in W$}
\end{equation}
\footnote{
	This follows naturally from the composition consistency property on the labelling map $l$:
	If $d_{1}: w \xrightarrow{a} w'$ and $d_{2}: w' \xrightarrow{a'} w''$, then $d_{2} \circ d_{1}: w \xrightarrow{a' \circ a} w''$.
	By the definition of $\ast$, $a * w = t(d_{1})$, $a' \ast (a \ast w) = t(d_{2})$, and $(a' \circ a) \ast w = t(d_{2} \circ d_{1})$.
	Therefore, since $d_{2} \circ d_{1}$ has the label $a' \circ a$, we see that $(a' \circ a) \ast w = a' \ast (a \ast w)$.
}

This means we can apply the minimum actions that make up an action to world states individually:

if $a = \hat{a}_{k}...\hat{a}_{1}$, then
\begin{align}
	a \ast w & = (\hat{a}_{k} \dots \hat{a}_{1}) \ast w                                       \\
	         & = (\hat{a}_{k} \dots \hat{a}_{2}) \ast (\hat{a}_{1} \ast w)                    \\
	         & = (\hat{a}_{k} \dots \hat{a}_{3}) \ast (\hat{a}_{2} \ast (\hat{a}_{1} \ast w)) \\
	         & = \dots
\end{align}

We can also split actions into short sequences of actions and apply those:

if $a = \hat{a}_{k}...\hat{a}_{1}$, then
\begin{align}
	a \ast w & = (\hat{a}_{k} \dots \hat{a}_{3}) \ast (\hat{a}_{2} \ast (\hat{a}_{1} \ast w))  \\
	         & = (\hat{a}_{k} \dots \hat{a}_{3}) \ast ((\hat{a}_{2} \circ \hat{a}_{1}) \ast w) \\
	         & = \dots
\end{align}

Equation \ref{eqn:circ_ast_compatibility} means that $\ast$ is an \emph{action} of the algebra $(\hat{A}^{\ast}, \circ)$ on the set $W$ \footnote{
	The action of an algebra is different to the actions of an agent.
	Yes, it's confusing!
}.


%%%%%%%%%%%%%%%%%%%%%%%%%%%%%%%%%%%%%%%%%%%%%
\subsection{Actions as (partial) functions.}

Consider all the transformations that are labelled by a particular action $a \in \hat{A}^{\ast}$.
Together these transformations form a partial function $f_{a}: W \to W$ because for any $w \in W$ either $a \ast w$ is undefined or $a \ast w$ is defined and there is a unique world state $w' \in W$ for which $a \ast w = w'$ (from action condition \ref{actcon:action_gives_single_outcome}).
Each of the transformations labelled by the action $a$ is an assignment of $f_{a}$ to one of the world states in $W$ ($f_{a}(w) \mapsto a \ast w = w'$).

Formally, we can curry\footnote{Currying allows us to transform an operation with multiple arguments into a collection of functions with single arguments by fixing all but one of the operation's arguments.} the action effect operator $\ast : \hat{A}^{\ast} \times W \to W$
\begin{equation}
	\textit{Curry}: (\ast: \hat{A}^{\ast} \times W \to W) \to (f_{a}: W \to W)
\end{equation}
to obtain a collection of (partial) functions
\begin{equation}
	\mathcal{T}_{\hat{A}^{\ast}} = \{f_{a}: W \to W \mid f_{a}(w) = a \ast w \text{ where defined, for each } a \in \hat{A}^{\ast} \}
\end{equation}

These function $f_{a}$:
\begin{enumerate}[(1)]
	\item are not generally surjective because for a given $w \in W$ there is not necessarily a transformation $d \in D$ with $l(d) = a$ and $t(d) = w$ \draftnote{blue}{awjdean}{Illustrative diagram}.

	\item are not generally injective because it is possible to have an environment where $f_{a}(w)=f_{a}(w')$ for some $w \in W$ different from $w' \in W$ \draftnote{blue}{awjdean}{Illustrative diagram}.
\end{enumerate}

\draftnote{blue}{Footnote}{
	We can also reproduce these functions using the formalism given by \autocite{caselles2019symmetry}, which describes the dynamics of the world in terms of a multivariate function $f: A \times W \to W$.
	If we let $f: A \times W \to W$ be the dynamics of the environment then the transformation caused by an action $a \in A$ on a world state $w \in W$ (where $a \ast w$ is defined) is given by $(a,w) \mapsto f(a,w) = a \ast w$. \draftnote{blue}{awjdean}{Put stuff like this in a separate "Other formalisms described using our formalism" section after Reproducing SBDRL ?}
}

%%%%%%%%%%%%%%%%%%%%%%%%%%%%%%%%%%%%%%%
\subsection{Reachable subworlds due to the actions of an agent}

\draftnote{blue}{To do}{
	Something (earlier) about how we need to specify $\hat{D}_{A}$ (you could have a transformation in $\hat{D}_{A}$ that's not in $\hat{D}$) - perhaps bundle this into world-agent pair.
}

To define reachable subworlds due to the actions of an agent, we take the reachable subworlds definition and replace $\hat{D}$ with $\hat{D}_{A}$.
For a world $\mathscr{W} = (W, \hat{D}, \hat{s}, \hat{t})$ containing an agent with minimum actions $\hat{A}$ and minimum action transformations $\hat{D}_{A}$,
\begin{align}
	 & W^{\hat{A}\to}(w) := \{ w' \in W \mid \text{there exists} \; d \in D_{A} \; \text{with} \; d: w \to w' \}                                                                 \\
	 & \hat{D}_{A}^{\hat{A}\to}(w) := \{ \hat{d} \in \hat{D}_{A} \mid s(\hat{d}) \in W^{\hat{A}\to}(w) \; \text{and} \; t(\hat{d}) \in W^{\hat{A}\to}(w) \}                      \\
	 & \mathscr{W}^{\hat{A}\to}(w) := (W^{\hat{A}\to}(w), \hat{D}_{A}^{\hat{A}\to}(w), \hat{s} \big|_{\hat{D}_{A}^{\hat{A}\to}(w)}, \hat{t} \big|_{\hat{D}_{A}^{\hat{A}\to}(w)})
\end{align}
where $W^{\hat{A}\to}(w)$ is the set of world states that are reachable from the world state $w \in W$ using the actions of the agent, $\hat{D}_{A}^{\hat{A}\to}(w)$ is the set of minimum transformations that are reachable from $w$ using the actions of the agent, and $\mathscr{W}^{\hat{A}\to}(w)$ is the reachable subworld of $\mathscr{W}$ from $w$ using the actions of the agent.
The natural reachable subworld due to the actions of the agent is denoted by $\mathscr{W}^{\hat{A}\to}(w^{*})$.


%%%%%%%%%%%%%%%%%%%%%%%%%%%%%%%%%%%%%%%
\subsection{From partial to total}

We will now turn our partial functions into total functions by adding a new state $\bot$ to our set $W$ of world states, where $\bot$ represents the \emph{undefined state}\footnote{
	This technique for turning partial functions into total functions is used in functional programming languages like Haskell \draftnote{blue}{awjdean}{Reference.}.
}.
\begin{equation}
	W^{\bot} = W \cup \{ \bot \}
\end{equation}

We also extend the effect $\ast : \hat{A}^{\ast} \times W \to W$ of actions on world states to $\ast^{\bot} : \hat{A}^{\ast} \times W^{\bot} \to  W^{\bot}$ so that if $a \ast w$ is defined then it remains the same as before, but if $a \ast w$ is undefined, then we set $a \ast^{\bot} w$ equal to our undefined state $\bot$:
\begin{align}
	 & \ast^{\bot} : \hat{A}^{\ast} \times W^{\bot} \to  W^{\bot} \quad\text{such that} \\
	 & a \ast^{\bot} w =
	\begin{cases}
		t(d) & \text{if, for $d \in D_{A}$, } d : w \xrightarrow{a} t(d) \text{ exists}, \\
		\bot & \text{otherwise.}
	\end{cases}
\end{align}
\footnote{
	Note that we have:
	\begin{equation}
		a \ast^{\bot} \bot = \bot \quad \text{for all $a \in \hat{A}^{\ast}$}
	\end{equation}
	which means that $\bot$ acts as an absorbing element for $\ast^{\bot}$.
}

We can also augmented our set $D_{A}$ of transformations to a set $D_{A}^{\bot}$, which includes the new transformations that send world states to $\bot$, so that we have underlying transformations that reflect the behaviour of $\ast^{\bot}$.
First, we define the set $\Delta^{\bot}$ of new transformations that terminate at $\bot$, which includes transformations for $a \ast^{\bot} \bot$:
\begin{alignat}{2}
	U             & {}={} &  & \{ (a,w) \in \hat{A}^{\ast} \times W \mid \not\exists (d: w \xrightarrow{a} t(d)) \in D_{A} \} \\
	\Delta^{\bot} & {}={} &  & \{ d^{\bot}: w \xrightarrow{a} \bot \mid (a, w) \in U \}                                       \\
	              &       &  & \cup \{ d^{\bot}: \bot \xrightarrow{a} \bot \text{ for all $a \in \hat{A}^{\ast}$} \}
\end{alignat}

\draftnote{blue}{awjdean}{Change symbol for temporary set $U$ ? - have a symbol that is used for a temporary thing that is never used again ?}
\begin{equation}
	D_{A}^{\bot} = D \cup \Delta^{\bot}
\end{equation}

Now we can extend our action labelling map $l: D_{A} \to \hat{A}^{\ast}$ to include our new $\bot$-terminating transformations:
\begin{align}
	 & l^{\bot} : D_{A}^{\bot} \to \hat{A}^{\ast} \quad\text{such that} \\
	 & l^{\bot}(d) =
	\begin{cases}
		l(d) & \text{if $d \in D_{A}$},                                            \\
		a    & \text{if $d \in \Delta^{\bot}$ and $d: s(d) \xrightarrow{a} \bot$.}
	\end{cases}
\end{align}

To simplify notation, we will use $W$, $\ast$, $D_{A}$, and $l$ to denote $W^{\bot}$, $\ast^{\bot}$, $D_{A}^{\bot}$, and $l^{\bot}$ respectively unless we want to explicitly differentiate between the use of $W$, $\ast$, $D_{A}$, and $l$ vs $W^{\bot}$, $\ast^{\bot}$, $D_{A}^{\bot}$, and $l^{\bot}$.

\draftnote{green}{To do}{
	\begin{enumerate}
		\item Footnote about potential loss of information about why an action was undefined  (e.g., if two distinct actions are undefined on the same state $w$ then their effects are indistinguishable after $\bot$ is reached because $\bot$ is absorbing - in this cases we could introduce $\bot_{1}, \bot_{2}, \bot_{3} \dots$ to separate out different "types" of undefined.
	\end{enumerate}
}

%%%%%%%%%%%%%%%%%%%%%%%%%%%%%%%%%%%%%%%%%
\subsection{$(\hat{A}^{\ast}, \circ)$, $\mathcal{T}_{\hat{A}^{\ast}}$, and $(\hat{A}^{\ast}, \circ, \ast)$}

There are three notable structures which we will explore in this section: $(\hat{A}^{\ast}, \circ)$, $\mathcal{T}_{\hat{A}^{\ast}}$, and $(\hat{A}^{\ast}, \circ, \ast)$.

%%%%%%%%%%%%%%%%%%%%%%%%%%%%%%%%%%%%%%%%%
\paragraph{$(\hat{A}^{\ast}, \circ)$.}
$(\hat{A}^{\ast}, \circ)$ is the free monoid generated by the set $\hat{A}$, where $\circ$ acts like the concatenation of strings.
This means there are no relations between the generators in $\hat{A}$ apart from the associativity and identity required for $(\hat{A}^{\ast}, \circ)$ being a monoid, and so $(\hat{A}^{\ast}, \circ)$ provides maximum flexibility in constructing elements\footnote{Note that the structure of $(\hat{A}^{\ast}, \circ)$ is unaffected by the effect $\ast$ of actions on world states.}.

\draftnote{blue}{awjdean}{Proof that $(\hat{A}^{\ast}, \circ)$ is closed ?}
\begin{proposition}
	$(\hat{A}^{\ast}, \circ)$ is a monoid.
\end{proposition}
\begin{proof}
	\begin{enumerate}[(1)]
		\item \textbf{Associativity.}
		      $\circ$ is associative for all $a \in \hat{A}^{\ast}$
		      \draftnote{blue}{awjdean}{Prove this explicitly.}
		\item \textbf{Identity element.}
		      The empty action $\epsilon$ acts as the identity element:
		      \begin{equation}
			      \epsilon \circ a = a \quad \text{and} \quad a \circ \epsilon = a \quad \text{for all } a \in \hat{A}^{\ast}.
		      \end{equation}
	\end{enumerate}
\end{proof}

%%%%%%%%%%%%%%%%%%%%%%%%%%%%%%%%%%%%%%%%%
\paragraph{$\mathcal{T}_{\hat{A}^{\ast}}$.}
Now we have introduced the undefined state $\bot$, the set $\mathcal{T}_{\hat{A}^{\ast}}$ is a set of (total) functions $f_{a}: W \to W$ with one $f_{a}$ for each action $a \in \hat{A}^{\ast}$ \footnote{
Making the $\bot$'s explicit, $\mathcal{T}_{\hat{A}^{\ast}}$ is now defined as:
\begin{equation}
	\begin{aligned}
		\mathcal{T}_{\hat{A}^{\ast}} = \{ & f_{a}: W^{\bot} \to W^{\bot} \mid                                \\
		                                  & f_{a}(w) = a \ast^{\bot} w \text{ for $a \in \hat{A}^{\ast}$} \}
	\end{aligned}
\end{equation}
}.

\begin{proposition}\label{prp:T_is_monoid}
	$\mathcal{T}_{\hat{A}^{\ast}}$ is a monoid under composition $\cdot: \mathcal{T}_{\hat{A}^{\ast}} \times \mathcal{T}_{\hat{A}^{\ast}} \to \mathcal{T}_{\hat{A}^{\ast}}$ of functions such that:
	\begin{equation}
		f_{a' \circ a} = f_{a'} \circ f_{a}
	\end{equation}
\end{proposition}
\begin{proof}
	\begin{enumerate}[(1)]
		\item \textbf{Associativity.}
		      Associativity of the composition of the partial functions $f_{a}$ follows from the associativity of $\circ: \hat{A}^{\ast} \times \hat{A}^{\ast} \to \hat{A}^{\ast}$:
		      \begin{align}
			      (f_{a''} \cdot f_{a'}) \cdot f_{a} & = f_{a'' \circ a'} \cdot f_{a}       \\
			                                         & = f_{(a'' \circ a') \circ a}         \\
			                                         & = f_{a'' \circ (a' \circ a)}         \\
			                                         & = f_{a''} \cdot f_{a' \circ a}       \\
			                                         & = f_{a''} \cdot (f_{a'} \cdot f_{a})
		      \end{align}
		\item \textbf{Identity element.}
		      The function $f_{\epsilon}$ correspondence to the empty action $\epsilon$ is the identity function\draftnote{blue}{awjdean}{Make this more explicit?}:
		      \begin{equation}
			      f_{\epsilon}(w) = \epsilon \ast w = w \quad \text{for all $w \in W$}
		      \end{equation}
	\end{enumerate}
\end{proof}

There exists a monoid homomorphism\footnote{
$\phi$ arises directly due to the \emph{universal property of free monoids}.
We take a mapping $a \mapsto f_{a}$ that assigns each minimum action $a \in \hat{A}$ to its corresponding effect function $f_{a}$.
Then, by the universal property of free monoids, $a \mapsto f_{a}$ uniquely extends to a monoid homomorphism $\phi : \hat{A}^{\ast} \to \mathcal{T}_{\hat{A}^{\ast}}$ that preserves the monoid structure.
\draftnote{blue}{awjdean}{Expand on this ?}
} from $(\hat{A}^{\ast}, \circ)$ to $\mathcal{T}_{\hat{A}^{\ast}}$:
\begin{equation}
	\phi : \hat{A}^{\ast} \to \mathcal{T}_{\hat{A}^{\ast}} \quad\text{defined by $\phi(a) = f_{a}$}
\end{equation}
that preserves the monoid structure
\begin{align}
	 & \phi(a' \circ a) = f_{a' \circ a} = f_{a'} \cdot f_{a} = \phi(a') \cdot \phi(a) \\
	 & \phi(\epsilon) = f_{\epsilon}
\end{align}

\draftnote{blue}{awjdean}{$\mathcal{T}_{\hat{A}^{\ast}}$ is an action monoid.}

Each function $f_{a}$ precisely represents the effect of an action $a \in \hat{A}^{*}$ on all possible world states.

Other properties of $f_{a} \in \mathcal{T}_{\hat{A}^{\ast}}$:
\begin{enumerate}[(1)]
	\item \textbf{Not generally invertible.}
	      There may not exist a function $f_{a}^{-1}$ such that $f_{a}^{-1} \circ f_{a} = f_{a} \circ f_{a}^{-1} = f_{\epsilon}$.
	\item \textbf{Not necessarily injective.}
	      It is possible for a function $f_{a}$ to map different world states to the same world state (i.e., it is possible to have $f_{a}(w) = f_{a'}(w)$).
	\item \textbf{Not necessarily surjective onto $W$.}
	      It is possible for there to be a world state that is not reachable through a function $f_{a}$ (i.e., it is possible for a $w \in W$ to exist such that there is no $w' \in W$ where $f_{a}(w') = w$ for some $a \in \hat{A}^{\ast}$).
\end{enumerate}

\draftnote{blue}{awjdean}{Note about the structure of $\mathcal{T}_{\hat{A}^{\ast}}$ is $\bot$ not introduced.}
\draftnote{blue}{awjdean}{$\mathcal{T}_{\hat{A}^{\ast}}$ may not be closed under composition due to partiality and redundancy, and therefore $\mathcal{T}_{\hat{A}^{\ast}}$ is not in general a monoid.}

%%%%%%%%%%%%%%%%%%%%%%%%%%%%%%%%%%%%%%%%%
\paragraph{$(\hat{A}^{\ast}, \circ, \ast)$.}
\begin{proposition}
	$(\hat{A}^{\ast}, \circ, \ast)$ is a monoid action of $\hat{A}^{\ast}$ on the set $W$.
\end{proposition}
\begin{proof}
	\begin{enumerate}[(1)]
		\item \textbf{Compatibility.}
		      The action effect operator $\ast$ is compatible with the composition operator $\circ$ since:
		      \begin{equation}
			      (a \circ b) \ast w = a \ast (b \ast w) \quad \text{for all } a, b \in A^{\ast}, \, w \in W.
		      \end{equation}

		\item \textbf{Identity action.}
		      The empty action $\epsilon$ acts as the identity on $W$:
		      \begin{equation}
			      \epsilon \ast w = w \quad \text{for all } w \in W.
		      \end{equation}

	\end{enumerate}
\end{proof}

\draftnote{blue}{awjdean}{Double check that $(\hat{A}^{\ast}, \circ, \ast)$ is a monoid.}

\draftnote{blue}{Consider}{
	What would the structures be with no $\bot$ use --> Category or partial transformation monoid ?
}

\draftnote{blue}{Consider - Other potentially useful properties of $(\hat{A}^{\ast}, \circ)$}{
	\begin{enumerate}
		\item Any monoid homomorphism from $(\hat{A}^{\ast}, \circ)$ to another monoid is fully determined by its action on the generators $\hat{A}$.
		      \begin{itemize}
			      \item Does this mean that if we change the minimum actions we'll be able to map generators to find out if there's an isomorphism ?
		      \end{itemize}
		\item $(\hat{A}^{\ast}, \circ)$ is closed.
		\item $(\hat{A}^{\ast}, \circ)$ is decidable (i.e., there exists an algorithm that can always provide a correct yes-or-no answer in a finite amount of time).
		      \begin{itemize}
			      \item Operations in Structures: In the context of $(\hat{A}^{\ast}, \circ)$, the questions "Is $s_{1} \circ s_{2} = s_{3}$" or "Does $s$ belong to $\hat{A}^{\ast}$?" are decidable because we can compute the concatenation $s_{1} \circ s_{2}$ and compare it to $s_{3}$, or verify $s$'s construction from the alphabet $\hat{A}$, respectively.
		      \end{itemize}
	\end{enumerate}
}

\draftnote{green}{To do}{
	\begin{enumerate}
		\item Appendix: Why $\hat{A}$ is not necessarily labelling elements of $\hat{D}$ only.
		\item Have I said that $\hat{D}_{A}$ contains $D_{\epsilon}$ by definition ?
		\item Is $l$ actually a collection of maps $l_{w}$ for each $w \in W$ ? --> endomorphism viewpoint ?
		\item Talk about why we've created action condition 1.
		\item \textbf{Footnote:} difference between the empty action and the no op action.
		\item Say we refer to the empty sequence as the empty action.
		\item \textbf{Mention:} We can use $s$, $t$ instead of having to define sub maps $s_{A}$ and $t_{A}$ because $s$ and $t$ act on the elements of $D$ and $D_{A}$ is just $D$ with some elements removed. - Put this as a footnote for the definition of subworlds; does this slightly alter the definition of subworld ?
		\item \textbf{Mention:} Minimum actions denoted by $\hat{hat}$.
		\item No-op action doesn't always exist.
		      \begin{compactitem}
			      \item Physically, the identity action $1 \in A$ corresponds to the no-op action (\textit{i.e.}, the world state does not change due to this action).
		      \end{compactitem}
		\item Action condition \ref*{actcon:action_gives_single_outcome} means that for each $w \in W$, there is a subset of $\hat{D}_{A}$ given by $\{d \mid d \in \hat{D}_{A}\textit{, } s(d)=w\}$; for each of these subsets, the elements of the subset are uniquely labelled (within the subset) by $l$.
		      In other words, the transformations in $D_{A}$ with source $s(d)$ are uniquely labelled by $\hat{l}$ for all $d \in D_{A}$.
		\item Summary
		      \begin{compactitem}
			      \item $a \sim a' \Leftrightarrow f_{a} = f_{a'} \Leftrightarrow \textit{for all } w \in W^{\bot}, \; a \ast w = a' \ast w$.
			      \item $\hat{A}^{\ast}/\sim \; \Leftrightarrow \mathcal{T}$ (set of distinct transformations $W^{\bot} \to W^{\bot}$.
		      \end{compactitem}
	\end{enumerate}
}

\draftnote{blue}{Questions/thoughts that have been answered:}{
	\begin{enumerate}
		\item $A$ is the set of words made from $\hat{A}$ - i.e., the set of finite sequences of elements of $A$, not infinite sequences - infinite sequences make things tricky (apparently).
		\item We want the set $A$ of actions to be the Kleene star operation of $\hat{A}$ ($A = \hat{A}^{\ast}$). $\hat{A}^{\ast}$ includes an empty sequence $\epsilon$ and our no-op action $1$ as separate sequences; if the no op action is an identity action, then $1 \sim \epsilon$.
	\end{enumerate}
}

%%%%%%%%%%%%%%%%%%%%%%%%%%%%%%%%%%%%%%%%%%%%%%%%%%%%%%%%%%%%%%%%%%%%%%%%%%%%%%%%%%%%%%%%%%%%%%
\section{
  An example world \texorpdfstring{$\mathscr{W}_{\alpha}$}{}
 }\label{sec:an_example_world}

So far we've seen a lot of definitions and mathematical descriptions, now we will illustrate these ideas through a toy example.
Our chosen toy problem is a $2\times 2$ cyclic grid world $\mathscr{W}_{\alpha}$, where all the transformations in $\mathscr{W}_{\alpha}$ are due to the actions of an agent embodied in the world.

%%%%%%%%%%%%%%%%%%%%%%%%%%%%%%%%%%%%%%%%%%%%%%%%%%%%%%%%%%%%%%%%%%%%%%%%%%%%%%%%%%%%%%%%%%%%%%
\subsection{World states of $\mathscr{W}_{\alpha}$}\label{sec:World states of example}

The world states of $\mathscr{W}_{\alpha}$ are shown in figure \ref{fig:2x2-cyclical-grid-world-states}.
So, for $\mathscr{W}_{\alpha}$, $W = \{ w_{0}. w_{1}, w_{2}, w_{3} \}$.
\begin{figure}[H]
	\centering
	\begin{subfigure}[b]{0.45\linewidth}
		\centering
		\includegraphics[width=0.5\linewidth]{2MathematicalFramework/Images/2x2_no_walls_world_states/w0.png}
		\caption{$w_{0}$}
		\vspace{0.25cm}
	\end{subfigure}
	\begin{subfigure}[b]{0.45\linewidth}
		\centering
		\includegraphics[width=0.5\linewidth]{2MathematicalFramework/Images/2x2_no_walls_world_states/w1.png}
		\caption{$w_{1}$}
		\vspace{0.25cm}
	\end{subfigure}
	\begin{subfigure}[b]{0.45\linewidth}
		\centering
		\includegraphics[width=0.5\linewidth]{2MathematicalFramework/Images/2x2_no_walls_world_states/w2.png}
		\caption{$w_{2}$}
	\end{subfigure}
	\begin{subfigure}[b]{0.45\linewidth}
		\centering
		\includegraphics[width=0.5\linewidth]{2MathematicalFramework/Images/2x2_no_walls_world_states/w3.png}
		\caption{$w_{3}$}
	\end{subfigure}
	\caption{
		The world states of a cyclical $2\times 2$ grid world $W_{(2,2)C}$.
		The position of the agent in the world is represented by the position of the circled A.
		\draftnote{blue}{awjdean}{Include a compass in the figure?}
	}
	\label{fig:2x2-cyclical-grid-world-states}
\end{figure}

%%%%%%%%%%%%%%%%%%%%%%%%%%%%%%%%%%%%%%%%%%%%%%%%%%%%%%%%%%%%%%%%%%%%%%%%%%%%%%%%%%%%%%%%%%%%%%
\subsection{Transformations of $\mathscr{W}_{\alpha}$}

We define that there are 24 minimum transformations in $\mathscr{W}_{\alpha}$: the 20 transformations of length 1 given in table \ref{tab:2x2_gridworld_length_1_transformations}\footnote{
	We have chosen these because we know the actions we will give the agent in section \ref{sec:Actions of the agent in example}; there is a link between the structure of a world from the agent's perspective and the actions the agent will perform - more on this later.
	However, these are the only transformations that are possible for a world with the world states given in section \ref{sec:World states of example}, the condition that all the transformations of the world are due to the actions of an agent with the actions provided in section \ref{sec:Actions of the agent in example}, and obeying action condition \ref{actcon:action_gives_single_outcome}.
	\draftnote{blue}{awjdean}{Put proof of this in appendices ?}
}
and the 4 trivial transformations $1_{w_{0}}$, $1_{w_{1}}$, $1_{w_{2}}$, and $1_{w_{3}}$.
So for $\mathscr{W}_{\alpha}$, $D_{\epsilon} = \{ 1_{w_{0}}, 1_{w_{1}}, 1_{w_{3}}, 1_{w_{4}} \}$ and $\hat{D} = \{ d_{1}, d_{2}, d_{3}, \dots, d_{20} \} \cup D_{\epsilon}$.

\begin{table}[H]
	\centering
	\begin{tabular}{|c|c c|}
		\hline
		Element of $\hat{D} \backslash D_{\epsilon}$ & Source  & Target  \\
		\hline
		$d_{1}$                                      & $w_{0}$ & $w_{0}$ \\
		$d_{2}$                                      & $w_{0}$ & $w_{1}$ \\
		$d_{3}$                                      & $w_{0}$ & $w_{1}$ \\
		$d_{4}$                                      & $w_{0}$ & $w_{2}$ \\
		$d_{5}$                                      & $w_{0}$ & $w_{2}$ \\
		$d_{6}$                                      & $w_{1}$ & $w_{1}$ \\
		$d_{7}$                                      & $w_{1}$ & $w_{0}$ \\
		$d_{8}$                                      & $w_{1}$ & $w_{0}$ \\
		$d_{9}$                                      & $w_{1}$ & $w_{3}$ \\
		$d_{10}$                                     & $w_{1}$ & $w_{3}$ \\
		$d_{11}$                                     & $w_{2}$ & $w_{2}$ \\
		$d_{12}$                                     & $w_{2}$ & $w_{0}$ \\
		$d_{13}$                                     & $w_{2}$ & $w_{0}$ \\
		$d_{14}$                                     & $w_{2}$ & $w_{3}$ \\
		$d_{15}$                                     & $w_{2}$ & $w_{3}$ \\
		$d_{16}$                                     & $w_{3}$ & $w_{3}$ \\
		$d_{17}$                                     & $w_{3}$ & $w_{1}$ \\
		$d_{18}$                                     & $w_{3}$ & $w_{1}$ \\
		$d_{19}$                                     & $w_{3}$ & $w_{2}$ \\
		$d_{20}$                                     & $w_{3}$ & $w_{2}$ \\
		\hline
	\end{tabular}
	\caption{The transformations of length 1 in $\mathscr{W}_{\alpha}$.}
	\label{tab:2x2_gridworld_length_1_transformations}
\end{table}

It's useful to look at the world diagram\footnote{
We have deliberately positioned the vertices of our world diagram where we have to mirror the structure we perceive it to have (e.g., the line from the $w_{0}-w_{1}$ plane is perpendicular to the $w_{0}$-$w_{2}$ plane); however many valid (isomorphic) world diagrams do not have this feature.
For example:
\includegraphics[width=0.5\linewidth]{2MathematicalFramework/Images/2x2_cyclical_min_trans_quirky.jpg}
}
containing the world states and the minimum transformations given in figure \ref{fig:2x2_cyclical_minimum_transformations}, since this displays the transformation structure of $\mathscr{W}_{\alpha}$ in a more intuitive way than giving the definitions of each minimum transformation as shown in $D_{\epsilon}$ and table \ref{tab:2x2_gridworld_length_1_transformations}.

\begin{figure}[H]
	\centering
	\includegraphics[width=0.5\linewidth]{2MathematicalFramework/Images/2x2_cyclical_minimum_transformations.jpeg}
	\caption{
		A world diagram of $\mathscr{W}_{\alpha}$ containing the transformations in $\hat{D}$.
		Transformations in $D_{\epsilon}$ are in red.
	}
	\label{fig:2x2_cyclical_minimum_transformations}
\end{figure}

\draftnote{blue}{awjdean}{Something about the structure of $D$ ?}

%%%%%%%%%%%%%%%%%%%%%%%%%%%%%%%%%%%%%%%%%%%%%%%%%%%%%%%%%%%%%%%%%%%%%%%%%%%%%%%%%%%%%%%%%%%%%%
\subsection{Introducing the agent}

We consider an agent with ideal sensors.
This means there is a one-to-one correspondence between world states visited by the agent and the agent's observations.
Since there are 4 world states, the agent has 4 possible observation states $O = \{ o_{0}, o_{1}, o_{2}, o_{3} \}$ where $b(w_{i}) = o_{i}$ for $i = \{0, 1, 2, 3 \}$.
The agent also has, once it has fully learned the structure of $\mathscr{W}_{\alpha}$, four representation states $Z = \{ z_{0}, z_{1}, z_{2}, z_{3} \}$ where $h(o_{i}) = z_{i}$ for $i = \{0, 1, 2, 3 \}$.
Therefore, we have a bijective map $f = hb$ where $f(w_{i}) = z_{i}$ for $i = \{0, 1, 2, 3 \}$.

\draftnote{blue}{awjdean}{Talk about states not visited by the agent here ?}

%%%%%%%%%%%%%%%%%%%%%%%%%%%%%%%%%%%%%%%%%%%%%%%%%%%%%%%%%%%%%%%%%%%%%%%%%%%%%%%%%%%%%%%%%%%%%%
\subsection{
	Actions of the agent in $\mathscr{W}_{\alpha}$
}\label{sec:Actions of the agent in example}

Our agent has the following minimum actions: $\hat{A} = \{ 1, U, D, L, R \}$ as shown in table \ref{tab:actions_of_agent_2_by_2_cyclical}.
\begin{table}[H]
	\centering
	\begin{tabular}{|c|c|}
		\hline
		Action                    & Element of $\hat{A}$ \\
		\hline
		Move up                   & $U$                  \\
		\hline
		Move down                 & $D$                  \\
		\hline
		Move right                & $R$                  \\
		\hline
		Move left                 & $L$                  \\
		\hline
		Do nothing (no op action) & $1$                  \\
		\hline
	\end{tabular}
	\caption{Minimum actions of agent in $\mathscr{W}_{\alpha}$.}
	\label{tab:actions_of_agent_2_by_2_cyclical}
\end{table}

Earlier we described the world $\mathscr{W}_{\alpha}$ as \emph{cyclical}.
Now we are dealing with actions in our toy example, we can finally state what we mean by a cyclical world.
A world $\mathscr{W} = $ is \emph{cyclical} if for any $a \in A$ and $w \in W$, there exists a positive integer $n \in \mathbb{N}^{+}$ such that $a^{n} \ast w = w$; in other words, a world is cyclical if the world returns to its starting state when any action is performed enough times\footnote{
	If the world returns to its starting state when only certain actions are applied enough times, then we say the world is \emph{cyclical with respect to those actions}.
	Sometimes, cyclical structure might be present for a certain actions (e.g., $a_{1}$, $a_{2}$) starting from certain world states (e.g., $w$, $w'$), but not starting from other world states; in these cases we can say the world is cyclic with respect to $(a_{1}, a_{2}; w, w')$.
}.
For our example world $\mathscr{W}_{\alpha}$, informally, this means that if the embodied agent goes through the top(/bottom/right/left) of side of the world then it will reappear at the bottom(/top/left/right) side.
This behaviour is similar to that found in games like Pac-Man \draftnote{blue}{awjdean}{Reference} or Snake \draftnote{blue}{awjdean}{Reference}.
A more precise example would be that if the embodied agent performs the up action while $\mathscr{W}_{\alpha}$ is in world state $w_{0}$ then $\mathscr{W}_{\alpha}$ will transform into world state $w_{2}$.

Before we label the transformations in $D$ with their associated actions, we must identify which transformations should be labelled and how (i.e, we must identify the elements of the sets $\hat{D}_{A}$ and $D_{A}$).
For our example world $\mathscr{W}_{\alpha}$, the procedure outlined in figure \ref{fig:action_labelling_procedure} in somewhat simplified because all the transformations in $\mathscr{W}_{\alpha}$ are due to the actions of the agent embodied in the world; this means that $\hat{D} = \hat{D}_{A}$\footnote{so $\hat{D}_{A} = \{ d_{1}, d_{2}, d_{3}, \dots, d_{20} \} \cup D_{\epsilon}$.}, and therefore $\mathscr{W} = \mathscr{W}_{A}$ and $D = D_{A}$.

The transformations of $\mathscr{W}_{\alpha}$ due to the agent's minimum actions (i.e., the effect of the minimum actions on the world states $W$) are shown in table \ref{tab:2x2_gridworld_minimum_actions}.
Since all transformations of $\mathscr{W}_{\alpha}$ are due to the actions of the agent, this table shows all the minimum transformations in $\mathscr{W}_{\alpha}$ and how they are labelled using the labelling map $l$.

\begin{table}[H]
	\centering
	\begin{tabular}{c c}
		                    & Minimum action                                                                                                                                                                                                        \\
		Initial world state & \begin{tabular}{c|c c c c c}
			                              & $1$     & $U$     & $D$     & $L$     & $R$     \\
			                      \hline
			                      $w_{0}$ & $w_{0}$ & $w_{2}$ & $w_{2}$ & $w_{1}$ & $w_{1}$ \\
			                      $w_{1}$ & $w_{1}$ & $w_{3}$ & $w_{3}$ & $w_{0}$ & $w_{0}$ \\
			                      $w_{2}$ & $w_{2}$ & $w_{0}$ & $w_{0}$ & $w_{3}$ & $w_{3}$ \\
			                      $w_{3}$ & $w_{3}$ & $w_{1}$ & $w_{1}$ & $w_{2}$ & $w_{2}$ \\
		                      \end{tabular} \\
	\end{tabular}
	\caption{
		Each entry in this table shows the outcome state of the agent performing the action given in the column label when in the world state given by the row label.
		\draftnote{blue}{awjdean}{Improve this caption - talk about the action effect operator ($\ast$).
			Fix centering of "Minimum action" and "Initial world state".}
	}
	\label{tab:2x2_gridworld_minimum_actions}
\end{table}

We can use table \ref{tab:2x2_gridworld_minimum_actions} to construct the labelling map $\hat{l}$ - the construction of $\hat{l}$ and $l$ and the generation of table \ref{tab:2x2_gridworld_minimum_actions} would happen during the agent's learning process.

\begin{table}[H]
	\centering
	\begin{tabular}{|c|cc||c|}
		\hline
		$d \in \hat{D}$ & $s(d)$  & $t(d)$  & $\hat{l}(d)$ \\
		\hline
		$d_{1}$         & $w_{0}$ & $w_{0}$ & $1$          \\
		$d_{2}$         & $w_{0}$ & $w_{1}$ & $R$          \\
		$d_{3}$         & $w_{0}$ & $w_{1}$ & $L$          \\
		$d_{4}$         & $w_{0}$ & $w_{2}$ & $D$          \\
		$d_{5}$         & $w_{0}$ & $w_{2}$ & $U$          \\
		$d_{6}$         & $w_{1}$ & $w_{1}$ & $1$          \\
		$d_{7}$         & $w_{1}$ & $w_{0}$ & $L$          \\
		$d_{8}$         & $w_{1}$ & $w_{0}$ & $R$          \\
		$d_{9}$         & $w_{1}$ & $w_{3}$ & $D$          \\
		$d_{10}$        & $w_{1}$ & $w_{3}$ & $U$          \\
		$d_{11}$        & $w_{2}$ & $w_{2}$ & $1$          \\
		$d_{12}$        & $w_{2}$ & $w_{0}$ & $U$          \\
		$d_{13}$        & $w_{2}$ & $w_{0}$ & $D$          \\
		$d_{14}$        & $w_{2}$ & $w_{3}$ & $R$          \\
		$d_{15}$        & $w_{2}$ & $w_{3}$ & $L$          \\
		$d_{16}$        & $w_{3}$ & $w_{3}$ & $1$          \\
		$d_{17}$        & $w_{3}$ & $w_{1}$ & $U$          \\
		$d_{18}$        & $w_{3}$ & $w_{1}$ & $D$          \\
		$d_{19}$        & $w_{3}$ & $w_{2}$ & $L$          \\
		$d_{20}$        & $w_{3}$ & $w_{2}$ & $R$          \\
		$1_{w_{0}}$     & $w_{0}$ & $w_{0}$ & $\epsilon$   \\
		$1_{w_{1}}$     & $w_{1}$ & $w_{1}$ & $\epsilon$   \\
		$1_{w_{2}}$     & $w_{2}$ & $w_{2}$ & $\epsilon$   \\
		$1_{w_{3}}$     & $w_{3}$ & $w_{3}$ & $\epsilon$   \\
		\hline
	\end{tabular}
	\caption{
		Labelling map $\hat{l}$ for $\mathscr{W}_{\alpha}$.
	}
	\label{tab:2x2_cyclical_labelling_with_min_actions}
\end{table}

The application of the labelling map $\hat{l}$ as a world diagram is shown in figure \ref{fig:2x2_cyclical_labelling_with_min_actions}\footnote{
	It's important to note that the no-op action $1$, where the agent does nothing, does not label the trivial transformations $1_{w}$, which are labelled by the empty action $\epsilon$.
	This means the no-op action does not act as an identity element for $(\hat{A}\ast, \circ)$.
}.

\begin{figure}[H]
	\centering
	\includegraphics[width=1\linewidth]{2MathematicalFramework/Images/2x2_cyclical_labelling_with_min_actions.png}
	\caption{
		World diagram for $\mathscr{W}_{\alpha}$ showing how $\hat{l}$ labels the minimum action transformations.
	}
	\label{fig:2x2_cyclical_labelling_with_min_actions}
\end{figure}

Remember, figure \ref{fig:2x2_cyclical_labelling_with_min_actions} only shows the minimum action transformations in $\hat{D}_{A}$ labelled by there associated minimum actions in $\hat{D}$ - we only show these transformations for simplicity.

When considering the full transformation structure of $\mathscr{W}_{\alpha}$, there will be transformations between any two world states; for example, $d_{9} \circ d_{1}: w_{0} \to w_{3}$.
In fact, there are a (countably) infinite number of transformations in $D$ between any two world states; for example, $D \circ R$, $D \circ R \circ 1^{n}$ ($n \in \mathbb{N}$), $1^{n} \circ D \circ R$ ($n \in \mathbb{N}$), $D \circ R \circ (L \circ R)^{n}$ ($n \in \mathbb{N}$) etc...
Similarly, there are a (countably) infinite number of actions in $A$ between any two world states\footnote{Although the set $A$ is less dense or as dense as $D$. \draftnote{blue}{awjdean}{Explore this further in appendices.}} - just take the actions in $A$ that labels each of the transformations in $D$ we just described.

%%%%%%%%%%%%%%%%%%%%%%%%%%%%%%%%%%%%%%%%%%%%%%%%%%%%%%%%%%%%%%%%%%%%%%%%%%%%%%%%%%%%%%%%%%%%%%
\draftnote{green}{To do}{
	\begin{enumerate}
		\item Show parts of: $\hat{A}^{\ast}$, $D$
		\item Summary at end: State what each of the things are shown in figure \ref{fig:action_labelling_procedure}.
		\item In section introduction --> say toy example inspired by Higgins (2018) + Caselles Dupres (?)
		\item Show example of minimum transformations in $\hat{D}_{A}$ - example with intermediate world states that are skipped over by transformations due to actions of agent.
		\item Change symbol for $\mathscr{W}_{\alpha}$ to include agent ? -- add agent to $(W, \hat{D}. s, t)$ ?
		\item Label the transformation diagrams with their symbols ($d_{1}$, $d_{2}$ etc...) ?
		\item Switch $U$, $D$, $R$, $L$ into $N$, $S$, $E$, $W$ --> makes sense when giving agent ability to rotate. What to do about $W$ also being the symbol for the set of world states ?
		      \begin{compactitem}
			      \item When doing the 'rotate 90 degrees and move forwards" agent, need to talk about how we usually only consider the transinformation of the world (arrows) for our specific agent, but in reality there are all possible transformations ? (otherwise $D$ looks different for agents with different actions - is that necessarily a bad thing or is it what we want ?)
		      \end{compactitem}
		\item \textbf{Appendices:} Size of $D$ and $A$: $D$ and $A$ countably infinite, $A$ more sparse than $D$.
		      \begin{compactitem}
			      \item Proof that $D$ and $A$ are countably infinite.
			      \item \textbf{Mention:} We only consider finite walks because we were worried that including infinite walks would complicate the work; how including infinite walks would affect things is a potential direction for future study. Out guess is that including infinite walks should not affect things.
			      \item Consider "Directed Walks $\&$ density of A vs D".
			      \item Asymptotic density of A vs D --> for $\mathscr{W}_{\alpha}$, I think it might be $\rho_{D}(A) = \frac{1}{5}$ because there are 4 elements in $\hat{A}$.
		      \end{compactitem}
		\item Example.
		      \begin{compactitem}
			      \item Diagram (l-map-for-transformations-from-and-to-w0).
			      \begin{compactitem}
				      \item This diagram shows how the minimum transformations $d$ with $d(s) = w_{0}$ or $d(t)= w_{0}$ are mapped to actions using the map $l$.
				      \item Figure ref[fig:l-map-for-transformations-from-and-to-w0] demonstrates how the map $l$ maps transformations in $D$ to the elements in $A$ for the transformations shown in Figure ref[fig:2x2-cyclical-min-actions-standard].
			      \end{compactitem}
		      \end{compactitem}
	\end{enumerate}
}

\draftnote{blue}{Questions/thoughts that have been answered}{
	\begin{enumerate}
		\item Are the elements $1_{w}$ in $\hat{D}$ ? At the moment - yes.
		      \begin{compactitem}
			      \item \textbf{Counter:} Making $\hat{D}$ not include $D_{\epsilon}$ would line the notation up with $\hat{A}$.
		      \end{compactitem}
	\end{enumerate}
}

%%%%%%%%%%%%%%%%%%%%%%%%%%%%%%%%%%%%%%%%%%%%%%%%%%%%%%%%%
\section{Structure through equivalence}
%%%%%%%%%%%%%%%%%%%%%%%%%%%%%%%%%%%%%%%%%%%%%%%%%%%%%%%%%
\subsection{Motivation}

We now have \draftnote{blue}{awjdean}{Improve this - include stuff like the set $\mathcal{T}$.}
\begin{enumerate}[(1)]
	\item an infinite set $D$ of transformations, which are walks in the directed multigraph $\mathscr{W}=(W, \hat{D}, s, t)$;

	\item an infinite set $D_{A}$ of action transformations, which are walks in the directed multigraph $\mathscr{W}_{A}=(W, \hat{D}_{A}, s, t)$;

	\item an infinite, but less dense, set $\hat{A}^{\ast}$ of actions;

	\item a labelling map $l: D_{A} \to \hat{A}^{\ast}$ that allows us to construct an infinite number of functions $f_{a}$ that describe the effect of the actions in $\hat{A}^{\ast}$; and

	\item a free monoid algebra $(\hat{A}^{\ast}, \; \circ)$.
\end{enumerate}


Our basic algebra of actions $(\hat{A}^{\ast}, \; \circ)$ is a (countably infinite) free monoid; this is the same for any world.
However, we expect that there must be more structure we can tease out of our $(\hat{A}^{\ast}, \; \circ)$ since it appears that different worlds have different transformation structures.
But how do we produce those structures from $(\hat{A}^{\ast}, \; \circ)$ ?
If we want to form other recognisable algebraic structures (e.g., groups, semigroups etc...), we need some way of saying collections of transformations (i.e., the actions that are elements of $\hat{A}^{\ast}$) are the same in some way, otherwise, there is no way to satisfy the requirements of these algebras (\textit{e.g.}, for the inverse property we need an element $a'$ in $\hat{A}^{\ast}$ for each $a \in \hat{A}^{\ast}$ such that $a' \circ a = \epsilon$ and $a \circ a' = \epsilon$).

To solve this issue and to enforce the structure of the world on the algebra $(\hat{A}^{\ast}, \; \circ)$, we will define an equivalence relation $\sim$ on the elements of $\hat{A}^{\ast}$.
This equivalence relation will enforce aspects of the structure of the world onto the free monoid $(\hat{A}^{\ast}, \; \circ)$ to give us a new algebra $(\hat{A}^{\ast}/\sim, \; \circ_{\sim})$ \draftnote{blue}{awjdean}{Do we actually need congruences?}, where it is now possible to satisfy properties like the inverse condition.

Additionally, considering the learning problem of the agent, as it stands, our agent would need to learn an infinite number of transformations (the elements of $D$) each of which is specific to individual world states or the effect of an infinite number of actions to learn the structure of the world (the structure of $(\hat{A}^{\ast}, \circ, \ast)$).
The structure of the new algebra $(\hat{A}^{\ast}/\sim, \; \circ_{\sim})$ should be easier for an agent to learn than learning every element in an infinite set.
We also believe agents can gain a better understanding of the structure of the world though learning the structure of the algebras we will construct using equivalence relations \draftnote{blue}{awjdean}{
	We actually think this is what agent's do when they learn the structure of the world
}.

\footnote{
	When we define an equivalence $\sim$ on $\hat{A}^{\ast}$ to get the set $\hat{A}^{\ast}/\sim$, we can think of our set $\hat{A}^{\ast}$ `folding over' on itself to make elements of $\hat{A}^{\ast}$ that are equivalent stick together.
}



\whendraft{
	\draftnote{red}{DIVIDER}{}
	\draftnote{blue}{awjdean}{Do something with this - something about how extracting the global algebra provides generalisation potential.
	}
	We can also describe the effect $\ast: \hat{A}^{\ast} \times W \to W$ of the actions of the agent on the world $\mathscr{W} = (W, \hat{D}, s, t)$ as the action of the (global) algebra $(\hat{A}^{\ast}, \circ)$ on the set $W$.
	If such a global algebra exists, then it is possible for our agent to learn the structure of $(\hat{A}^{\ast}, \circ)$ and so understand the transformation structure of its actions without considering its current world state; this could have powerful generalisation capabilities.
	\draftnote{blue}{awjdean}{
		I currently think that a global algebra does always exist, but that sometimes it's horrible and doesn't really abstract away the effect of the actions from the world states since it requires a map that connects to the world states to tell the algebra on which world states things are and aren't defined.
	}
	\draftnote{red}{DIVIDER}{}
}

%%%%%%%%%%%%%%%%%%%%%%%%%%%%%%%%%%%%%%%%%%%%%%%%%%%%%%%%%
\subsection{An equivalence relation: $\sim$}

We define an equivalence relation on the elements of $\hat{A}^{\ast}$ that says two actions are equivalent (our sense of the actions being the same) if they lead to the same end world state when performed in any initial world state. \draftnote{blue}{awjdean}{Footnote about how this was inspired by \autocite{caselles2020sensory} is our mathematical interpretation of equality in \autocite{Higgins2018}.}

\begin{definition}[Equivalence of actions under $\sim$]
	Given two actions $a, a' \in \hat{A}^{\ast}$,
	\begin{equation}
		a \sim a' \iff a \ast w = a' \ast w \quad \text{ for all $w \in W$}
	\end{equation}
\end{definition}

\draftnote{blue}{awjdean}{Footnote about this being true for actions leading to the undefined state (and why).}

\begin{proposition}
	$\sim$ is an equivalence relation on $\hat{A}^{\ast}$.
\end{proposition}
\begin{proof}
	\draftnote{blue}{awjdean}{Check/improve this.}
	\textbf{Reflexive.}
	If $a \sim a'$ then $a \ast w = a' \ast w$ for all $w \in W$, and so $a \sim a$.

	\textbf{Transitive.}
	If $a \sim a'$ and $a' \sim a''$, then $a \ast w = a' \ast w$ for all $w \in W$ and $a' \ast w = a'' \ast w$ for all $w \in W$.
	Therefore, $a \ast w = a'' \ast w$ for all $w \in W$ and so $a \sim a''$.

	\textbf{Symmetric.}
	If $a \sim a'$, then $a \ast w = a' \ast w$ for all $w \in W$.
	Therefore $a' \ast w = a \ast w$ for all $w \in W$, and so $a' \sim a$.
\end{proof}

We define the canonical projection map $\pi_{A}: \hat{A}^{\ast} \to \hat{A}^{\ast}/\sim$ that sends actions in $\hat{A}^{\ast}$ to their equivalence classes under $\sim$ in the quotient set $\hat{A}^{\ast}/\sim$:
\begin{equation}
	\pi_{A}: \hat{A}^{\ast} \to \hat{A}^{\ast}/\sim \text{ where } \pi_{A}(a) = [a]_{\sim}
\end{equation}
This projection map $\pi_{A}$ effectively collapses actions with identical effects into a single representative.
The equivalent classes $[a]_{\sim}$, where $[a]_{\sim}$ denotes the equivalence class of $a$ due to $\sim$, are
\begin{equation}
	[a]_{\sim} = \{ a' \in \hat{A}^{\ast} \mid a' \sim a \}
\end{equation}

Sometimes we will drop the $[a]_{\sim}$ in favour of $a \in \hat{A}^{\ast}/\sim$ for ease.

\paragraph{Composition of actions}
We define the composition of elements in $\hat{A}^{\ast}/\sim$ using our action composition operator $\circ$ as
\begin{equation}
	\begin{aligned}
		 & \circ_{\sim}: (\hat{A}^{\ast}/\sim) \times (\hat{A}^{\ast}/\sim) \to (\hat{A}^{\ast}/\sim) \quad \text{such that} \\
		 & [a']_{\sim} \circ_{\sim} [a]_{\sim} = [a' \circ a]_{\sim} \quad \text{for $a,a' \in \hat{A}^{\ast}$}
	\end{aligned}
\end{equation}


We need to show that the operation $\circ_{\sim}$ on the equivariance classes does not depend on the representative element chosen for those classes; this means we can consistently use any element of an equivalence class as a representative of that class.

\begin{proposition}\label{prp:circ_sim_well_defined}
	$\circ_{\sim}$ is well-defined on $\hat{A}^{\ast}/\sim$.
\end{proposition}
\begin{proof}
	\begin{enumerate}[(1)]
		\item \textbf{Problem.}
		      For $\circ_{\sim}$ to be well-defined on $\hat{A}^{*}/\sim$, we need that\footnote{
			      \textbf{Well-definedness requirement.}
			      An operation on a quotient set
			      \begin{equation}
				      (\hat{A}^{*}/\sim) \times (\hat{A}^{*}/\sim) \to \hat{A}^{*}/\sim
			      \end{equation}
			      is \emph{well-defined} if take different representatives of the same equivalence classes doesn't change the resulting equivalence class.
			      Mathematically,
			      \begin{equation}
				      \begin{aligned}
					               & [a]_{\sim} = [b]_{\sim} \; \text{and} \; [a']_{\sim} = [b']_{\sim} \\
					      \implies & [a' \circ a]_{\sim} = [b' \circ b]_{\sim}
				      \end{aligned}
			      \end{equation}
		      }
		      \begin{equation}
			      \begin{aligned}
				       & \text{If $[a]_{\sim} = [b]_{\sim}$ and $[a']_{\sim} = [b']_{\sim}$, then} \\
				       & [a' \circ a]_{\sim} = [b' \circ b]_{\sim}
			      \end{aligned}
		      \end{equation}

		\item \textbf{Initial assumptions.}
		      For $a, b, a', b' \in \hat{A}^{\ast}$:
		      \begin{itemize}
			      \item $a \sim b$ means:
			            \begin{equation}
				            a \ast w = b \ast w \quad \text{for all } w\in W
			            \end{equation}
			      \item $a' \sim b'$ means:
			            \begin{equation}
				            a' \ast w = b' \ast w \quad \text{for all } w\in W
			            \end{equation}
		      \end{itemize}

		\item \textbf{Goal.}
		      We want to show $[a' \circ a]_{\sim} = [b' \circ b]_{\sim}$.
		      \begin{align}
			                    & [a' \circ a]_{\sim} = [b' \circ b]_{\sim}                               \\
			      \Rightarrow{} & (a' \circ a) \sim (b' \circ b)                                          \\
			      \Rightarrow{} & (a' \circ a) \ast w = (b' \circ b) \ast w \quad \text{for all } w \in W
		      \end{align}

		\item \textbf{Proof.}
		      We have:
		      \begin{equation}
			      (a' \circ a) \ast w = a' \ast (a \ast w).
		      \end{equation}
		      Similarly, we have:
		      \begin{equation}
			      (b' \circ b) \ast w = b' \ast (b \ast w).
		      \end{equation}
		      Since $a \sim b$, we have $a \ast w = b \ast w$ for all $w \in W$.
		      Therefore,
		      \begin{align}
			          & (a' \circ a) \ast w \\
			      ={} & a' \ast (a \ast w)  \\
			      ={} & a' \ast (b \ast w)
		      \end{align}
		      Since $a' \sim b'$, we have $a' \ast w = b' \ast w$ for all $w \in W$.
		      Therefore,
		      \begin{align}
			          & (a' \circ a) \ast w  \\
			      ={} & b' \ast (b \ast w)   \\
			      ={} & (b' \circ b) \ast w.
		      \end{align}


		\item \textbf{Conclusion.}
		      We have shown that $(a' \circ a) \ast w = (b' \circ b) \ast w$, and therefore $(a' \circ a) \sim (b' \circ b)$, and hence $\circ_{\sim}$ is well-defined on $\hat{A}^{\ast}/\sim$.
	\end{enumerate}
\end{proof}

\begin{proposition}\label{prp:circ_sim_closed}
	$\circ_{\sim}: (\hat{A}^{\ast}/\sim) \times (\hat{A}^{\ast}/\sim) \to \hat{A}^{\ast}/\sim$ is closed.
\end{proposition}
\begin{proof}
	$\circ_{\sim}$ is closed $\circ_{\sim}$ is the images under $\pi_{A}$ of the concatenation operation $\circ$ in $\hat{A}^{\ast}$, and all elements of $\hat{A}^{\ast}$ have equivalence classes in $\hat{A}^{\ast}/\sim$.
\end{proof}


Now we have proved that $\circ_{\sim}$ is well-defined on $\hat{A}^{\ast}/\sim$ we denote $\circ_{\sim}$ by $\circ$ when it is obvious what we mean\footnote{$\circ_{\sim}$ can be thought of as the composition operator $\circ$ after it has been pulled through the map $\pi_{A}$.}.

\paragraph{Effect of equivalent actions on world states.}
We define the effect of an element of  $\hat{A}^{\ast}/\sim$ on world states as
\begin{equation}
	\begin{aligned}
		 & \ast_{\sim}: (\hat{A}^{\ast}/\sim) \times W \to W \quad \text{such that} \\
		 & [a]_{\sim} \ast_{\sim} w = a \ast w
	\end{aligned}
\end{equation}

\begin{proposition}
	$\ast_{\sim}$ is well-defined on $\hat{A}^{\ast}/\sim$.
\end{proposition}
\begin{proof}
	To prove that $\ast_{\sim}$ is well-defined, we need to show that if two actions $a, b \in \hat{A}^{\ast}$ are equivalent (i.e., $a \sim b$), then they have the same effect on any world state $w$.

	\begin{enumerate}[(1)]
		\item \textbf{Initial assumption.}
		      Let $a \sim b$.
		      By definition of the equivalence relation $\sim$, we have
		      \begin{align}
			       & a \ast w = b \ast w \quad \text{for all } w \in W
		      \end{align}

		\item \textbf{Definition of $\ast_{\sim}$.}
		      By definition, the effect of $[a]_{\sim}$ on a world $w$ is given by
		      \begin{equation}
			      [a]_{\sim} \ast_{\sim} w = a \ast w
		      \end{equation}
		      Similarly, the effect of $[b]_{\sim}$ on a world $w$ is given by
		      \begin{equation}
			      [b]_{\sim} \ast_{\sim} w = b \ast w
		      \end{equation}

		\item \textbf{Goal.}
		      We want to prove that
		      \begin{equation}
			      [a]_{\sim} \ast_{\sim} w = [b]_{\sim} \ast_{\sim} w
		      \end{equation}
		      Which is equivalent to proving that:
		      \begin{equation}
			      a \ast w = b \ast w.
		      \end{equation}

		\item \textbf{Proof.}
		      Since $a \sim b$, we have $a \ast w = b \ast w$ by the definition of $\sim$.

		\item \textbf{Conclusion.}
		      We have shown that $a \ast w = b \ast w$.
		      This implies that
		      \begin{equation}
			      [a]_{\sim} \ast_{\sim} w = [b]_{\sim} \ast_{\sim} w
		      \end{equation}
		      Therefore, $\ast_{\sim}$ is well-defined on $\hat{A}^{\ast}/\sim$.
	\end{enumerate}
\end{proof}

Now we have proved that $\ast_{\sim}$ is well-defined on $\hat{A}^{\ast}/\sim$ we denote $\ast_{\sim}$ by $\ast$ where it is obvious what we mean\footnote{$\ast_{\sim}$ can be thought of as the composition operator $\ast$ after it has been pulled through the map $\pi_{A}$.}.

%%%%%%%%%%%%%%%%%%%%%%%%%%%%%%%%%%%%%%%%%%%%%%%
\subsection{New structures from $\sim$}

Applying our projection map $\pi_{A}$ to the structures $(\hat{A}^{\ast}, \circ)$, $(\hat{A}^{\ast}, \circ, \ast)$, and $\mathcal{T}_{\hat{A}^{\ast}}$ gives us these new structures:
\begin{align}
	 & \pi_{A}(\text{ }(\hat{A}^{\ast}, \circ)\text{ }) = (\hat{A}^{\ast}/\sim, \circ_{\sim})                    \\
	 & \pi_{A}(\text{ }(\hat{A}^{\ast}, \circ, \ast)\text{ }) = (\hat{A}^{\ast}/\sim, \circ_{\sim}, \ast_{\sim}) \\
	 & \pi_{A}(\text{ }\mathcal{T}_{\hat{A}^{\ast}}\text{ }) = \mathcal{T}_{\hat{A}^{\ast}/\sim}
\end{align}
\footnote{Technically $\pi_{A}(\text{ }\mathcal{T}_{\hat{A}^{\ast}}\text{ }) = \mathcal{T}_{\hat{A}^{\ast}/\sim}$ implies that the functions $f_{a}$ induced by individual actions $a \in \hat{A}^{\ast}$ correspond exactly to the functions $f_{[a]_{\sim}}$ induced by the equivalence classes in $\hat{A}^{\ast}/\sim$.
We will prove this soon.}


\subsection{$(\hat{A}^{\ast}/\sim, \circ_{\sim})$.}
Applying the equivalence relation $\sim$ using the projection $\pi_{A}$ to the free monoid $(\hat{A}^{\ast}, \circ)$ gives the quotient monoid $(\hat{A}^{\ast}/\sim, \circ_{\sim})$\footnote{
	$(\hat{A}^{\ast}/\sim, \circ_{\sim})$ is the homomorphic image of $(\hat{A}^{\ast}, \circ)$ through the projection $\pi_{A}$.
}, whose elements are the equivalence classes $[a]_{\sim}$ for all $a \in \hat{A}^{\ast}$.

\begin{propositionE}\label{prp:circ_sim_associative}
	$\circ_{\sim}$ is associative on $\hat{A}^{\ast}/\sim$.
\end{propositionE}
\begin{proofE}
	Associativity of $\circ_{\sim}$ comes from the associativity of $\circ$.
	\begin{align}
		     & ([a'']_{\sim} \circ_{\sim} [a']_{\sim}) \circ_{\sim} [a]_{\sim} \\
		= \; & [a'' \circ a']_{\sim} \circ_{\sim} [a]_{\sim}                   \\
		= \; & [ (a'' \circ a') \circ a ]_{\sim}                               \\
		= \; & [ a'' \circ (a' \circ a) ]_{\sim}                               \\
		= \; & [a'']_{\sim} \circ_{\sim} [a' \circ a]_{\sim}                   \\
		= \; & [a'']_{\sim} \circ_{\sim} ([a']_{\sim} \circ_{\sim} [a]_{\sim})
	\end{align}
\end{proofE}


\begin{proposition}\label{prp:A_sim_identity}
	$[\epsilon]_{\sim}$ is the identity element of $(\hat{A}^{\ast}/\sim, \; \circ_{\sim})$.
\end{proposition}
\begin{proof}
	For any $[a]_{\sim} \in \hat{A}^{\ast}/\sim$, we have
	\begin{align}
		 & [\epsilon]_{\sim} \circ_{\sim} [a]_{\sim} = [\epsilon \circ a]_{\sim} = [a]_{\sim}
		 & [a]_{\sim} \circ_{\sim} [\epsilon]_{\sim} = [a \circ \epsilon]_{\sim} = [a]_{\sim}
	\end{align}
	Therefore, $[\epsilon]_{\sim}$, where $\epsilon$ is the empty action, acts as the identity.
\end{proof}


\begin{proposition}\label{prp:A_sim_is_monoid}
	$(\hat{A}^{\ast}/\sim, \circ_{\sim})$ is a monoid.
\end{proposition}
\begin{proof}
	\begin{enumerate}[(1)]
		\item \textbf{Closure and well-definedness.}
		      We need to show that the operation $\circ_{\sim}$ is well-defined and closed on $\hat{A}^{\ast}/\sim$.
		      $\circ_{\sim}$ is well-defined on $\hat{A}^{\ast}/\sim$ from proposition \ref{prp:circ_sim_well_defined} and closed from proposition \ref{prp:circ_sim_closed}.

		\item \textbf{Associativity.}
		      Given by proposition \ref{prp:circ_sim_associative}.

		\item \textbf{Identity Element.}
		      Given by proposition \ref{prp:A_sim_identity}.

		\item \textbf{Conclusion.}
		      The structure $(\hat{A}^{\ast}/\sim, \circ_{\sim})$ is a monoid.
	\end{enumerate}
\end{proof}

Now can now show that $\pi_{A}$ preserves the monoid structure of $(\hat{A}^{\ast}, \circ)$, and so $\pi_{A}$ is a monoid homomorphism.
\begin{proposition}
	$\pi_{A}$ is a monoid homomorphism from $(\hat{A}^{\ast}, \circ)$ to $(\hat{A}^{\ast}/\sim, \circ_{\sim})$.
\end{proposition}
\begin{proof}
	\begin{enumerate}[(1)]
		\item \textbf{Preservation of the Operation.}
		      For any $a, a' \in \hat{A}^{\ast}$, the projection $\pi_{A}$ satisfies:
		      \begin{align}
			      \pi_{A}(a \circ a') & = [a \circ a']_{\sim}                 \\
			                          & = [a]_{\sim} \circ_{\sim} [a']_{\sim} \\
			                          & = \pi_{A}(a) \circ_{\sim} \pi_{A}(a')
		      \end{align}
		      Thus:
		      \begin{equation}
			      \pi_{A}(a \circ a') = \pi_{A}(a) \circ_{\sim} \pi_{A}(a').
		      \end{equation}

		\item \textbf{Preservation of the Identity Element.}
		      The identity element in $(\hat{A}^{\ast}, \circ)$ is $\epsilon$, and the identity element in $(\hat{A}^{\ast}/\sim, \circ_{\sim})$ is $[\epsilon]_{\sim}$.
		      The projection satisfies:
		      \begin{equation}
			      \pi_{A}(\epsilon) = [\epsilon]_{\sim}.
		      \end{equation}

		\item \textbf{Conclusion.}
		      The projection $\pi_{A}$ is a monoid homomorphism.
	\end{enumerate}
\end{proof}

If the number of distinct effects on $W$ is finite, then $(\hat{A}^{\ast}/\sim, \circ_{\sim})$ will have a finite number of elements.

%%%%%%%%%%%%%%%%%%%%%%%%%%%%%%%%%%%%%%%%%%%
\subsection{$(\hat{A}^{\ast}/\sim, \circ_{\sim}, \ast_{\sim})$.}

\begin{proposition}
	$(\hat{A}^{\ast}/\sim, \circ_{\sim}, \ast_{\sim})$ is a monoid action of $(\hat{A}^{\ast}/\sim, \circ_{\sim})$ on the set $W$.
\end{proposition}

\begin{proof}
	\begin{enumerate}[(1)]
		\item \textbf{Compatibility.}
		      The action effect operator $\ast_{\sim}$ is compatible with the composition operator $\circ_{\sim}$ since:
		      \begin{equation}
			      ([a']_{\sim} \circ_{\sim} [a]_{\sim}) \ast_{\sim} w = [a']_{\sim} \ast_{\sim} ([a]_{\sim} \ast_{\sim} w) \quad \text{for all } [a]_{\sim}, [a']_{\sim} \in \hat{A}^{\ast}/\sim, \, w \in W.
		      \end{equation}

		      \textit{Proof:} By the definitions of $\circ_{\sim}$ and $\ast_{\sim}$, we have:
		      \begin{align}
			      ([a']_{\sim} \circ_{\sim} [a]_{\sim}) \ast_{\sim} w & = [a' \circ a]_{\sim} \ast_{\sim} w                   \\
			                                                          & = (a' \circ a) \ast w                                 \\
			                                                          & = a' \ast (a \ast w)                                  \\
			                                                          & = [a']_{\sim} \ast_{\sim} ([a]_{\sim} \ast_{\sim} w).
		      \end{align}
		      Therefore, the compatibility condition holds.

		\item \textbf{Identity action.}
		      The identity element $[\epsilon]_{\sim}$ acts as the identity on $W$:
		      \begin{equation}
			      [\epsilon]_{\sim} \ast_{\sim} w = w \quad \text{for all } w \in W.
		      \end{equation}

		      \textit{Proof:} Using the definition of $\ast_{\sim}$ and knowing that $\epsilon$ is the empty action sequence, we get:
		      \begin{align}
			      [\epsilon]_{\sim} \ast_{\sim} w & = \epsilon \ast w \\
			                                      & = w.
		      \end{align}
		      Thus, $[\epsilon]_{\sim}$ acts as the identity on $W$.

		\item \textbf{Conclusion.}
		      $(\hat{A}^{\ast}/\sim, \circ_{\sim}, \ast_{\sim})$ is a monoid action on the set $W$.
	\end{enumerate}
\end{proof}


%%%%%%%%%%%%%%%%%%%%%%%%%%%%%%%%%%%%%%%%%%%%%%%%%%
\subsection{$\mathcal{T}_{\hat{A}^{\ast}/\sim}$}

To explore the effect of applying the equivalence relation $\sim$ to the functions $f_{a} \in \mathcal{T}_{\hat{A}^{\ast}}$, we must first find the equivalent to applying $\sim$ on the functions $f_{a}$.

\begin{proposition}\label{prp:equivalence_equivalence_on_functions}
	$a \sim a' \iff f_{a}(w) = f_{a'}(w)$ for all $w \in W$.
\end{proposition}
\begin{proof}
	\begin{enumerate}[(1)]
		\item \textbf{$a \sim a' \implies f_{a}(w) = f_{a'}(w)$ for all $w \in W$.}
		      Let $a \sim a'$. By the definition of $\sim$, this means:
		      \begin{equation}
			      a \ast w = a' \ast w \quad \text{for all } w \in W.
		      \end{equation}
		      The functions $f_{a}$ and $f_{a'}$ are defined as:
		      \begin{equation}
			      f_{a}(w) = a \ast w, \quad f_{a'}(w) = a' \ast w.
		      \end{equation}
		      From the assumption $a \ast w = a' \ast w$, we have:
		      \begin{equation}
			      f_{a}(w) = f_{a'}(w) \quad \text{for all } w \in W.
		      \end{equation}
		      Therefore, the functions $f_{a}$ and $f_{a'}$ are identical:
		      \begin{equation}
			      f_{a} = f_{a'}.
		      \end{equation}

		\item \textbf{$f_{a}(w) = f_{a'}(w)$ for all $w \in W$ $\implies$ $a \sim a'$.}
		      The functions $f_{a}$ and $f_{a'}$ are defined as:
		      \begin{equation}
			      f_{a}(w) = a \ast w, \quad f_{a'}(w) = a' \ast w.
		      \end{equation}
		      Since $f_{a}(w) = f_{a'}(w)$, it follows that:
		      \begin{equation}
			      a \ast w = a' \ast w \quad \text{for all } w \in W.
		      \end{equation}
		      By the definition of $\sim$, we conclude that:
		      \begin{equation}
			      a \sim a'
		      \end{equation}

		\item \textbf{Conclusion.}
		      Combining both directions, we have:
		      \begin{equation}
			      a \sim a' \iff f_{a}(w) = f_{a'}(w) \quad \text{for all } w \in W.
		      \end{equation}
	\end{enumerate}
\end{proof}

\draftnote{blue}{awjdean}{Does potential redundancy mean that $\{ f_{a} \mid a \in \hat{A}^{\ast} \}$ does not necessarily form a monoid ?}

As before with $\ast$, we can curry the action $\ast_{\sim}: (\hat{A}^{\ast}/\sim) \times W \to W$ to give a set of functions $f_{[a]_{\sim}}$.
We denote this set by $\mathcal{T}_{\hat{A}^{\ast}/\sim}$:
\begin{equation}
	\begin{aligned}
		 & \mathcal{T}_{\hat{A}^{\ast}/\sim} = \{ f_{[a]_{\sim}} \mid a \in \hat{A}^{\ast} \} \quad \text{where} \\
		 & f_{[a]_{\sim}}(w) = [a]_{\sim} \ast w = a \ast w \quad \text{for $a \in \hat{A}^{\ast}$}
	\end{aligned}
\end{equation}

The set $\mathcal{T}_{\hat{A}^{\ast}/\sim}$ is the set $\mathcal{T}_{\hat{A}^{\ast}}$ but with identical functions in $\mathcal{T}_{\hat{A}^{\ast}}$ now collapsed into a single representative in $\mathcal{T}_{\hat{A}^{\ast}/\sim}$.

\begin{proposition}
	$\mathcal{T}_{\hat{A}^{\ast}/\sim}$ is a monoid.
\end{proposition}
\begin{proof}
	The proof is trivial: $\mathcal{T}_{\hat{A}^{\ast}}$ is a monoid from proposition \ref{prp:T_is_monoid}.
	From proposition \ref{prp:equivalence_equivalence_on_functions}, the application of $\sim$ just means that functions where $f_{a}=f_{a'}$ are represented by a single function $f_{[a]_{\sim}}$ in $\mathcal{T}_{\hat{A}^{\ast}/\sim}$, which does not affect the properties of a monoid.
\end{proof}

Similarly to what we did with $\phi: \hat{A}^{\ast} \to \mathcal{T}_{\hat{A}^{\ast}}$, we define a map
\begin{equation}
	\begin{aligned}
		 & \Phi : \hat{A}^{\ast}/\sim \to \text{End}(W) \quad \text{where} \\
		 & \Phi([a]_{\sim}) = f_{[a]_{\sim}}
	\end{aligned}
\end{equation}
\footnote{$\text{End}(W)$ is the set of functions from $W$ to itself.
	Functions from a set to itself are called \emph{endomorphisms}.}

\begin{proposition}
	$\Phi$ is an injective homomorphism. \draftnote{blue}{awjdean}{Put $[]_{\sim}$'s in.}
\end{proposition}
\begin{proof}
	\begin{enumerate}[(1)]
		To prove that $\Phi$ is an injective homomorphism we need to show that $\Phi$ is injective and a homomorphism.
		\item \textbf{Homomorphism.}
		      \begin{enumerate}[(a)]
			      \item \textbf{Goal.}
			            We want
			            \begin{equation}
				            \Phi([a']_\sim \circ_\sim [a]_\sim) = \Phi([a']_\sim) \circ \Phi([a]_\sim) \quad \textit{for all $[a]_\sim, [a']_\sim \in \hat{A}^{\ast} / \sim$}
			            \end{equation}
			      \item \textbf{Proof.}
			            \begin{itemize}
				            \item LHS:
				                  \begin{equation}
					                  \Phi([a']_\sim \circ_\sim [a]_\sim) = \Phi([a' \circ a]_\sim) = f_{[a' \circ a]}
				                  \end{equation}
				            \item RHS:
				                  \begin{equation}
					                  \Phi([a']_\sim) \circ \Phi([a]_\sim) = f_{[a']} \circ f_{[a]}
				                  \end{equation}
				            \item Proof that $f_{[a' \circ a]} = f_{[a']} \circ f_{[a]}$.
				                  For any $w \in W$:
				                  \begin{align}
					                  (f_{[a']} \circ f_{[a]})(w) & = f_{[a']}(f_{[a]}(w)) \\
					                                              & = f_{[a']}(a \ast w)   \\
					                                              & = a' \ast (a \ast w).  \\
					                  f_{[a' \circ a]}(w)         & = (a' \circ a) \ast w  \\
					                                              & = a' \ast (a \ast w).
				                  \end{align}
				            \item Therefore, $\Phi$ preserves the (monoid) operation and so is a (monoid) homomorphism.
			            \end{itemize}
		      \end{enumerate}

		\item \textbf{Injective.}
		      \begin{enumerate}[(a)]
			      \item \textbf{Goal.}
			            We want to show that:
			            \begin{equation}
				            \text{If } \Phi([a]_{\sim}) = \Phi([a']_{\sim}) \text{, then } [a]_{\sim} = [a']_{\sim}
			            \end{equation}

			      \item \textbf{Proof.}
			            \begin{itemize}
				            \item Suppose $\Phi([a]_\sim) = \Phi([a']_\sim)$ (i.e., $f_{[a]} = f_{[a']}$).
				            \item From \ref{prp:equivalence_equivalence_on_functions}, for all $w \in W$:
				                  \begin{equation}
					                  f_{[a]}(w) = f_{[a']}(w) \implies a \ast w = a' \ast w
				                  \end{equation}
				            \item Therefore, $a \ast w = a' \ast w$ for all $w \in W$, which means $a \sim a'$, and so $[a]_{\sim} = [a']_{\sim}$.
				            \item Therefore, $\Phi$ is injective.
			            \end{itemize}
		      \end{enumerate}
	\end{enumerate}
\end{proof}

\draftnote{blue}{awjdean}{
	Cayley's theorem extends to monoids in that every monoid can be embedded into the monoid of functions on itself.
	Does this mean we can represent states $w \in W$ by the action that reaches them like with groups? Watch Cayley diagram vid !
}

We can now use the infectivity of $\Phi$ to establish an upper bound on the number of elements in $\hat{A}^{\ast}/\sim$\footnote{The number of elements in a set $S$ is denoted by $|S|$.}.

\begin{proposition}
	$|\hat{A}^{\ast}/\sim| \leq |W|^{|W|}$.
\end{proposition}
\begin{proof}
	\begin{enumerate}[(1)]
		\item \textbf{Relation between $\hat{A}^{\ast}/\sim$ and endofunctions on $W$.}
		      \begin{itemize}
			      \item Since $\Phi$ is injective, there is a one-to-one correspondence between elements of $\hat{A}^{\ast}/\sim$ and their images in $\text{End}(W)$ under $\Phi$.
			      \item Each $[a]_{\sim} \in \hat{A}^{\ast}$ corresponds uniquely to the function $f_{[a]_{\sim}} \in \text{End}(W)$.
		      \end{itemize}

		\item \textbf{Counting the number of functions in $\text{End}(W)$.}
		      \begin{itemize}
			      \item Each element $w \in W$ can be mapped to any of the $|W|$ elements in $W$.
			      \item Since there are $|W|$ elements in $W$, the total number of mapping permutations, and therefore the total number of functions from $W$ to $W$, is $|W|^{|W|}$.
		      \end{itemize}

		\item \textbf{Establishing the upper bound.}
		      \begin{itemize}
			      \item Since $\Phi$ is injective, the number $|\hat{A}^{\ast}/\sim|$ of elements in $\hat{A}^{\ast}/\sim$ is less than or equal to the number of functions from $W$ to $W$.
			      \item Therefore:
			            \begin{equation}
				            |\hat{A}^{\ast}/\sim| \leq |W|^{|W|}
			            \end{equation}
		      \end{itemize}
	\end{enumerate}
\end{proof}

If we restrict the codomain of the map $\Phi$ to only the functions from $W$ to $W$ that are due to an action sequence in $\hat{A}^{\ast}$, then the restricted map will become a bijection (and therefore an isomorphism).
The codomain required for this restriction is exactly our set $\mathcal{T}_{\hat{A}^{\ast}/\sim}$ (i.e., $\Phi(\hat{A}^{\ast}/\sim) = \mathcal{T}_{\hat{A}^{\ast}/\sim}$).

\begin{proposition}\label{prp:A_to_T_isomorphism}
	\begin{equation}
		\begin{aligned}
			 & \Phi_{R} : \hat{A}^{\ast}/\sim \to \mathcal{T}_{\hat{A}^{\ast}/\sim} \quad \text{where} \\
			 & \Phi_{R}([a]_{\sim}) = f_{[a]_{\sim}}
		\end{aligned}
	\end{equation}
	is an isomorphism.
\end{proposition}
\begin{proof}
	To prove that $\Phi_{R}$ is an isomorphism we need to show that $\Phi$ is injective, surjective and a homomorphism.
	\begin{enumerate}[(1)]
		\item \textbf{Homomorphism.}
		      This follows from the fact that $\Phi$ is a homomorphism.

		\item \textbf{Injective.}
		      This follows from the fact that $\Phi$ is injective.

		\item \textbf{Surjective.}
		      \begin{itemize}
			      \item $\Phi(\hat{A}^{\ast}/\sim) = \mathcal{T}_{\hat{A}^{\ast}/\sim}$.
			      \item Therefore $\Phi_{R}$ maps $\hat{A}^{\ast}/\sim$ onto $\mathcal{T}_{\hat{A}^{\ast}/\sim}$.
			      \item For every $g \in \mathcal{T}_{\hat{A}^{\ast}/\sim}$, there exists an $[a]_{\sim} \in \hat{A}^{\ast}/\sim$ such that $\Phi_{R}([a]_{\sim}) = g$.
		      \end{itemize}
	\end{enumerate}
\end{proof}

Proposition \ref{prp:A_to_T_isomorphism} means that the monoid $(\hat{A}^{\ast}/\sim, \circ_{\sim})$ is isomorphic to the monoid $(\mathcal{T}_{\hat{A}^{\ast}/\sim}, \cdot)$.


%%%%%%%%%%%%%%%%%%%%%%%%%%%%%%%%%%%%%%%%%%%%%%%
\section{
  Example: the equivalence condition on \texorpdfstring{$\mathscr{W}_{\alpha}$}{example world}
 }\label{sec:the_equivalence_condition_on_example_world}

Now let's see what happens with our example world $\mathscr{W}_{\alpha}$, when we apply the equivalence relation $\sim$.
Figure \ref{fig:2x2_cyclical_equivalence_min_actions} shows what happens to Figure \ref{fig:2x2_cyclical_labelling_with_min_actions} when we apply our equivalence relation $\sim$ to our example world $\mathscr{W}_{\alpha}$.

\begin{figure}[H]
	\centering
	\includegraphics[width=0.5\linewidth]{2MathematicalFramework/Images/2x2_cyclical_equivalence_min_actions.jpeg}
	\caption{World diagram showing the minimum action equivalence classes under $\sim$ for the set $\hat{A} \cup \epsilon$.}
	\label{fig:2x2_cyclical_equivalence_min_actions}
\end{figure}

\draftnote{blue}{awjdean}{
	Show diagrams with some non-minimum action transformations.
	Make it clear that equivalence classes contain more elements: $\{\epsilon, 1 \dots \}$.
}

\draftnote{blue}{awjdean}{
	Map showing actions being sent to their equivalence classes under $\sim$: $D_{A}$ -($l$)-> $\hat{A}^{\ast}$ -($\pi_{A}$)-> $\hat{A}^{\ast}/\sim$.
}

%%%%%%%%%%%%%%%%%%%%%%%%%%%%%%%%%%%%%%%%%%%%%%%
\draftnote{red}{DIVIDER}{}
\draftnote{green}{To do}{
	\begin{enumerate}
		\item Equivalence relation $\sim$ for transformations and how it compares to equivalence relation $\sim$ for actions.
		\item \textbf{Mention:} Our use of an equivalence relation was inspired by \cite{caselles2020sensory}, which uses a similar equivalence relation to equate action sequences that cause the same final observation state after each action sequence is performed from an initial observation state.
	\end{enumerate}
}
\draftnote{red}{DIVIDER}{}
