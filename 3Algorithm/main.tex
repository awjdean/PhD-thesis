\chapter{
Algorithm \draftnote{blue}{(change name)}{}
(V1.1)
}

The latest version of the code for this chapter can be found at
\begin{center}
\url{https://github.com/awjdean/CayleyTableGeneration}
\end{center}

%%%%%%%%%%%%%%%%%%%%%%%%%%%%%%%%%%%%%%%%%%%%%
\section{Motivation}
\draftnote{blue}{To do}{
	\begin{enumerate}
		\item Why create the algorithm ?
		      \begin{itemize}
			      \item We wanted a way to explore structures.
		      \end{itemize}
	\end{enumerate}
}

To gain an intuition for the structure of different worlds and to illustrate our theoretical work with examples, we developed an algorithm that uses an agent’s minimum actions to generate the algebraic structure of the transformations of a world due to the agent's actions.
We display this structure as a generalised Cayley table (a multiplication table for the distinct elements of the algebra).

After generating the algorithm, we also check the following properties of the algebra algorithmically: (1) the presence of identity, including the presence of left and right identity elements separately, (2) the presence of inverses, including the presence of left and right inverses separately for each element, (3) associativity, (4) commutativity, and (5) the order of each element in the algebra.

%%%%%%%%%%%%%%%%%%%%%%%%%%%%%%%%%%%%%%%%%%%%%
\section{The algorithm}
\whendraft{
	\textbf{Include in algorithm description:}
	\begin{enumerate}
        \item What each of the objects in the algorithm (e.g., $L$) are during the algorithm (e.g., $L \subseteq \hat{A}^{*}$).
        \item What each of the objects are when the algorithm terminates.
		\item Make up of $\mathcal{E}$.
		\item We store actions as their unique sequences of minimum actions.
		\item All we need from $\mathscr{W}$ is the set $W$ of world states and the map $\hat{*}: \hat{A} \times W \to W$.
		      We treat $\hat{*}$ as a part of $\mathscr{W}$.
		\item How we use the $\operatorname{Combine}$ operator to evaluate $\circ$.
		\item What each of elements of $\mathcal{E}$ are at the termination of the algorithm.
		\item $\operatorname{Seq} : \hat{A}^* \to (\hat{A})^n, \quad \operatorname{Seq}(a) = (\hat{a}_{n}, \hat{a}_{n-1}, \dots, \hat{a}_{1})$
		\item Explain how assignment to $\mathcal{E}$ works - assignment occurs for each of the constituents of $\mathcal{E}$.
		\item Mention that the only bit of information needed from $\mathscr{W}$ is the mapping: $\hat{*}: \hat{A} \times W \to W$ that takes a minimum action and applies it to a world state.


	\begin{enumerate}
		\item Change name of `states Cayley table' and `actions Cayley table` ?
		      \begin{itemize}
			      \item Cayley table is already a thing.
		      \end{itemize}
	\end{enumerate}

    \item ``transformations generated by $\hat{A}$under composition form a sub (semi-)monoid of the full transformation monoid on $W^{\bot}$''.

    \item Can represent each $f_{a}$ by a tuple $(f_{a}(w_{0}), f_{a}(w_{1}), \cdots , f_{a}(w_{n}), f_{a}(\bot))$.

    \item "Cayley tables can be used to characterize the operation of any finite magma"
\end{enumerate}
}

%%%%%%%%%%%%%%%%%%%%%%%%%%%%%%%%%%%%%%%%%%%%%%%
\subsection{Description}
Algorithm \ref{alg:GenerateEquivClasses} can be split into three parts: (1) the initialisation step, (2) the expansion step, and (3) the termination condition.

\textbf{Initialisation.}
We start by finding the distinct minimum actions in $\hat{A}$ by calculating their actions functions $f_{\hat{a}}: w \mapsto \hat{a} \ast w$ and checking if they are unique in $\mathcal{T}$.
We can then initialise our set $L$ of actions whose induced action functions (stored in $\mathcal{T}$) are unique as $L \gets \hat{A}/\sim$ of distinct minimum actions, we can initialise our set $\mathcal{T}$ of unique action functions as $\mathcal{T}\gets \{f_{\hat{a}}\}$ for all $\hat{a} \in \hat{A}/\sim$, and we can initialise our map $\rho$ that stores the correspondence between distinct actions in $L$ and their unique action functions in $\mathcal{T}$ as $\rho \gets (\hat{a} \mapsto f_{\hat{a}})$ for all $\hat{a} \in \hat{A}/\sim$.

\textbf{Expansion.}
In each iteration of the expansion step we attempt to expand the set $L$ of distinct actions by left composing minimum actions onto the distinct actions we've previously discovered to get new candidates for distinct actions: $a_{C} = \hat{a} \circ l$ for every $\hat{a} \in \hat{A}/\sim$ and for every $l \in L$.
These candidate are then checked for distinctness using the same process as for minimum actions: find the action function $f_{a_{C}}$ for the candidate $a_{C}$ and check if it is already in $\mathcal{T}$; if it's not, then store $a_{C}$, $f_{a_{C}}$ and the correspondence between them in $L$, $\mathcal{T}$, and $\rho$.
In reality, in every iteration of the expansion step we only need to create candidates using the actions in $L$ with the greatest length when expressed in terms of their unique sequence of minimum actions rather than creating candidates using every action in $L$; this is because if the largest actions in $L$ at the start of an expansion iteration have a length of $n$, then all elements with a length of less than $n$ have already been checked for distinctness due to the creation of candidates at each expansion iteration using a single minimum action and so increasing the length of candidates by one at each iteration.

\textbf{Termination.}
The algorithm halts when an expansion iteration produces no new distinct action functions.

For our algorithm to successfully generate the complete transformation algebra of a world, the world must contain a finite number of states, the agent must have a finite number of minimum actions, and all the transformations of the world must be due to the actions of the agent.

%%%%%%%%%%%%%%%%%%%%%%%%%%%%%%%%%%%%%%%%%%%%%
\subsection{Preliminaries}
\draftnote{blue}{To do}{
\begin{enumerate}
    \item Definition of Cayley table. - put this in motivation section ?
    \item Something about how the Cayley table containing all the necessary information for the algebra.
\end{enumerate}
}


%%%%%%%%%%%%%%%%%%%%%%%%%%%%%%%%%%%%%%%%%%%%%
\subsection{Design decisions}

\paragraph{Problem:}
[?] Assessing equality between two actions.
\\\textit{Solution:}


\paragraph{Problem:}
Calculating the result of the $\circ$ operator.
\\\textit{Solution:}



%%%%%%%%%%%%%%%%%%%%%%%%%%%%%%%%%%%%%%%%%%%%%
%%%%%%%%%%%%%%%%%%%%%%%%%%%%%%%%%%%%%%%%%%%%%
%%%%%%%%%%%%%%%%%%%%%%%%%%%%%%%%%%%%%%%%%%%%%

\paragraph{Problem:}
How do we know if the algebra produced is complete?
\\\textit{Solution:}
\begin{itemize}
	\item
\end{itemize}


\paragraph{Problem:}
Dealing with undefined actions.
\\\textit{Solution:}
\begin{itemize}
	\item Describe how we implemented the undefined state.
\end{itemize}




%%%%%%%%%%%%%%%%%%%%%%%%%%%%%%%%%%%%%%%%%%%%%
\section{Pseudocode}
\whendraft{
	\noindent\rule{\textwidth}{1mm}
	\textbf{To do:}
	\begin{enumerate}
		\item [visual] Span algorithms across entire page.
	\end{enumerate}
	\noindent\rule{\textwidth}{1mm}
}


\begin{table*}[htbp]
  \begin{fullwidth}
    \centering
    \begin{tabularx}{\linewidth}{lX}
      \toprule
      \textbf{Symbol}                    & \textbf{Description} \\
      \midrule
      $\hat{A}$                          & The set of minimum actions of the agent. Elements of $\hat{A}$ are given a $\hat{\ }$. \\
      $\mathscr{W} = (W, \; \hat{\ast})$ & The world characterised by a set $W$ of world states and a minimum action effect map $\hat{\ast}$. \\
      $W$                                & The set of world states. \\
      $\hat{\ast}$                       & The minimum effect map $\hat{A} \times W \to W$ that sends a minimum action-world state pair to the resultant world state after the agent performs that minimum action in the world state. \\
      $L$                                & The set of equivalence class labels. \\
      $E$                                & The set of processed elements... \\
      $\pi$                              & A map $E \to L$ ... \\
      $\mathcal{E} = (L, \; E, \; \pi)$  & The equivalence classes. \\
      $\mathcal{T}$                      & A set of functions $f_{a}: W \to W$... \\
      $\rho$                             & A map $L \to \mathcal{T}$ ... \\
      $\hat{A}/\sim$                     & The set of minimum actions that are distinct under $\sim$. \\
      $n$                                & Counter that represents the length ... \\
      $N_{L}$                            & The number of new equivalence class labelling elements ... \\
      $A_{C}$                            & The set of candidates ... \\
      $\mathcal{L}_{n}(L)$               & A filtering map $L \to \{ l \in L \mid |l| = n \}$... \\
      $\operatorname{Combine}$           & The combine operator ... \\
      $T_{A}$                            & The Cayley table. \\
      \bottomrule
    \end{tabularx}
    \caption{Key for pseudocode.}
  \end{fullwidth}
\end{table*}



%%%%%%%%%%%%%%%%%%%%%%%%%%%%%%%%%%%%%%%%%%%%%
\subsection{Generating the equivalence classes}

\begin{algorithm}
	\caption{
		Generate the equivalence classes $\mathcal{E} = (L, \; E, \; \pi: E \to L)$ for $\sim$ and the set of action functions $\mathcal{T} = \{f_{l}: W \to W\}$ for a world $\mathscr{W} = (W, \; \hat{\ast})$ that is characterised by a set $W$ of world states and a minimum action effect map $\hat{\ast}$.
	}
        \label{alg:GenerateEquivClasses}
	\hrulefill
	\begin{algorithmic}[1]
		\Procedure{GenerateEquivClasses}{$\hat{A}$, \; $\mathscr{W}$}
		\Statex \Comment{Initialise equivalence classes object $\mathcal{E}$.}
		\State $L \gets \emptyset$
		\State $E \gets \emptyset$
		\State $\pi \gets (\emptyset \to \emptyset)$
		\State $\mathcal{E} \gets (L, \; E, \; \pi)$.

		\Statex \Comment{Initialise $\mathcal{T}$ and $\rho$}
		\State $\mathcal{T} \gets \emptyset$
		\State $\rho: \gets (\emptyset \to \emptyset)$

		\Statex \Comment{Find distinct minimum actions.}
		\For{\textbf{each} $\hat{a} \in \hat{A}$}
		\State $(\mathcal{E}, \; \mathcal{T}, \; \rho) \gets$ \Call{ProcessCandidate}{$\hat{a}$, \; $\mathcal{T}$, \; $\rho$, \; $\mathcal{E}$, \; $\mathscr{W}$}
		\EndFor
		\State $\hat{A}/\sim \; \gets L$

		\Statex \Comment{Iteratively create candidate actions sequences for equivalence class labelling elements then check if they are successful candidates.}
		\State $n \gets 0$
		\State $N_{L} \gets 0$
		\Statex \Comment{If no new labelling actions found in a set of candidate elements, then halt the algorithm.}
		\While{$N_{L} \neq |L|$}
		\State $n \gets n + 1$
		\State $N_{L} \gets |L|$
		\State $A_{C} \gets \Call{GenerateCandidates}{\mathcal{L}_{n}(L), \;  \hat{A}/\sim}$

		\For{\textbf{each} $a_{C} \in A_{C}$}
		\State $(\mathcal{E}, \; \mathcal{T}, \; \rho) \gets$ \Call{ProcessCandidate}{$a_{C}$, \; $\mathcal{T}$, \; $\rho$, \; $\mathcal{E}$, \; $\mathscr{W}$}
		\EndFor
		\EndWhile
		\State \Return $\mathcal{E}, \; \mathcal{T}$
		\EndProcedure
	\end{algorithmic}
\end{algorithm}


\begin{algorithm}[H]
	\caption{
		Generate new action sequences that are candidates for equivalence class labelling elements.
	}
	\hrulefill
	\begin{algorithmic}[1]
		\Procedure{GenerateCandidates}{$L$, \;  $\hat{A}/\sim$}
		\State $A_{C} \gets \emptyset$
		\ForAll{$(l, \; \hat{a}) \in L \times \hat{A}/\sim$}
		\State $a' = \operatorname{Combine}(\hat{a}, \; l)$
		\State $A_{C} \gets A_{C} \cup \{a'\}$
		\EndFor
		\State \Return $A_{C}$
		\EndProcedure
	\end{algorithmic}
\end{algorithm}


\begin{algorithm}[H]
	\caption{
		Process a candidate $a_{C}$ for being an equivalence class labelling element.
		If $a_{C}$ is a successful candidate, then create a new equivalence classes labelled by $a_{C}$.
		If $a_{C}$ is found to be in another equivalence class then add it to that equivalence class.
	}
	\hrulefill
	\begin{algorithmic}[1]
		\Procedure{ProcessCandidate}{$a_{C}$, \; $\mathcal{T}$, \; $\rho$, \; $\mathcal{E}$, \; $\mathscr{W}$}
		\State $f_{a_{C}} = \Call{ComputeActionFunction}{a_{C}, \; \mathscr{W}}$

		\If{$f_{a_{C}} \in \mathcal{T}$}
		\Statex \Comment{Add $a_{C}$ to equivalence class in $\mathcal{E}$ with class label that has the same action function $f_{a_{C}}$.}
		\State $\mathcal{E} \gets (\; L, \; E \cup \{a_{C}\}, \; \pi \cup \pi' \;)$, where $\pi': \{a_{C}\} \to L$ such that $\pi'(a_{C}) = \rho^{-1}(f_{a_{C}})$.
		\Else
		\Statex \Comment{Create new equivalence class in $\mathcal{E}$ labelled by $a_{C}$.}
		\State $\mathcal{E}' \gets (\; L \cup \{a_{C}\}, \; E \cup \{a_{C}\}, \; \pi \cup \pi' \;)$, where $\pi': \{a_{C}\} \to \{a_{C}\}$ such that $\pi'(a_{C}) = a_{C}$.

		\Statex \Comment{Add $f_{a_{C}}$ to $\mathcal{T}$.}
		\State $\mathcal{T}' \gets \mathcal{T} \cup \{f_{a_{C}}\}$

		\Statex \Comment{Update $\rho$ to send $a_{C}$ to $f_{a_{C}}$.}
		\State $\rho' \gets \rho \cup \rho''$ where $\rho'': \{a_{C}\} \to \{f_{a_{C}}\}$ such that $\rho''(a_{C}) = f_{a_{C}}$
		\EndIf
		\State \Return $(\mathcal{E}', \; \mathcal{T}', \; \rho')$
		\EndProcedure
	\end{algorithmic}
\end{algorithm}



\begin{algorithm}[H]
	\caption{Compute the action function $f_{a}: W \to W$ that sends $w \mapsto a \ast w$.}
        \label{alg:ComputeActionFunction}
	\hrulefill
	\begin{algorithmic}[1]
		\Procedure{ComputeActionFunction}{$a$, \; $\mathscr{W}$}
		\State $f_{a} \gets (\emptyset \to \emptyset)$
		\ForAll{$w \in W$}
		\State $w_{a} \gets$ \Call{GenerateActionOutcome}{$a$, \; $w$, \; $\hat{\ast}$}
		\State $f_{a} \gets f_{a} \cup f_{a}'$ where $f_{a}': \{w\} \to \{w_{a}\}$ such that $f_{a}'(w) = w_{a}$
		\EndFor
		\State \Return $f_{a}$
		\EndProcedure
	\end{algorithmic}
\end{algorithm}


\begin{algorithm}[H]
	\caption{
		Generate the outcome state of a world $\mathscr{W}$ when an action sequence $a$ is applied to the world in an initial state $w$.
	}
	\label{alg:GenerateActionOutcome}
	\hrulefill
	\begin{algorithmic}[1]
		\Procedure{GenerateActionOutcome}{$a$, \; $w$, \; $\hat{\ast}$}
		\State $w_{a} \gets w$
		\For{$i \gets 1, \; \dots \;, \; n$}
		\State $w_{a} \gets \hat{a}_{i} \; \hat{\ast} \; w_{a}$ where $\operatorname{Seq}(a) = (\hat{a}_n, \; \hat{a}_{n-1}, \; \dots \;, \; \hat{a}_1)$
		\EndFor
		\State \Return $w_{a}$
		\EndProcedure
	\end{algorithmic}
\end{algorithm}


%%%%%%%%%%%%%%%%%%%%%%%%%%%%%%%%%%%%%%%%%%%%%
\subsection{Generating the Cayley table}

\begin{algorithm}[H]
	\caption{
		Generate the Cayley table $T_{A}$
	}
        \label{alg:GenerateCayley}
	\hrulefill
	\begin{algorithmic}[1]
		\Procedure{GenerateCayley}{$\mathcal{E}$}
		\State $T_{A} \gets$ Empty $|L| \times |L|$ table with rows and columns labelled by the elements of $L$.
		\Statex \Comment{Fill actions Cayley table.}
		\For{\textbf{each} $l_{row} \in L$}
		\For{\textbf{each} $l_{col} \in L$}
		\Statex \Comment{Get action function for the combined element.}
		\State $f_{(l_{row} \; \circ \; l_{col})} \gets$ \Call{ComputeCompositionActionFunction}{$l_{row}$, \; $l_{col}$, \; $\mathcal{T}$, \; $\rho$}
		\State $l \gets \rho^{-1}(f_{(l_{row} \; \circ \; l_{col})})$
		\State $T_{A}[l_{row}][l_{col}] \gets l$

		\whendraft{
			\Statex \Comment{TODO: Add $a$ to the equivalence class as $\operatorname{Combine}(l_{row}, l_{col})$?}
			\Statex \Comment{TODO: Include equivalence class check before generating the combined action function ?}
		}

		\EndFor
		\EndFor

		\State \Return $T_{A}$
		\EndProcedure
	\end{algorithmic}
\end{algorithm}


\begin{algorithm}[H]
	\caption{
		Compute the action function for the combination $l_{L} \circ l_{R}$ of two actions by combining their action functions.
	}
	\hrulefill
	\begin{algorithmic}[1]
		\Procedure{ComputeCompositionActionFunction}{$l_{L}$, \; $l_{R}$, \; $\mathcal{T}$, \; $\rho$}
		\Statex \Comment{Get action functions for $l_{L}$ and $l_{R}$.}
		\State $f_{L} \gets \rho(l_{R})$
		\State $f_{R} \gets \rho(l_{R})$

		\State $f_{a} \gets (\emptyset \to \emptyset)$
		\Statex \Comment{Compute the combined action function $f_{a}$}
		\State $f_{a} \gets (\emptyset \to \emptyset)$
		\ForAll{$w_{R, \; I} \in \operatorname{Dom}(f_{R})$}
		\State $w_{R, \; F} \gets f_{R}(w_{R, \; I})$
		\ForAll{$w_{L, \; I} \in \operatorname{Dom}(f_{L})$}
		\If{$w_{L, \; I} = w_{R, \; F}$}
		\State $w_{L, \; F} \gets f_{L}(w_{L, \; I})$
		\State $f_{a} \gets f_{a} \cup f_{a}'$ where $f_{a}': \{w_{R, \; I}\} \to \{w_{L, \; F}\}$ such that $f_{a}'(w_{R, \; I}) = w_{L, \; F}$
		\EndIf
		\EndFor
		\EndFor
		\State \Return $f_{a}$
		\EndProcedure
	\end{algorithmic}
\end{algorithm}

%%%%%%%%%%%%%%%%%%%%%%%%%%%%%%%%%%%%%%%%%%%%%
\whendraft{
	\section{Example [\textbf{To do}]}
}

%%%%%%%%%%%%%%%%%%%%%%%%%%%%%%%%%%%%%%%%%%%%%
\section{Mathematical justification}
\whendraft{
	\noindent\rule{\textwidth}{1mm}
	\textbf{To do:}
	\begin{enumerate}
		\item Proof that algorithm halts when it has all necessary elements and not before.
        \item Different name for `action function' ?
	\end{enumerate}
	\noindent\rule{\textwidth}{1mm}
}

%%%%%%%%%%%%%%%%%%%%%%%%%%%%%%%%%%%%%%%%%%%%%
\subsection{Preliminaries}
\whendraft{
\noindent\rule{\textwidth}{1mm}
\textbf{To do:}
\begin{enumerate}
    \item Pseudocode of word description.
    \item Pseudocode of simplified mathematical version of algorithm with use of $\epsilon$, and $\hat{A}$ instead of $\hat{A}/\sim$. - put in margin ?
    \item Footnote:
    Mathematically,
    \begin{enumerate}[(a)]
        \item we can use the set $\hat{A}$ of minimum actions to create new candidate elements in the iterations of our expansion step instead of the set $\hat{A}/\sim$ of distinct minimum actions.
        
        \item we can initialise the algorithm with $L = \{ \epsilon \}$, $\mathcal{T} = \{f_{\epsilon}\}$, $\rho(\epsilon) = f_{\epsilon}$, and then go straight to the expansion step since the first iteration of the expansion step will check the actions $\hat{a} \circ \epsilon$ for all $\hat{a} \in \hat{A}$ and $\hat{a} \circ \epsilon = \hat{a}$ for all $\hat{a} \in \hat{A}$ by the definition of $\epsilon$.
    \end{enumerate}

    In practical terms,
    \begin{enumerate}[(a)]
        \item using the set $\hat{A}/\sim$ of distinct minimum actions reduces the number of candidates elements we need to check for distinctness in each iteration of the expansion step since $(\hat{a}_{1} \circ a) \sim (\hat{a}_{2} \circ a)$ if $\hat{a}_{1} \sim \hat{a}_{2}$;
        
        \item we already calculate the $f_{\hat{a}}$ for all $\hat{a} \in \hat{A}$ to get our set of distinct minimum actions $\hat{A}/\sim$, so it's natural to set $L \gets \hat{A}/\sim$, $\mathcal{T} \gets \{f_{\hat{a}}\}$ for all $\hat{a} \in \hat{A}/\sim$, and $\rho \gets (\hat{a} \mapsto f_{\hat{a}})$ for all $\hat{a} \in \hat{A}/\sim$.
    \end{enumerate}
    We also need to be a little careful with $\epsilon$ because, even though $\epsilon$ is always in $\hat{A}^{*}$ for mathematical reasons, if none of the agent's actions are equivalent to $\epsilon$, then we might not want to include it in our algebra of the transformations of the actions of the agent.
    \draftnote{blue}{awjdean}{what does this mean for the Cayley table? I think you'd have $\epsilon$ as an identity element, but there would be no identity elements as entries in the table other than in the row and column of $\epsilon$.}
\end{enumerate}
\noindent\rule{\textwidth}{1mm}
}


%%%%%%%%%%%%%%%%%%%%%%%%%%%%%%%%%%%%%%%%%%%
\subsection{Algorithm \ref{alg:GenerateEquivClasses} halts}

\begin{proposition}
    Algorithm \ref{alg:GenerateEquivClasses} always halts for finite $W$.
\end{proposition}
\begin{proof}
    \textbf{$|L|$ increases monotonically until its final iteration.}
    During each expansion iteration, the set $L$ either grows or remains the same.
    So at each expansion iteration, with a new distinct action is discovered and added to $L$ or no new distinct actions are discovered and the algorithm halts.

    \textbf{Upper bound on $|L|$.}
    Since, for a world with $|W|$ world states there are at most $(|W| + 1)^(|W| + 1)$ distinct transformations $W^{\bot} \to W^{\bot}$, the set $L$ can have at most $(|W| + 1)^(|W| + 1)$ elements.

    \textbf{Halting.}
    If $L$ stops growing for one expansion iteration, then the algorithm halts.
    If $L = (|W| + 1)^(|W| + 1)$, then there are no more unique action functions and therefore no more distinct actions under $\sim$ and so  algorithm \ref{alg:GenerateEquivClasses} halts.
    $|L|$ increases monotonically, therefore either $|L|$ reaches $(|W| + 1)^(|W| + 1)$ and halts or the algorithm halts for another reason.
\end{proof}

%%%%%%%%%%%%%%%%%%%%%%%%%%%%%%%%%%%%%
\subsection{When algorithm \ref{alg:GenerateEquivClasses} halts, $L \cong \hat{A}^{*}/\sim$}

\begin{proposition}
    When algorithm \ref{alg:GenerateEquivClasses} halts, the set $L$ contains all distinct actions under $\sim$ in $\hat{A}^{*}$ (i.e., there is an isomorphism $L \cong \hat{A}^{*}/\sim$ between the elements of $L$ and the elements of $\hat{A}^{*}/\sim$ upon termination).
\end{proposition}
\begin{proof}
    \textbf{Proof by contradiction.}
    Assume algorithm \ref{alg:GenerateEquivClasses} halts with a set $L$ of discovered distinct actions, but that there exists at least one distinct transformation that is not represented in $L$.

    Let $b$ be the shortest undiscovered distinct action.
    We can write $b$ as $b = \hat{a} \circ b'$ where $\hat{a} \in \hat{A}/\sim$ and $b' \in \hat{A}^{\ast}$.

    Since $b$ is the shortest undiscovered distinct action, either $b'$ must have been discovered and be represented in $L$ or $b'$ is not discovered and $b$ is not the shortest undiscovered distinct action, since $b'$ is shorter than $b$, which is a contradiction.

    Given that $b'$ is known and represented in $L$, algorithm \ref{alg:GenerateEquivClasses} will have left composed $b'$ with every distinct minimum action in $\hat{A}/\sim$, including $\hat{a}$.
    Therefore, algorithm \ref{alg:GenerateEquivClasses} must have discovered $\hat{a} \circ b' = a$, which is a contradiction.
\end{proof}


%%%%%%%%%%%%%%%%%%%%%%%%%%%%%%%%%%%%%%%%%%%%%
\subsection{
Complexity analysis
}

Let $|W| = n$ (so $|W^{\bot}| = n+1$) and let $|\hat{A}/\sim| = m$.

In the worst case, algorithm \ref{alg:GenerateEquivClasses} must discover all $|W^{\bot}|^{|W^{\bot}|}$ functions $W^{\bot} \to W^{\bot}$.
The algorithm generates new candidate actions by composing discovered distinct actions in $L$ with distinct minimum actions in $\hat{A}/\sim$; in the worst case scenario that would mean $m (n+1)^{(n+1)}$ candidate elements to check.
Checking each candidate element costs at least $\mathcal{O}(n)$ time because the the action function for each candidate must be determined for each of the $|W^{\bot}|$ states.
Therefore, an estimate for the upper bound on the complexity of algorithm \ref{alg:GenerateEquivClasses} is
\begin{equation}
    \mathcal{O}(m \cdot (n+1)^{(n+1)} \cdot n)
\end{equation}





%%%%%%%%%%%%%%%%%%%%%%%%%%%%%%%%%%%%%%%%%%%%%
\section{Minimum actions affect the algebra}
\whendraft{
	\begin{enumerate}
		\item Have a world $\mathscr{W}$ with loads of transformations, then have different agent actions (different labelling map).
	\end{enumerate}
}
