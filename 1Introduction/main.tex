\chapter{
  Introduction
  (V0.0)
 }

\draftnote{green}{Structure}{
	\begin{enumerate}
		\item What is artificial intelligence?
		\item What is representation learning?
		\item
	\end{enumerate}
}

\draftnote{blue}{About this thesis}{
	\begin{enumerate}
		\item We will introduce the necessary preliminaries as we go rather.
		\item Description of how the work was carried out.
	\end{enumerate}
}


\paragraph{Include: About this thesis.}
When explaining a work of this length, understanding how the work was carried out can help explain why certain decisions were made and particular directions taken.

Initially, our aim was to extend work on Symmetry-based disentangled representation learning [citations] to more general
In particular, we felt that their derivation of the equivariance principle should

We also felt that the language Category Theory seemed like an ideal candidate for describing the structure of an agent's representations because...

A discussion with Hugo Caselles-Dupré about his work in [citations] where he explained "we couldn't figure out how to deal with walls" was particularly insightful.

Our plan was


Distilling thoughts and beleifs into rigourous mathematical frameworks has led to new insights

That said, we tend to deal with idealisd situtations (toy problems and scenarios) that are easier to deal with mathematically, but are are much more simple than real-world situations.

Another benefit of building a generalised mathematical framework is its results (if derived correctly) should be true whatever fancy new learning algorithm is the cutting edge in the future.





\draftnote{blue}{Chapter structure}{
	\begin{enumerate}
		\item Overview.
		\item Preliminaries.
		\item Discussion.
		\item Summary.
	\end{enumerate}
}






%%%%%%%%%%%%%%%%%%%%%%%%%%%%%%
\chapter{
(OLD) Introduction (to be converted over)
}\label{chp:Introduction}
%%%%%%%%%%%%%%%%%%%%%%%%%%%%%%
\section{Towards nature-based artificial intelligence}

\textit{Artificial intelligence} (AI) has progressed significantly in recent years due to massive increases in available computational power allowing the development and training of new deep learning algorithms \autocite{amodei2018ai}.
However, the best-performing learning algorithms often suffer from poor data efficiency and lack the levels of robustness and generalisation (\textit{i.e.,} poor transfer learning) that is characteristic of nature-based intelligence; reinforcement learning algorithms such as DeepMind’s AlphaZero algorithm \autocite{Silver2017a, Silver2017b} can successfully achieve superhuman level performance on many different games, however, they cannot transfer knowledge learnt during training on one game to another game: the system must be trained on each game individually\footnote{This algorithm shows generalisation during training, but no post-training generalisation (\textit{i.e.,} the same algorithm can learn to play different games with no code change, but knowledge learnt while training for one game cannot be transferred to a different but similar game).}.
Performance of current learning algorithms is also heavily dependent on intensive problem-specific feature engineering, which reduces the generalisation of their solutions to new problems and the speed at which novel applications can be constructed or tested.
Finally, current learning algorithms struggle, compared to nature-based intelligence, to learn tasks in dynamic, real-world environments by continually adapting their knowledge through interaction with their environment without experiencing catastrophic forgetting (\textit{i.e.,} they struggle with continual learning) \autocite{Sauders2018}; this is a long standing problem in reinforcement learning research and robotics.
A possible reason for the poor performance of these algorithms in replicating nature-based intelligence is that they are good at prediction using statistical, ‘model-free’ pattern recognition but not at building models, which can then be used to understand the world.
Processes that allow agents to build models of their environment, then continually update these models as time progresses, and the environment evolves, without explicit supervision appear the most promising path towards solutions to the described problems and towards human-like Artificial General Intelligence\footnote{Model-free approaches could be used to support model-building methods by accelerating inferences in perception and cognition, similar to how the occipital lobe processes data from the eye.} \autocite{Lake2017, Bengio2013}.

Improving the ability of an agent to generalise should allow for faster learning with improved data efficiency, which means that the artificial agent needs fewer data points to learn - especially important if the agent is learning in a real-world continual learning environment.
Improved generalisation can also increase the robustness of an agent's learning; this means that the agent reacts with greater consistency if it encounters data that is different, but similar, to the data that was used to train the agent.
This makes the agent's actions more predictable, since the agent should react in a similar way in real-world situations as it reacts in simulations or in controlled environments.
Therefore, it becomes easier to reduce the chance of the agent performing undesirable actions (\textit{e.g.}, an artificial agent running a self-driving car causing the car to drive off a bridge).
Robustness becomes increasingly important as artificial agents are used more and more for industrial applications.

%%%%%%%%%%%%%%%%%%%%%%%%%%%%%%
\section{Representation learning}

Much of the success of machine learning algorithms has been due to labour-intensive feature engineering of low-level sensory data, such as the creation of task-specific or data-specific data prepossessing pipelines which transform data into forms that are easier for current machine learning algorithms to learn.
This feature engineering relies on human-curated, task-specific prior knowledge because current learning algorithms struggle to identify and disentangle the underlying explanatory factors of the data.
If learning algorithms were less dependant on labour-intensive feature engineering, then novel applications could be constructed faster, and trained more efficiently.

\textit{Representation learning} (RL) is a process that transforms data into useful, (usually) more abstract features, which can faithfully characterise the data, into a form where it is easier to extract useful information for other algorithms, such as classifiers or predictors.
The transformed data is known as a \textit{representation} of the original data.
RL algorithms can be thought of as seeking to remove the need for human-curated, task-specific feature engineering by performing useful feature engineering automatically\footnote{This also means that, ideally, RL would be an unsupervised learning process.}.
RL is a wide domain that has had success in many applications across a range of areas, such as machine vision, speech recognition, natural language processing, and reinforcement learning (see \autocite{Bengio2013} for examples).

In state RL (a particular case of RL) features are low dimensional, evolve through time and are influenced by the actions of agents \autocite{Lesort2018StateRL}.
More formally, an agent uses sensors to form observations of the states of the environment (the observation process); these observations can then be used to create representations of the environment (the inference process) \autocite{Russell2010}.
Mathematically, this process can be described as a mapping from aspects of the world state (\textit{i.e.,} the environment) to the observations followed by a mapping from the observations to representations which can then be used by learning or control algorithms (see figure \ref{fig:intro-flow-chart}).
State RL has been combined with reinforcement learning to enable algorithms to learn more efficiently and with improved ability to transfer learnings to other tasks (improved generalisation) \autocite{Munk2016}.
It is hypothesised that the success of state RL is because the learned abstract representations are convenient to describe general priors about the world in a way that is easier to interpret and utilise while not being task specific; this makes the representations useful for solving a range of tasks \autocite{Bengio2013}.
It has also been argued that representations that express general priors about the world are necessary for progress towards Artificial General Intelligence.

\begin{figure}[H]
    \centering
    \includegraphics[width = \textwidth]{1Introduction/Old/Images/fig-intro-flow-chart.jpg}
    \caption{Diagram showing mappings from the world state to a state representation.}
    \label{fig:intro-flow-chart}
\end{figure}

%%%%%%%%%%%%%%%%%%%%%%%%%%%%%%
\section{Symmetries}

Due to their successes in Physics and other scientific disciplines, symmetries have been proposed as important aspects of the world that should be preserved in agents' representations, and as a method to disentangle representations \autocite{Higgins2018}.
The study of algebraic structures and symmetries and how these structures relate to the world around us has been extremely successful in the creation of accurate, interpretable representations of the universe in Physics; since symmetry transforms are prevalent at every level of abstraction they have been used to make predictions, test theories and discover previously unknown relationships between concepts.

This research is being conducted, in part, on the belief that there are properties of the transitions between world states that are true for any structure or, at least, that are true for large classes of structures, which share similar characteristics.
We believe that these universal properties have the potential to improve generalisation in the RL of a range of structures.

We believe that \autocite{Higgins2018}'s approach, of describing the structure of world states and their transitions in state representations using abstract mathematics (\textit{i.e.}, their use of groups and group representations), and \autocite{caselles2019symmetry}'s insights, of using \autocite{Higgins2018}'s approach to describe specific types of an agent's actions, can be recovered and extended by a more general mathematical framework built from first principles.
We believe that this mathematical framework will enable the description, and potentially the discovery, of universal properties of transitions between world states.
The discovery of such universal properties, aside from being interesting in their own right, could impact RL by informing improvements to RL algorithms (\textit{e.g.}, improving an agent's ability to generalise to new environments, or improving the performance of an agent in particular environments), by providing explanations for why certain algorithms can or cannot learn to the represent the details of particular structures, and by leading to the unification of different ideas in RL under a single framework.


%%%%%%%%%%%%%%%%%%%%%%%%%%%%%%%%%%%%%
\section{Mathematical background}\label{sec:Mathematical background}

%%%%%%%%%%%%%%%%%%%%%%%%%%%%%%
\subsection{Symmetries}

A symmetry is an intrinsic property of a mathematical object, where the object remains invariant to (unchanged by) certain transformations (\textit{e.g.}, rotations, reflections, more abstract transformations etc...) \autocite{Mathworld-group-theory}.

Mathematically, a symmetry is a mapping of an object to itself, which preserves the structure of the object.
If this object is a set, which has no additional structure, then a symmetry is a permutation of the elements in the set; this is equivalent to a bijective map from the set to itself.

%%%%%%%%%%%%%%%%%%%%%%%%%%%%%%
\subsection{Groups}

The concept of (full) symmetries can be described in abstract terms by the mathematical theory of groups.

\begin{definition}[Group]
    A group is a set $G$ together with a binary operation $\cdot$ that satisfies:
    \begin{itemize}
        \item \textbf{Closure/Totality:} $\forall g,h \in G, \ \ g \cdot h \in G$.
        \item \textbf{Associativity:} $\forall g,h,l \in G, \ \ (g \cdot h) \cdot l = g \cdot (h \cdot l)$.
        \item \textbf{Identity:} there exists an element $e \in G$ such that $e \cdot g = g \cdot e = g \ \ \forall g \in G$.
        \item \textbf{Inverse/Invertibility:} for each $g \in G$, there exists an element $g^{-1} \in G$ such that $g \cdot g^{-1} = g^{-1} \cdot g = e$.
    \end{itemize}
\end{definition}

Two groups $G$ and $H$ can be combined using the \textit{direct product} to give another group $G \times H$ with elements $(g, h)$ where $g \in G$ and $h \in H$.

To describe the effect of symmetry transforms on an object, a \textit{group action} is used.
The elements in the set $X$ in definition ref[def:group-action] would be aspects of the object (\textit{e.g.}, the corners of a shape) and the elements of $G$ are the transformations.

\begin{definition}[(left) Group action]\label{def:group-action}
     Let $(G, \cdot)$ be a group with identity element $1$, and let $X$ be a set.
     A (left) group action $\alpha$ of $G$ on $X$ is a function $\alpha: G \times X \to X$ that satisfies:
     \begin{enumerate}
         \item (Identity) $\alpha(1, x) = x$ for all $x \in X$;
         \item (Compatibility) $\alpha(g, \alpha(h,x)) = \alpha(g \cdot h, x)$ for all $g,h \in G$ and all $x \in X$.
     \end{enumerate}
\end{definition}

An important type of group action is a \textit{group representation}.
This is when a group acts on a vector space through invertible linear maps \autocite{Mathworld-group}.
The idea of group representations is notable because it can allow the representation of group transforms as the set of $n\times x$ invertible matrices along with the operation of matrix multiplication; matrices are commonly used in current ML algorithms.


%%%%%%%%%%%%%%%%%%%%%%%%%%%%%%
\subsection{Relaxing the properties of groups}\label{sec:Relaxing the properties of groups}

It might be desirable to use algebraic structures that do not require one or more of the properties of groups.
These more flexible algebraic structures can be used to describe processes that do not adhere to group properties.
See ref[fig:properties-of-groups-table] for a description of these more flexible group-like structures.

\begin{table}[H]
\begin{tabularx}{\textwidth}{|X|c|c|c|c|c|}
\hline
\rowcolor[HTML]{CFE2F3} 
\textbf{Group-like structures} & \textbf{Totality} & \textbf{Associativity} & \textbf{Identity} & \textbf{Invertibility} & \textbf{Commutativity} \\ \hline
Semigroupoid                  & \cellcolor[HTML]{F4CCCC}N  & \cellcolor[HTML]{D9EAD3}Y & \cellcolor[HTML]{F4CCCC}N  & \cellcolor[HTML]{F4CCCC}N  & \cellcolor[HTML]{F4CCCC}N  \\ \hline
Small Category                & \cellcolor[HTML]{F4CCCC}N  & \cellcolor[HTML]{D9EAD3}Y & \cellcolor[HTML]{D9EAD3}Y & \cellcolor[HTML]{F4CCCC}N  & \cellcolor[HTML]{F4CCCC}N  \\ \hline
Groupoid                      & \cellcolor[HTML]{F4CCCC}N  & \cellcolor[HTML]{D9EAD3}Y & \cellcolor[HTML]{D9EAD3}Y & \cellcolor[HTML]{D9EAD3}Y & \cellcolor[HTML]{F4CCCC}N  \\ \hline
Magma                         & \cellcolor[HTML]{D9EAD3}Y  & \cellcolor[HTML]{F4CCCC}N  & \cellcolor[HTML]{F4CCCC}N  & \cellcolor[HTML]{F4CCCC}N  & \cellcolor[HTML]{F4CCCC}N  \\ \hline
Quasigroup                    & \cellcolor[HTML]{D9EAD3}Y  & \cellcolor[HTML]{F4CCCC}N  & \cellcolor[HTML]{F4CCCC}N  & \cellcolor[HTML]{D9EAD3}Y  & \cellcolor[HTML]{F4CCCC}N  \\ \hline
Unital Magma                  & \cellcolor[HTML]{D9EAD3}Y  & \cellcolor[HTML]{F4CCCC}N  & \cellcolor[HTML]{D9EAD3}Y  & \cellcolor[HTML]{F4CCCC}N  & \cellcolor[HTML]{F4CCCC}N  \\ \hline
Loop                          & \cellcolor[HTML]{D9EAD3}Y  & \cellcolor[HTML]{F4CCCC}N  & \cellcolor[HTML]{D9EAD3}Y  & \cellcolor[HTML]{D9EAD3}Y  & \cellcolor[HTML]{F4CCCC}N  \\ \hline
Semigroup                     & \cellcolor[HTML]{D9EAD3}Y  & \cellcolor[HTML]{D9EAD3}Y  & \cellcolor[HTML]{F4CCCC}N  & \cellcolor[HTML]{F4CCCC}N  & \cellcolor[HTML]{F4CCCC}N  \\ \hline
Inverse Semigroup             & \cellcolor[HTML]{D9EAD3}Y  & \cellcolor[HTML]{D9EAD3}Y  & \cellcolor[HTML]{F4CCCC}N  & \cellcolor[HTML]{D9EAD3}Y  & \cellcolor[HTML]{F4CCCC}N  \\ \hline
Monoid                        & \cellcolor[HTML]{D9EAD3}Y  & \cellcolor[HTML]{D9EAD3}Y  & \cellcolor[HTML]{D9EAD3}Y  & \cellcolor[HTML]{F4CCCC}N  & \cellcolor[HTML]{F4CCCC}N  \\ \hline
Commutative Monoid            & \cellcolor[HTML]{D9EAD3}Y  & \cellcolor[HTML]{D9EAD3}Y  & \cellcolor[HTML]{D9EAD3}Y  & \cellcolor[HTML]{F4CCCC}N  & \cellcolor[HTML]{D9EAD3}Y  \\ \hline
Group                         & \cellcolor[HTML]{D9EAD3}Y  & \cellcolor[HTML]{D9EAD3}Y  & \cellcolor[HTML]{D9EAD3}Y  & \cellcolor[HTML]{D9EAD3}Y  & \cellcolor[HTML]{F4CCCC}N  \\ \hline
Abelian Group                 & \cellcolor[HTML]{D9EAD3}Y  & \cellcolor[HTML]{D9EAD3}Y  & \cellcolor[HTML]{D9EAD3}Y  & \cellcolor[HTML]{D9EAD3}Y  & \cellcolor[HTML]{D9EAD3}Y  \\ \hline
\end{tabularx}
\caption{Properties of various group-like algebraic structures (figure from \autocite{group_like_table}.).}
\label{fig:properties-of-groups-table}
\end{table}


Commutativivity is the property where for a group-like algebraic structure $(G, \circ)$ and for any $g,h \in G$ then $g \circ h = h \circ g$.

%%%%%%%%%%%%%%%%%%%%%%%%%%%%%%
\subsection{Category theory}

A category $C$ is a class of objects $ob(C)$ together with a class of morphisms that relate the objects to each other.
Two objects are associated with every morphism: the source, where the morphism starts, and the target, where the morphism ends.
Morphisms can be combined using a partial binary operation (`composition'); the composition $f \circ g$ of two morphisms $f, g$ is defined when the target of $g$ is the source of $f$.
Morphisms obey associativity: $h \circ (g \circ f) = (h \circ g) \circ f$.
The properties of morphisms are shown in fig \ref{fig:morphisms}.

\begin{figure}[H]
    \centering
    \includegraphics[scale = 0.8]{1Introduction/Old/Images/fig-morphisms.png}
    \caption{The morphism properties of composition and associativity \autocite{associativity2019}.
    $f$, $g$, and $h$ are morphisms of a category.
    $A$, $B$, $C$ and $D$ are objects of the category.}
    \label{fig:morphisms}
\end{figure}

A functor $F: C \to D$ between two categories $C$ and $D$ is a mapping that preserves identity morphisms and composition of morphisms \autocite{functor2020}:
\begin{itemize}
    \item Associates each object $X \in ob(C)$ to an object $F(X) \in D$.
    \item Associates each morphism $f: X \to Y$ in $C$ to a morphism $F(f): F(X) \to F(Y)$ in $D$ such that $F$ satisfies:
    \begin{itemize}
        \item $F(id_{X}) = id_{F(X)}$ for every object $X \in ob(C)$.
        \item $F(g \circ f) = F(g) \circ F(f)$ for all morphisms $f: X \to Y$ and $g: Y \to Z$ in $C$.
    \end{itemize}
\end{itemize}

A natural transform $\eta$ between two functors $F: C \to D$ and $G: C \to D$ consists of the morphisms $\eta_{x}: F(x) \to G(x)$ for each object $x \in C$ that obey the property $G(f) \circ \eta_x = \eta_y \circ F(f)$ for each morphism $f: x \to y$ in $C$ \autocite{natural_transform2020}.

A homomorphism is a map $f: A \to B$ between sets $A, B$ that preserves the operation $\cdot$ by satisfying: $f(x, y) = f(x) \cdot f(y)$ for every pair $x, y \in A$.

An isomorphism is a bijective homomorphism (\textit{i.e.,} a morphism that has an inverse that is also a morphism).


%%%%%%%%%%%%%%%%%%%%%%%%%%%%%%
\section{What makes a good representation?}\label{sec:What makes a good representation?}

%%%%%%%%%%%%%%%%%%%%%%%%%%%%%%
\subsection{Priors for representations}

\autocite{Bengio2013} proposed that representations that express general priors about the world could be useful for a large range of AI and robotics tasks.
Examples of possible general priors, or properties, proposed include:
        % #TODO: define mathematical formalism for these conditions (what is $f$ ?, what is $x$ ? etc...)
\begin{enumerate}
    \item \textbf{Smoothness.}
    The assumption that the function $f$ being learned satisfies the property: $x \approx y \implies f(x) \approx f(y)$, where $x,y$ are input data points.
    Smoothness exploits the principle of local generalisation, which is the assumption that the target function is smooth enough so that training examples can be used to explicitly map out the variations of the target function.
    
    \item \textbf{Multiple explanatory factors.}
    The data generating the distribution is generated by many different underlying factors.
    In many cases, what is learnt about one of the underlying factors can generalise to the other factors.
    
    \item \textbf{Hierarchical organisation of explanatory factors.}
    Concepts used to describe the world can be defined in terms of other concepts.
    Concepts that are higher in the hierarchy are considered to be more abstract because they are constructed from concepts at lower levels of the hierarchy rather than being built using the raw input data.
    
    \item \textbf{Semi-supervised learning.}
    For a supervised learning problem with input data $X$ and targets $Y$ to predict, a subset of the underlying factors that explain the distribution of $X$ explain how the targets $Y$ can be obtained from $X$ (\textit{i.e.}, some underlying features of the input data $X$ can be used to identify the target for a specific piece of data).
    Therefore, representations that describe the distribution $P(X)$ of $X$ tend to be useful when learning $P(Y \mid X)$.
    This can be viewed as the unsupervised learning task of producing a representation of $X$ sharing statistical strength with the supervised learning task of predicting the targets $Y$ given the data $X$.
    
    \item \textbf{Manifolds.}
    The probability mass concentrates near regions that form a much lower dimensional space than the original data space.
    These lower dimensional spaces can be interpreted as manifolds\footnote{In other words, the behaviour space of the model has a much lower dimensionality than the input space to the model.}.
    This is a key assumption of the manifold learning paradigm of RL (see section \ref{sec:manifold-learning}).
    
    \item \textbf{Temporal and spatial coherence.}
    Consecutive observations or spatially close observations result in a small movement on the surface of the high-density representation manifold.

    Also, it has been proposed that different underlying explanatory factors of the data distribution could change at different time scales and different spatial scales; this could possibly be exploited to disentangle explanatory factors (see section \ref{sec:Disentangled representations}).
    
    It is also expected that some explanatory factors of the data would be represented by a collection of values (\textit{e.g.}, the $x$, $y$, and $z$ positions of an object in space) which would spatially and temporally move together.
    Additionally, it is sometimes assumed that many concepts of interest change (relatively) slowly; these concepts can be captured in a representation by making the constructed representation change slowly through penalising changes in values over space and time.

    \item \textbf{Sparsity.}
    For any given observation $x \in X$, only a small fraction of possible factors are relevant and this should be reflected in representations of the data distribution that produced the observations.
    
    This sparsity property could appear in representations by underlying features, or combinations of underlying features, that are often zero for a given observation, or by the value of most of the extracted underlying features in the representations not changing significantly due to small variations in observations.
    
    A potential method for enforcing this sparsity property on representations is by penalising the magnitude of the Jacobian matrix (of derivatives) of the function mapping input to representation.
    
    \item \textbf{Simplicity of factor dependencies.}
    In high-level representations, it is sometimes considered good if the underlying factors are related to each other through simple, typically linear, dependencies.
    This also has the benefit of making the construction of algorithms easier since computers are very good at linear algebra manipulation (\textit{e.g.}, matrix multiplication).
    
    \item \textbf{Expressiveness.} Good representations can accurately capture a large number of input configurations. This can be measured by counting the number of parameters the model requires compared to the number of input regions it can distinguish \autocite{luo2019expressiveness}.
\end{enumerate}


% \begin{itemize}
%     \item Temporal coherence prior
%     \begin{itemize}
%         \item ``The principle of identifying slowly moving/changing factors in temporal/spatial data has been investigated by many as a principle for finding useful representations.''
        
%         \item Simplest expression of temporal coherence prior: penalise the squared difference between underlying feature values at times $t$ and $t+1$.
        
%         \item Can also penalise separate underlying factors by different time scales, and the difference in these time scales could help in disentangling the explanatory factors.
        
%         \item Would also expect that some factors would be represented by collections of numbers rather than a single scalar.
%         These collections of number would be expected to tend to move together  (see ``structured sparsity penalities'').
%     \end{itemize}


    
%     \item Disentangling features
%     \begin{itemize}
%         \item Low-level features recovered. Then subsets of low level-features used to give higher-level features with greater invariance.
%     \end{itemize}
% \end{itemize}

%%%%%%%%%%%%%%%%%%%%%%%%%%%%%%
\subsection{Disentangled representations}\label{sec:Disentangled representations}

Many of these potential priors can be viewed as ways for the learner to separate out some of the underlying factors of variation in the data so that certain changes in the input data only change some aspects of the representation of that input data point, but do not change other aspects; representations with this property are called \textit{disentangled representations}.
\textit{Disentangled RL} aims to produce representations that disentangle as many factors of some original data as possible, while discarding as little information as possible from the original data \autocite{Bengio2013}.

%%%%%%%%%%%%%%%%%%%%%%%%%%%%%%
\subsection{Generalisation}

As stated previously, the smoothness property allows learning algorithms to use the principle of local generalisation by using training data to map out the variations in the, assumed to be, smooth target function.
Local generalisation relies on local interpolation between neighbouring training examples to make predictions for unseen data.
However, this smoothness assumption falls down when it encounters data with many interacting underlying features, which are relevant to the prediction of the target function, because the number of changes in the target function commonly grows exponentially with the number of relevant interacting underlying features when the data are represented in the raw input space; this is a case of the \textit{curse of dimensionality}.

%%%%%%%%%%%%%%%%%%%%%%%%%%%%%%
\subsection{Abstraction through deep architectures}

\textit{Deep architectures} have a natural hierarchy, which can lead to abstract representations as more abstract concepts are constructed in terms of less abstract concepts.
More abstract concepts are generally invariant to local changes of the initial input data compared to less abstract concepts.
This generally makes representation that capture more abstract concepts highly non-linear functions of the raw input.

Deep architectures also promote the re-use of features, which encourages the construction of more abstract representation at deeper levels of the architecture.
Also, since the number of ways to use features at different depths of a deep architecture grows exponentially as the depth of the architecture grows (\textit{i.e.}, the number of ways to use different parts of the architecture grows exponentially with the depth of the architecture), using deep architectures can give improvements in computational efficiency (fewer nodes to visit for the model give its output) and statistical efficiency (fewer model parameters to learn), which both improve computational performance.

%%%%%%%%%%%%%%%%%%%%%%%%%%%%%
\section{Paradigms for RL}\label{sec:Paradigms for RL}

In this section we provide an overview of some approaches to RL:
\begin{enumerate}
    \item \textbf{Probabilistic models.}
    This approach to RL seeks to directly parametrise the decoding map (\textit{i.e.}, the generative map).        % #TODO: change name to more general version ?
    
    \item \textbf{Parametrising a representation function.}
    This approach to RL uses seeks to directly parametrise the encoding map.
    
    \item \textbf{Manifold learning.}
    This approach to RL seeks to characterise lower-dimensional regions in the input space, which can be approximated by manifolds.
    
    \item \textbf{State representation learning.}
    This approach to RL deals with an agent, which is embedded in the world, attempting to learn a representation of the world around it.
\end{enumerate}

%%%%%%%%%%%%%%%%%%%%%%%%%%%%%
\subsection{Probabilistic models}

In the probabilistic modelling approach, RL is viewed as the process of recovering a set of latent random variables that describe a distribution of observed data.
These methods seek to learn a generative map. 
Let $x$ the observed data (\textit{i.e.}, the input data), and let $h$ be the hidden latent variables.
We can treat $p(x, h)$ as a probabilistic model of the joint space of the latent variables and the observed data.
An inference process is used to determine the probability distribution $p(h \mid x)$ of the latent variable given the observed data; in statistical interference, $p(h \mid x)$ is known as the \textit{posterior distribution}.
Learning is interpreted as estimating a set of model parameters that (locally) maximises the regularised likelihood of the training data.
Feature vectors are derived from the posterior distribution, usually by taking the expectation, the marginal probability, or the most likely value of the latent variables\footnote{See section 6 of \autocite{Bengio2013} for more information}.

% There are two main modelling paradigms when considering the inference of latent variables: directed graphical methods and undirected graphical methods.
% These methods differ in how they parametrise the joint distribution $p(x,h)$.

In many scenarios, the posterior distribution from probabilistic modelling techniques can become very complicated and intractable, and so sampling or approximate inference techniques become necessary; these techniques can lead to a build up of errors in the probabilistic representation \autocite{Bengio2013}.

%%%%%%%%%%%%%%%%%%%%%%%%%%%%%
\subsection{Parametrising a representation function}

An alternative method of representation learning is to learn a direct encoding, which is a parametric map from the observations (inputs) to their representations.

The learning of this paramedic mapping is usually performed in the auto-encoder framework.
The (basic) auto-encoder framework consists of an \textit{encoder}, which is a feature-extracting function $f_{\theta}$ that computes a feature vector $h=f_{\theta}(x)$ from an input $x$, and a \textit{decoder} $g_{\theta}$, which maps the feature vector back into the input space to produce a reconstruction $r=g_{\theta}(f_{\theta}(x)$.
During learning, the set of parameters $\theta$ of the encoder and decoder are simultaneous trained to minimise a reconstruction error $L(x,r)$ over training examples.
For a data set $\{x^{(1)}, x^{(2)}, ..., x^{(T)}\}$ of data points $x^{(t)}$, the feature-vectors $h^{(t)}$ (\textit{i.e.}, representations) are computed as $h^{(t)} = f_{\theta}(x^{(t)}(x^{(t)})$, and reconstructed as $r^{(t)}=g_{\theta}(f_{\theta}(x^{(t)})$; the encoder and decoder are then trained to find values of the parameters $\theta$ that minimise the reconstruction error $\sum_{t} L(x^{(t)}, g_{\theta}(f_{\theta}(x^{(t)}))$ \autocite{Bengio2013}.

%%%%%%%%%%%%%%%%%%%%%%%%%%%%%%
\subsection{Manifold learning}\label{sec:manifold-learning}

Manifold learning seeks to utilise the \textit{manifold hypothesis} as a prior.
The manifold hypothesis states that real-world data $x \in X$ presented in high-dimensional input spaces $R^{d_{x}}$ are expected to concentrate in the vicinity of a manifold $M$, which has a dimensionality $d_{M}$ where $d_{M} << d_{x}$, embedded in $R^{d_{x}}$.

Once the notion of a representation has been created then the notion of the manifold $M$ follows by considering how variations in the input space $X$ cause variations in the learned representation space.
Some directions of variation in the input space (to first order approximation) may be well preserved in the manifold - these directions can be considered to be the tangent directions of the manifold - while some directions of variation in the input space (to first order approximation) may not be well preserved - these directions can be considered to be directions that are orthogonal to the manifold.
Using these ideas of orthogonal and tangent directions in the manifold, an intrinsic coordinate system can be built on the embedded manifold.

It is important to note that all data points do not need to lie on the manifold; the manifold hypothesis just expects the probability density of the data points to sharply fall off with distance from the manifold.
There might also be several disconnected data manifolds in $R^{d_{x}}$, each with its own dimension.

Data manifolds for complex real-world domains are expected to be highly non-linear.

Local tangent spaces at points along the manifold can be interpreted as capturing locally-valid transformations in the training data.
When considering the data manifold for a classification problem, taking these local transformations from a point on the data manifold are not expected to change the class output \autocite{Bengio2013}.

In the manifold learning approach, the (unsupervised) learning task can be viewed as modelling the structure of the manifold $M$.
The manifold learning approach to RL takes a geometric perspective on RL and allows the use of mathematical tools, which have been developed for work with manifolds.

%%%%%%%%%%%%%%%%%%%%%%%%%%%%%%
\subsection{Generalisation in the manifold learning paradigm}

If dimensionality reduction is desirable, disentangling the underlying factors of variation allows sensible dimensionality reduction by moving in the local directions of variation least represented by the training data towards the approximate boundary of the manifold; this boundary will have a lower dimension than the manifold and has been successfully used as a generalisation technique for various models, depending on the distance to the boundary \autocite{transtrum2015sloppiness}.
This method has had more success than globally removing the variations least represented in the training data, as with principle component analysis.


%%%%%%%%%%%%%%%%%%%%%%%%%%%%%%
\subsection{State representation learning}

State representation learning (SRL) is a particular case of representation learning that deals with an agent, which is embedded in the world, learning representations of its world states.
The goal of SRL is to produce a representation that captures features of an agent's observations, which are given by the agent's sensors, and are generated by the agent's actions (or lack of actions).

Good state representations capture the features of the world that are useful for the task(s) the agent is performing.
In good state representations these features are low dimensional, evolve through time, and are influenced by the actions of agents.

SRL has been combined with reinforcement learning to enable reinforcement learning algorithms to learn more efficiently and with improved ability to transfer learning to other tasks \autocite{Munk2016}; this improved efficiency can be crucial in real-world applications, where experimenting an action can be extremely costly.
In these cases, the reinforcement learning algorithm is given the state representation produced from the SRL algorithm as an input instead of being given raw data from the reinforcement learning agent's sensors.

It is hypothesised that the success of SRL is because the learned abstract representations are convenient to describe general priors about the world in a way that is easier for the reinforcement learning algorithm to interpret and utilise, while not being task specific; this lack of task specificity makes the state representations useful for solving a range of tasks \autocite{Bengio2013}.

It is important to note that SRL can be used as a method to learn a representation of any data distribution (\textit{e.g.}, a representation of an object), not just world states.
This can be done by considering an agent that can manipulate the data distribution through its actions during the learning process in order to learn a representation of the distribution, but that the agent itself is not embodied in the world (as is usually the case in SRL) or that the agent is embodied in the world but that the aspect of the world state that contains the agent is unchanged by the agent's actions.

%%%%%%%%%%%%%%%%%%%%%%%%%%%%%%
\subsection{Formalism}

There is a \textit{world} (sometimes called an \textit{environment}) where an agent can perform actions $a^{t} \in A$ at a time step $t$, where $A$ is the agent's \textit{action space}.
$A$ can be continuous or discrete.
Each action $a^{t} \in A$ causes the world to transition from the true \textit{world state} $w^{t} \in W$ at time $t$ to the unknown world state $w^{t+1}$ if $w^{t+1} \in W$, which it is usually assumed to be.
The agent may also receive a reward $r^{t+1}$ when it reaches state $w^{t+1}$.
The agent obtains an observation $o^{t+1} \in O$ of $w^{t+1} \in W$, where $O$ is the agent's \textit{observation space}.
$o^{t}$ is (usually) the raw information provided from the agent's sensors.
The state representation learning task is to learn representation states $z^{t} \in Z$, where $Z$ is the $K$-dimensional representation space, with characteristics similar to the true world states $w^{t}$ using only the observations $o^{t}$ \autocite{Lesort2018StateRL}.

%%%%%%%%%%%%%%%%%%%%%%%%%%%%%%
\section{SBDRL}\label{sec:SBDRL}

Inspired by their use in Physics, \autocite{Higgins2018}'s symmetry-based disentangled RL proposed that symmetries are important, disentanglable aspects of the structure of the world state that should be preserved when mapping world states to a representation.
They use group representation theory and the actions of symmetry groups to propose a formal definition of a vector space representation that reflects the symmetries of the world state with different symmetries represented by disjoint parts of the representation (symmetry-based disentangled vector representations); this means that a transformation (\textit{i.e.}, an action) that changes certain aspects of the world state, but leaves other aspects unchanged (\textit{i.e.}, a symmetry transform) should change the corresponding aspects of the disentangled representation but leave other aspects of the representation unchanged.

\autocite{caselles2019symmetry} developed symmetry-based disentangled RL further by showing, mathematically, that constructing a symmetry-based disentangled representation requires interaction with the environment (\textit{i.e.}, transitions are required instead of still samples); this is similar to ideas about active sensing.
It is important to note that this conclusion is intrinsic to the SBDRL problem and not necessarily required in the natural world.



\chapter{(OLD) Paper introduction (to be converted over)}

Artificial intelligence (AI) has progressed significantly in recent years due to massive increases in available computational power allowing the development and training of new deep learning algorithms \autocite{Amodei2018,thompson2020computational}.
However, the best-performing learning algorithms often suffer from poor data efficiency and lack the levels of robustness and generalisation that are characteristic of nature-based intelligence.
The brain appears to solve complex tasks by learning efficient, low-dimensional representations through the simplification of the tasks by focusing on the aspects of the environment that are important to the performance of each task \autocite{Niv2019,mack2020ventromedial,jha2023extracting,op2001inferotemporal,shepard1987toward,edelman1997learning}.
Furthermore, there is evidence that representations in nature and in artificial intelligence networks support generalisation and robustness \autocite{flesch2022orthogonal, Bernardi2020,ito2022compositional,momennejad2020learning,lehnert2020reward,alonso2013associative,kokkola2019double}.

In its most general form, a \textit{representation} is an encoding of data.
The encoded form of the data is usually more useful than the original data in some way.
For example, the representation of the data could be easier to transfer than the original data, it could improve the efficiency or performance of an algorithm or biological system that uses the representation \autocite{Lesort2018StateRL, Bengio2013}, or the representation could be a recreation of a partially observable data distribution through the combination of many samples of the distribution \autocite{Boutilier1996, Zhang2019}.

In machine learning the use of representations has been shown to improve generalisation capabilities and task performance across a range of areas such as computer vision, natural language processing, and reinforcement learning \autocite{Bengio2013,botteghi2022unsupervised,mohamed2022self,voulodimos2018deep}.
Representation learning algorithms can be thought of as seeking to remove the need for human-curated, task-specific feature engineering by performing useful feature engineering automatically.
Representations even emerge naturally in neural networks trained to perform multiple tasks \autocite{Johnston2023}.
\textit{Representation learning} is the process of devising a method to transform the input data into its encoded form.
\whendraft{[!This sentence is currently drafted out! - needs reference] The input data to the representation learning process can be received in full at the same time or over many samples, which can be dependent or independent.}

An \textit{agent} can interact with its world.
In this work, we consider an artificial agent interacting with a world to learn a representation of that world.
In particular, we are interested in the representation that the agent should ideally have learnt by the end of the representation learning process.
Some properties of the world can only be learnt in the representation of the agent if the agent interacts with its environment \autocite{caselles2019symmetry,thomas2018disentangling}.
Agents have an internal state (the agent's representation), which can be viewed as the agent's understanding of the world, and an external state, which is made up of everything in the world, not in the agent's internal state.
The agent accesses information about its external state through \textit{sensory states} (or \textit{sensors}) and affects the world through \textit{action states} (or \textit{actions}).
This boundary between the external and internal states is known as a Markov blanket \autocite{Palacios2020, Kirchhoff2018}.
The world can be treated as a data distribution, which the agent samples using its sensors and interacts with using its action states.
The agent continually updates its internal state using information gained from its sensors as it learns more about the world (representation learning).
The agent is attempting to create a representation of this data distribution by applying various operations to the distribution and observing the outcome of these operations.
The operations that the agent applies to the distribution are the \textit{actions} of the agent, and each sample that the agent takes of the distribution using its sensors is an \textit{observation} of the world.
For our purposes, it is important to note that the agent does not have to be embodied within the world it is interacting with; it only needs to be able to sample observations of the world using its sensors and to be able to interact with the world using its actions.

The area of representation learning that deals with representations of worlds that are influenced by the actions of agents is called \textit{state representation learning}.
State representation learning is a particular case of representation learning where features are low-dimensional, evolve over time, and are influenced by agents' actions \autocite{Lesort2018StateRL}.
More formally, an agent uses sensors to form observations of the states of the world (the observation process); these observations can then be used to create representations of the world (the inference process).
State representation learning has been combined with reinforcement learning to enable algorithms to learn more efficiently and with an improved ability to transfer learning to other tasks (improved generalisation) \autocite{munk2016learning}.
It has been hypothesised that the success of state representation learning is due to the way that the learnt abstract representations describe general priors about the world in a way that is easier to interpret and utilise while not being task-specific; this makes the representations useful for solving a range of tasks \autocite{Bengio2013}.

Reinforcement learning is a decision-making algorithm that involves an agent interacting with its environment to maximise the total amount of some reward signal it receives \autocite{sutton2018reinforcement,li2017deep,arulkumaran2017deep,nian2020review}.
We choose to explore the structure of worlds containing features found in common reinforcement learning scenarios because representations of a world from an agent interacting with that world, as found in our work, are often given to reinforcement learning algorithms to improve the performance of these reinforcement learning algorithms.

The question of what makes a `good' representation is vital to representation learning.
\autocite{Higgins2018} argues that the symmetries of a world are important structures that should be present in the representation of that world.
The study of symmetries takes us away from studying objects directly to studying the transformations of those objects and using information about these transformations of objects to discover properties about the objects themselves \autocite{Higgins2022}.
The field of Physics experienced such a paradigm shift due to the work of Emmy Noether, who proved that conservation laws correspond to continuous symmetry transformations \autocite{Noether1918}, and the field of AI is undergoing a similar shift.

The exploitation of symmetries has led to many successful deep-learning architectures.
Examples include convolutional layers \autocite{LeCun1995}, which use transitional symmetries to outperform humans in image recognition tasks \autocite{Dai2021}, and graph neural networks \autocite{Battaglia2018}, which use permutation groups.
Symmetries not only can provide a useful indicator of what an agent has learnt but incorporating symmetries into learning algorithms regularly reduces the size of the problem space, leading to greater learning efficiency and improved generalisation \autocite{Higgins2022}.
In fact, a large majority of neural network architectures have been shown to be described as stacking layers that deal with different symmetries \autocite{Bronstein2021}.
The main methods used to integrate symmetries into a representation are to build symmetries into the architecture of learning algorithms \autocite{Baek2021, Batzner2022}, use data-augmentation that encourages the model to learn symmetries \autocite{Chen2020, Kohler2020}, or to adjust the model's learning objective to encourage the representation to exhibit certain symmetries \autocite{burgess2018understanding, Jaderberg2016}.
The mathematical concept of symmetries can be abstracted to algebraic structures called \textit{groups}.

Two main types of symmetries are used in AI: \textit{invariant} symmetries, where a representation does not change when certain transformations are applied to it, and \textit{equivariant} symmetries, where the representation reflects the symmetries of the world.
Historically, learning of representations that are invariant to certain transformations has been a successful line of research \autocite{Krizhevsky2012, Hu2018, Silver2016, Espeholt2018}.
In building these invariant representations, the agent effectively learns to ignore the invariant transformation since the representation is unaffected by the transformation.
It has been suggested that this approach can lead to a more narrow intelligence, where an agent becomes good at solving a small set of tasks but struggles with data efficiency and generalisation when tasked with new learning problems \autocite{marcus2018deep, Cobbe2019}.
Instead of ignoring certain transformations, the equivariant approach attempts to preserve symmetry transformations in the agent's representation in such a  way that the symmetry transformations of the representation match the symmetry transformations of the world.
It has been hypothesised that the equivariant approach is likely to produce representations that can be reused to solve a more diverse range of tasks since no transformations are discarded \autocite{Higgins2022}.
Equivariant symmetry approaches are commonly linked with \textit{disentangling representations} \autocite{Bengio2013}, in which the agent's representation is separated into subspaces that are invariant to different transformations.
Disentangled representation learning, which aims to produce representations that separate the underlying structure of the world into disjoint parts, has been shown to improve the data efficiency of learning \autocite{raffin2019decoupling,wang2022disentangled}.

Inspired by their use in physics, symmetry-based disentangled representations (SBDRs) were proposed by \autocite{Higgins2018} as a formal mathematical definition of disentangled representations.
SBDRs are built on the assumption that the symmetries of the world state are important aspects of that world state that need to be preserved in an agent's internal representation (\textit{i.e.,} the symmetries that are present in the world state should also be present in the agent's internal representation state).
They describe symmetries of the world state as ``transformations that change only some properties of the underlying world state while leaving all other properties invariant'' 
\cite[page 1]{Higgins2018}.
For example, the $y$-coordinate of an agent moving parallel to the $x$-axis on a 2D Euclidean plane does not change.
SBDRL has gained traction in AI in recent years \autocite{Park2022learning,Quessard2020learning,Miyato2022unsupervised,Wang2022surprising,Keurti2023homomorphism,Zhu2021commutative,Wang2021self,Pfau2020disentangling,caselles2019symmetry,Mercatali2022,Marchetti2023}.
However, symmetry-based disentangled representation learning (SBDRL) only considers actions that form groups and so cannot take into account, for example, irreversible actions \autocite{Higgins2018}.
\autocite{caselles2019symmetry} showed that a symmetry-based representation cannot be learned using only a training set composed of the observations of world states; instead, a training set composed of transitions between world states, as well as the observations of the world states, is required.
In other words, SBDRL requires the agent to interact with the world.
This is in agreement with the empirically proven idea that active sensing of a world can be used to make non-invertible mappings from the world state to the representation state into invertible mappings \autocite{soatto2011steps}, and gives mathematical credence to SBDRL.

We agree with \autocite{Higgins2018} that symmetry transformations are important structures to include in an agent's representation, but want to take their work one step further: \textit{we posit that the relationships of transformations of the world due to the actions of the agent should be included in the agent's representation of the world}.
We will show that only including transformations of the actions of an agent that form groups in an agent's representation of a world would lose important information about the world since we demonstrate that features of many worlds cause transformations due to the actions of an agent that do not form group structures.
We generalise some important results, which have been put forward for worlds with transformations of the actions of an agent that form groups, to worlds with transformations of the actions of an agent that do not form groups.
We believe that including the relationships of these transformations in the agent's representation has the potential for powerful learning generalisation properties:
(a) Take the following thought experiment: Consider an agent that has learned the structure of the transformations due to its actions of a world $W$. Now consider a world $W'$, which is identical to world $W$ in every way except observations of the state of the world from a collection of the agent's sensors are shifted by a constant amount $\epsilon$ (if the sensors were light sensors, then $W$ and $W'$ would only be different in colour).
So the observations of $W'$ in state $s'$ would be given by $o_{s'} = o_{s} + \epsilon$, where $o_{s'}$ is the observation of $W'$ in state $s'$, and $o_{s}$ is the observation of $W$ in the state $s$.
The relationship between the transformations of the world due to the agent's actions is the same in both worlds, but the observations are different.
Therefore, the agent only needs to learn to adjust for the shift $\epsilon$ in the data from some of its sensors to be able to learn a good representation of $W'$.
In fact, the agent might have already learned that the transformations of the world due to particular actions cause relative changes to certain sensor values rather than depending on the raw sensor values.
(b) \whendraft{[Have this as a proof later in the paper - does it work for irreversible actions]} If the agent completely learns the structure of the world from one state $s$ then it knows it from all states of the world.
By taking any sequence of actions that go from $s$ to some new state and then applying that sequence of actions to the relations that the agent possesses for state $s$, the agent can produce the relationship between actions from the new state
(the relationship between the actions is dependent on the current state, similar to how local symmetries in physics are dependent on space-time coordinates).
\whendraft{[?tree pruning diagram - cutting off unreachable states?]}
(c) Another form of generalisation could be due to the action algebra being independent of the starting state; in other words, the relationship between transformations due to the agent's actions from one state is the same as from any other state in the world - the relationships between actions have been generalised to every state in the world.
This would allow the agent to learn the effect of its actions faster since the relationship between actions is the same in any state.
(d) In a partially observable world, previously explored relationships between actions could be extrapolated to unexplored areas.
If these areas have the same or similar structure to the relationships between actions, then the agent could generalise what it has learnt previously to unexplored parts of the world.

We also believe that a general framework for exploring the algebra of the transformations of worlds containing an agent, as proposed in this paper, has the potential to be used as a tool in the field of explainable AI.
With such a framework, we are able to predict which algebraic structures should appear in the agent's representation at the end of the learning process.
Being able to predict the structures that should be present in an agent's representation in certain worlds and using certain learning algorithms would be a powerful explanatory tool.
For example, if there is a sharp improvement in an agent's performance at a task at a certain point of learning, it could be the case that certain algebraic structures are present in the agent's representation after the sharp improvement in performance that were not present before the sharp improvement; if so, then it could be argued that the sharp increase in performance is due to the 'discovery' of the algebraic structure in the agent's representation.

\whendraft{
\noindent\rule{\textwidth}{1mm}

\textbf{MOVE THIS TO INTRO.}
The first two points of the definition define a symmetry-based representation, while the third point provides the disentanglement property \autocite{caselles2019symmetry}.
\noindent\rule{\linewidth}{1pt}
}

We aim to help answer the question of which features should be present in a `good' representation by, as suggested by \autocite{Higgins2018}, looking at the transformation properties of worlds.
However, while \autocite{Higgins2018} only considered symmetry transformations, we aim to go further and consider the full algebra of various worlds.
We propose a mathematical framework to describe the transformations of the world and, therefore, describe the features we expect to find in the representation of an artificial agent.
We derive \autocite{Higgins2018}'s SBDRs using our framework; we then use category theory to generalise elements of their work, namely their equivariance condition and their definition of disentangling, to worlds where the transformations of the world cannot be described using \autocite{Higgins2018}'s framework.
This paper aims to make theoretical contributions to representation learning and does not propose new learning algorithms.
More specifically, our contributions are the following:
\begin{enumerate}
    \item We propose a general mathematical framework for describing the transformation of worlds due to the actions of an agent that can interact with the world.
    
    \item We derive the SBDRs proposed by \autocite{Higgins2018} from our framework and in doing so identify the limitations of SBDRs in their current form.
    
    \item We use our framework to explore the structure of the transformations of worlds for classes of worlds containing features found in common reinforcement learning scenarios.
    Our contributions are to the field of representation learning and not the field of reinforcement learning.
    We also present the code used to work out the algebra of the transformations of worlds due to the actions of an agent.
    
    \item We generalise the equivariance condition and the definition of disentangling given by \autocite{Higgins2018} to worlds that do not satisfy the conditions for SBDRs.
    This generalisation is performed using category theory.
\end{enumerate}


This paper is structured as follows:
In Section ref[sec:Mathematical framework for an agent in an environment] we define our framework and then describe how it deals with generalised worlds, which consist of distinct world states connected by transitions that describe the dynamics of the world.
We define transitions and some of their relevant properties.
Then we define the actions of an agent in our framework.
In Section ref[sec:Reproducing SBDRL] our framework is used to reproduce the SBDRL framework given by \autocite{Higgins2018}.
This is achieved by defining an equivalence relation that makes the actions of an agent equivalent if the actions produce the same outcome if performed while the world is in any state.
In Section ref[sec:Beyond SBDRs], we apply our framework to worlds exhibiting common reinforcement learning scenarios that cannot be described fully using SBDRs and study the algebraic structures exhibited by the dynamics of these worlds.
In Section ref[sec:Generalising SBDRL], we generalise two important results of \autocite{Higgins2018} - the equivariance condition and the disentangling definition - to worlds with transformations with algebras that do not fit into the SBDRL paradigm.
We finish with a discussion in Section ref[sec:discussion and conclusion].

\whendraft{
\noindent\rule{\linewidth}{1pt}
\begin{proposition}
    The structure of the representation of an embodied agent depends on the actions it has available.
\end{proposition}
\begin{proof}
    Show algebra table for NSEW agent, then show algebra table for rotating + forwards agent, then prove that the algebraic structure is not isomorphic.
    \begin{itemize}
        \item Do combinations of actions correspond? For example, Rotate clockwise once, then move forwards == move east?
    \end{itemize}
\end{proof}
\noindent\rule{\linewidth}{1pt}
}


