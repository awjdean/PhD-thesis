%%%%%%%%%%%%%%%%%%%%%%%%%%%%%%%%%%%%%%%%%%%%%%%
\chapter{Action-homogeneous worlds and generalisation potential}

\draftnote{blue}{awjdean}{
Hypothesis: action-homogeneous worlds are easier to learn.
Can we design a method that an agent could use to work out the algebra for these worlds ?
Could use the (old) algorithm we developed because each time we move to a new state it's as if we're returning to where we began (kinda - states are still distinct).
}


\whendraft{
	\begin{proposition}
		If $A/\sim$ is a group, then $a' \circ a * w = w$ ($w \in W$, $a, a' \in A/\sim$) does not mean that $a' \circ a * w' = w'$ for all $w' \in W$.
	\end{proposition}
	\begin{proof}
		Proof by counter example.
		Consider the world given in Figure ?
		[insert figure here]
		The algebra of the actions of an agent with minimum actions $\{1, a, b\}$ is given by the following action Cayley table:
		[insert action Cayley table]
	\end{proof}
}

\begin{world_condition}[Action homogeneity]\label{wldcon:action-homogeneity}
	For every pair $(w_{1}, w_{2}) \in W^{2}$, there exists a bijective map $\sigma_{(w_{1},w_{2})}: W \to W$ such that $\sigma_{(w_{1},w_{2})}(w_{1})=w_{2}$ and such that:

	\begin{enumerate}
		\item for every $d \in D_{A}$ with $d: s(d) \xrightarrow{a} t(d)$, there exists a $d' \in D_{A}$ with $d': \sigma_{(w_{1}, w_{2})}(s(d)) \xrightarrow{a} \sigma_{(w_{1}, w_{2})}(t(d))$;

		      % Is this part needed --> implied by first condition due to $\sigma$ being a bijection ?
		\item for every $d \in D_{A}$ with $d: s(d) \xrightarrow{a} t(d)$, there exists a $d' \in D_{A}$ with $d': \sigma^{-1}_{(w_{1}, w_{2})}(s(d)) \xrightarrow{a} \sigma^{-1}_{(w_{1}, w_{2})}(t(d))$.
	\end{enumerate}
\end{world_condition}

World condition ref[wldcon:action-homogeneity] means that action sequences have the same result for any initial world state.
Essentially, this means that the world looks the same from any world state with respect to the relationships of actions.
We call worlds with world condition ref[wldcon:action-homogeneity] \textit{action-homogeneous worlds}.

\begin{definition}[Weak equivalence $\sim_{w}$]\label{def:weak action equivalence}
	For $a,a' \in A$ and $w \in W$, $a \sim_{w} a$ if $a * w = a' * w$ or $a$ and $a'$ are both restricted actions with respect to $w$.
\end{definition}

\begin{remark}
	In definition \ref{def:weak action equivalence}, for the case where $a, a'$ are both restricted actions, then we say $a * w = a' * w$.
	The meaning of $a * w$ where $a$ is restricted on $w$ will be defined later.
\end{remark}

\begin{proposition}
	$\sim_{w}$ is an equivalence relation.
\end{proposition}
\begin{proof}
	To show that $\sim_{w}$ is an equivalence relation, we need to show that the relation is (a) reflexive, (b) transitive, and (c) symmetric.
	(a) For a binary relation $R$ over a set $X$ to be reflexive: $x R x$ for every $x \in X$.
	For $w \in W$, if $a * w$ is defined then $a * w = a * w$ from the properties of $=$ for any $a \in A$.
	Therefore, $a \sim_{w} a$.
	(b) For a binary relation $R$ over a set $X$ to be transitive: if $a R b$ and $b R c$ then $a R c$ for all $a,b,c \in X$.
	If $a \sim_{w} a'$ and $a' \sim_{w} a''$, then $a * w = a' * w$, $a' * w = a'' * w$ for all $w \in W$.
	Combining these two equations gives $a * w = a'' * w$.
	(c) For a binary relation $R$ over a set $X$ to be reflexive: if $a R b$, then $b R a$ for all $a,b \in X$.
	If $a \sim_{w} a'$, then $a * w = a' * w$. Therefore $a' * w = a * w$, and so $a' \sim_{W} a$.
\end{proof}

For action-homogeneous worlds, the following properties hold:
\begin{enumerate}
	\item If a world is action-homogeneous, then $a \sim_{w} a'$ means $a \sim a'$ - this makes it much easier to learn the algebra of action-homogeneous worlds;

	\item The number of elements in $A/\sim$ is always equal to the number of states in the world since for any state $w \in W$, there is exactly one action $a \in A/\sim$ for which $a * w = w'$ where $w'$ is any state in $W$; therefore the number of elements in the action algebra is equal to the number of states in $W$;

	\item If $a * w_{i} = w_{j}$, then there exists a world state $w_{k} \in W$ such that $a * w_{k} = w_{i}$.
	      In other words, if $a$ is defined from one world state it is defined from all world states;

	\item If $a' * (a * w) = a * (a' * w) = w$ for any $w \in W$, then $a' * (a * w) = a * (a' * w) = w$ for all $w \in W$.
	      In other words, reversible actions imply inverse actions.
\end{enumerate}

\whendraft{
	\textbf{[Insert info about generalisation.]}
	\begin{itemize}
		\item If we know that, from a particular initial state, a particular sequence of actions has a particular outcome and we know that a different sequence of actions from the same initial state has the same outcome then we know that those two action sequences will have the same outcome in from every initial state (and therefore will be equivalent?!).
		      \begin{itemize}
			      \item Illustrate this using the counter-example - see PhD Notebook 1 notes.
		      \end{itemize}
	\end{itemize}
}
