\chapter{Algorithm (change name)}
\begin{table}[H]
	\centering
	\begin{tabularx}{\textwidth}{@{}lX@{}}
		\toprule
		\textbf{Symbol}                         & \textbf{Description}                                                                                     \\
		\midrule
		$T_{S}$                                 & The states Cayley table.                                                                                 \\
		$\mathcal{E}$ = ($L$, \; $E$, \; $\pi$) & The equivalence classes.                                                                                 \\
		$L$                                     & The set of equivalence class labels.                                                                     \\
		$E$                                     & The set of processed elements. These elements have been assigned to a equivalence class labelled by $L$. \\
		$\pi: E \to L$                          & A map $E \to L$ that sends each processed element to its equivalence class.                              \\
		$w^{*}$                                 & Initial world state.                                                                                     \\
		$\hat{A}$                               & The set of minimum actions of the agent. Elements of $\hat{A}$ are given a $\; \hat{ } \;$.              \\
		$\hat{*}$                               & The effect operator for minimum actions.                                                                 \\
		$\operatorname{Combine}$                & The combine operator that combines two sequences of minimum actions.                                     \\
		$A_{C}$                                 & The set of elements that are candidates for new equivalence classes labelling elements in $L$.           \\
		$T_{A}$                                 & The actions Cayley table.                                                                                \\
		\bottomrule
	\end{tabularx}
	\caption{Key for pseudocode.}
	\label{tab:pseudocode_key}
\end{table}



%%%%%%%%%%%%%%%%%%%%%%%%%%%%%%%%%%%%%%%%%%%%%
\section{Motivation}
\whendraft{
	\begin{enumerate}
		\item Why create the algorithm ?
		      \begin{itemize}
			      \item We wanted a way to explore structures.
		      \end{itemize}

	\end{enumerate}
}

%%%%%%%%%%%%%%%%%%%%%%%%%%%%%%%%%%%%%%%%%%%%%
\section{Description [/preliminaries?]}
\whendraft{
	\textbf{Include in algorithm description:}
	\begin{enumerate}
		\item Make up of $\mathcal{E}$.
		\item We store actions as their unique sequences of minimum actions.
		\item All we need from $\mathscr{W}$ is the set $W$ of world states and the map $\hat{*}: \hat{A} \times W \to W$.
		      We treat $\hat{*}$ as a part of $\mathscr{W}$.
		\item How we use the $\operatorname{Combine}$ operator to evaluate $\circ$.
		\item What each of elements of $\mathcal{E}$ are at the termination of the algorithm.
		\item $\operatorname{Seq} : \hat{A}^* \to (\hat{A})^n, \quad \operatorname{Seq}(a) = (\hat{a}_{n}, \hat{a}_{n-1}, \dots, \hat{a}_{1})$
		\item Explain how assignment to $\mathcal{E}$ works - assignment occurs for each of the constituents of $\mathcal{E}$.
		\item What is states Cayley table?
		\item What is actions Cayley table?
		\item Mention that the only bit of information needed from $\mathscr{W}$ is the mapping: $\hat{*}: \hat{A} \times W \to W$ that takes a minimum action and applies it to a world state.
	\end{enumerate}

	\begin{enumerate}
		\item Change name of `states Cayley table' and `actions Cayley table` ?
		      \begin{itemize}
			      \item Cayley table is already a thing.
		      \end{itemize}
	\end{enumerate}
}


%%%%%%%%%%%%%%%%%%%%%%%%%%%%%%%%%%%%%%%%%%%%%
\subsection{The initial}


The definition of $\sim$ is ...





%%%%%%%%%%%%%%%%%%%%%%%%%%%%%%%%%%%%%%%%%%%%%
\section{Mathematical preliminaries}
\whendraft{
	\begin{enumerate}
		\item Definition of Cayley table. - put this in motivation section ?
	\end{enumerate}
}


\paragraph{What is an actions Cayley table?}


\paragraph{What is a states Cayley table?}


\paragraph{}

%%%%%%%%%%%%%%%%%%%%%%%%%%%%%%%%%%%%%%%%%%%%%
\subsection{Design decisions}

\paragraph{Problem:}
Assessing equality between two actions.
\\\textit{Solution:}
When assessing equality between two actions, we use the fact that every action can be expressed as a unique composition of minimum actions.
When we store actions in $E$ or $L$ we store them as a tuple of their minimum actions: if $a = \hat{a}_{n} \circ \hat{a}_{n-1} \circ \dots \circ \hat{a}_{1}$ then we store action $a$ as $(\hat{a}_{n}, \; \hat{a}_{n-1}, \; \dots, \; \hat{a}_{1})$ when $a$ is stored in $E$ and $L$.
Then, when we need to assess if two actions are the same, we can compare their minimum action tuples.

\paragraph{Problem:}
Calculating the result of the $\circ$ operator.
\\\textit{Solution:}
The $\circ$ operator finds the result of composing two actions.
We define an operator $\operatorname{Combine}$ which combines two minimum action tuples into a single minimum action tuple as:
\begin{equation}
	\begin{aligned}
		 & \operatorname{Combine}: \hat{A}^{k} \times \hat{A}^{p} \to \hat{A}^{k+p}, \text{ such that}                                                                                                     \\
		 & \operatorname{Combine}((\hat{a}_{k}, \; \dots, \; \hat{a}_{1}), \; (\hat{b}_{p}, \; \dots, \; \hat{b}_{1})) = (\hat{a}_{k}, \; \dots, \; \hat{a}_{1}, \; \hat{b}_{p}, \; \dots, \; \hat{b}_{1})
	\end{aligned}
\end{equation}
Due to the associativity of $\circ$
\begin{equation}
	a \circ a' = \operatorname{Seq}^{-1}(\;\operatorname{Combine}(\;\operatorname{Seq}(a), \; \operatorname{Seq}(a')\;)\;)
\end{equation}
\footnote{
	Recall $\operatorname{Seq}: \hat{A}^{*} \to (\hat{A}^{*})^{n}$ such that $\operatorname{Seq}(a) = (\hat{a}_{n}, \; \dots, \; \hat{a}_{1})$ if $a = \hat{a}_{n} \circ \dots \circ \hat{a}_{1}$.
}
But since we're representing the elements of $\hat{A}^{*}$ that we come across as minimum action tuples, in our algorithm
\begin{equation}
	a \circ a' = \operatorname{Combine}(a, \; a')
\end{equation}

\paragraph{Problem:}
Calculating the effect of an action on the initial world state.
\\\textit{Solution:}
In algorithm \ref{alg:GenerateActionOutcome}, we need to calculate the effect of applying any action in $\hat{A}^{*}$ that we come across.
But how can we do this without programming the (indefinitely many) labelled transformations into our construction of the world $\mathscr{W}$?
Once again, we can use that any action $a$ can be represented by a minimum action tuple.
We apply each minimum action in order (moving from the RHS to the LHS of the tuple) to get the resulting world state of applying $a$ directly to the world state:
\begin{equation}
	\begin{aligned}
		a * w = & (\hat{a}_{n} \circ \dots \circ \hat{a}_{2} \circ \hat{a}_{1}) * w \\
		=       & \hat{a}_{n} * ( \dots (\hat{a}_{2} * (\hat{a}_{1} * w)) \dots)
	\end{aligned}
\end{equation}




%%%%%%%%%%%%%%%%%%%%%%%%%%%%%%%%%%%%%%%%%%%%%
%%%%%%%%%%%%%%%%%%%%%%%%%%%%%%%%%%%%%%%%%%%%%
%%%%%%%%%%%%%%%%%%%%%%%%%%%%%%%%%%%%%%%%%%%%%

\paragraph{Problem:}
How do we know if the algebra produced is complete?
\\\textit{Solution:}
\begin{itemize}
	\item
\end{itemize}


\paragraph{Problem:}
Dealing with undefined actions.
\\\textit{Solution:}
\begin{itemize}
	\item Describe how we implemented the undefined state.
\end{itemize}

\paragraph{Problem:}
[Something where the solution requires equivalence class breaking]
What if two actions appeared to be equivalent under $\sim$ at one point in the algorithm but are not actually equivalent under $\sim$ ?
\\\textit{Solution:}
\begin{itemize}
	\item
\end{itemize}


%%%%%%%%%%%%%%%%%%%%%%%%%%%%%%%%%%%%%%%%%%%%%
\section{Pseudocode}
\whendraft{
	\noindent\rule{\textwidth}{1mm}
	\textbf{To do:}
	\begin{enumerate}
		\item [visual] Span algorithms across entire page.
	\end{enumerate}
	\noindent\rule{\textwidth}{1mm}
}

%%%%%%%%%%%%%%%%%%%%%%%%%%%%%%%%%%%%%%%%%%%%%
\subsection{Generating states Cayley table and equivalence classes}

\begin{algorithm}[H]
	\caption{Generate states Cayley table $T_{S}$ and equivalence classes $\mathcal{E} = (L, \; E, \; \pi: E \to L)$.}
	\hrulefill
	\begin{algorithmic}[1]
		\Procedure{GenerateStatesCayley}{$\hat{A}$, \; $w^{*}$, \; $\mathscr{W}$}
		\Statex \Comment{Step 1: Initialise structures.}
		\State $\mathcal{E} \gets$ \Call{InitialiseEquivClasses}{$\hat{A}$, \; $w^{*}$, \; $\mathscr{W}$}
		\State $T_{S} \gets$ \Call{FillStateCayley}{$L$, \; $w^{*}$, \; $\mathscr{W}$}
		\State $A_{C} \gets$ \Call{FindCandidates}{$\mathcal{E}$, \; $w^{*}$, \; $\mathscr{W}$}

		\Statex \Comment{Step 2: Iteratively find and process candidates for states Cayley table row-column / equivalence class labelling elements.}
		\While{$A_{C} \neq \emptyset$}
		\For{\textbf{each} $a_{C} \in A_{C}$}
		\State $l \gets$ \Call{FindEquivElement}{$a_{C}$, \; $T_{S}$, \; $\mathcal{E}$, \; $w^{*}$, \; $\mathscr{W}$}

		\If{$l$}
		\Statex \Comment{$a_{c}$ is a member of the equivalence class $[l]_{\sim}$, so add it to that class.}
		\State $\mathcal{E} \gets (\; L, \; E \cup \{a_{C}\}, \; \pi \cup \pi' \;)$, where $\pi': \{a_{C}\} \to L$ such that $\pi'(a_{C}) = l$.

		\Else
		\Statex \Comment{Check if $a_{C}$ breaks any existing equivalence classes.}
		\State $\mathcal{E}_{B} \gets ( \; L_{B}, \; E_{B}, \; \pi_{B}: E_{B} \to L_{B} \; ) \gets$ \Call{FindBrokenEquivClasses}{$a_{C}$, \; $\mathcal{E}$, \; $w^{*}$, \; $\mathscr{W}$}

		\If{$L_{B} \neq \emptyset$}
		\Comment{Equivalence class(es) broken.}
		\State $\mathcal{E} \gets (\; L, \; E \setminus E_{B}, \; \pi|_{E \setminus E_{B}} \;)$
		\State $\mathcal{E} \gets (\; L \cup{ }L_{B}, \; E \cup E_{B}, \; \pi \cup \pi_{B} \;)$
		\EndIf

		\Statex \Comment{Create equivalence class for $a_{C}$.}
		\State $\mathcal{E} \gets (\; L \cup \{a_{C}\}, \; E \cup \{a_{C}\}, \; \pi \cup \pi' \;)$, where $\pi': \{a_{C}\} \to \{a_{C}\}$ such that $\pi'(a_{C}) = a_{C}$.
		\State Add elements of $\{ a_{C} \} \cup L_{B}$ as row-column labels in $T_{S}$.
		\State $T_{S} \gets$ \Call{FillStateCayley}{$L$, \; $w^{*}$, \; $\mathscr{W}$}
		\EndIf
		\EndFor
		\State $A_{C} \gets$ \Call{FindCandidates}{$\mathcal{E}$, \; $w^{*}$, \; $\mathscr{W}$}
		\EndWhile
		\State \Return $T_{S}, \; \mathcal{E}$
		\EndProcedure
	\end{algorithmic}
\end{algorithm}


\begin{algorithm}[H]
	\caption{Initialise equivalence classes.}
	\hrulefill
	\begin{algorithmic}[1]
		\Procedure{InitialiseEquivClasses}{$\hat{A}$, \; $w^{*}$, \; $\mathscr{W}$}
		\State $L \gets \emptyset$
		\State $E \gets \emptyset$
		\State $\pi \gets (\emptyset \to \emptyset)$
		\State $\mathcal{E} \gets (L, \; E, \; \pi)$.
		\For{\textbf{each} $\hat{a} \in \hat{A}$}
		\State $\mathcal{E} \gets$ \Call{AddToEquivClassesInitial}{$(\hat{a}, \;)$, \; $\mathcal{E}$, \; $w^{*}$, \; $\mathcal{W}$}
		\EndFor
		\State \Return $\mathcal{E}$
		\EndProcedure
	\end{algorithmic}
\end{algorithm}


\begin{algorithm}[H]
	\caption{
		Check if an element belongs to any of the equivalence classes in $\mathcal{E}$.
		If it does, then add the element to the relevant equivalence class.
		If it doesn't, then create a new equivalence class in $\mathcal{E}$ with $a$ as the class labelling element.
	}
	\hrulefill
	\begin{algorithmic}[1]
		\Procedure{AddToEquivClassesInitial}{$a$, \; $\mathcal{E}$, \; $w^{*}$, \; $\mathscr{W}$}
		\State $w_{a} \gets$ \Call{GenerateActionOutcome}{$a$, \; $w^{*}$, \; $\mathscr{W}$}
		\State $\texttt{class$\_$found} \gets \texttt{False}$

		\For{\textbf{each} $l \in L$}
		\State $w_{l} \gets$ \Call{GenerateActionOutcome}{$l$, \; $w^{*}$, \; $\mathscr{W}$}
		\If{$w_{a} = w_{l}$}
		\Statex \Comment{Add $a$ to $l$ equivalence class in $\mathcal{E}$.}
		\State $\mathcal{E} \gets (\; L, \; E \cup \{a\}, \; \pi \cup \pi' \;)$, where $\pi': \{a\} \to L$ such that $\pi'(a) = l$.
		\State $\texttt{class$\_$found} \gets \texttt{True}$
		\State \textbf{break}
		\EndIf
		\EndFor

		\If{\textbf{not} \texttt{class$\_$found}}
		\Statex \Comment{Create new equivalence class in $\mathcal{E}$ labelled by $a$.}
		\State $\mathcal{E} \gets (\; L \cup \{a\}, \; E \cup \{a\}, \; \pi \cup \pi' \;)$, where $\pi': \{a\} \to \{a\}$ such that $\pi'(a) = a$.
		\EndIf

		\State \Return $\mathcal{E}$
		\EndProcedure
	\end{algorithmic}
\end{algorithm}


\begin{algorithm}[H]
	\caption{
		Generate the outcome state of a world $\mathscr{W}$ when an action sequence $a$ is applied to the world in an initial state $w$.
	}
	\label{alg:GenerateActionOutcome2}
	\hrulefill
	\begin{algorithmic}[1]
		\Procedure{GenerateActionOutcome}{$a$, \; $w$, \; $\mathscr{W}$}
		\State $w_{a} \gets w$
		\For{$i \gets 1, \dots, n$}
		\State $w_{a} \gets \hat{a}_{i} \; \hat{\ast} \; w_{a}$ where $a = (\hat{a}_n, \; \hat{a}_{n-1}, \; \dots, \; \hat{a}_1)$ and $\hat{\ast} \in \mathscr{W}$.
		\EndFor
		\State \Return $w_{a}$
		\EndProcedure
	\end{algorithmic}
\end{algorithm}


\begin{algorithm}[H]
	\caption{
		Fill the entries of the states Cayley table.
	}
	\hrulefill
	\begin{algorithmic}[1]
		\Procedure{FillStateCayley}{$L$, \; $w^{*}$, \; $\mathscr{W}$}
		\State $T_{S} \gets$ Empty $|L| \times |L|$ table with rows and columns labelled by the elements of $L$.

		\For{\textbf{each} $l_{row} \in L$}
		\For{\textbf{each} $l_{col} \in L$}
		\State $a \gets \operatorname{Combine}(l_{col}, \; l_{row})$
		\State $w_{a} \gets$ \Call{GenerateActionOutcome}{$a$, \; $w^{*}$, \; $\mathscr{W}$}
		\State $T_{S}[l_{row}][l_{col}] \gets w_{a}$
		\EndFor
		\EndFor
		\State \Return $T_{S}$
		\EndProcedure
	\end{algorithmic}
\end{algorithm}


\begin{algorithm}[H]
	\caption{Search for elements that are candidates for new Cayley table row-column elements / equivalence class labelling elements.}
	\hrulefill
	\begin{algorithmic}[1]
		\Procedure{FindCandidates}{$\mathcal{E}$, \; $w^{*}$, \; $\mathscr{W}$}
		\State $A_{C} \gets \emptyset$
		\For{\textbf{each} $l_{row} \in L$}
		\For{\textbf{each} $l_{col} \in L$}
		\State $a_{C} \gets \operatorname{Combine}(l_{col}, \; l_{row})$
		\Statex \Comment{Check if $a_{C}$ has already been processed as a candidate element.}
		\If{$a_{C} \not\in E$}
		\State $A_{C} \gets A_{C} \cup \{a_{C}\}$
		\EndIf
		\EndFor
		\EndFor
		\State \Return $A_{C}$
		\EndProcedure
	\end{algorithmic}
\end{algorithm}


\begin{algorithm}[H]
	\caption{
		Search for an element in $T_{S}$ that is equivalent to $a_{C}$.
		If it exists, there should only be one such element.
	}
	\hrulefill
	\begin{algorithmic}[1]
		\Procedure{FindEquivElement}{$a_{C}$, \; $T_{S}$, \; $\mathcal{E}$, \; $w^{*}$, \; $\mathscr{W}$}
		\Statex \Comment{If $a_{C}$ in sorted elements, then return equiv class immediately}
		\State $(a_{C})_{row} \gets$ \Call{GenerateElementRow}{$a_{C}$, \; $\mathcal{E}$, \; $w^{*}$, \; $\mathscr{W}$}
		\State $(a_{C})_{col} \gets$ \Call{GenerateElementColumn}{$a_{C}$, \; $\mathcal{E}$, \; $w^{*}$, \; $\mathscr{W}$}

		\For{$l \in L$}
		\Statex \Comment{Retrieve the row labelled by $l$ from $T_{S}$.}
		\State $l_{row} \gets$ \Call{GetCayleyRow}{$l$, \; $T_{S}$}
		\Statex \Comment{Retrieve the column labelled by $l$ from $T_{S}$.}
		\State $l_{col} \gets$ \Call{GetCayleyColumn}{$l$, \; $T_{S}$}

		\If{$(a_{C})_{row} = l_{row}$ \textbf{and} $(a_{C})_{col} = l_{col}$}
		\State \Return $l$
		\EndIf
		\EndFor
		\EndProcedure

		\Statex
		\Procedure{GenerateElementRow}{$a_{C}$, \; $L$, \; $w^{*}$, \; $\mathscr{W}$}
		\State $a_{row} \gets \{ \; \}$
		\For{\textbf{each} $l_{col} \in L$}
		\State $a \gets \operatorname{Combine}(l_{col}, \; a_{C})$
		\State $w_{a} \gets$ \Call{GenerateActionOutcome}{$a$, \; $w^{*}$, \; $\mathscr{W}$}
		\State $a_{row}[l_{col}] \gets w_{a}$
		\EndFor
		\State \Return $a_{row}$
		\EndProcedure

		\Statex
		\Procedure{GenerateElementColumn}{$a_{C}$, \; $L$, \; $w^{*}$, \; $\mathscr{W}$}
		\hrulefill
		\State $a_{col} \gets \{ \; \}$
		\For{\textbf{each} $l_{row} \in L$}
		\State $a \gets \operatorname{Combine}(a_{C}, \; l_{row})$
		\State $w_{a} \gets$ \Call{GenerateActionOutcome}{$a$, \; $w^{*}$, \; $\mathscr{W}$}
		\State $a_{col}[l_{row}] \gets w_{a}$
		\EndFor
		\State \Return $a_{col}$
		\EndProcedure
	\end{algorithmic}
\end{algorithm}


\begin{algorithm}[H]
	\caption{
		Find equivalence classes in $\mathcal{E} = (L, \; E, \; \pi: E \to L)$ that are broken by $a_{C}$.
	}
	\hrulefill
	\begin{algorithmic}[1]
		\Procedure{FindBrokenEquivClasses}{$a_{C}$, \; $\mathcal{E}$, \; $w^{*}$, \; $\mathscr{W}$}
		\State $L_{B} \gets \emptyset$
		\State $E_{B} \gets \emptyset$
		\State $\pi_{B} \gets (\emptyset \to \emptyset)$
		\State $\mathcal{E}_{B} \gets (L_{B}, \; E, \; \pi)$
		\For{\textbf{each} $l \in L$}
		\State $a \gets \operatorname{Combine}(l, \; a_{C})$
		\State $w_{a} \gets$ \Call{GenerateActionOutcome}{$a$, \; $w^{*}$, \; $\mathscr{W}$}
		\For{\textbf{each} $b \in E$ where $\pi(b) = l$}
		\State $a' = \operatorname{Combine}(b, \; a_{C})$
		\State $w_{a'} \gets$ \Call{GenerateActionOutcome}{$a'$, \; $w^{*}$, \; $\mathscr{W}$}

		\If{$w_{a} \neq w_{a'}$}
		\Statex \Comment{Equivalence class $[l]_{\sim}$ broken by $a_{C}$ so add $b$ to $\mathcal{E}_{B}$.}
		\State \texttt{class$\_$found} $\gets$ \texttt{False}

		\For{\textbf{each} $l_{B} \in L_{B}$}
		\State $a'' \gets \operatorname{Combine}(l_{B}, \; a_{C})$
		\State $w_{a''} \gets$ \Call{GenerateActionOutcome}{$a''$, \; $w^{*}$, \; $\mathscr{W}$}

		\If{$w_{a'} = w_{a''}$}
		\Statex \Comment{Add $b$ to $l_{B}$ equivalence class in $\mathcal{E}_{B}$ labelled by $l_{B}$.}
		\State $\mathcal{E}_{B} \gets ( \; L_{B}, \; E_{B} \cup \{b\}, \; \pi \cup \pi' \; )$, where $\pi': \{b\} \to L$ such that $\pi'(b) = l_{B}$.
		\State \texttt{class$\_$found} $\gets$ \texttt{True}
		\State \textbf{break}
		\EndIf
		\EndFor
		\If{\textbf{not} \texttt{class$\_$found}}
		\Statex \Comment{Create new equivalence class in $\mathcal{E}_{B}$ labelled by $a$.}
		\State $\mathcal{E}_{B} \gets (\; L \cup \{b\}, \; E \cup \{b\}, \; \pi \cup \pi' \;)$, where $\pi': \{b\} \to \{b\}$ such that $\pi'(b) = b$.
		\EndIf
		\EndIf
		\EndFor
		\EndFor

		\State \Return $( \; L_{B}, \; E_{B}, \; \pi_{B}: E_{B} \to L_{B} \; )$
		\EndProcedure
	\end{algorithmic}
\end{algorithm}

%%%%%%%%%%%%%%%%%%%%%%%%%%%%%%%%%%%%%%%%%%%%%
\subsection{Generating actions Cayley table}

\begin{algorithm}[H]
	\caption{
		Generate actions Cayley table $T_{A}$.
	}
	\hrulefill
	\begin{algorithmic}[1]
		\Procedure{GenerateActionsCayley}{$T_{S}$, \; $\mathcal{E}$}
		\State $T_{A} \gets$ Empty $|L| \times |L|$ table with rows and columns labelled by the elements of $L$.
		\Statex \Comment{Fill actions Cayley table.}
		\For{\textbf{each} $l_{row} \in L$}
		\For{\textbf{each} $l_{col} \in L$}
		\Statex \Comment{Find class labelling equivalent to $l_{col} \circ l_{row}$ by finding the equivalence class of $\operatorname{Combine}(l_{col}, \; l_{row})$ and getting the class label.}
		\State $a \gets \operatorname{Combine}(l_{col}, \; l_{row})$
		\State $T_{A}[l_{row}][l_{col}] \gets \pi(a)$
		\EndFor
		\EndFor

		\State \Return $T_{A}$
		\EndProcedure
	\end{algorithmic}
\end{algorithm}

%%%%%%%%%%%%%%%%%%%%%%%%%%%%%%%%%%%%%%%%%%%%%
\whendraft{
	\section{Example [\textbf{To do}]}
	\begin{enumerate}
		\item Go through code step-by-step.
	\end{enumerate}
}

%%%%%%%%%%%%%%%%%%%%%%%%%%%%%%%%%%%%%%%%%%%%%
\whendraft{
	\section{Mathematical justification}
	\noindent\rule{\textwidth}{1mm}
	\textbf{To do:}
	\begin{enumerate}
		\item Proof that algorithm halts when it has all necessary elements and not before.
	\end{enumerate}
	\noindent\rule{\textwidth}{1mm}
}
